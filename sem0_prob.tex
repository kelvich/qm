%This is a LaTeX template for homework assignments
\documentclass[a4paper]{article}
\usepackage[utf8]{inputenc}
\usepackage[russian]{babel}
\usepackage{amsmath}

\begin{document}

\section*{Основные понятия теории вероятности}

\begin{enumerate}

\item Выпишите 25 первых цифр десятичного представления числа $\pi$.
\begin{enumerate}
    \item Вы выбираете наугад цифру из этого набора. Какова вероятность получить каждую из 10 цифр.
    \item Какая наиболее вероятная цифра? Медианная? Средняя?
    \item Найдите стандартное отклонение для этого распределения.
\end{enumerate}

\item Стрелка сломанного спидометра может свободно двигаться, отскакивая от обоих ограничителей. Таким образом если ударить по стрелке, то она остановится на любом угле от 0 до $\pi$.
\begin{enumerate}
    \item Каково распределение вероятности $\rho(\theta)$? Убедитесь, что полная вероятность равна единице.
    \item Вычислите $<\theta>$, $<\theta^2>$ и $\sigma$.
    \item Вычислите $<sin \theta>$, $<cos \theta>$, и $<cos^2\theta>$.
\end{enumerate}

\item Рассмотрим тот же сломанный спидометр из прошлой задачи теперь интересуясь $x$-координатой стрелки.
\begin{enumerate}
    \item Каково распределение вероятности $\rho(x)$? Нарисуйте его от $-2r$ до $2r$.
    \item Вычислите $<x>$, $<x^2>$ и $\sigma$. Как эти величины связаны с аналогичными из прошлой задачи?
\end{enumerate}

\item Рассмотрим гауссово распределение
\begin{equation*}
\rho(x) = A e^{- \lambda (x-a)^2}.
\end{equation*}
\begin{enumerate}
    \item Нормируйте это распределение на единицу.
    \item Вычислите $<x>$, $<x^2>$ и $\sigma$.
    \item Нарисуйте график.
    \item Найидте $\sigma$ в случае если $\rho(x)$ — распределение вероятности нахождения электрона в основном состоянии атома водорода.
\end{enumerate}

\item Игла Бюффона. Иголка длины $l$ бросается случайным образом на лист бумаги с начерченными параллельными линиями с расстоянием $l$ между линиями. Какова вероятность иголке пересечь линию?

\end{enumerate}

\end{document}
