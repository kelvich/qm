\documentclass[11pt,a4paper]{exam} 
\usepackage{mathtools}
\usepackage[utf8]{inputenc}
\usepackage[russian]{babel}

\newcommand{\rightchoice}{\choice +++ }


\begin{document}

\tableofcontents

\section{ ГЛАВА 1. ОСНОВНЫЕ ПРИНЦИПЫ КВАНТОВОЙ МЕХАНИКИ }
\subsection{ Математические основы квантовой механики }

\begin{questions}

\question Какой формулой определяется скалярное произведение $( {{\psi }_{1}},{{\psi }_{2}} )$ элементов, входящих в линейное пространство комплексных функций одной переменной, определенных на интервале $[a,b]$?
\begin{choices}
\choice $\int\limits_{a}^{b}{{{\psi }_{1}}^{*}(x){{\psi }_{2}}(x)dx}$      
\choice $\int\limits_{a}^{b}{{{\psi }_{1}}(x){{\psi }_{2}}^{*}(x)dx}$      
\choice $\int\limits_{a}^{b}{{{\psi }_{1}}^{*}(x){{\psi }_{2}}^{*}(x)dx}$     
\choice $\int\limits_{a}^{b}{{{\psi }_{1}}(x){{\psi }_{2}}(x)dx}$
\end{choices}

\question В линейном пространстве комплексных функций одной переменной выбран дискретный орто-нормированный базис. Две функции $\,{{\psi }_{1}}(x)$ и $\,{{\psi }_{2}}(x)$ заданы своими координатами в этом базисе $\{{{c}_{i}}\}$ и $\{{{d}_{i}}\}$ соответственно. Какой формулой определяется скалярное произведение $\left( {{\psi }_{1}},{{\psi }_{2}} \right)$?
\begin{choices}
\choice $\sum\limits_{i}{{{c}_{i}}^{*}{{d}_{i}}^{*}}$    
\choice $\sum\limits_{i}{{{c}_{i}}{{d}_{i}}^{*}}$     
\choice $\sum\limits_{i}{{{c}_{i}}^{*}{{d}_{i}}}$     
\choice $\sum\limits_{i}{{{c}_{i}}{{d}_{i}}}$
\end{choices}

\question Какой формулой правильно выражается перестановочное свойство скалярного произведения двух произвольных элементов линейного пространства?
\begin{choices}
\choice $\left( {{\psi }_{1}},{{\psi }_{2}} \right)=\left( {{\psi }_{2}},{{\psi }_{1}} \right)$     
\choice $\left( {{\psi }_{1}},{{\psi }_{2}} \right)={{\left( {{\psi }_{2}},{{\psi }_{1}} \right)}^{*}}$          
\choice $\left( {{\psi }_{1}},{{\psi }_{2}} \right)=-\left( {{\psi }_{2}},{{\psi }_{1}} \right)$   
\choice $\left( {{\psi }_{1}},{{\psi }_{2}} \right)=-{{\left( {{\psi }_{2}},{{\psi }_{1}} \right)}^{*}}$
\end{choices}

\question Выбрать неравенство, справедливое для любой функции $\psi $ 
\begin{choices}
\choice $\left( \psi ,\psi  \right)<0$     
\choice $\left( \psi ,\psi  \right)\le 0$          
\choice $\left( \psi ,\psi  \right)>0$    
\choice $\left( \psi ,\psi  \right)\ge 0$
\end{choices}

\question Выбрать правильное равенство (здесь ${{\psi }_{1}}$, ${{\psi }_{2}}$ и ${{\psi }_{3}}$ - про-извольные элементы линейного пространства, ${{\alpha }_{1}}$ и ${{\alpha }_{2}}$ - произвольные комплексные числа).
\begin{choices}
\choice $\left( ({{\alpha }_{1}}{{\psi }_{1}}+{{\alpha }_{2}}{{\psi }_{2}}),{{\psi }_{3}} \right)={{\alpha }_{1}}\left( {{\psi }_{1}},{{\psi }_{3}} \right)+{{\alpha }_{2}}\left( {{\psi }_{2}},{{\psi }_{3}} \right)$
\choice $\left( ({{\alpha }_{1}}{{\psi }_{1}}+{{\alpha }_{2}}{{\psi }_{2}}),{{\psi }_{3}} \right)={{\alpha }_{1}}{{\left( {{\psi }_{1}},{{\psi }_{3}} \right)}^{*}}+{{\alpha }_{2}}{{\left( {{\psi }_{2}},{{\psi }_{3}} \right)}^{*}}$
\choice $\left( ({{\alpha }_{1}}{{\psi }_{1}}+{{\alpha }_{2}}{{\psi }_{2}}),{{\psi }_{3}} \right)=\alpha _{1}^{*}\left( {{\psi }_{1}},{{\psi }_{3}} \right)+\alpha _{2}^{*}\left( {{\psi }_{2}},{{\psi }_{3}} \right)$
\choice все вышеперечисленное неверно
\end{choices}

\question Оператор $\hat{A}$, действующий в некотором линейном пространстве, является линейным, если для любых элементов ${{\psi }_{1}}$и ${{\psi }_{2}}$ и произвольных комплексных чисел $\alpha $ и $\beta $ имеет место равенство:
\begin{choices}
\choice $\hat{A}\left( \alpha {{\psi }_{1}}+\beta {{\psi }_{2}} \right)={{\alpha }^{*}}\hat{A}{{\psi }_{1}}+{{\beta }^{*}}\hat{A}{{\psi }_{2}}$    
\choice $\hat{A}\left( \alpha {{\psi }_{1}}+\beta {{\psi }_{2}} \right)=\alpha \hat{A}{{\psi }_{1}}+\beta \hat{A}{{\psi }_{2}}$
\choice $\hat{A}\left( \alpha {{\psi }_{1}}+\beta {{\psi }_{2}} \right)=\alpha \hat{A}{{\psi }_{1}}^{*}+\beta \hat{A}{{\psi }_{2}}^{*}$      
\choice $\hat{A}\left( \alpha {{\psi }_{1}}+\beta {{\psi }_{2}} \right)={{\alpha }^{*}}\hat{A}{{\psi }_{1}}^{*}+{{\beta }^{*}}\hat{A}{{\psi }_{2}}^{*}$
\end{choices}

\question Какова размерность линейного пространства многочленов степени не выше пятой?
\begin{choices}
\choice 5         
\choice 6         
\choice 7         
\choice $\infty $
\end{choices}

\question Какой из перечисленных операторов, действующих в линейном пространстве комплексных функций одной переменной, является линейным?
\begin{choices}
\choice комплексного сопряжения        
\choice взятия действительной части
\choice возведения по модулю в квадрат    
\choice никакой из перечисленных
\end{choices}

\question Какой из ниже перечисленных операторов, действующих в линейном пространстве дифферен-цируемых функций одной переменной, не является линейным
\begin{choices}
\choice дифференцирования  
\choice четности    
\choice возведения в квадрат  
\choice умножения на функцию $f(x)$
\end{choices}

\question В результате действия произведения операторов дифференцирования $\hat{D}=d/dx$ и возве-дения в квадрат $\hat{S}$ на функцию ${{e}^{-x}}$ получится функция
\begin{choices}
\choice $\hat{D}\hat{S}{{e}^{-x}}=-2{{e}^{2x}}$    
\choice $\hat{D}\hat{S}{{e}^{-x}}=2{{e}^{2x}}$     
\choice $\hat{D}\hat{S}{{e}^{-x}}=-2{{e}^{-2x}}$      
\choice $\hat{D}\hat{S}{{e}^{-x}}=2x{{e}^{{{x}^{2}}}}$
\end{choices}

\question Операторы $\hat{A}$ и $\hat{B}$, действующие в некотором линейном пространстве, комму-тируют, если для любого элемента этого пространства $\psi $ имеет место равенство:
\begin{choices}
\choice $\hat{A}\hat{B}\,\psi =\hat{B}\hat{A}\,\psi $ 
\choice $\left( \hat{A}+\hat{B} \right)\psi =\left( \hat{B}+\hat{A} \right)\psi $      
\choice $\hat{A}\,\psi ={{\hat{B}}^{+}}\psi $         
\choice $\hat{A}\,\psi =\hat{B}\,\psi $
\end{choices}

\question Как определяется оператор четности $\hat{P}$, действующий в пространстве функций одной переменной
\begin{choices}
\choice $\hat{P}f(x)=-f(x)$      
\choice $\hat{P}f(x)=-f(-x)$     
\choice $\hat{P}f(x)=f(-x)$      
\choice $\hat{P}f(x)=f(x)$
\end{choices}

\question Даны три оператора, действующие в линейном пространстве функций одной переменной $x$: четности ($\hat{P}$), дифференцирования ($\hat{D}$), умножения на $x$ ($\hat{x}$). Какая пара из них не коммутирует?
\begin{choices}
\choice только $\hat{P}$ и $\hat{D}$      
\choice только $\hat{D}$ и $\hat{x}$      
\choice только $\hat{x}$ и $\hat{P}$      
\choice Все не коммутируют
\end{choices}

\question Даны три оператора, действующие в линейном пространстве функций одной переменной $x$: четности ($\hat{P}$), возведения в квадрат ($\hat{S}$), умножения на $x$ ($\hat{x}$). Какая пара из них коммутирует?
\begin{choices}
\choice только $\hat{P}$ и $\hat{S}$      
\choice только $\hat{S}$ и $\hat{x}$      
\choice только $\hat{x}$ и $\hat{P}$      
\choice Все коммутируют
\end{choices}

\question Коммутатор операторов $d/dx$ и умножения на функцию $f(x)$ равен
\begin{choices}
\choice оператору $d/dx$,                 
\choice оператору умножения на функцию $f(x)$
\choice оператору умножения на функцию ${f}'(x)$      
\choice оператору ${{d}^{2}}/d{{x}^{2}}$
\end{choices}

\question Коммутатор операторов четности $\hat{P}$ и умножения на функцию $f(x)$ равен
\begin{choices}
\choice оператору $\hat{P}$   
\choice оператору $f(x)\hat{P}$       
\choice оператору $f(-x)\hat{P}$ 
\choice опера-тору $[f(-x)-f(x)]\hat{P}$
\end{choices}

\question Оператором, обратным оператору четности является
\begin{choices}
\choice оператор четности           
\choice оператор однократного дифференцирования
\choice оператор возведения в квадрат        
\choice оператор двукратного дифференцирования.
\end{choices}

\question Произведение операторов $d/dx$ и ${{d}^{2}}/d{{x}^{2}}$ на произвольную функцию $f(x)$ действует так:
\begin{choices}
\choice $\frac{{{d}^{2}}f(x)}{d{{x}^{2}}}\frac{df(x)}{dx}$     
\choice $\frac{{{d}^{2}}f(x)}{d{{x}^{2}}}+\frac{df(x)}{dx}$    
\choice $\frac{{{d}^{3}}f(x)}{d{{x}^{3}}}$      
\choice $\frac{{{d}^{4}}f(x)}{d{{x}^{4}}}$
\end{choices}

\question Сумма операторов $d/dx$ и ${{d}^{2}}/d{{x}^{2}}$ на произвольную функцию $f(x)$ дейст-вует так:
\begin{choices}
\choice $\frac{{{d}^{2}}f(x)}{d{{x}^{2}}}\frac{df(x)}{dx}$     
\choice $\frac{{{d}^{2}}f(x)}{d{{x}^{2}}}+\frac{df(x)}{dx}$    
\choice $\frac{{{d}^{3}}f(x)}{d{{x}^{3}}}$      
\choice $\frac{{{d}^{4}}f(x)}{d{{x}^{4}}}$
\end{choices}

\question $\hat{A}$ - произвольный оператор. Как определяется оператор ${{\hat{A}}^{-1}}$, обратный оператору $\hat{A}$ (здесь $\hat{I}$ - единичный оператор, $\hat{O}$ - нулевой оператор)?
\begin{choices}
\choice $\hat{A}{{\hat{A}}^{-1}}={{\hat{A}}^{-1}}\hat{A}=\hat{O}$       
\choice $\hat{A}+{{\hat{A}}^{-1}}={{\hat{A}}^{-1}}+\hat{A}=\hat{O}$
\choice $\hat{A}{{\hat{A}}^{-1}}={{\hat{A}}^{-1}}\hat{A}=\hat{I}$       
\choice $\hat{A}+{{\hat{A}}^{-1}}={{\hat{A}}^{-1}}+\hat{A}=\hat{I}$
\end{choices}

\question Чему равен оператор ${{\left( \hat{A}\hat{B} \right)}^{-1}}$?
\begin{choices}
\choice ${{\hat{A}}^{-1}}{{\hat{B}}^{-1}}$      
\choice ${{\hat{A}}^{-1}}+{{\hat{B}}^{-1}}$     
\choice ${{\hat{A}}^{-1}}-{{\hat{B}}^{-1}}$     
\choice ${{\hat{B}}^{-1}}{{\hat{A}}^{-1}}$
\end{choices}

\question Оператору, действующему в двумерном линейном пространстве, при некотором выборе бази-са отвечает матрица $\left( \begin{matrix}
   1 & 2  \\
   3 & 4  \\
\end{matrix} \right)$. Какая функция получится при действии этого оператора на функцию, имеющую в выбранном базисе координаты 1 и 2
\begin{choices}
\choice нулевая         
\choice с координатами 5 и 13 
\choice с координатами 5 и 11      
\choice с координатами 6 и 11
\end{choices}

\question Операторам $\hat{A}$ и $\hat{B}$, действующим в двумерном пространстве отвечают сле-дующие матрицы: $\hat{A}=\left( \begin{matrix}
   1 & 2  \\
   3 & 4  \\
\end{matrix} \right)$, $\hat{B}=\left( \begin{matrix}
   5 & 6  \\
   7 & 8  \\
\end{matrix} \right)$. Какая матрица отвечает оператору, равному сумме операторов $\hat{A}$ и $\hat{B}$?
\begin{choices}
\choice $\left( \begin{matrix}
   7 & 10  \\
   12 & 16  \\
\end{matrix} \right)$      
\choice $\left( \begin{matrix}
   6 & 8  \\
   12 & 16  \\
\end{matrix} \right)$      
\choice $\left( \begin{matrix}
   6 & 8  \\
   10 & 12  \\
\end{matrix} \right)$      
\choice $\left( \begin{matrix}
   7 & 9  \\
   10 & 12  \\
\end{matrix} \right)$
\end{choices}

\question Операторам $\hat{A}$ и $\hat{B}$, действующим в двумерном пространстве отвечают следую-щие матрицы: $\hat{A}=\left( \begin{matrix}
   1 & 2  \\
   3 & 4  \\
\end{matrix} \right)$, $\hat{B}=\left( \begin{matrix}
   5 & 6  \\
   7 & 8  \\
\end{matrix} \right)$. Какая матрица отвечает оператору, равному произведению операторов $\hat{A}$ и $\hat{B}$?
\begin{choices}
\choice $\left( \begin{matrix}
   21 & 28  \\
   33 & 54  \\
\end{matrix} \right)$      
\choice $\left( \begin{matrix}
   19 & 21  \\
   44 & 56  \\
\end{matrix} \right)$      
\choice $\left( \begin{matrix}
   20 & 29  \\
   42 & 51  \\
\end{matrix} \right)$      
\choice $\left( \begin{matrix}
   19 & 22  \\
   43 & 50  \\
\end{matrix} \right)$
\end{choices}

\question Какая из нижеперечисленных матриц совпадает с обратной?
\begin{choices}
\choice $\left( \begin{matrix}
   1 & 0  \\
   0 & -1  \\
\end{matrix} \right)$      
\choice $\left( \begin{matrix}
   0 & 1  \\
   1 & 0  \\
\end{matrix} \right)$      
\choice $\left( \begin{matrix}
   0 & -i  \\
   i & 0  \\
\end{matrix} \right)$      
\choice все
\end{choices}

\question Чему равен коммутатор двух операторов, заданных матрицами $\hat{A}=\left( \begin{matrix}
   1 & 2  \\
   3 & 4  \\
\end{matrix} \right)$ и $\hat{B}=\left( \begin{matrix}
   3 & 4  \\
   5 & 6  \\
\end{matrix} \right)$?
\begin{choices}
\choice $\left[ \hat{A}\hat{B} \right]=\left( \begin{matrix}
   8 & 13  \\
   18 & 24  \\
\end{matrix} \right)$      
\choice $\left[ \hat{A}\hat{B} \right]=0$    
\choice $\left[ \hat{A}\hat{B} \right]=\left( \begin{matrix}
   3 & 6  \\
   10 & 15  \\
\end{matrix} \right)$      
\choice $\left[ \hat{A}\hat{B} \right]=\left( \begin{matrix}
   7 & 15  \\
   12 & 22  \\
\end{matrix} \right)$
\end{choices}

\question Чему равен коммутатор двух операторов, заданных матрицами $\hat{A}=\left( \begin{matrix}
   1 & 2  \\
   3 & 4  \\
\end{matrix} \right)$) и $\hat{B}=\left( \begin{matrix}
   2 & 4  \\
   6 & 8  \\
\end{matrix} \right)$?
\begin{choices}
\choice $\left[ \hat{A}\hat{B} \right]=\left( \begin{matrix}
   8 & 13  \\
   18 & 24  \\
\end{matrix} \right)$      
\choice $\left[ \hat{A}\hat{B} \right]=0$    
\choice $\left[ \hat{A}\hat{B} \right]=\left( \begin{matrix}
   3 & 6  \\
   10 & 15  \\
\end{matrix} \right)$      
\choice $\left[ \hat{A}\hat{B} \right]=\left( \begin{matrix}
   7 & 15  \\
   12 & 22  \\
\end{matrix} \right)$
\end{choices}

\question Известен результат действия оператора $\hat{A}$, действующего в двумерном пространстве, на базисные элементы ${{e}_{1}}$ и ${{e}_{2}}$ ортонормированного базиса: $\hat{A}{{e}_{1}}={{a}_{1}}{{e}_{1}}$, $\hat{A}{{e}_{2}}={{a}_{2}}{{e}_{2}}$, где ${{a}_{1}}$ и ${{a}_{2}}$ - некоторые числа. Какова матрица оператора $\hat{A}$?
\begin{choices}
\choice $\left( \begin{matrix}
   {{a}_{1}} & {{a}_{1}}{{a}_{2}}  \\
   {{a}_{2}}{{a}_{1}} & {{a}_{2}}  \\
\end{matrix} \right)$      
\choice $\left( \begin{matrix}
   {{a}_{1}}{{a}_{2}} & 0  \\
   0 & {{a}_{2}}{{a}_{1}}  \\
\end{matrix} \right)$      
\choice $\left( \begin{matrix}
   {{a}_{1}} & 0  \\
   0 & {{a}_{2}}  \\
\end{matrix} \right)$      
\choice $\left( \begin{matrix}
   0 & {{a}_{1}}  \\
   {{a}_{2}} & 0  \\
\end{matrix} \right)$
\end{choices}

\question Известен результат действия оператора $\hat{A}$, действующего в двумерном пространстве, на базисные элементы ${{e}_{1}}$ и ${{e}_{2}}$ ортонормированного базиса: $\hat{A}{{e}_{1}}={{\alpha }_{1}}{{e}_{1}}+{{\alpha }_{2}}{{e}_{2}}$, $\hat{A}{{e}_{2}}={{\beta }_{1}}{{e}_{1}}+{{\beta }_{2}}{{e}_{2}}$, где ${{\alpha }_{1}}$, ${{\alpha }_{2}}$, ${{\beta }_{1}}$ и ${{\beta }_{2}}$ - некоторые числа. Какова матрица оператора $\hat{A}$?
\begin{choices}
\choice $\left( \begin{matrix}
   {{\alpha }_{1}} & {{\beta }_{1}}  \\
   {{\alpha }_{2}} & {{\beta }_{2}}  \\
\end{matrix} \right)$      
\choice $\left( \begin{matrix}
   {{\alpha }_{1}} & {{\alpha }_{2}}  \\
   {{\beta }_{1}} & {{\beta }_{2}}  \\
\end{matrix} \right)$      
\choice $\left( \begin{matrix}
   {{\alpha }_{1}} & {{\beta }_{1}}  \\
   {{\beta }_{2}} & {{\alpha }_{2}}  \\
\end{matrix} \right)$      
\choice $\left( \begin{matrix}
   {{\alpha }_{1}} & {{\beta }_{2}}  \\
   {{\beta }_{1}} & {{\alpha }_{2}}  \\
\end{matrix} \right)$
\end{choices}

\question В некотором линейном пространстве выбран ортонормированный базис $\{{{f}_{i}}\}$. Какой формулой определяются матричные элементы матрицы некоторого линейного оператора $\hat{A}$, действующего в этом пространстве?
\begin{choices}
\choice ${{a}_{kn}}=({{f}_{k,}}\hat{A}{{f}_{n}})$  
\choice ${{a}_{kn}}={{({{f}_{k,}}{{\hat{A}}^{+}}{{f}_{n}})}^{*}}$ 
\choice ${{a}_{kn}}=(\hat{A}{{f}_{k,}}{{f}_{n}})$  
\choice ${{a}_{kn}}={{({{\hat{A}}^{+}}{{f}_{k,}}{{f}_{n}})}^{*}}$
\end{choices}

\question Чему равен интеграл $\int\limits_{-\infty }^{+\infty }{f(x)\delta ({{x}^{2}}-4)dx}$ (где $\delta (...)$ - дельта-функция, $f(x)$ - непрерывная функция координаты)?
\begin{choices}
\choice $4\left( f(2)+f(-2) \right)$      
\choice $4\left( f(2)-f(-2) \right)$      
\choice $\frac{1}{4}\left( f(2)+f(-2) \right)$      
\choice $\frac{1}{4}\left( f(2)-f(-2) \right)$
\end{choices}

\question Чему равна функция $\delta (ax)$ (где $\delta (...)$ - дельта-функция, $a$ - некоторое действи-тельное число)?
\begin{choices}
\choice $a\delta (x)$      
\choice $\frac{1}{a}\delta (x)$     
\choice $|a|\delta (x)$    
\choice $\frac{1}{|a|}\delta (x)$ 
\end{choices}

\question Чему равен интеграл $\int\limits_{-\infty }^{+\infty }{\cos \pi x\left( \delta (x)+\delta (x-1)+\delta (x-2) \right)dx}$ (где $\delta (...)$ - дельта-функция)?
\begin{choices}
\choice $1$       
\choice 2         
\choice -1        
\choice -2
\end{choices}

\question Оператор $\hat{A}$ является эрмитовым, если для произвольных элементов $f$ и $g$ линей-ного пространства выполнено равенство
\begin{choices}
\choice $\left( f,\hat{A}g \right)=\left( \hat{A}f,g \right)$     
\choice $\left( f,\hat{A}g \right)=\left( g,\hat{A}f \right)$     
\choice $\left( f,\hat{A}g \right)=\left( \hat{A}g,f \right)$      
\choice $\left( f,\hat{A}g \right)={{\left( f,\hat{A}g \right)}^{*}}$
\end{choices}

\question Оператор $\hat{A}$ является эрмитовым, если для произвольных элементов $f$ и $g$ линей-ного пространства выполнено равенство
\begin{choices}
\choice $\left( f,\hat{A}g \right)=\left( \hat{A}f,g \right)$     
\choice $\left( f,\hat{A}g \right)=\left( g,\hat{A}f \right)$     
\choice $\left( f,\hat{A}g \right)=\left( \hat{A}g,f \right)$      
\choice $\left( f,\hat{A}g \right)=\left( \hat{A}f,\hat{A}g \right)$
где $\left( ...\ ,\ ... \right)$ - скалярное произведение элементов пространства.
\end{choices}

\question Дан оператор $\hat{A}$. Как определяется оператор ${{\hat{A}}^{T}}$, транспонированный оператору $\hat{A}$?
\begin{choices}
\choice $\left( f,\hat{A}g \right)=\left( {{{\hat{A}}}^{T}}f,g \right)$ 
\choice $\left( f,\hat{A}g \right)=\left( g,{{{\hat{A}}}^{T}}f \right)$ 
\choice $\left( f,\hat{A}g \right)=\left( f,\hat{A}{{}^{T}}g \right)$      
\choice $\left( f,\hat{A}g \right)=\left( {{{\hat{A}}}^{T}}g,f \right)$
($f$ и $g$  произвольные элементы линейного пространства).
\end{choices}

\question Оператор $\hat{A}$ называется эрмитовым, если 
\begin{choices}
\choice $\hat{A}={{\hat{A}}^{-1}}$     
\choice ${{\hat{A}}^{+}}={{\hat{A}}^{-1}}$      
\choice $\hat{A}={{\hat{A}}^{+}}$      
\choice $\hat{A}={{\hat{A}}^{T}}$
\end{choices}

\question Какой из трех операторов – четности ($\hat{P}$), дифференцирования ($\hat{D}$), умножения на координату $x$ ($\hat{x}$), действующих в линейном пространстве функций одной действитель-ной переменной $x$, не является эрмитовым?
\begin{choices}
\choice только $\hat{P}$      
\choice только $\hat{D}$      
\choice только $\hat{x}$      
\choice все эти опера-торы не эрмитовы
\end{choices}

\question Чему равен оператор ${{\left( \hat{A}\hat{B} \right)}^{+}}$?
\begin{choices}
\choice ${{\hat{A}}^{+}}{{\hat{B}}^{+}}$     
\choice ${{\hat{A}}^{+}}+{{\hat{B}}^{+}}$    
\choice ${{\hat{A}}^{+}}-{{\hat{B}}^{+}}$    
\choice ${{\hat{B}}^{+}}{{\hat{A}}^{+}}$
\end{choices}

\question $\hat{A}$ и $\hat{B}$ - эрмитовы операторы. В каком случае оператор $\hat{A}\hat{B}$ бу-дет эрмитовым?
\begin{choices}
\choice если $\hat{A}$ и $\hat{B}$ линейны   
\choice если $\hat{A}$ и $\hat{B}$ коммутируют  
\choice если $\hat{A}={{\hat{B}}^{-1}}$   
\choice всегда
\end{choices}

\question $\hat{A}$ - произвольный оператор. Какой из нижеперечисленных операторов будет неэрми-товым?
\begin{choices}
\choice ${{\hat{A}}^{+}}A$    
\choice $i\left( \hat{A}-{{{\hat{A}}}^{+}} \right)$      
\choice $\hat{A}-{{\hat{A}}^{+}}$      
\choice $\hat{A}+{{\hat{A}}^{+}}$
\end{choices}

\question Чему равен оператор ${{\left( {{{\hat{A}}}^{+}} \right)}^{+}}$?
\begin{choices}
\choice $\hat{A}$    
\choice ${{\hat{A}}^{++}}$    
\choice ${{\hat{A}}^{+2}}$    
\choice ${{\hat{A}}^{-1}}$
\end{choices}

\question Пусть $\hat{A}$ - эрмитов оператор. Чему равен оператор ${{\left( i\hat{A} \right)}^{+}}$, где $i$ - мнимая единица
\begin{choices}
\choice $\hat{A}$       
\choice $-i\hat{A}$        
\choice $i\hat{A}$         
\choice $-\hat{A}$ 
\end{choices}

\question Какое соотношение имеет место для матричных элементов ${{a}_{ij}}$ матрицы некоторого эрмитового оператора?
\begin{choices}
\choice ${{a}_{ij}}={{a}_{ji}}$  
\choice ${{a}_{ij}}=a_{ij}^{*}$  
\choice ${{a}_{ij}}=a_{ji}^{*}$  
\choice ${{a}_{ij}}=a_{ji}^{-1}$
где $a_{ji}^{-1}$ - матричные элементы матрицы обратного оператора.
\end{choices}

\question Пусть оператор $\hat{B}$ является оператором, эрмитово сопряженным оператору $\hat{A}$: $\hat{B}={{\hat{A}}^{+}}$. Как связаны друг с другом матричные элементы операторов $\hat{A}$ и $\hat{B}$?
\begin{choices}
\choice ${{b}_{ij}}={{a}_{ji}}$  
\choice ${{b}_{ij}}=a_{ij}^{*}$  
\choice ${{b}_{ij}}=a_{ji}^{*}$  
\choice ${{b}_{ij}}=a_{ji}^{-1}$
\end{choices}

\question Какая из четырех матриц является матрицей эрмитового оператора?
\begin{choices}
\choice $\left( \begin{matrix}
   1 & i  \\
   i & 2  \\
\end{matrix} \right)$      
\choice $\left( \begin{matrix}
   1 & 1  \\
   1 & 2-i  \\
\end{matrix} \right)$      
\choice $\left( \begin{matrix}
   1 & i  \\
   -i & 2  \\
\end{matrix} \right)$      
\choice $\left( \begin{matrix}
   1 & 2  \\
   3 & 4  \\
\end{matrix} \right)$
\end{choices}

\question Какая из четырех матриц не является матрицей эрмитового оператора?
\begin{choices}
\choice $\left( \begin{matrix}
   1 & 1-i  \\
   2+i & 2  \\
\end{matrix} \right)$      
\choice $\left( \begin{matrix}
   1 & 2  \\
   2 & 4  \\
\end{matrix} \right)$      
\choice $\left( \begin{matrix}
   2 & 5+2i  \\
   5-2i & 0  \\
\end{matrix} \right)$      
\choice $\left( \begin{matrix}
   1 & 1  \\
   1 & 1  \\
\end{matrix} \right)$
\end{choices}

\question Оператор $id/dx$, действующий в пространстве функций, заданных на интервале $(-\infty ,+\infty )$, в котором определено скалярное произведение, является
\begin{choices}
\choice эрмитовым 
\choice унитарным    
\choice совпадающим со своим обратным  
\choice нели-нейным
\end{choices}

\question Даны три оператора: четности ($\hat{P}$), дифференцирования ($\hat{D}$), умножения на мнимую единицу $i$ ($\hat{i}$), действующие в линейном пространстве функций одной переменной $x$ ($x\in (-\infty ,\infty )$), в котором определено скалярное произведение. Какой из них является эр-митовым?
\begin{choices}
\choice только $\hat{P}$      
\choice только $\hat{D}$      
\choice только $\hat{i}$      
\choice Все 
\end{choices}

\question Собственные значения любого эрмитового оператора 
\begin{choices}
\choice положительны    
\choice отрицательны       
\choice вещественны     
\choice чисто мнимы
\end{choices}

\question Собственные функции эрмитового оператора, отвечающие различным собственным значени-ям 
\begin{choices}
\choice ортогональны             
\choice отличаются числовым сомножителем
\choice совпадают             
\choice являются комплексно сопряженными друг к другу
\end{choices}

\question Собственное значение оператора вырождено, если
\begin{choices}
\choice этому значению отвечает одна собственная функция
\choice этому значению отвечает две или более линейно независимых собственных функции
\choice это значение равно нулю
\choice это значение не равно нулю 
\end{choices}

\question Собственное значение эрмитового оператора вырождено. Какое утверждение относительно линейно независимых собственных функции, отвечающих одному и тому же собственному значе-нию,  справедливо?
\begin{choices}
\choice всегда ортогональны
\choice никогда не ортогональны 
\choice могут быть выбраны так, чтобы были ортогональны
\choice являются комплексно сопряженными друг к другу
\end{choices}

\question Пусть ${{f}_{1}}(x)$ и ${{f}_{2}}(x)$ - собственные функции некоторого оператора, отве-чающие собственным значениям ${{a}_{1}}$ и ${{a}_{2}}$. Функция ${{C}_{1}}{{f}_{1}}(x)+{{C}_{2}}{{f}_{2}}(x)$ (${{C}_{1}}\quad \text{}\quad {{C}_{2}}$ - произвольные числа)
\begin{choices}
\choice будет собственной функцией того же оператора
\choice будет собственной функцией того же оператора только в том случае, когда ${{a}_{1}}={{a}_{2}}$
\choice никогда не будет собственной функцией того же оператора
\choice будет собственной или несобственной этого оператора в зависимости от функций ${{f}_{1}}(x)$ и ${{f}_{2}}(x)$.
\end{choices}

\question Если в качестве базиса в линейном пространстве выбрать собственные функции некоторого эрмитового оператора, то его матрица 
\begin{choices}
\choice равна единичной    
\choice кратна единичной      
\choice диагональная 
\choice антисимметричная
\end{choices}

\question Для любого эрмитового оператора $\hat{A}$, действующего в некотором линейном простран-стве, можно выбрать такой базис, в котором матрица оператора $\hat{A}$ является
\begin{choices}
\choice кратной единичной  
\choice антисимметричной   
\choice нулевой      
\choice диагональной
\end{choices}

\question Спектр собственных значений оператора является дискретным. Это значит, что
\begin{choices}
\choice оператор имеет бесконечное количество собственных значений
\choice оператор имеет бесконечное количество положительных собственных значений
\choice собственные значения можно пересчитать, даже если их число бесконечно
\choice собственным значением является любое число из некоторого интервала значений.
\end{choices}

\question Спектр собственных значений оператора является непрерывным. Это значит, что
\begin{choices}
\choice оператор не имеет собственных значений
\choice оператор имеет конечное число собственных значений
\choice собственные значения можно пересчитать, даже если их число бесконечно
\choice собственным значением является любое число из некоторого интервала значений
\end{choices}

\question Сколько собственных значений имеет оператор, заданный матрицей $\left( \begin{matrix}
   1 & i  \\
   -i & 2  \\
\end{matrix} \right)$?
\begin{choices}
\choice одно   
\choice два    
\choice три    
\choice четыре
\end{choices}

\question Чему равны собственные значения оператора, заданного матрицей $\left( \begin{matrix}
   0 & i  \\
   -i & 0  \\
\end{matrix} \right)$?
\begin{choices}
\choice +1 и –1      
\choice 0 и 1     
\choice 0 и –1       
\choice –$i$ и +$i$
\end{choices}

\question Оператор задан матрицей
$\left( \begin{matrix}
   1 & 2 & 3i  \\
   2 & 5 & 2-3i  \\
   -3i & 2+3i & 6  \\
\end{matrix} \right)$.
Какие утверждения относительно свойств собственных значений этого оператора справедливы?
\begin{choices}
\choice все три чисто мнимы      
\choice все три вещественны
\choice все четыре чисто мнимы      
\choice все четыре вещественны
\end{choices}

\question Чему равны собственные значения единичного оператора?
\begin{choices}
\choice 1 и -1 
\choice 1и 2      
\choice 1      
\choice 0
\end{choices}

\question Чему равны собственные значения оператора четности?
\begin{choices}
\choice 1 и -1 
\choice 1и 2      
\choice 1      
\choice 0
\end{choices}

\question Какова кратность вырождения собственного значения -1 оператора четности, действующего в линейном пространстве многочленов степени не выше шестой?
\begin{choices}
\choice 3         
\choice 4         
\choice 5         
\choice 6
\end{choices}

\question Какие функции являются собственными функциями оператора четности, отвечающими собст-венному значению -1?
\begin{choices}
\choice Все четные
\choice все нечетные 
\choice все функции, положительные во всех точках
\choice все функции, отрицательные во всех точках
\end{choices}

\question Является ли функция $\cos x$ собственной функцией оператора четности, действующего в пространстве функций одной переменной $x$, и если да, то какому собственному значению она от-вечает?
\begin{choices}
\choice не является  
\choice является, +1 
\choice является, -1 
\choice является, 0
\end{choices}

\question Является ли функция $\cos \vartheta $ ($\vartheta $ - полярный угол сферической системы ко-ординат) собственной функцией оператора четности, действующего в пространстве функций трех переменных $x,y,z$, и если да, то какому собственному значению она отвечает?
\begin{choices}
\choice не является  
\choice является, +1 
\choice является, -1 
\choice является, 0
\end{choices}

\question Является ли функция $\sin \vartheta {{e}^{-i\varphi }}$ ($\vartheta $ и $\varphi $ - полярный и азимутальный угол сферической системы координат) собственной функцией оператора четности, действующего в пространстве функций трех переменных $x,y,z$, и если да, то какому собственному значению она отвечает?
\begin{choices}
\choice не является  
\choice является, +1 
\choice является, -1 
\choice является, 0
\end{choices}

\question Является ли функция $\cos \varphi $ ($\varphi $ - азимутальный угол сферической системы ко-ординат) собственной функцией оператора четности, действующего в пространстве функций трех переменных $x,y,z$, и если да, то какому собственному значению она отвечает?
\begin{choices}
\choice не является  
\choice является, +1 
\choice является, -1 
\choice является, 0
\end{choices}

\question Привести матрицу оператора к диагональному виду значит
\begin{choices}
\choice заменить элементы, находящиеся не на главной диагонали, нулями
\choice выбрать другой базис, в котором матрица оператора кратна единичной
\choice выбрать другой базис, в котором матрица оператора равна единичной
\choice выбрать другой базис, в котором матрица оператора диагональна
\end{choices}

\question Что такое шпур матрицы оператора?
\begin{choices}
\choice сумма всех элементов его матрицы
\choice сумма всех элементов, находящихся в матрице этого оператора выше главной диагонали
\choice сумма всех элементов, находящихся в матрице этого оператора ниже главной диагонали
\choice сумма всех диагональных элементов матрицы этого оператора 
\end{choices}

\question При изменении базиса линейного пространства матрицы всех операторов изменяются. Какие характеристики матрицы инварианты по отношению к выбору базиса?
\begin{choices}
\choice шпур
\choice произведение диагональных элементов
\choice разность элементов, расположенных выше и ниже главной диагонали
\choice ни одна из перечисленных
\end{choices}

\question При изменении базиса линейного пространства матрицы всех операторов изменяются. Какие характеристики матрицы инварианты по отношению к выбору базиса?
\begin{choices}
\choice сумма всех элементов матрицы
\choice произведение всех элементов матрицы
\choice произведение диагональных элементов
\choice ни одна из перечисленных
\end{choices}

\question При изменении базиса линейного пространства матрицы всех операторов изменяются. Какие характеристики матрицы инварианты по отношению к выбору базиса?
\begin{choices}
\choice сумма всех элементов матрицы
\choice детерминант
\choice произведение элементов матрицы
\choice ни одна из перечисленных
\end{choices}

\question Оператор задан матрицей $\left( \begin{matrix}
   1 & 2 & 3  \\
   2 & 3 & 4  \\
   3 & 4 & 5  \\
\end{matrix} \right)$. Чему равна сумма всех собственных значений этого оператора?
\begin{choices}
\choice 1      
\choice 3      
\choice 5      
\choice 9
\end{choices}

\question Оператор задан матрицей $\left( \begin{matrix}
   1 & 2  \\
   2 & 3  \\
\end{matrix} \right)$. Чему равно произведение всех собственных значений этого оператора?
\begin{choices}
\choice 1      
\choice 2      
\choice -1     
\choice -2
\end{choices}

\question Даны матрицы четырех операторов. У какого из них сумма собственных значений равна ну-лю?
\begin{choices}
\choice $\left( \begin{matrix}
   1 & -1  \\
   -1 & 1  \\
\end{matrix} \right)$      
\choice $\left( \begin{matrix}
   1 & 1  \\
   1 & 1  \\
\end{matrix} \right)$      
\choice $\left( \begin{matrix}
   1 & 1  \\
   1 & -1  \\
\end{matrix} \right)$      
\choice $\left( \begin{matrix}
   -1 & 1  \\
   1 & -1  \\
\end{matrix} \right)$
\end{choices}

\question Если эрмитовы операторы $\hat{A}$ и $\hat{B}$ коммутируют, то 
\begin{choices}
\choice любая собственная функция одного из операторов является также собственной функцией другого оператора
\choice операторы не имеют общих собственных функций
\choice операторы имеют общие собственные функции, число которых меньше размерности пространства, в котором действуют эти операторы
\choice существует полная система общих собственных функций этих операторов
\end{choices}

\question Оператор $\hat{A}$, действующий в некотором линейном пространстве, является унитарным, если
\begin{choices}
\choice $\hat{A}={{\hat{A}}^{+}}$      
\choice $\hat{A}={{\left( {{{\hat{A}}}^{+}} \right)}^{+}}$     
\choice $\hat{A}={{\hat{A}}^{-1}}$     
\choice ${{\hat{A}}^{+}}={{\hat{A}}^{-1}}$
\end{choices}

\question Оператор $\hat{A}$, действующий в некотором линейном пространстве, является унитарным. Чему равно произведение $\hat{A}{{\hat{A}}^{+}}$?
\begin{choices}
\choice оператору $\hat{A}$   
\choice оператору ${{\hat{A}}^{2}}$ 
\choice нулевому оператору 
\choice единич-ному оператору
\end{choices}

\question Оператор $\hat{A}$, действующий в некотором линейном пространстве, является унитарным. Чему равен определитель матрицы этого оператора?
\begin{choices}
\choice 1      
\choice -1     
\choice ${{e}^{i\alpha }}$, где $\alpha $ - некоторое действительное число    
\choice ${{e}^{i\alpha }}$, где $\alpha $ - некоторое чисто мнимое число
\end{choices}

\question Пусть для любых двух элементов ${{f}_{1}}$ и ${{f}_{2}}$ линейного пространства и неко-торого оператора $\hat{A}$, действующего в этом пространстве, выполнено условие $({{f}_{1}},{{f}_{2}})=(\hat{A}{{f}_{1}},\hat{A}{{f}_{2}})$ (в таком случае говорят, что оператор $\hat{A}$ сохраняет скалярное произведение элементов пространства). Какое свойство оператора $\hat{A}$ обязательно имеет место?
\begin{choices}
\choice он нелинейный      
\choice он эрмитов     
\choice он унитарный     
\choice он совпадает со своим обратным
\end{choices}

\question Для любых двух элементов ${{f}_{1}}$ и ${{f}_{2}}$ линейного пространства и некоторого оператора $\hat{A}$, действующего в этом пространстве, выполнено условие $({{f}_{1}},{{f}_{2}})=({{\hat{A}}^{+}}{{f}_{1}},\hat{A}{{f}_{2}})$. Какое свойство оператора $\hat{A}$ обязательно имеет место?
\begin{choices}
\choice он нелинейный       
\choice он эмитов   
\choice он унитарный     
\choice он совпадает со своим обратным
\end{choices}

\question Собственные значения унитарного оператора
\begin{choices}
\choice действительны            
\choice чисто мнимы
\choice квадраты их модулей равны 1    
\choice все равны 1
\end{choices}

\question Оператору, действующему в двумерном линейном пространстве, при некотором выборе бази-са отвечает матрица $\left( \begin{matrix}
   \cos \alpha  & -\sin \alpha   \\
   \sin \alpha  & \cos \alpha   \\
\end{matrix} \right)$, где $\alpha $ - некоторый угол. Этот оператор
\begin{choices}
\choice эрмитов      
\choice унитарный    
\choice совпадает со своим обратным    
\choice четный
\end{choices}

\question Какая из нижеперечисленных матриц унитарна?
\begin{choices}
\choice $\frac{1}{\sqrt{2}}\left( \begin{matrix}
   1 & -1  \\
   1 & 1  \\
\end{matrix} \right)$      
\choice $\frac{1}{\sqrt{2}}\left( \begin{matrix}
   1 & -1  \\
   -1 & 1  \\
\end{matrix} \right)$      
\choice $\frac{1}{\sqrt{2}}\left( \begin{matrix}
   -1 & 1  \\
   1 & -1  \\
\end{matrix} \right)$      
\choice ни одна из них
\end{choices}

\question Пусть в некотором линейном пространстве выбраны два ортонормированных базиса $\{{{f}_{i}}\}$ и $\{{{e}_{i}}\}$. Матрицей перехода от одного базиса к другому называется матрица ${{S}_{ki}}$, составленная из коэффициентов разложения одной базисной системы по другой: ${{f}_{i}}=\sum\limits_{k}{{{S}_{ki}}{{e}_{k}}}$. Матрица ${{S}_{ki}}$
\begin{choices}
\choice эрмитова     
\choice унитарна     
\choice совпадает со своей обратной    
\choice диагональна
\end{choices}

\question Существует ли оператор (кроме единичного), который является одновременно эрмитовым и унитарным, и если да, то что это за оператор?
\begin{choices}
\choice не существует         
\choice да, оператор поворота вектора вокруг некоторой оси
\choice да, оператор четности    
\choice да, оператор умножения на координату
\end{choices}

\end{questions}


\subsection{ Общие свойства собственных функций и собственных значений операторов физических ве-личин }


\begin{questions}

\question Оператор некоторой физической величины имеет дискретный спектр собственных значений. Интеграл от квадрата модуля собственных функций
\begin{choices}
\choice сходится
\choice расходится
\choice для некоторых состояний сходится, для некоторых расходится
\choice это зависит от оператора
\end{choices}

\question Оператор некоторой физической величины имеет непрерывный спектр собственных значений. Интеграл от квадрата модуля собственных функций
\begin{choices}
\choice сходится
\choice расходится
\choice для некоторых состояний сходится, для некоторых расходится
\choice это зависит от оператора
\end{choices}

\question Оператор некоторой физической величины действует в пространстве функций трех переменных и имеет непрерывный спектр собственных значений. Какова размерность линейного пространства, в котором действует оператор?
\begin{choices}
\choice 3      
\choice 4      
\choice $\infty $       
\choice мало информации для ответа
\end{choices}

\question Какова размерность нормированных на единицу собственных функций ${f_n}(x)$ оператора физической величины $A$, имеющего дискретный спектр собственных значений?
\begin{choices}
\choice        
\choice 
\choice безразмерны     
\choice нормировать собственные функции ${f_n}(x)$ на единицу нельзя
\end{choices}

\question Какова размерность нормированных на единицу собственных функций ${f_a}(x)$ оператора физической величины $A$, имеющего непрерывный спектр собственных значений?
\begin{choices}
\choice        
\choice 
\choice безразмерны     
\choice нормировать собственные функции ${f_a}(x)$ на единицу нельзя
\end{choices}

\question Какова размерность нормированных на дельта-функцию от $a$ собственных функций ${f_n}(x)$ оператора физической величины $A$, имеющего дискретный спектр собственных значе-ний?
\begin{choices}
\choice        
\choice 
\choice безразмерны     
\choice нормировать собственные функции ${f_n}(x)$ на дельта-функцию нельзя
\end{choices}

\question Какова размерность нормированных на дельта-функцию от $a$ собственных функций ${f_a}(x)$ оператора физической величины $A$, имеющего непрерывный спектр собственных значе-ний?
\begin{choices}
\choice        
\choice 
\choice безразмерны     
\choice нормировать собственные функции ${f_a}(x)$ на дельта-функцию нельзя
\end{choices}

\question Эрмитов оператор действует в линейном пространстве функций одной переменной. Какое ут-верждение всегда справедливо для собственных функций ${f_1}(x)$ и ${f_2}(x)$ этого оператора, от-вечающих разным собственным значениям?
\begin{choices}
\choice ${f_1}(x)$ и ${f_2}(x)$ отличаются множителем    
\choice $\int {{f_1}^*(x){f_2}^*(x)xdx = } 0$
\choice $\int {{f_1}(x){f_2}(x)dx = } 0$              
\choice $\int {{f_1}^*(x){f_2}(x)dx = } 0$
\end{choices}

\question Линейно независимые собственные функции эрмитового оператора, отвечающие одному и то-му же вырожденному собственному значению
\begin{choices}
\choice ортогональны
\choice не ортогональны
\choice вообще говоря, не ортогональны, но могут быть выбраны так, чтобы были ортогональны
\choice это зависит от оператора
\end{choices}

\question Что означает утверждение, что выбор собственных функций оператора, отвечающих вырож-денному собственному значению, является неоднозначным?
\begin{choices}
\choice эти функции определены с точностью до множителя
\choice любые их линейные комбинации также будут собственными функциями
\choice уравнение на собственные значения в этом случае не имеет решений
\choice произведение двух собственных функций также будет собственной функцией
\end{choices}

\question Какой формулой выражается условие нормировки собственных функций ${f_n}(x)$ оператора физической величины, имеющего дискретный спектр собственных значений ${a_n}$,
\begin{choices}
\choice $\int {{f_n}(x)f_k^*(x')dxdx' = } {\delta _{nk}}$ (где ${\delta _{nk}}$ - дельта-символ)
\choice $\int {{f_n}(x)f_k^*(x)dx = } {\delta _{nk}}$(где ${\delta _{nk}}$ - дельта-символ)
\choice $\sum\limits_n {{f_n}^*(x){f_n}(x')}  = \delta (x - x')$ (где $\delta (x - x')$ - дельта-функция)
\choice $\sum\limits_{nn'} {{f_n}^*(x){f_{n'}}(x')}  = \delta (x - x')$ (где $\delta (x - x')$ - дельта-функция)
\end{choices}

\question Какой формулой выражается условие нормировки собственных функций ${f_a}(x)$ оператора физической величины, имеющего непрерывный спектр собственных значений $a$?
\begin{choices}
\choice $\int {{f_a}(x)f_{a'}^*(x')dxdx' = } \delta (a - a')$ (где $\delta (a - a')$ - дельта-функция)
\choice $\int {{f_a}(x)f_a^*(x')da = } \delta (x - x')$ (где $\delta (x - x')$ - дельта-функция)
\choice $\int {{f_a}^*(x){f_{a'}}(x)} dx = \delta (a - a')$ (где $\delta (a - a')$ - дельта-функция)
\choice $\int {{f_a}(x)f_{a'}^*(x')dada' = } \delta (x - x')$ (где $\delta (x - x')$ - дельта-функция)
\end{choices}

\question Какой формулой выражается условие полноты системы собственных функций ${f_n}(x)$ опе-ратора физической величины, имеющего дискретный спектр собственных значений?
\begin{choices}
\choice $\int {{f_n}(x)f_{n'}^*(x)dx = } {\delta _{nn'}}$ (где ${\delta _{nn'}}$ - дельта-символ)
\choice $\sum\limits_n {{f_n}^*(x){f_n}(x')}  = \delta (x - x')$ (где $\delta (x - x')$ - дельта-функция)
\choice $\int {{f_n}^*(x){f_{n'}}(x')dxdx' = } {\delta _{nn'}}$ (где ${\delta _{nn'}}$ - дельта-символ)
\choice $\sum\limits_{nn'} {{f_n}^*(x){f_{n'}}(x')}  = \delta (x - x')$ (где $\delta (x - x')$ - дельта-функция)
\end{choices}

\question Какой формулой выражается условие полноты системы собственных функций ${f_a}(x)$ опе-ратора физической величины, имеющего непрерывный спектр собственных значений?
\begin{choices}
\choice $\int {{f_a}(x)f_{a'}^*(x')dxdx' = } \delta (a - a')$ (где $\delta (a - a')$ - дельта-функция)
\choice $\int {{f_a}(x)f_a^*(x')da = } \delta (x - x')$ (где $\delta (x - x')$ - дельта-функция)
\choice $\int {{f_a}^*(x){f_{a'}}(x)} dx = \delta (a - a')$ (где $\delta (a - a')$ - дельта-функция)
\choice $\int {{f_a}(x)f_{a'}^*(x')dada' = } \delta (x - x')$ (где $\delta (x - x')$ - дельта-функция)
\end{choices}

\question Оператор физической величины $A$ имеет непрерывный спектр собственных значений $a$ и собственных функций ${f_a}(x)$ (${f_a}(x)$ нормированы на $\delta $-функцию от $a$). Разложение волновой функции частицы $\Psi (x,t)$ по собственным функциям имеет вид $\Psi (x,t) = \int {C(a,t){f_a}(x)da} \,$, где $C(a)$ - коэффициенты разложения. Какую размерность имеют функции $C(a)$?
\begin{choices}
\choice $\frac{1}{{\sqrt {\rm{A}} }}$     
\choice     
\choice безразмерны     
\choice 
\end{choices}

\question Разложение волновой функции квантовой системы $\Psi (x)$ по ортонормированным собст-венным функциям оператора некоторой физической величины $\,{f_n}(x)$ имеет вид $\Psi (x) = \frac{1}{{\sqrt {15} }}\,{f_1}(x) - \frac{1}{{\sqrt 3 }}{f_2}(x) + i\sqrt {\frac{6}{{15}}} \,{f_3}(x) + \frac{1}{{\sqrt 5 }}{f_4}(x)$. Что моно сказать о нормировке этой функции?
\begin{choices}
\choice ненормирована         
\choice нормирована на 1
\choice нормирована на –1     
\choice нормирована на 2
\end{choices}

\question Разложение волновой функции квантовой системы $\Psi (x)$ по ортонормированным собст-венным функциям оператора некоторой физической величины ${f_n}(x)$ имеет вид $\Psi (x) = \frac{1}{{\sqrt 2 }}{f_1}(x) - \frac{1}{{\sqrt 2 }}{f_2}(x) + \frac{1}{{\sqrt 2 }}\,{f_3}(x) - \frac{1}{{\sqrt 2 }}{f_4}(x)$. Что моно сказать о нормировке этой функции?
\begin{choices}
\choice нормирована на 0         
\choice нормирована на 1
\choice нормирована на 2         
\choice нормирована на 4
\end{choices}

\question Оператор физической величины $A$ имеет дискретный спектр собственных значений ${a_n}$ и собственных функций ${f_n}(x)$. Какая из ниже перечисленных формул выражает собой разложе-ние волновой функции частицы $\Psi (x,t)$ по собственным функциям?
\begin{choices}
\choice $\Psi (x,t) = \sum\limits_n {{C_n}(t){f_n}(x)} $       
\choice $\Psi (x,t) = \int {{C_n}(t){f_n}(x)} dn$
\choice ${f_n}(x) = \int {{C_n}(t)\Psi (x,t)dt} $        
\choice ${f_n}(x) = \int {\sum\limits_k {{C_{kn}}(t)\Psi (x,t)} } dt$
\end{choices}

\question Оператор физической величины $A$ имеет непрерывный спектр собственных значений $a$ и собственных функций ${f_a}(x)$. Какая из ниже перечисленных формул выражает собой разложение волновой функции частицы $\Psi (x,t)$ по собственным функциям?
\begin{choices}
\choice $\Psi (x,t) = \int {C(a,t){f_a}(x)da} \,$     
\choice $\Psi (x,t) = \int {C(x',t){f_a}(x')} dx'$
\choice ${f_a}(x) = \int {{C_a}(x,t)\Psi (x,t)dt} \,$ 
\choice ${f_a}(x) = \int {{C_a}(x)\Psi (x,t)da} $
\end{choices}

\question Оператор физической величины $A$ имеет дискретный спектр собственных значений ${a_n}$ и собственных функций ${f_n}(x)$. По какой из нижеперечисленных формул можно найти коэффи-циенты разложения нормированной волновой функции частицы $\Psi (x,t)$ по собственным функци-ям ${f_n}(x)$?
\begin{choices}
\choice ${C_n}(t) = \int {f_n^*(x)\Psi (x,t)dt} $        
\choice ${C_n}(t) = \int {f_n^*(x)\Psi (x,t)dx} $
\choice $\Psi (x,t) = \sum\limits_n {{C_n}(t){f_n}(x)} $       
\choice ${C_n}(t) = \int {f_n^*(x){\Psi ^*}(x,t)dx} $
\end{choices}

\question Оператор физической величины $A$ имеет дискретный спектр собственных значений ${a_n}$ и собственных функций ${f_n}(x)$. По какой из нижеперечисленных формул можно найти коэффи-циенты разложения нормированной волновой функции частицы $\Psi (x,t)$ по собственным функци-ям ${f_n}(x)$?
\begin{choices}
\choice ${C_n}(t) = \int {{f_n}(x)\Psi (x,t)dx} $        
\choice ${C_n}(t) = \int {{f_n}(x){\Psi ^*}(x,t)dx} $
\choice ${C_n}(t) = \int {f_n^*(x)\Psi (x,t)dx} $        
\choice ${f_n}(x) = \int {\sum\limits_k {{C_{kn}}(t)\Psi (x,t)} } dt$
\end{choices}

\question Оператор физической величины $A$ имеет непрерывный спектр собственных значений $a$ и собственных функций ${f_a}(x)$ (нормированных на $\delta $-функцию от $a$). По какой из нижепе-речисленных формул можно найти коэффициенты разложения нормированной волновой функции частицы $\Psi (x,t)$ по собственным функциям ${f_a}(x)$?
\begin{choices}
\choice $C(a,t) = \int {\Psi (x,t)f_a^*(x)dx} $    
\choice $C(a,t) = \int {\Psi (x,t)f_{a'}^*(x)da'} $
\choice $C(a,t) = \int {\Psi (x,t){f_a}^*(x)dt} $     
\choice $C(a,t) = \int {{\Psi ^*}(x,t){f_a}^*(x)dt} $
\end{choices}

\question Разложение волновой функции квантовой системы $\Psi (x,t)$ по ортонормированным собст-венным функциям ${f_n}(x)$ оператора некоторой физической величины имеет вид $\Psi (x,t) = \sum {{C_n}(t)} {f_n}(x)\,$. Функция $\Psi (x,t)$ нормирована на единицу: $\int {|\Psi (x,t){|^2}dx = 1} $. Что можно сказать об условии нормировки коэффициентов?
\begin{choices}
\choice $\sum\limits_n {{C_n}}  = \sqrt 2 $     
\choice ${\sum\limits_n {\left| {{C_n}} \right|} ^2} = 1$         
\choice ${\sum\limits_n {\left| {{C_n}} \right|} ^2} = 2$      
\choice $\sum\limits_n {{C_n}}  = 2$
\end{choices}

\question Разложение волновой функции квантовой системы $\Psi (x,t)$ по ортонормированным собст-венным функциям ${f_n}(x)$ оператора некоторой физической величины имеет вид $\Psi (x,t) = \sum {{C_n}(t)} {f_n}(x)\,$. Функция $\Psi (x,t)$ нормирована на 2: $\int {|\Psi (x,t){|^2}dx = 2} $. Что мож-но сказать об условии нормировки коэффициентов?
\begin{choices}
\choice ${\sum\limits_n {\left| {{C_n}} \right|} ^2} = \sqrt 2 $     
\choice ${\sum\limits_n {\left| {{C_n}} \right|} ^2} = 1$         
\choice ${\sum\limits_n {\left| {{C_n}} \right|} ^2} = 2$      
\choice $\sum\limits_n {{C_n}}  = 2$
\end{choices}

\question Оператор величины $A$ имеет непрерывный спектр собственных значений $a$ и собствен-ных функций ${f_a}(x)$ (${f_a}(x)$ нормированы на $\delta $-функцию от $a$). Разложение нормированной на единицу волновой функции частицы $\Psi (x,t)$ по собственным функциям имеет вид $\Psi (x,t) = \int {C(a,t){f_a}(x)da} \,$, где $C(a)$ - коэффициенты разложения. Что можно сказать об интеграле $\int {|C(a,t){|^2}da} \,$?
\begin{choices}
\choice расходится         
\choice сходится и равен 1
\choice сходится и равен 2 
\choice мало информации для ответа
\end{choices}

\question Квантовомеханическая система находится в состоянии с нормированной на единицу волновой функцией $\Psi (x,t)$. Разложение этой функции по нормированным собственным функциям оператора физической величины $\hat A$, имеющего дискретный спектр собственных значений, имеет вид: $\Psi (x,t) = \sum {{C_n}(t)} {f_n}(x)\,$, где ${f_n}(x)$ собственная функция оператора $\hat A$, отвечающая собственному значению ${a_n}$. Пусть собственное значение ${a_n}$ не вырождено. Вероятность того, что в момент времени $t$ величина $A$ имеет значение ${a_n}$, равна
\begin{choices}
\choice ${C_n}(t)$   
\choice $\left| {{C_n}(t)} \right|$ 
\choice ${\left| {{C_n}(t)} \right|^2}$   
\choice ${\mathop{\rm Re}\nolimits} \left( {{C_n}(t)} \right)$
\end{choices}

\question Квантовомеханическая система находится в состоянии с нормированной на единицу волно-вой функцией $\Psi (x,t)$. Разложение этой функции по нормированным собственным функциям оператора физической величины $\hat A$, имеющего дискретный спектр собственных значений, имеет вид: $\Psi (x,t) = \sum {{C_n}(t)} {f_n}(x)\,$, где ${f_n}(x)$ собственная функция оператора $\hat A$, отвечающая собственному значению ${a_n}$. Чему равно среднее значение результатов многих измерений величины $A$ в момент времени $t$?
\begin{choices}
\choice $\overline {A(t)}  = {\sum\limits_n {{a_n}\left| {{C_n}(t)} \right|} ^2}$      
\choice $\overline {A(t)}  = \sum\limits_n {{a_n}\left| {{C_n}(t)} \right|} $
\choice $\overline {A(t)}  = \sum\limits_n {{a_n}{{\left| {{C_n}(t)} \right|}^3}} $    
\choice $\overline {A(t)}  = {\sum\limits_n {{a_n}{\mathop{\rm Re}\nolimits} \left( {{C_n}(t)} \right)} ^2}$
\end{choices}

\question Нормированная волновая функция частицы в некоторый момент времени равна $\Psi (x)$. Оператор физической величины $A$ имеет дискретный спектр невырожденных собственных значе-ний ${a_n}$ и отвечающих им собственных функций ${f_n}(x)$ (${f_n}(x)$ нормированы на едини-цу). Какой из нижеперечисленных формул определяется вероятность того, что при измерении вели-чины $A$ будет обнаружено $A = {a_k}$?
\begin{choices}
\choice ${\left| {\int {\Psi (x){f_k}(x)dx} } \right|^2}$         
\choice ${\left| {\int {{\Psi ^2}(x){f_k}(x)dx} } \right|^2}$
\choice ${\left| {\int {{\Psi ^*}(x){f_k}(x)dx} } \right|^2}$        
\choice ни одной из этих формул
\end{choices}

\question Оператор физической величины $A$ имеет непрерывный спектр собственных значений $a$ и собственных функций ${f_a}(x)$ (${f_a}(x)$ нормированы на $\delta $-функцию от $a$). Разложение нормированной волновой функции частицы $\Psi (x,t)$ по собственным функциям имеет вид $\Psi (x,t) = \int {C(a,t){f_a}(x)da} \,$, где $C(a)$ - коэффициенты разложения, причем функция $C(a)$ ко-нечна во всех точках. Чему равна вероятность того, что при измерении физической величины $A$ будет получено некоторое значение $a$?
\begin{choices}
\choice $|C(a)|$     
\choice $|C(a){|^2}$    
\choice нулю      
\choice $\int {|{f_a}(x){|^2}dx} \,$
\end{choices}

\question Оператор физической величины $A$ имеет непрерывный спектр собственных значений $a$ и собственных функций ${f_a}(x)$ (${f_a}(x)$ нормированы на $\delta $-функцию от $a$). Разложение волновой функции частицы $\Psi (x,t)$ по собственным функциям имеет вид $\Psi (x,t) = \int {C(a,t){f_a}(x)da} \,$, где $C(a)$ - коэффициенты разложения, причем функция $C(a)$ конечна во всех точках. Чему равна вероятность того, что при измерении физической величины $A$ будет полу-чено некоторое значение из малого интервала $da$ вблизи значения ${a_0}$?
\begin{choices}
\choice $|C({a_0}){|^2}$      
\choice $|C({a_0}){|^2}da$    
\choice нулю, так как интервал $da$ - мал 
\choice $\int {|{f_{{a_0}}}(x){|^2}dx} \,$
\end{choices}

\question Оператор физической величины $A$ имеет непрерывный спектр собственных значений $a$ и собственных функций ${f_a}(x)$ (${f_a}(x)$ нормированы на $\delta $-функцию от $a$). Разложение волновой функции частицы $\Psi (x,t)$ по собственным функциям имеет вид $\Psi (x,t) = \int {C(a,t){f_a}(x)da} \,$, где $C(a)$ - коэффициенты разложения, причем функция $C(a)$ конечна во всех точках. Чему равно среднее значение результатов измерений физической величины $A$?
\begin{choices}
\choice $\overline A  = \int {a|C(a)|da} $         
\choice $\overline A  = \int {a|\Psi (x,t)|dx} $
\choice $\overline A  = \int {a|\Psi (x,t){|^2}dx} $     
\choice $\overline A  = \int {a|C(a){|^2}da} $
\end{choices}

\question Квантовая система описывается нормированной волновой функцией $\Psi (x,t)$. Физической величине $A$ отвечает квантово-механический оператор $\hat A$. По какой формуле можно вычис-лить среднее значение результатов многих измерений величины $A$ над ансамблем тождественных квантовых систем?
\begin{choices}
\choice $\overline A  = \int {\hat A|} \psi (x,t){|^2}dx$      
\choice $\overline A  = \int | \psi (x,t){|^2}\hat Adx$  
\choice $\overline A  = \int {{\psi ^*}(x,t)\hat A} \psi (x,t)dx$    
\choice $\overline A  = \hat A\int {|\psi (x,t){|^2}} dx$
\end{choices}

\question Оператор физической величины $A$ имеет непрерывный спектр собственных значений $a$ и собственных функций ${f_a}(x)$ (${f_a}(x)$ нормированы на $\delta $-функцию от $a$). Разложение волновой функции частицы $\Psi (x,t)$ по собственным функциям имеет вид $\Psi (x,t) = \int {C(a,t){f_a}(x)da} \,$, где $C(a)$ - коэффициенты разложения, причем функция $C(a)$ конечна во всех точках. Выбрать верное равенство
\begin{choices}
\choice $\int {a|C(a){|^2}da}  = \int {\hat A} \Psi (x)dx$        
\choice $\int {a|C(a){|^2}da}  = \int {{\Psi ^*}(x)\hat A} \Psi (x)dx$
\choice $\int {aC(a)da}  = \int {{\Psi ^*}(x)\hat A} \Psi (x)dx$        
\choice $\int {aC(a)da}  = \int {\hat A} \Psi (x)dx$
\end{choices}

\question Известны собственные значения ${a_i}$ оператора некоторой физической величины $A$ и отвечающие им нормированные на единицу собственные функции: ${a_1} \leftrightarrow {f_1}(x)$,  (две линейно независимых функции), ${a_3} \leftrightarrow {f_3}(x)$. Задано разложение нормиро-ванной волновой функции квантовой системы $\Psi (x)$ по собственным функциям $\Psi (x) = {C_1}{f_1}(x) + {C_{21}}{f_{21}}(x) + {C_{22}}{f_{22}}(x) + {C_3}{f_3}(x)$ (где $C$ - некоторые чис-ла). Измеряют физическую величину $A$. С какой вероятностью значение ${a_2}$ можно получить при измерениях, проводимых над ансамблем тождественных квантовых систем?
\begin{choices}
\choice        
\choice $w({a_2}) = |{C_{21}}/{C_{22}}{|^2}$
\choice $w({a_2}) = |{C_{21}} + {C_{22}}{|^2}$     
\choice $w({a_2}) = |{C_{21}}{|^2} + |{C_{22}}{|^2}$
\end{choices}

\question Что означает, что физическая величина $A$ имеет в некотором состоянии квантовой системы определенное значение (или, как иногда говорят, является измеримой)?
\begin{choices}
\choice результаты многократных измерений величины $A$ над ансамблем тождественных квантовых систем дадут одинаковые результаты
\choice в данном состоянии квантовой системы величину $A$ можно измерить
\choice результаты многократных измерений величины $A$ в одной и той же квантовой системе дадут одинаковые результаты
\choice величина $A$ в принципе является наблюдаемой
\end{choices}

\question Волновая функция некоторой квантовой системы в некоторый момент времени совпадает с $n$-ой собственной функцией оператора физической величины $\hat A$. При измерении физиче-ской величины $A$ в этот момент времени будут получены
\begin{choices}
\choice $n + 1$-ое и $n - 1$-ое собственные значения с одинаковыми вероятностями
\choice $n$-ое собственное значение с единичной вероятностью
\choice все собственные значения с равными вероятностями
\choice все собственные значения с номерами, меньшими или равными $n$
\end{choices}

\question Физическая величина $A$ имеет в состоянии с волновой функцией $\Psi (x,t)$ определенное значение, если
\begin{choices}
\choice $\Psi $ не зависит от времени
\choice $\Psi (x,t)$ совпадает с одной из собственных функций оператора этой физической величины $\hat A$
\choice $\Psi (x,t)$ является собственной функцией оператора Гамильтона системы
\choice $\Psi (x,t)$ является собственной функцией оператора импульса
\end{choices}

\question Собственные значения и отвечающие им нормированные собственные функции оператора некоторой физической величины $A$ равны: ${a_1} = 1$ и ${f_1}(x) = B\sin (x/a)$ (первое собственное значение и отвечающая ему собственная функция), ${a_2} = 2$ и ${f_2}(x) = B\sin (2x/a)$, ${a_3} = 3$ и ${f_3}(x) = B\sin (3x/a)$, …. (где $a$ и $B$ - некоторые числа, одинаковые для всех функций). Волновая функция частицы в некоторый момент времени равна $\Psi (x) = C\sin (2x/a)\cos (5x/a)$. Какие значения величины $A$ можно обнаружить при измерениях в этот момент времени?
\begin{choices}
\choice 1 и 2     
\choice любое целое положительное число      
\choice 2 и 5     
\choice 3 и 7
\end{choices}

\question Собственные значения и отвечающие им нормированные собственные функции оператора некоторой физической величины $A$ равны: ${a_1} = 1$ и ${f_1}(x) = B\sin (x/a)$ (первое собственное значение и отвечающая ему собственная функция), ${a_2} = 2$ и ${f_2}(x) = B\sin (2x/a)$, ${a_3} = 3$ и${f_3}(x) = B\sin (3x/a)$, …. (где $a$ и $B$ - некоторые числа, одинаковые для всех функций). Волновая функция частицы в некоторый момент времени равна $\Psi (x) = C\sin (2x/a)\cos (5x/a)$. Чему равно среднее значение величины $A$ в этот момент времени?
\begin{choices}
\choice 5      
\choice 6      
\choice 7      
\choice 8
\end{choices}

\question Собственные значения и отвечающие им нормированные собственные функции оператора некоторой физической величины $A$ равны: ${a_1} = 1$ и ${f_1}(x) = B\sin (x/a)$ (первое собственное значение и отвечающая ему собственная функция), ${a_2} = 2$ и ${f_2}(x) = B\sin (2x/a)$, ${a_3} = 3$ и ${f_3}(x) = B\sin (3x/a)$, …. (где $a$ и $B$ - некоторые числа, одинаковые для всех функций). Волновая функция частицы в некоторый момент времени равна $\Psi (x) = C\sin (2x/a)\cos (5x/a)$. Какие значения величины ${A^2}$ можно обнаружить при измерениях в этот момент времени?
\begin{choices}
\choice 4 и 25    
\choice любое целое положительное число      
\choice 9 и 49    
\choice 1 и 4
\end{choices}

\question Собственные значения и отвечающие им нормированные собственные функции оператора некоторой физической величины $A$ равны: ${a_1} = 1$ и ${f_1}(x) = B\sin (x/a)$ (первое собственное значение и отвечающая ему собственная функция), ${a_2} = 2$ и ${f_2}(x) = B\sin (2x/a)$, ${a_3} = 3$ и${f_3}(x) = B\sin (3x/a)$, …. (где $a$ и $B$ - некоторые числа, одинаковые для всех функций). Волновая функция частицы в некоторый момент времени равна $\Psi (x) = C\sin (2x/a)\cos (5x/a)$. Чему равно среднее значение величины ${A^2}$ в этот момент времени?
\begin{choices}
\choice 28        
\choice 29        
\choice 30        
\choice 31
\end{choices}

\question Собственные значения и отвечающие им нормированные собственные функции оператора некоторой физической величины $A$ равны: ${a_n} = n$ и ${f_n}(\varphi ) = \left( {1/\sqrt {2\pi } } \right){e^{in\varphi }}$ ($n = 0, \pm 1, \pm 2,...$, $0 < \varphi  < 2\pi $). Волновая функция частицы равна $\Psi (\varphi ) = C{\cos ^2}\varphi $. Какие значения физической величины $A$ можно обна-ружить при измерениях и с какими вероятностями? 
\begin{choices}
\choice $ - 2,\;\;0,\;\;2$ с вероятностями 1/6, 4/6, 1/6 соответственно
\choice $ - 2,\;\;0,\;\;2$ с одинаковыми вероятностями
\choice  с вероятностями 1/2 и 1/2
\choice $ - 1,\;\;0,\;\;1$ с вероятностями 1/6, 4/6, 1/6 соответственно
\end{choices}

\question Собственные значения и отвечающие им нормированные собственные функции оператора некоторой физической величины $A$ равны: ${a_n} = n$ и ${f_n}(\varphi ) = \left( {1/\sqrt {2\pi } } \right){e^{in\varphi }}$ ($n = 0, \pm 1, \pm 2,...$, $0 < \varphi  < 2\pi $). Волновая функция частицы равна $\Psi (\varphi ) = C{\cos ^2}\varphi $. Какие значения физической величины ${A^2}$ можно обнаружить при измерениях и с какими вероятностями? 
\begin{choices}
\choice  с одинаковыми вероятностями
\choice  с вероятностями 4/6 и 2/6 соответственно
\choice  с вероятностями 4/6 и 2/6 соответственно 
\choice  с вероятностями 2/6 и 4/6 соответственно
\end{choices}

\question Собственные значения и отвечающие им нормированные собственные функции оператора некоторой физической величины $A$ равны: ${a_n} = n$ и ${f_n}(\varphi ) = \left( {1/\sqrt {2\pi } } \right){e^{in\varphi }}$ ($n = 0, \pm 1, \pm 2,...$, $0 < \varphi  < 2\pi $). Волновая функция частицы равна $\Psi (\varphi ) = C{\cos ^2}\varphi $. Найти среднее значение результатов многих измерений физической величины $A$. 
\begin{choices}
\choice -1/2      
\choice 0      
\choice 1/2       
\choice 1 
\end{choices}

\question Собственные значения и отвечающие им нормированные собственные функции оператора некоторой физической величины $A$ равны: ${a_n} = n$ и ${f_n}(\varphi ) = \left( {1/\sqrt {2\pi } } \right){e^{in\varphi }}$ ($n = 0, \pm 1, \pm 2,...$, $0 < \varphi  < 2\pi $). Волновая функция частицы равна $\Psi (\varphi ) = C{\cos ^2}\varphi $. Найти среднее значение результатов многих измерений физической величины ${A^2}$. 
\begin{choices}
\choice 1/3    Б.2/3    
\choice 3/3       
\choice 4/3 
\end{choices}

\question Собственные значения и отвечающие им нормированные собственные функции оператора некоторой физической величины $A$ равны: ${a_n} = n$ и ${f_n}(\varphi ) = \left( {1/\sqrt {2\pi } } \right){e^{in\varphi }}$ ($n = 0, \pm 1, \pm 2,...$, $0 < \varphi  < 2\pi $). В каком из нижеперечислен-ных состояний физическая величина $A$ имеет определенное значение? 
\begin{choices}
\choice $\Psi (\varphi ) = C\cos \varphi $         
\choice $\Psi (\varphi ) = C{e^{ - i\varphi }}$
\choice $\Psi (\varphi ) = C\sin \varphi $         
\choice ни в одном из перечисленных
\end{choices}

\question Собственные значения и отвечающие им нормированные собственные функции оператора некоторой физической величины $A$ равны: ${a_n} = n$ и ${f_n}(\varphi ) = \left( {1/\sqrt {2\pi } } \right){e^{in\varphi }}$ ($n = 0, \pm 1, \pm 2,...$, $0 < \varphi  < 2\pi $). В каком из нижеперечислен-ных состояний физическая величина ${A^2}$ имеет определенное значение? 
\begin{choices}
\choice $\Psi (\varphi ) = C\cos \varphi $         
\choice $\Psi (\varphi ) = C{\cos ^2}\varphi $
\choice $\Psi (\varphi ) = C{\cos ^3}\varphi $        
\choice ни в одном из перечисленных
\end{choices}

\question Собственные значения и отвечающие им нормированные собственные функции оператора некоторой физической величины $A$ равны: ${a_n} = n$ и ${f_n}(\varphi ) = \left( {1/\sqrt {2\pi } } \right){e^{in\varphi }}$ ($n = 0, \pm 1, \pm 2,...$, $0 < \varphi  < 2\pi $). В каком из нижеперечислен-ных состояний можно с определенностью обнаружить, что $A = 0,5$? 
\begin{choices}
\choice $\Psi (\varphi ) = C\left( {1 + {e^{i\varphi }}} \right)$       
\choice $\Psi (\varphi ) = C{\cos ^2}\varphi $
\choice $\Psi (\varphi ) = C\left( {1 + 0,5{e^{i\varphi }}} \right)$       
\choice такого состояния не сущест-вует
\end{choices}

\question Собственные значения и отвечающие им нормированные собственные функции оператора некоторой физической величины $A$ равны: ${a_n} = n$ и ${f_n}(\varphi ) = \left( {1/\sqrt {2\pi } } \right){e^{in\varphi }}$ ($n = 0, \pm 1, \pm 2,...$, $0 < \varphi  < 2\pi $). Волновая функция частицы равна $\Psi (\varphi ) = C\cos \varphi $. Какие значения физической величины $A$ можно обнаружить при измерениях и с какими вероятностями? 
\begin{choices}
\choice  с вероятностями 1/2 и 1/2     
\choice  с вероятностями 1/2 и 1/2
\choice  с вероятностями 1/2 и 1/2     
\choice $ - 1,\;\;0,\;\;1$ с вероятностями 1/3, 1/3 и 1/3 
\end{choices}

\question Оператор некоторой физической величины $A$ имеет следующие собственные значения и отвечающие им нормированные собственные функции: ${a_1} \leftrightarrow {f_1}(x)$, ${a_2} \leftrightarrow {f_2}(x)$, ${a_3} \leftrightarrow {f_3}(x)$ и т.д. В состоянии с какой волновой функ-цией $\psi (x)$ при измерении физической величины $A$ с единичной вероятностью можно обнару-жить значение $({a_1} + {a_2})/2$?
\begin{choices}
\choice $\psi (x) = \left( {{f_1}(x) + {f_2}(x)} \right)/\sqrt 2 $      
\choice $\psi (x) = \left( {{f_1}(x) + {f_2}(x) + {f_3}(x)} \right)/\sqrt 3 $
\choice $\psi (x) = {f_1}(x) + {f_2}(x)$        
\choice такого состояния не существует
\end{choices}

\question Оператор некоторой физической величины $A$ имеет следующие собственные значения и отвечающие им собственные функции: ${a_1} \leftrightarrow {f_1}(x)$, ${a_2} \leftrightarrow {f_2}(x)$, ${a_3} \leftrightarrow {f_3}(x)$ и т.д. В некоторый момент времени волновая функция системы имеет вид $\Psi (x) = \frac{1}{2}\,{f_1}(x) + \frac{1}{3}{f_2}(x)$. Какие значения величины $A$ можно обнаружить в этом состоянии и с какими вероятностями? 
\begin{choices}
\choice ${a_1},\;{a_2},\;{a_3}$ с одинаковыми вероятностями
\choice ${a_1}$ с вероятностью 1/4 и ${a_2}$ с вероятностью 1/9 
\choice ${a_1}$ с вероятностью 4/13 и ${a_2}$ с вероятностью 9/13
\choice ${a_1}$ с вероятностью 9/13 и ${a_2}$ с вероятностью 4/13
\end{choices}

\question В некоторый момент времени нормированная волновая функция системы имеет вид $\Psi (x) = \frac{1}{{\sqrt 3 }}\,{f_{A = 1}}(x) + \sqrt {\frac{2}{3}} \,{f_{A = 3}}(x)$, где ${f_{A = 1}}(x)$ и $\,{f_{A = 3}}(x)$ - нормированные собственные функции оператора физической величины $\hat A$, отве-чающие собственным значениям $A = 1$ и $A = 3$, соответственно. Среднее значение величины фи-зической величины $A$ в этот момент равно 
\begin{choices}
\choice $2$       
\choice $7/3$        
\choice $5/3$     
\choice $4/3$
\end{choices}

\question В некоторый момент времени нормированная волновая функция системы имеет вид $\Psi (x) = \frac{1}{{\sqrt 3 }}\,{f_{A = 1}}(x) + \sqrt {\frac{2}{3}} \,{f_{A = 3}}(x)$, где ${f_{A = 1}}(x)$ и $\,{f_{A = 3}}(x)$ - нормированные собственные функции оператора физической величины $\hat A$, отве-чающие собственным значениям $A = 1$ и $A = 3$, соответственно. Среднее значение величины фи-зической величины ${A^2}$ в этот момент равно 
\begin{choices}
\choice $15/3$    
\choice $17/3$    
\choice $19/3$    
\choice $21/3$
\end{choices}

\question Физическая величина $A$ в некоторой квантовой системе может принимать два значения 1 и 4. В результате проведения многократных измерений над ансамблем тождественных квантовых сис-тем оказалось, что $\overline A  = 2$ ($\overline A $ - среднее значение результатов этих эксперимен-тов). Чему равны вероятности обнаружения возможных значений величины $A$ в этом эксперимен-те?
\begin{choices}
\choice $w(1) = 1/4,\quad w(4) = 3/4$        
\choice $w(1) = 1/3,\quad w(4) = 2/3$
\choice $w(1) = 3/4,\quad w(4) = 1/4$        
\choice $w(1) = 2/3,\quad w(4) = 1/3$
\end{choices}

\question Физическая величина $A$ в некоторой квантовой системе может принимать три значения 1, 4 и 5 с вероятностями $w(1) = 1/6,\quad w(4) = 1/3,\quad w(5) = 1/2$. Чему равно среднее значение ре-зультатов многих измерений величины $A$?
\begin{choices}
\choice $\overline A  = 2$    
\choice $\overline A  = 3$    
\choice $\overline A  = 4$    
\choice $\overline A  = 5$
\end{choices}

\question Физическая величина $A$ в некоторой квантовой системе может принимать три значения 1, 4 и 5 с вероятностями $w(1) = 1/6,\quad w(4) = 1/3,\quad w(5) = 1/2$. Чему равно среднее значение ре-зультатов многих измерений величины ${A^2}$?
\begin{choices}
\choice $\overline {{A^2}}  = 16$      
\choice $\overline {{A^2}}  = 17$      
\choice $\overline {{A^2}}  = 18$      
\choice $\overline {{A^2}}  = 19$
\end{choices}

\question Волновые функции некоторой квантовой системы определены на интервале $[0,1]$. Собственные функции оператора некоторой физической величины ${f_n}(x)$, отвечающие дискретным собственным значениям ${a_n}$, равны ${f_n}(x) = {C_n}\sin (\pi nx)$ (индекс $n$ пробегает значения $n = 1,\;2,\;3,\;...$, ${C_n}$ - постоянные). Чему равна вероятность того, что при измерениях величины $A$ можно обнаружить значение $A = {a_2}$, в состоянии с волновой функцией $\Psi (x) = Bx(x - 1)$?
\begin{choices}
\choice $w({a_3}) = 1/4$      
\choice $w({a_3}) = 1/3$         
\choice $w({a_3}) = 1/2$      
\choice $w({a_3}) = 0$
\end{choices}

\question Оператор некоторой физической величины $A$ имеет невырожденные собственные значения ${a_n}$ и отвечающие им нормированные собственные функции ${f_n}(x)$. Какой должна быть нор-мированная волновая функция $\psi (x)$ квантовой системы, чтобы при измерении физической вели-чины $A$ в ней с равными вероятностями можно было обнаружить значения ${a_1}$ и ${a_2}$, а с вдвое большей вероятностью - ${a_3}$?
\begin{choices}
\choice $\psi (x) = \left( {{f_1}(x) - {f_2}(x) + \sqrt 2 i{f_3}(x)} \right)/\sqrt 6 $       
\choice $\psi (x) = \left( {{f_1}(x) - {f_2}(x) - \sqrt 2 {f_3}(x)} \right)/6$
\choice $\psi (x) = \left( {{f_1}(x) + {f_2}(x) - \sqrt 2 i{f_3}(x)} \right)/2$        
\choice такого состоя-ния не существует
\end{choices}

\question Волновая функция квантовой системы имеет вид $\psi (x) \sim \sin x + \cos x$ ($ - \pi  < x < \pi $). Чему равна средняя четность этого состояния?
\begin{choices}
\choice +1     
\choice –1     
\choice 0      
\choice среднюю четность в этом состоянии нельзя определить
\end{choices}

\question Волновая функция квантовой системы имеет вид $\psi (x) \sim \sin x + 2\cos x$ ($ - \pi  < x < \pi $). Чему равна средняя четность этого состояния?
\begin{choices}
\choice $3/5$  
\choice $2/5$     
\choice $1/5$     
\choice среднюю четность в этом состоянии нельзя определить
\end{choices}

\question Пространство состояний частицы представляет собой пространство функций одной перемен-ной, определенных на отрезке $[ - a,a]$. Частица находится в состоянии с нормированной волновой функцией $\psi (x)$. Известно, что $\int\limits_{ - a}^a {{\psi ^*}( - x)\psi (x)dx = \alpha } $. Чему равна средняя четность рассматриваемого состояния?
\begin{choices}
\choice $\overline P  = \alpha $    
\choice $\overline P  =  - \alpha $    
\choice $\overline P  = \alpha /2$     
\choice $\overline P  =  - \alpha /2$
\end{choices}

\question Оператор физической величины $A$ имеет непрерывный спектр собственных значений $a$ и собственных функций ${f_a}(x)$ (${f_a}(x)$ нормированы на $\delta $-функцию от $a$). Разложение волновой функции квантовой системы $\Psi (x,t)$, взятой в некоторый момент времени, по собствен-ным функциям содержит только две собственных функции: $\Psi (x,t) = {f_{{a_1}}}(x) + \,{f_{{a_2}}}(x)$. Какие значения физической величины $A$ можно обнаружить при измерениях над ансамблем тождественных квантовых систем?
\begin{choices}
\choice любые, так как спектр собственных значений оператора непрерывен
\choice определенное значение $\left( {{a_1} + {a_2}} \right)/2$
\choice только ${a_1}$ и ${a_2}$
\choice только ${a_1}$, ${a_2}$ и $\left( {{a_1} + {a_2}} \right)/2$
\end{choices}

\question Оператор некоторой физической величины $A$ имеет невырожденные собственные значения ${a_n}$ и отвечающие им нормированные собственные функции ${f_n}(x)$. В состоянии с волновой функцией ${\Psi _1}(x)$ при измерении величины $A$ могут быть получены ${a_1}$ и ${a_2}$, в со-стоянии с волновой функцией ${\Psi _2}(x)$ при измерении величины $A$ могут быть получены ${a_2}$ и ${a_3}$. Какие значения величины $A$ можно измерить в состоянии суперпозиции ${C_1}{\Psi _1}(x) + {C_2}{\Psi _2}(x)$?
\begin{choices}
\choice ${a_1}$ и ${a_2}$        
\choice в зависимости от коэффициентов: либо ${a_1}$, ${a_2}$ и ${a_3}$, либо ${a_1}$ и ${a_2}$
\choice ${a_1}$, ${a_2}$ и ${a_3}$        
\choice в зависимости от коэффициентов: либо ${a_1}$, ${a_2}$ и ${a_3}$, либо ${a_1}$ и ${a_3}$  
\end{choices}

\end{questions}

\subsection{ Координата и импульс. Различные представления волновой функции }

\begin{questions}

\question Состояние частицы описывается нормированной волновой функцией $\Psi (\vec r,t)$. Какое из нижеследующих утверждений справедливо?
\begin{choices}
\rightchoice ${\left| {\Psi (\vec r,t)} \right|^2}dV$ - вероятность найти частицу в момент времени $t$ в объеме $dV$ около точки $\vec r$
\choice ${\left| {\Psi (\vec r,t)} \right|^2}dt$ - вероятность найти частицу в точке $\vec r$ в интервале времени ($t,t + dt$)
\choice ${\left| {\Psi (\vec r,t)} \right|^2}dVdt$ - вероятность найти частицу в интервале ($t,t + dt$) в объеме $dV$ в около точки $\vec r$
\choice все утверждения неправильны
\end{choices}

\question Состояние частицы описывается нормированной волновой функцией $\psi (x,y,z)$. Какова вероятность того, что $x$-координата частицы больше нуля?
\begin{choices}
\choice $w(x > 0) = \int\limits_0^\infty  {dx|\psi (x,y,z){|^2}} $         
\choice $w(x > 0) = \int\limits_0^\infty  {dx\int\limits_0^\infty  {dy\int\limits_0^\infty  {dz} } |\psi (x,y,z){|^2}} $
\rightchoice $w(x > 0) = \int\limits_0^\infty  {dx\int\limits_{ - \infty }^\infty  {dy\int\limits_{ - \infty }^\infty  {dz} } |\psi (x,y,z){|^2}} $     
\choice $w(x > 0) = \int\limits_{ - \infty }^\infty  {dy\int\limits_{ - \infty }^\infty  {dz} } |\psi (x > 0,y,z){|^2}$
\end{choices}

\question Состояние частицы описывается нормированной волновой функцией $\psi (x,y,z)$. Какова вероятность того, что $x$-координата частицы больше нуля, а $y$-координата лежит в интервале от $y$ до $y + dy$ ($dy$ - малая величина)?
\begin{choices}
\choice $dw(x > 0,y \in dy) = dy\int\limits_0^\infty  {dx|\psi (x,y,z){|^2}} $      
\choice $dw(x > 0,y \in dy) = dy\int\limits_0^\infty  {dx\int\limits_0^\infty  {dz} } |\psi (x,y,z){|^2}$
\choice $dw(x > 0,y \in dy) = dy\int\limits_{ - \infty }^\infty  {dx\int\limits_0^\infty  {dz} } |\psi (x,y,z){|^2}$ 
\rightchoice $dw(x > 0,y \in dy) = dy\int\limits_0^\infty  {dx\int\limits_{ - \infty }^\infty  {dz} } |\psi (x,y,z){|^2}$
\end{choices}

\question Состояние частицы описывается нормированной волновой функцией $\psi (r)$, где $r = \sqrt {{x^2} + {y^2} + {z^2}} $ - модуль радиус вектора. Какова вероятность обнаружить частицу в сферическом слое ${r_1} \le r \le {r_2}$?
\begin{choices}
\choice $\int\limits_{{r_1}}^{{r_2}} {{{\left| \psi  \right|}^2}} dr$      
\choice $\int\limits_{{r_1}}^{{r_2}} {{{\left| \psi  \right|}^2}{r^2}} dr$    
\choice $2\pi \int\limits_{{r_1}}^{{r_2}} {{{\left| \psi  \right|}^2}} {r^2}dr$     
\rightchoice $4\pi \int\limits_{{r_1}}^{{r_2}} {{{\left| \psi  \right|}^2}{r^2}} dr$
\end{choices}

\question Состояние частицы описывается нормированной волновой функцией $\psi (x,y,z) = A\exp ( - r/a)$, где $A$ и $a$ - некоторые действительные числа, $r = \sqrt {{x^2} + {y^2} + {z^2}} $ модуль радиус вектора. Какое значение $r$ является наиболее вероятным?
\begin{choices}
\choice $r = 0$      
\choice $r = A$      
\rightchoice $r = a$      
\choice $r = 2a$
\end{choices}

\question Состояние частицы описывается нормированной волновой функцией $\psi (x)$. Какой формулой определяется средняя координата частицы в этом состоянии?
\begin{choices}
\choice $\bar x = \int {x|\psi (x)|dx} $     
\rightchoice $\bar x = \int {x|\psi (x){|^2}dx} $    
\choice $\bar x = \int {|\psi (x)|dx} $   
\choice $\bar x = \int {|\psi (x){|^2}dx} $
\end{choices}

\question Состояние частицы описывается нормированной волновой функцией $\psi (x) = A\exp ( - {x^2}/2{a^2})$, где $A$ и $a$ - некоторые действительные числа. Средняя координата частицы в этом состоянии равна
\begin{choices}
\choice $\bar x = 2a$      
\choice $\bar x = a$    
\choice $\bar x = \sqrt 2 a$     
\rightchoice $\bar x = 0$
\end{choices}

\question Частица находится в состоянии с нормированной волновой функцией $\psi (x) = A(x - b)\exp ( - {(x - b)^2}/2{a^2})$, где $A$, $b$ и $a$ - некоторые действительные числа. Средняя координата частицы в этом состоянии равна
\begin{choices}
\rightchoice $\bar x = b$    
\choice $\bar x = a$    
\choice $\bar x = \sqrt 2 b$     
\choice нулю
\end{choices}

\question Частица находится в состоянии с нормированной волновой функцией $\psi (x) = A(x + b)\exp ( - {x^2}/{a^2})$, где $A$, $b$ и $a$ - некоторые действительные числа. Какой величиной определяется дисперсия (разброс) координаты в этом состоянии?
\begin{choices}
\choice $A$       
\choice $b$
\rightchoice $a$       
\choice в этом состоянии координата имеет определенное значение
\end{choices}

\question Действие оператора координаты $\hat x$ на произвольную функцию $\psi (x)$ в координатном представлении определяется соотношением
\begin{choices}
\rightchoice $\hat x\psi  = x\psi $      
\choice $\hat x\psi  = \psi /x$     
\choice $\hat x\psi  = \frac{{d\psi }}{{dx}}$      
\choice $\hat x\psi  = \int\limits_{ - \infty }^\infty  {dx\psi (x)} $
\end{choices}

\question Чему равны собственные значения оператора координаты $\hat x$?
\begin{choices}
\choice любому целому числу         
\choice любому положительному целому числу
\rightchoice любому действительному числу      
\choice любому положительному действительному числу
\end{choices}

\question Даны три функции: 
(1) $f(x) = \delta (x{\rm{ - }}a)$, (2) $f(x) = (1/2)\delta (x{\rm{ - }}a) + (1/2)\delta (x + a)$, (3) $f(x) = (1/4)\delta (x{\rm{ - }}a) + (3/4)\delta (x + a)$. 
Какие из них являются собственными функциями оператора квадрата координаты?
\begin{choices}
\rightchoice только (1)      
\choice только (1) и (2)      
\choice ни одна из них     
\choice все.
\end{choices}

\question Что можно сказать о собственных функциях оператора квадрата координаты? 
\begin{choices}
\choice любая из них будет собственной для оператора координаты
\choice ни одна из них не будет собственной для оператора координаты
\rightchoice их можно выбрать так, чтобы они были собственными оператора координаты
\choice бессмысленный вопрос
\end{choices}

\question Что можно сказать о собственных функциях оператора координаты? 
\begin{choices}
\rightchoice любая из них будет собственной для оператора квадрата координаты
\choice ни одна из них не будет собственной для оператора квадрата координаты
\choice их можно выбрать так, чтобы они были собственными оператора квадрата координаты
\choice бессмысленный вопрос
\end{choices}

\question Спектр собственных значений оператора координаты 
\begin{choices}
\choice дискретен          
\rightchoice непрерывен
\choice дискретно-непрерывен     
\choice бессмысленный вопрос
\end{choices}

\question Какова кратность вырождения собственных значений оператора координаты?
\begin{choices}
\rightchoice 1      
\choice 2      
\choice 3      
\choice $\infty $
\end{choices}

\question Какова кратность вырождения собственных значений оператора квадрата координаты?
\begin{choices}
\choice 1      
\rightchoice 2      
\choice 3      
\choice $\infty $
\end{choices}

\question Даны волновые функции в координатном представлении ряда состояний частицы. В каких из них квадрат координаты частицы имеет определенное значение ($a$ - действительное число)?
\begin{choices}
\rightchoice $\delta ({x^2} - {a^2})$    
\choice $\cos \left\{ {\frac{{x - a}}{a}} \right\}$         
\choice $\exp \left\{ { - i\frac{x}{a}} \right\}$        
\choice ни в одном
\end{choices}

\question Интеграл от квадрата собственной функции оператора координаты
\begin{choices}
\choice сходится              
\rightchoice расходится   
\choice зависит от собственного значения  
\choice равен нулю
\end{choices}

\question Рассматривается пространство состояний одномерной частицы. Какова размерность нормированных на $\delta $-функцию от координаты собственных функций оператора координаты частицы в координатном представлении?
\begin{choices}
\choice длина
\rightchoice 1/длина
\choice длина$^{1/2}$
\choice длина$^{-1/2}$
\end{choices}

\question Нормированная на $\delta$-функцию от координаты собственная функция оператора координаты ${f_a}(x)$, отвечающая собственному значению $a$, в координатном представлении равна
\begin{choices}
\choice ${f_a}(x) = \exp (ix/a)$       
\choice ${f_a}(x) = \delta (x/a)$
\choice ${f_a}(x) = \sin (x/a)$     
\rightchoice ${f_a}(x) = \delta (x{\rm{ - }}a)$
\end{choices}

\question Даны волновые функции в координатном представлении ряда состояний частицы. В каком из них координата частицы имеет определенное значение ($a$ - действительное число)?
\begin{choices}
\rightchoice $\delta (x - a)$      
\choice $\cos \left\{ {\frac{{x - a}}{a}} \right\}$         
\choice $\exp \left\{ { - i\frac{x}{a}} \right\}$        
\choice $\delta ({x^2} - {a^2})$ 
\end{choices}

\question Состояние частицы в координатном представлении описывается волновой функцией $c\delta (x - a) + d\delta (x - b)$, где $a$, $b$, $c$ и $d$ - некоторые действительные числа. При измерении координаты частицы будут получены
\begin{choices}
\choice $a$ с вероятностью ${c^2}$ и $b$ с вероятностью ${d^2}$
\choice $c$ с вероятностью ${a^2}/({a^2} + {b^2})$ и $d$ с вероятностью ${b^2}/({a^2} + {b^2})$
\choice $c$ с вероятностью ${a^2}$ и $d$ с вероятностью ${b^2}$
\rightchoice $a$ с вероятностью ${c^2}/({c^2} + {d^2})$ и $b$ с вероятностью ${d^2}/({c^2} + {d^2})$ 
\end{choices}

\question В каком из нижеперечисленных состояний радиус-вектор частицы будет определенным? (здесь $a,b,c$ - произвольные действительные числа)
\begin{choices}
\choice $\sin ax\sin by\sin cz$           
\choice такого состояния не существует
\rightchoice $\delta (x - a)\delta (y - b)\delta (z - c)$        
\choice $\delta ({x^2} - {a^2})\delta ({y^2} - {b^2})\delta ({z^2} - {c^2})$
\end{choices}

\question Состояние частицы описывается волновой функцией $\delta (x - a) + 2\delta (x - 2a)$, где $a$ - число. Найти среднюю координату частицы в этом состоянии.
\begin{choices}
\rightchoice $\bar x = 9a/5$    
\choice $\bar x = 7a/5$    
\choice $\bar x = a/2$     
\choice $\bar x = 5a/3$
\end{choices}

\question Состояние частицы волновой функцией $\delta (x - a) + 2\delta (x - 2a)$, где $a$ - число. Найти средний квадрат координаты частицы в этом состоянии.
\begin{choices}
\choice $\overline {{x^2}}  = 15{a^2}/5$  
\rightchoice $\overline {{x^2}}  = 17{a^2}/5$  
\choice $\overline {{x^2}}  = 19{a^2}/5$  
\choice $\overline {{x^2}}  = 21{a^2}/5$
\end{choices}

\question Состояние частицы волновой функцией $\delta (x - a) + 2\delta (x - 2a)$, где $a$ - число. Найти квадрат средней координаты частицы в этом состоянии.
\begin{choices}
\choice ${\bar x^2} = 49{a^2}/25$   
\choice ${\bar x^2} = {a^2}/4$      
\choice ${\bar x^2} = 25{a^2}/9$ 
\rightchoice ${\bar x^2} = 81{a^2}/25$
\end{choices}

\question Действие оператора проекции импульса ${\hat p_x}$ на ось $x$ на произвольную функ-цию $\psi (x)$ в координатном представлении определяется соотношением
\begin{choices}
\choice ${\hat p_x}\psi (x) = {p_x}\psi (x)$       
\choice ${\hat p_x}\,\psi (x) =  - {\hbar ^2}\frac{{{d^2}\psi (x)}}{{d{x^2}}}$
\rightchoice ${\hat p_x}\psi (x) =  - i\hbar \frac{{d{\kern 1pt} \psi (x)}}{{d{\kern 1pt} x}}$    
\choice ${\hat p_x}\,\psi (x) = \,i\hbar \frac{{d{\kern 1pt} \psi (x)}}{{d{\kern 1pt} {p_x}}}$
\end{choices}

\question Оператор проекции импульса частицы на ось $x$ является
\begin{choices}
\choice унитарным             
\rightchoice эрмитовым
\choice совпадающим со своим обратным     
\choice нелинейным 
\end{choices}

\question Чему равны собственные значения оператора проекции импульса на ось $x$?
\begin{choices}
\rightchoice любому действительному числу      
\choice любому положительному действительному числу
\choice любому целому числу         
\choice любому положительному целому числу
\end{choices}

\question Собственная функция ${f_p}(x)$ оператора импульса ${\hat p_x}$, отвечающая собст-венному значению $p$, в координатном представлении равна
\begin{choices}
\choice ${f_p}(x) = \sin \left\{ {\frac{{px}}{\hbar }} \right\}$     
\choice ${f_p}(x) = \cos \left\{ {\frac{{px}}{\hbar }} \right\}$
\rightchoice ${f_p}(x) = \exp \left\{ {\frac{{ipx}}{\hbar }} \right\}$    
\choice ${f_p}(x) = \exp \left\{ { - \frac{{ipx}}{\hbar }} \right\}$
\end{choices}

\question Нормированная на $\delta $-функцию от импульса собственная функция ${f_p}(x)$ опе-ратора импульса ${\hat p_x}$, отвечающая собственному значению $p$, в координатном пред-ставлении равна
\begin{choices}
\choice ${f_p}(x) = \frac{1}{{{{(2\pi \hbar )}^{3/2}}}}{e^{i\frac{{px}}{\hbar }}}$
\rightchoice ${f_p}(x) = \frac{1}{{{{(2\pi \hbar )}^{1/2}}}}{e^{i\frac{{px}}{\hbar }}}$
\choice ${f_p}(x) = {(2\pi \hbar )^{1/2}}{e^{i\frac{{px}}{\hbar }}}$    
\choice ${f_p}(x) = {(2\pi \hbar )^{3/2}}{e^{i\frac{{px}}{\hbar }}}$
\end{choices}

\question Рассматривается пространство состояний одномерной частицы. Какова размерность нор-мированных на $\delta $-функцию от импульса собственных функций оператора импульса в координатном представлении?
\begin{choices}
\choice        
\choice 
\choice     
\choice 
\end{choices}

\question Частица находится в состоянии с некоторой нормированной волновой функцией $\psi (x)$. Известно, что функция $\psi (x)$ действительна и положительна во всех точках. Что можно сказать о среднем импульсе частицы в этом состоянии?
\begin{choices}
\choice обязательно равен нулю         
\choice обязательно не равен нулю
\choice обязательно положителен     
\rightchoice информации для ответа недостаточно
\end{choices}

\question Частица находится в состоянии с нормированной волновой функцией $\psi (x) = A(x + b)\exp ( - {x^2}/{a^2})$, где $A$, $b$ и $a$ - некоторые действительные числа. Средний импульс частицы в этом состоянии равен
\begin{choices}
\rightchoice нулю      
\choice $\bar p = \hbar /a$      
\choice $\bar p = \hbar /b$      
\choice $\bar p = \hbar /\sqrt {ab} $
\end{choices}

\question Какова кратность вырождения собственных значений оператора проекции импульса на ось $x$?
\begin{choices}
\rightchoice 1      
\choice 2      
\choice 3      
\choice $\infty $
\end{choices}

\question Какова кратность вырождения собственных значений оператора квадрата проекции импульса на ось $x$?
\begin{choices}
\choice 1      
\rightchoice 2      
\choice 3      
\choice $\infty $
\end{choices}

\question Даны волновые функции в координатном представлении ряда состояний частицы. В ка-ком из них импульс частицы имеет определенное значение (здесь $p$ - действительное число)?
\begin{choices}
\choice $\sin \left( {px/\hbar } \right)$      
\choice $\cos \left( {px/\hbar } \right)$    
\choice ${\rm{exp}}\left( { - px/\hbar } \right)$     
\rightchoice $\exp \left( {ipx/\hbar } \right)$ 
\end{choices}

\question В каком из нижеперечисленных состояний квадрат импульса имеет определенное значе-ние?
\begin{choices}
\choice $\sin \left\{ {\frac{{px}}{\hbar }} \right\}$    
\choice ${\rm{tg}}\left\{ {\frac{{px}}{\hbar }} \right\}$      
\choice $\sin \left\{ {\frac{{px}}{\hbar }} \right\} + \cos \left\{ {\frac{{2px}}{\hbar }} \right\}$     
\choice ${\rm{ctg}}\left\{ {\frac{{px}}{\hbar }} \right\}$ 
\end{choices}

\question В каком из нижеперечисленных состояний квадрат импульса не имеет определенного значения ($p \ne 0$)?
\begin{choices}
\choice $\sin \left\{ {\frac{{px}}{\hbar }} \right\}$    
\choice ${\rm{tg}}\left\{ {\frac{{px}}{\hbar }} \right\}$      
\choice $\cos \left\{ {\frac{{px}}{\hbar }} \right\}$    
\choice $\exp \left( {ipx/\hbar } \right)$
\end{choices}

\question Будет ли собственная функция оператора импульса собственной функцией оператора квадрата импульса?
\begin{choices}
\choice да     
\choice нет    
\choice это зависит от состояния    
\choice это зависит от гамильтониана
\end{choices}

\question Будет ли собственная функция оператора квадрата импульса собственной функцией опе-ратора импульса?
\begin{choices}
\choice да     
\choice нет    
\choice это зависит от состояния    
\choice это зависит от гамильтониана
\end{choices}

\question Даны три функции:
(1) $f(x) = {e^{ipx/\hbar }}$, (2) $f(x) = (1/2){e^{ipx/\hbar }} + (1/2){e^{ - ipx/\hbar }}$, (3) $f(x) = (1/4){e^{ipx/\hbar }} + (3/4){e^{ - ipx/\hbar }}$. 
Какие из них будут собственными функциями оператора квадрата импульса?
\begin{choices}
\choice только (1)      
\choice только (1) и (2)      
\choice ни одна из них     
\choice все
\end{choices}

\question Состояние частицы в координатном представлении описывается волновой функцией $\exp ( - 2ibx)$ (где $b$ - некоторое действительное число). Проводят измерение проекции им-пульса частицы на ось $x$. Какие значения могут быть при этом получены?
\begin{choices}
\choice любые с одинаковыми вероятностями          
\choice $2\hbar b$ с единичной вероятностью
\choice $ - 2\hbar b$ с единичной вероятностью        
\choice $ - 2b$ с единичной вероятно-стью
\end{choices}

\question Состояние частицы в координатном представлении описывается волновой функцией $\exp (ipx) + 2\exp (2ipx)$, где $p$ - некоторое действительное число. При измерении импульса частицы будут получены
\begin{choices}
\choice $p$с вероятностью 1/3, $2p$ с вероятностью 2/3
\choice $\hbar p$ с вероятностью 1/3, $2\hbar p$ с вероятностью 2/3
\choice $p$ с вероятностью 1/5, $2p$ с вероятностью 4/5  
\choice $\hbar p$с вероятностью 1/5, $2\hbar p$ с вероятностью 4/5
\end{choices}

\question Состояние частицы в координатном представлении описывается волновой функцией $2\exp (ipx) + \exp (2ipx)$, где $p$ - некоторое действительное число. Чему равен средний им-пульс частицы в этом состоянии?
\begin{choices}
\choice $\overline p  = 6\hbar p/4$    
\choice $\overline p  = 6\hbar p/5$    
\choice $\overline p  = 6p/4$    
\choice $\overline p  = 6p/5$ 
\end{choices}

\question Состояние частицы в координатном представлении описывается волновой функцией $2\exp (ipx) + \exp (2ipx)$, где $p$ - некоторое действительное число. Чему равен квадрат сред-него импульса частицы в этом состоянии?
\begin{choices}
\choice ${\overline p ^2} = 36{p^2}/25$   
\choice ${\overline p ^2} = 36p/16$ 
\choice ${\overline p ^2} = 36{\hbar ^2}{p^2}/25$  
\choice ${\overline p ^2} = 36{\hbar ^2}{p^2}/16$ 
\end{choices}

\question Состояние частицы в координатном представлении описывается волновой функцией $2\exp (ipx) + \exp (2ipx)$, где $p$ - некоторое действительное число. Чему равен средний квад-рат импульса частицы в этом состоянии?
\begin{choices}
\choice $\overline {{p^2}}  = 8{\hbar ^2}{p^2}/5$  
\choice $\overline {{p^2}}  = 8{\hbar ^2}{p^2}/3$  
\choice $\overline {{p^2}}  = 7{\hbar ^2}{p^2}/5$  
\choice $\overline {{p^2}}  = 7{\hbar ^2}{p^2}/3$ 
\end{choices}

\question Что можно сказать о собственных функциях оператора квадрата импульса? 
\begin{choices}
\choice любая из них будет собственной для оператора импульса
\choice ни одна из них не будет собственной для оператора импульса
\choice их можно выбрать так, чтобы они были собственными оператора импульса
\choice бессмысленный вопрос
\end{choices}

\question Что можно сказать о собственных функциях оператора импульса? 
\begin{choices}
\choice любая из них будет собственной для оператора квадрата импульса
\choice ни одна из них не будет собственной для оператора квадрата импульса
\choice их можно выбрать так, чтобы они были собственными оператора квадрата импульса
\choice бессмысленный вопрос
\end{choices}

\question Какая из трех функций: 
(1) $f(x) = {e^{ipx/\hbar }}$, (2) $f(x) = {e^{ipx/\hbar }} - {e^{ - ipx/\hbar }}$, (3) $f(x) = {e^{ - ipx/\hbar }}$. 
будет собственной функцией оператора квадрата импульса, но не будет собственной функцией оператора импульса?
\begin{choices}
\choice только (1)      
\choice только (2)      
\choice (2) и (3)       
\choice таких функций не существует
\end{choices}

\question Какая из трех функций: 
(1) $f(x) = {e^{ip{x^2}/\hbar }}$, (2) $f(x) = {e^{ipx/\hbar }} - {e^{ - ipx/\hbar }}$, (3) $f(x) = (1/4){e^{ipx/\hbar }} - (3/4){e^{ - ipx/\hbar }}$. 
будет собственной функцией оператора импульса, но не будет собственной функцией операто-ра квадрата импульса?
\begin{choices}
\choice только (1)      
\choice только (2)      
\choice (2) и (3)       
\choice таких функций не существует
\end{choices}

\question Даны три функции: 
(1) $f(x) = {\left( {{e^{ipx/\hbar }}} \right)^2}$, (2) $f(x) = {e^{ip{x^2}/\hbar }}$, (3) $f(x) = {e^{i{p^2}x/\hbar }}$. 
Какие из них будут собственными функциями оператора квадрата импульса?
\begin{choices}
\choice только (1)      
\choice (1) и (2)    
\choice (1) и (3)    
\choice все
\end{choices}

\question Состояние частицы описывается нормированной волновой функцией $\psi (x,t)$. Разло-жим эту функцию по нормированным на дельта-функцию от импульса собственным функциям оператора импульса ${f_p}(x)$: $\psi (x,t) = \int {dp} \,C(p,t){f_p}(x)$. Вероятность того, что в момент времени $t$ импульс частицы лежит в интервале $p\; \div \;p + dp$, где $dp$ - некото-рый малый интервал, равна
\begin{choices}
\choice нулю      
\choice $\left| {C(p,t)} \right|dp$    
\choice ${\left| {{f_p}(x)} \right|^2}dp$    
\choice ${\left| {C(p,t)} \right|^2}dp$
\end{choices}

\question Состояние частицы описывается нормированной волновой функцией $\psi (x,t)$. Разло-жим эту функцию по нормированным на дельта-функцию от импульса собственным функциям оператора импульса ${f_p}(x)$: $\psi (x,t) = \int {dp} \,C(p,t){f_p}(x)$. Вероятность того, что при измерении импульса частицы в момент времени $t$ будет обнаружено некоторое значение ${p_0}$, равна
\begin{choices}
\choice $\left| {{\varphi _{{p_0}}}(x)} \right|$      
\choice $\left| {C({p_0},t)} \right|$     
\choice нулю      
\choice ${\left| {C({p_0},t)} \right|^2}$
\end{choices}

\question Частица находится в состоянии, в котором ее координата $x$ имеет определенное значе-ние $a$. Проводят измерение проекции импульса частицы на ось $x$. Какие значения можно при этом получить и с какими вероятностями?
\begin{choices}
\choice любые действительные значения с одинаковыми вероятностями
\choice любые положительные действительные значения с одинаковыми вероятностями
\choice $\hbar /a$ и $ - \hbar /a$ с одинаковыми вероятностями
\choice $\hbar /a$ с единичной вероятностью
\end{choices}

\question Частица находится в состоянии, в котором ее импульс имеет определенное значение ${p_0}$. Проводят измерение координаты частицы. Какие значения можно при этом получить и с какими вероятностями?
\begin{choices}
\choice любые действительные значения с одинаковыми вероятностями
\choice значение $\hbar /{p_0}$ с единичной вероятностью
\choice значение $ - \hbar /{p_0}$ с единичной вероятностью
\choice значения $\hbar /{p_0}$ и $ - \hbar /{p_0}$ с одинаковыми вероятностями
\end{choices}

\question Коммутатор операторов координаты и проекции импульса $\left[ {\hat x\,,{{\hat p}_x}} \right]$ равен
\begin{choices}
\choice $ - i\hbar $    
\choice $\hat x$     
\choice $i\hbar $    
\choice 0
\end{choices}

\question Какая из нижеперечисленных функций является общей собственной функцией операто-ров $\hat x$ и ${\hat p_x}$ (здесь $a,b$ - произвольные действительные числа)?
\begin{choices}
\choice $\delta (x - a)\sin bx$        
\choice $\delta (x - a)\exp ( - ibx)$
\choice $\delta (x - a)\exp (ibx)$        
\choice такой функции не существует 
\end{choices}

\question Коммутируют ли операторы координаты $\hat x$ и четности $\hat P$ 
\begin{choices}
\choice да     
\choice нет    
\choice зависит от состояния     
\choice зависит от гамильтониана
\end{choices}

\question Коммутатор операторов координаты $\hat x$ и четности $\hat P$ равен
\begin{choices}
\choice $\left[ {\hat x\,,\hat P} \right] = 0$     
\choice $\left[ {\hat x\,,\hat P} \right] = 2\hat x$     
\choice $\left[ {\hat x\,,\hat P} \right] = 2\hat P$  
\choice $\left[ {\hat x\,,\hat P} \right] = 2\hat x\hat P$
\end{choices}

\question Имеют ли операторы координаты и четности, действующие в пространстве функций од-ной переменной, полную систему общих собственных функций?
\begin{choices}
\choice да
\choice нет
\choice в некоторых случаях имеют, в некоторых нет
\choice это зависит от размерности пространства
\end{choices}

\question Имеют ли операторы координаты и четности, действующие в пространстве функций од-ной переменной, общие собственные функции?
\begin{choices}
\choice да
\choice нет
\choice в некоторых случаях имеют, в некоторых нет
\choice это зависит от размерности пространства
\end{choices}

\question Какая из нижеперечисленных функций является общей собственной функцией операто-ров координаты и четности?
\begin{choices}
\choice $\delta (x)$    
\choice $\delta ({x^2} - {a^2})$    
\choice такой функции не существует    
\choice ${\delta ^2}(x - a)$
\end{choices}

\question Коммутатор операторов квадрата координаты ${\hat x^2}$ и четности $\hat P$ равен
\begin{choices}
\choice $\left[ {{{\hat x}^2},\hat P} \right] = 0$    
\choice $\left[ {{{\hat x}^2},\hat P} \right] = 2{\hat x^2}$      
\choice $\left[ {{{\hat x}^2},\hat P} \right] = 2\hat P$    
\choice $\left[ {{{\hat x}^2},\hat P} \right] = 2{\hat x^2}\hat P$
\end{choices}

\question Имеют ли операторы квадрата координаты и четности полную систему общих собствен-ных функций?
\begin{choices}
\choice да     
\choice нет    
\choice зависит от состояния     
\choice зависит от гамильтониана
\end{choices}

\question Даны три функции
(1) $f(x) = \delta (x{\rm{ - }}a)$, (2) $f(x) = (1/2)\delta (x{\rm{ - }}a) + (1/2)\delta (x + a)$, (3) $f(x) = (1/4)\delta (x{\rm{ - }}a) + (3/4)\delta (x + a)$. 
Какие из них являются общими собственными функциями операторов квадрата координаты и четности?
\begin{choices}
\choice только (1)      
\choice только (2)      
\choice только (3)      
\choice ни одна
\end{choices}

\question Каковы свойства четности собственных функций оператора квадрата координаты?
\begin{choices}
\choice все четные
\choice все обладают определенной четностью
\choice таких функций не существует
\choice могут быть выбраны так, чтобы обладали определенной четностью
\end{choices}

\question Какая из нижеследующих функций является общей собственной функцией операторов ${\hat p_x}$, ${\hat p_y}$ и ${\hat p_z}$ ($a,b,c$ - произвольные действительные числа)?
\begin{choices}
\choice $\sin ax\sin by\sin cz$           
\choice такой функции не существует
\choice $\exp (iax)\exp (iby)\exp (icz)$        
\choice $\exp (ax)\exp (by)\exp (cz)$ 
\end{choices}

\question В каком из нижеперечисленных состояний частица имеет определенный вектор импуль-са ($a,\;b,\;c$ - произвольные действительные числа)?
\begin{choices}
\choice $\exp (ax)\exp (by)\exp (cz)$        
\choice $\cos ax\cos by\cos cz$
В.$\exp (iax)\exp (iby)\exp (icz)$        
\choice таких состояний не существует 
\end{choices}

\question Состояние частицы описывается волновой функцией $\psi (x,y,z) = a\delta (x - b)\exp (icy)\sin (dz)$, где $a$, $b$, $c$ и $d$ - некоторые действительные числа. Какие из величин $x$, $y$, $z$, ${p_x}$, ${p_y}$, ${p_z}$ имеют в этом состоянии определенные значения?
\begin{choices}
\choice $y$ и ${p_x}$      
\choice $x$ и ${p_y}$      
\choice $z$ и ${p_z}$      
\choice никакие из перечисленных
\end{choices}

\question Квадрат модуля нормированной волновой функции частицы в импульсном представле-нии определяет вероятности
\begin{choices}
\choice различных значений координаты частицы
\choice различных значений координаты и импульса частицы
\choice различных значений энергии частицы
\choice различных значений импульса частицы 
\end{choices}

\question Переходу к импульсному представлению отвечает
\begin{choices}
\choice введение в задачу классического импульса
\choice дифференцирование волновой функции по координате
\choice дифференцирование волновой функции по импульсу
\choice разложение волновой функции по собственным функциям оператора импульса
\end{choices}

\question Состояние частицы описывается волновой функцией $\psi (x,t)$. По какой формуле мож-но найти волновую функцию этого состояния в импульсном представлении $C(p,t)$?
\begin{choices}
\choice $C(p,t) = \psi (p,t)$             
\choice $C(p,t) = \frac{1}{{\sqrt {2\pi \hbar } }}\int {\psi (x,t){e^{ - i\frac{{px}}{\hbar }}}dx} $
\choice $C(p,t) = \frac{1}{{\sqrt {2\pi \hbar } }}\int {\psi (x,t)\sin \left( {px/\hbar } \right)dx} $      
\choice $C(p,t) = \frac{1}{{\sqrt {2\pi \hbar } }}\int {\psi (x,t){e^{i\frac{{px}}{\hbar }}}dx} $
\end{choices}

\question Дана волновая функция некоторого состояния частицы в импульсном представлении $C(p,t)$. По какой формуле можно найти волновую функцию этого состояния в координатном представлении $\psi (x,t)$:
\begin{choices}
\choice $\Psi (x,t) = C(x,t)$             
\choice $\psi (x,t) = \frac{1}{{\sqrt {2\pi \hbar } }}\int {C(p,t){e^{ - i\frac{{px}}{\hbar }}}dp} $  
\choice $\psi (x,t) = \frac{1}{{\sqrt {2\pi \hbar } }}\int {C(p,t){e^{i\frac{{px}}{\hbar }}}dp} $        
\choice $\psi (x,t) = \frac{1}{{\sqrt {2\pi \hbar } }}\int {C(p,t)\sin \left( {px/\hbar } \right)dp} $
\end{choices}

\question Волновая функция состояния частицы в координатном представлении имеет вид $\Psi (x) = \delta \left( {x - a} \right)$, где $a$ - некоторое действительное число. Какой будет волновая функция этого состояния в импульсном представлении?
\begin{choices}
\choice $C(p) = \frac{1}{{\sqrt {2\pi \hbar } }}\exp \left( { - i\frac{{pa}}{\hbar }} \right)$     
\choice $C(p) = \frac{1}{{\sqrt {2\pi \hbar } }}\exp \left( {i\frac{{pa}}{\hbar }} \right)$
\choice $C(p) = \delta \left( {p - (\hbar /a)} \right)$        
\choice $C(p) = \frac{1}{{\sqrt {2\pi \hbar } }}\sin \left( {\frac{{pa}}{\hbar }} \right)$
\end{choices}

\question В каких из нижеперечисленных состояний частицы
(1) $C(p) = \frac{1}{{\sqrt {2\pi \hbar } }}\sin \left( {\frac{{pa}}{\hbar }} \right)$, (2) $C(p) = \frac{1}{{\sqrt {2\pi \hbar } }}\exp \left( {i\frac{{pa}}{\hbar }} \right)$, (3) $C(p) = \delta \left( {p - (\hbar /a)} \right)$
ее координата имеет определенное значение (здесь $C(p)$ - волновые функции состояний в им-пульсном представлении, $a$ - число)
\begin{choices}
\choice только в (1) 
\choice только в (2) 
\choice только в (3) 
\choice ни в одном из них
\end{choices}

\question В каких из нижеперечисленных состояний частицы
(1) $C(p) = \frac{1}{{\sqrt {2\pi \hbar } }}\sin \left( {\frac{{pa}}{\hbar }} \right)$, (2) $C(p) = \frac{1}{{\sqrt {2\pi \hbar } }}\exp \left( {i\frac{{pa}}{\hbar }} \right)$, (3) $C(p) = \delta \left( {p - (\hbar /a)} \right)$
ее импульс имеет определенное значение (здесь $C(p)$ - волновые функции состояний в им-пульсном представлении, $a$ - число)
\begin{choices}
\choice только в (1) 
\choice только в (2) 
\choice только в (3) 
\choice ни в одном из них
\end{choices}

\question Дано разложение волновой функции $\psi (x,t) = \frac{1}{{\sqrt {2\pi \hbar } }}\int {C(p,t){e^{i\frac{{px}}{\hbar }}}dp} $, где $x$ - координата частицы. Что в этом равенстве есть волновая функция в импульсном представлении?
\begin{choices}
\choice $\psi (x,t)$    
\choice $C(p,t)$     
\choice ${e^{i\frac{{px}}{\hbar }}}$         
\choice $dp$
\end{choices}

\question Собственная функция ${f_{{p_1}}}(p)$ оператора импульса, отвечающая собственному значению ${p_1}$, в импульсном представлении равна
\begin{choices}
\choice ${f_{{p_1}}}(p) = \exp \left( { - i\frac{{(p - {p_1})}}{\hbar }} \right)$         
\choice ${f_{{p_1}}}(p) = \delta \left( {p - {p_1}} \right)$
\choice ${f_{{p_1}}}(p) = \cos \left( {\frac{{(p - {p_1})}}{\hbar }} \right)$       
\choice ${f_{{p_1}}}(p) = \exp \left( {i\frac{{(p - {p_1})}}{\hbar }} \right)$
\end{choices}

\question Собственная функция ${f_a}(p)$ оператора координаты, отвечающая собственному зна-чению $a$, в импульсном представлении равна
\begin{choices}
\choice ${f_a}(p) = \exp \left( {i\frac{{pa}}{\hbar }} \right)$      
\choice ${f_a}(p) = \delta \left( {p - a} \right)$
\choice ${f_a}(p) = \delta \left( {p - (\hbar /a)} \right)$    
\choice ${f_a}(p) = \exp \left( { - i\frac{{pa}}{\hbar }} \right)$
\end{choices}

\question Оператор координаты $\hat x$ в импульсном представлении – это
\begin{choices}
\choice $\hat x =  - i\hbar \frac{d}{{d{p_x}}}$       
\choice умножение на координату $x$
\choice $i\hbar \frac{d}{{d{p_x}}}$          
\choice умножение на импульс ${p_x}$
\end{choices}

\question Оператор импульса ${\hat p_x}$ в импульсном представлении – это
\begin{choices}
\choice ${\hat p_x} =  - i\hbar \frac{d}{{d{p_x}}}$         
\choice умножение на импульс ${p_x}$
\choice ${\hat p_x} = i\hbar \frac{d}{{d{p_x}}}$         
\choice умножение на координату $x$ 
\end{choices}

\question Оператор физической величины $A$ в $a$-представлении – это
\begin{choices}
\choice деление на $a$        
\choice дифференцирование по $a$
\choice умножение на ${a^2}$        
\choice умножение на $a$
\end{choices}

\question Оператор величины $A$ имеет дискретный спектр собственных значений и собствен-ных функций. Собственная функция оператора $\hat A$, отвечающая собственному значению ${a_1}$, в $a$-представлении равна (здесь ${\delta _{...,...}}$ - дельта-символ, $\delta (... - ...)$ - дельта-функция)
\begin{choices}
\choice $\exp ( - i(a - {a_1})\hbar )$          
\choice $\exp ( - i{a_1}\hbar )$
\choice ${\delta _{a,{a_1}}}$          
\choice $\delta (a - {a_1})$
\end{choices}

\question Оператор величины $A$ имеет непрерывный спектр собственных значений и собствен-ных функций. Собственная функция оператора $\hat A$, отвечающая собственному значению ${a_1}$, в $a$-представлении равна (здесь ${\delta _{...,...}}$ - дельта-символ, $\delta (... - ...)$ - дельта-функция)
\begin{choices}
\choice $\exp ( - i(a - {a_1})\hbar )$          
\choice $\exp ( - i{a_1}\hbar )$
\choice ${\delta _{a,{a_1}}}$          
\choice $\delta (a - {a_1})$
\end{choices}

\question Пусть физической величине $A$ отвечает оператор ${\hat A_x}$, действие которого в координатном представлении выражается формулой ${\hat A_x}f(x) = g(x)$, где $f(x)$ и $g(x)$ - функции координаты. Какой формулой определяется действие оператора величины $A$ в им-пульсном представлении?
\begin{choices}
\choice ${\hat A_p}f(p) = g(p)$       
\choice ${\hat A_p}g(p) = f(p)$         
\choice ${\hat A_p}{C_g}(p) = {C_f}(p)$        
\choice ${\hat A_p}{C_f}(p) = {C_g}(p)$
(${C_f}(p)$ и ${C_g}(p)$ - волновые функции состояний $f(x)$ и $g(x)$ в импульсном представ-лении) 
\end{choices}

\question Оператор величины $A$ имеет дискретный спектр собственных значений, а оператор величины $B$ - непрерывный спектр собственных значений и собственных функций. Собственная функция оператора $\hat A$ в $b$-представлении зависит от
\begin{choices}
\choice непрерывного аргумента и дискретного индекса
\choice непрерывного аргумента и непрерывного индекса как от параметра  
\choice дискретного аргумента и непрерывного индекса как от параметра
\choice дискретного аргумента и дискретного индекса
\end{choices}

\question Как связаны собственная функция ${f_a}(b)$ оператора $\hat A$ в $b$-представлении и собственная функция ${g_b}(a)$ оператора $\hat B$ в $a$-представлении?
\begin{choices}
\choice никак     
\choice ${f_a}(b) = {g_b}(a)$    
\choice ${f_a}(b) =  - {g_b}(a)$    
\choice ${f_a}(b) = g_b^*(a)$
\end{choices}

\question Оператора $\hat A$ и $\hat B$ коммутируют и существует собственная функция опера-тора $\hat A$, которая не является собственной функцией оператора $\hat B$. Какое утвержде-ние обязательно имеет место?
\begin{choices}
\choice такого быть не может           
\choice собственные значения $\hat A$ вырождены
\choice собственные значения $\hat B$ вырождены       
\choice собственные значения $\hat A$ и $\hat B$ вырождены
\end{choices}

\end{questions}

\subsection{ Зависимость физических величин от времени. Уравнение Шредингера }

\begin{questions}

\question Частица находится во внешнем поле $U(\vec r,t)$. Какой из приведенных формул опреде-ляется оператор Гамильтона частицы $\hat H$ в координатном представлении?
\begin{choices}
\choice $\hat H =  - \frac{{i\hbar }}{m}\nabla  + U(\vec r,t)$    
\choice $\hat H =  - \frac{{{\hbar ^2}}}{{2m}}\Delta  - U(\vec r,t)$
\choice $\hat H = U(\vec r,t)$         
\choice $\hat H =  - \frac{{{\hbar ^2}}}{{2m}}\Delta  + U(\vec r,t)$
\end{choices}

\question Какой из нижеперечисленных операторов является оператором Гамильтона свободной одномерной частицы в импульсном представлении?
\begin{choices}
\choice оператор умножения на $ - \frac{{{\hbar ^2}{p^2}}}{{2m}}$    
\choice оператор умно-жения на $ - \frac{{{\hbar ^2}{x^2}}}{{2m}}$
\choice оператор умножения на $\frac{{{p^2}}}{{2m}}$     
\choice оператор дифференцирования $ - \frac{{{\hbar ^2}}}{{2m}}\frac{{{d^2}}}{{d{p^2}}}$ 
\end{choices}

\question Какое из нижеследующих уравнений является временным уравнением Шредингера для волновой функции частицы?
\begin{choices}
\choice $i\hbar \frac{{\partial {\kern 1pt} \Psi }}{{\partial {\kern 1pt} t}} = \hat H{\kern 1pt} \Psi $ 
\choice $i\hbar \frac{{\partial {{\kern 1pt} ^2}\Psi }}{{\partial {\kern 1pt} {t^2}}} = \hat H{\kern 1pt} \Psi $  
\choice $i\hbar \frac{{\partial {\kern 1pt} \Psi }}{{\partial {\kern 1pt} t}} = {\hat H^2}{\kern 1pt} \Psi $   
\choice $i\hbar \frac{{\partial {{\kern 1pt} ^2}\Psi }}{{\partial {\kern 1pt} {t^2}}} = {\hat H^2}{\kern 1pt} \Psi $
\end{choices}

\question Частица находится во внешнем поле $U(\vec r,t)$. Какое из нижеследующих уравнений является временным уравнением Шредингера для волновой функции этой частицы?
\begin{choices}
\choice $i\hbar \frac{{\partial {\kern 1pt} \Psi }}{{\partial {\kern 1pt} t}} = \left( { - \frac{{{\hbar ^2}}}{{2m}}\Delta  + U(\vec r,t){\kern 1pt} } \right){\kern 1pt} \Psi $       
\choice $i\hbar \frac{{\partial {\kern 1pt} \Psi }}{{\partial {\kern 1pt} t}} = U(\vec r,t){\kern 1pt} \Psi $
\choice $i\hbar \frac{{\partial {\kern 1pt} \Psi }}{{\partial {\kern 1pt} t}} = \left( { - \frac{{{\hbar ^2}}}{{2m}}\Delta  - U(\vec r,t){\kern 1pt} } \right){\kern 1pt} \Psi $       
\choice $\left( { - \frac{{{\hbar ^2}}}{{2m}}\Delta  + U(\vec r,t){\kern 1pt} } \right){\kern 1pt} \Psi  = \,E{\kern 1pt} \Psi $
\end{choices}

\question Для однозначного нахождения решения временного уравнения Шредингера нужно задать:
\begin{choices}
\choice волновую функцию в начальный момент времени
\choice волновую функцию и ее первую производную по времени в начальный момент времени
\choice волновую функцию, ее первую и вторую производные по времени в начальный момент вре-мени
\choice волновую функцию, ее первую, вторую и третью производные по времени в начальный момент времени
\end{choices}

\question Частица находится во внешнем поле $U(\vec r)$. Какое из нижеследующих уравнений яв-ляется стационарным уравнением Шредингера для энергий и волновых функций стационарных состояний этой частицы?
\begin{choices}
\choice $i\hbar \frac{{\partial {\kern 1pt} \Psi }}{{\partial {\kern 1pt} t}} = \left( { - \frac{{{\hbar ^2}}}{{2m}}\Delta  + U(\vec r){\kern 1pt} } \right){\kern 1pt} \Psi $      
\choice $\left( { - \frac{{{\hbar ^2}}}{{2m}}\Delta  + U(\vec r){\kern 1pt} } \right){\kern 1pt} \Psi  = \,E{\kern 1pt} \Psi $
\choice $i\hbar \frac{{\partial {\kern 1pt} \Psi }}{{\partial {\kern 1pt} t}} = E{\kern 1pt} \Psi $            
\choice $ - \frac{{{\hbar ^2}}}{{2m}}\Delta {\kern 1pt} \Psi  = \,U(\vec r){\kern 1pt} \Psi $ 
\end{choices}

\question Какие из ниже перечисленных уравнений или законов являются уравнениями на собст-венные значения и собственные функции какого-либо оператора?
\begin{choices}
\choice временное уравнение Шредингера    
\choice стационарное уравнение Шредингера
\choice закон сохранения вероятности         
\choice принцип суперпозиции
\end{choices}

\question Гамильтониан некоторой квантовой системы не зависит от времени. Собственные функ-ции ${f_n}(x)$ и собственные значения ${E_n}$ этого гамильтониана известны. Какой из нижеследующих формул описывается общее решение временного уравнения Шредингера $\Psi (x,t)$?
\begin{choices}
\choice $\Psi (x,t) = \sum\limits_n {{C_n}} \,{f_n}(x){e^{ - i\frac{{{E_n}t}}{\hbar }}}$           
\choice $\Psi (x,t) = \sum\limits_n {{C_n}} \,{f_n}(x){e^{i\frac{{{E_n}t}}{\hbar }}}$
\choice $\Psi (x,t) = \sum\limits_n {{C_n}} \,{f_n}(x){e^{ - \frac{{{E_n}t}}{\hbar }}}\,$       
\choice $\Psi (x,t) = \sum\limits_n {{C_n}} \,{f_n}(x){e^{\frac{{{E_n}t}}{\hbar }}}\,$
\end{choices}

\question Гамильтониан некоторой квантовой системы не зависит от времени. Собственные функ-ции ${f_n}(x)$ и собственные значения ${E_n}$ этого гамильтониана известны. Какой из ниже перечисленных функций определяется волновая функция стационарного состояния системы?
\begin{choices}
\choice $\Psi (x,t) = {f_n}(x){e^{ - i\frac{{{E_n}t}}{\hbar }}}\,$   
\choice $\Psi (x,t) = {f_n}(x)\,$   
\choice $\Psi (x,t) = {e^{ - i\frac{{{E_n}t}}{\hbar }}}\,$    
\choice $\Psi (x,t) = \sum\limits_n {{C_n}} \,{f_n}(x){e^{ - i\frac{{{E_n}t}}{\hbar }}}$
\end{choices}

\question Гамильтониан частицы не зависит от времени. Будут ли зависеть от времени волновые функции стационарных состояний частицы?
\begin{choices}
\choice нет                
\choice да
\choice это зависит от начальных условий  
\choice это зависит от гамильтониана
\end{choices}

\question Гамильтониан частицы не зависит от времени. Существуют ли среди состояний частицы такие, в которых вероятности тех или иных физических величин зависят от времени?
\begin{choices}
\choice нет    
\choice да     
\choice это зависит от гамильтониана      
\choice бессмысленный вопрос
\end{choices}

\question Потенциальная энергия частицы не зависит от времени. Какое начальное условие опре-деляет стационарное состояние частицы (здесь ${f_n}(x)$ - собственные функции оператора Гамильтона частицы, отвечающие разным собственным значениям)?
\begin{choices}
\choice $\Psi (x,t = 0) = {f_3}(x)$          
\choice $\Psi (x,t = 0) = \left( {{f_4}(x) - {f_2}(x)} \right)/\sqrt 2 $
\choice $\Psi (x,t = 0) = {f_4}(x) - {f_3}(x)$        
\choice никакое из перечисленных
\end{choices}

\question Потенциальная энергия частицы не зависит от времени. Волновая функция частицы в начальный момент времени $\Psi (x,t = 0)$ совпадает с одной из собственных функций оператора Гамильтона частицы. Как зависит от времени среднее значение координаты частицы в этом состоянии?
\begin{choices}
\choice растет 
\choice убывает   
\choice не зависит от времени 
\choice осциллирует
\end{choices}

\question Потенциальная энергия частицы не зависит от времени. Известно, что частица находится в состоянии с определенной энергией. Как зависит от времени среднее значение координаты частицы?
\begin{choices}
\choice растет 
\choice убывает   
\choice не зависит от времени 
\choice это зависит от состояния
\end{choices}

\question Потенциальная энергия частицы не зависит от времени. Частица находится в состоянии с определенной энергией. Как зависит от времени среднее значение импульса частицы в этом состоянии?
\begin{choices}
\choice растет 
\choice убывает   
\choice не зависит от времени 
\choice осциллирует
\end{choices}

\question Потенциальная энергия частицы не зависит от времени. Частица находится в состоянии, в котором энергия может принимать два значения ${E_1}$ и ${E_2}$ с определенными веро-ятностями. Как зависит от времени средне значение физической величины $A$ в этом состоя-нии? Известно, что $\left( {{f_1},\hat A{f_2}} \right) \ne 0$. Здесь ${f_1}$ и ${f_2}$  собствен-ные функции гамильтониана, отвечающие собственным значениям ${E_1}$ и ${E_2}$.
\begin{choices}
\choice растут 
\choice убывают   
\choice не зависят от времени 
\choice осциллируют
\end{choices}

\question Потенциальная энергия частицы не зависит от времени. Частица находится в состоянии, в котором энергия может принимать два значения ${E_1}$ и ${E_2}$ с определенными веро-ятностями. Как зависит от времени средне значение физической величины $A$ в этом состоя-нии? Известно, что $\left( {{f_1},\hat A{f_2}} \right) = 0$. Здесь ${f_1}$ и ${f_2}$  собствен-ные функции гамильтониана, отвечающие собственным значениям ${E_1}$ и ${E_2}$.
\begin{choices}
\choice растут 
\choice убывают   
\choice не зависят от времени 
\choice осциллируют
\end{choices}

\question Потенциальная энергия частицы не зависит от времени. Частица находится в состоянии, в котором энергия может принимать два значения ${E_1}$ и ${E_2}$ с определенными веро-ятностями. С какой частотой осциллирует среднее значение координаты частицы в этом состоя-нии? Известно, что $\left( {{f_1},\hat x{f_2}} \right) \ne 0$, где ${f_1}$ и ${f_2}$  собственные функции гамильтониана, отвечающие собственным значениям ${E_1}$ и ${E_2}$.
\begin{choices}
\choice $\left( {{E_1} + {E_2}} \right)/\hbar $ 
\choice $\left( {{E_1} - {E_2}} \right)/\hbar $ 
\choice $\sqrt {{E_1}{E_2}} /\hbar $      
\choice не зависит от времени
\end{choices}

\question Гамильтониан частицы не зависит от времени. Частица находится в стационарном со-стоянии. Как зависят от времени вероятности различных значений некоторой физической вели-чины, оператор которой не коммутирует с оператором Гамильтона?
\begin{choices}
\choice не зависят от времени 
\choice растут 
\choice убывают      
\choice осциллируют
\end{choices}

\question Гамильтониан системы не зависит от времени. Среднее значение физической величины в некотором состоянии зависит от времени. Какое из нижеследующих утверждений относительно свойств этого состояния и оператора физической величины обязательно справедливо?
\begin{choices}
\choice энергия системы в этом состоянии имеет определенное значение
\choice оператор физической величины не коммутирует с оператором Гамильтона
\choice оператор этой физической величины коммутирует с оператором Гамильтона
\choice импульс системы в этом состоянии имеет определенное значение
\end{choices}

\question Гамильтониан квантовой системы не зависит от времени. Частица находится в стацио-нарном состоянии. Какое утверждение обязательно является верным? 
\begin{choices}
\choice волновая функция этого состояния не зависит от времени
\choice импульс частицы в этом состоянии имеет определенное значение
\choice энергия частицы в этом состоянии имеет определенное значение
\choice оператор импульса коммутирует с оператором Гамильтона
\end{choices}

\question Гамильтониан квантовой системы не зависит от времени. Как зависят от времени вероятности различных значений энергии системы?
\begin{choices}
\choice растут 
\choice убывают   
\choice не зависят от времени 
\choice это зависит от состояния
\end{choices}

\question Свободная частица находится в некотором нестационарном состоянии. Как зависят от времени вероятности различных значений энергии частицы?
\begin{choices}
\choice растут 
\choice убывают   
\choice не зависят от времени 
\choice это зависит от состояния
\end{choices}

\question Гамильтониан квантовой системы не зависит от времени. Как зависит от времени сред-нее значение координаты в некотором состоянии?
\begin{choices}
\choice растет 
\choice убывает   
\choice не зависит от времени 
\choice это зависит от состояния
\end{choices}

\question Гамильтониан частицы не зависит от времени. Частица находится в таком состоянии, в котором среднее значение любой физической величины не зависит от времени. Измеряют энер-гию частицы. Что будет обнаружено в результате измерений?
\begin{choices}
\choice любое число из некоторого интервала значений
\choice все собственные значения гамильтониана с равными вероятностями
\choice некоторое собственное значение гамильтониана с единичной вероятностью
\choice информации для ответа не достаточно
\end{choices}

\question Какой формулой определяется оператор производной оператора $\hat B$ физической ве-личины по времени? Здесь $\hat H$ - оператор Гамильтона, $\hat x$ - оператор координаты, $\hat p$ - оператор импульса. 
\begin{choices}
\choice $\hat \dot B = \frac{{\partial \hat B}}{{\partial t}}$ 
\choice $\hat \dot B = \frac{{\partial \hat B}}{{\partial t}} + \frac{i}{\hbar }\left[ {\hat H,\hat B} \right]$      
\choice $\hat \dot B = \frac{{\partial \hat B}}{{\partial t}} + \frac{i}{\hbar }\left[ {\hat p,\hat B} \right]$      
\choice $\hat \dot B = \frac{{\partial \hat B}}{{\partial t}} + \frac{i}{\hbar }\left[ {\hat p,\hat B} \right]$
\end{choices}

\question Какой формулой определяется оператор $\hat \dot x$? Здесь $\hat H$ - оператор Гамиль-тона, $\hat x$ - оператор координаты, $\hat p$ - оператор импульса. 
\begin{choices}
\choice $\hat \dot x = \frac{{\partial \hat x}}{{\partial t}}$    
\choice $\hat \dot x = \frac{{\hat p}}{{2m}}$      
\choice $\hat \dot x = \sqrt {\frac{{2\hat H}}{m}} $     
\choice $\hat \dot x = \frac{{\hat p}}{m}$
\end{choices}

\question Какой формулой определяется оператор $\hat \dot p$? Здесь $\hat H$ - оператор Гамиль-тона, $\hat x$ - оператор координаты, $\hat p$ - оператор импульса, $\hat U$ - оператор потен-циальной энергии. 
\begin{choices}
\choice $\hat \dot p = \frac{{\partial \hat p}}{{\partial t}}$    
\choice $\hat \dot p = \frac{{2\hat H}}{m}$     
\choice $\hat \dot p =  - \frac{{\partial \hat U}}{{\partial x}}$    
\choice $\hat \dot p =  - \frac{{\partial \hat U}}{{\partial t}}$
\end{choices}

\question Какой формулой определяется оператор $\hat \dot H$? Здесь $\hat H$ - оператор Гамиль-тона, $\hat x$ - оператор координаты, $\hat p$ - оператор импульса, $\hat U$ - оператор потен-циальной энергии. 
\begin{choices}
\choice $\hat \dot H = \frac{{\partial \hat H}}{{\partial t}}$    
\choice $\hat \dot H = \frac{{\hat p}}{m}$      
\choice $\hat \dot H = m\hat x$     
\choice $\hat \dot H = \frac{{\partial \hat U}}{{\partial x}}$
\end{choices}

\question Частица движется в потенциале $U(x) = \alpha {x^2}$. Найти оператор $\hat \dot P$, где $\hat P$ - оператор четности. Здесь $\hat H$ - оператор Гамильтона, $\hat x$ - оператор коор-динаты, $\hat p$ - оператор импульса. 
\begin{choices}
\choice $\hat \dot P = \frac{{\partial \hat P}}{{\partial t}}$    
\choice $\hat \dot P = 0$     
\choice $\hat \dot P = \frac{{\hat p\hat x}}{\hbar }$    
\choice $\hat \dot P = \frac{{m\hat H{{\hat x}^2}}}{{{\hbar ^2}}}$
\end{choices}

\question Частица движется в потенциале $U(x) = \alpha x$. Найти оператор $\hat \dot P$, где $\hat P$ - оператор четности. Здесь $\hat H$ - оператор Гамильтона, $\hat x$ - оператор координаты, $\hat p$ - оператор импульса, $\hat U$ - оператор потенциальной энергии. 
\begin{choices}
\choice $\hat \dot P = \frac{{\partial \hat P}}{{\partial t}}$    
\choice $\hat \dot P = 0$     
\choice $\hat \dot P = \frac{{2i\alpha \hat x\hat P}}{\hbar }$    
\choice $\hat \dot P = \frac{{i\alpha \hat x\hat P}}{\hbar }$
\end{choices}


\question Найти производную по времени единичного оператора $\hat \dot I$.
Здесь $\hat P$ - оператор четности, $\hat H$ - оператор Гамильтона, $\hat U$ - оператор потен-циальной энергии. 
\begin{choices}
\choice $\hat \dot I = \hat P$      
\choice $\hat \dot I = \hat O$      
\choice $\hat \dot I = {\hat P^{ - 1}}$      
\choice $\hat \dot I = {\hat P^ + }$
\end{choices}

\question Что означает, что физическая величина $A$ для некоторой квантовой системы является интегралом движения? 
\begin{choices}
\choice совпадают результаты всех измерений величины $A$, выполненных в разные моменты вре-мени над ансамблем квантовых систем
\choice не меняется оператор Гамильтона квантовой системы
\choice не зависит от времени среднее значение результатов многих измерений величины $A$ в лю-бом состоянии
\choice не зависит от времени волновая функция квантовой системы
\end{choices}

\question Физическая величина является интегралом движения, если
\begin{choices}
\choice оператор этой величины не зависит от времени
\choice оператор этой величины не зависит от времени и коммутирует с оператором импульса
\choice оператор этой величины не зависит от времени и коммутирует с оператором координаты
\choice оператор этой величины не зависит от времени и коммутирует с оператором Гамильтона 
\end{choices}

\question Среднее значение физической величины $A$ в некотором состоянии не зависит от вре-мени. Является ли величина $A$ интегралом движения?
\begin{choices}
\choice да     
\choice нет    
\choice бессмысленный вопрос     
\choice мало информации
\end{choices}

\question Среднее значение физической величины $A$ в любом состоянии не зависит от времени. Является ли величина $A$ интегралом движения?
\begin{choices}
\choice да     
\choice нет    
\choice бессмысленный вопрос     
\choice мало информации
\end{choices}

\question Если оператор некоторой физической величины не зависит от времени и коммутирует с оператором Гамильтона, то
\begin{choices}
\choice среднее значение этой величины в любом состоянии не зависит от времени
\choice эта величина имеет определенное значение в любом состоянии
\choice эта величина есть энергия
\choice среднее значение этой величины не зависит от времени только в стационарных состояниях
\end{choices}

\question Оператор Гамильтона частицы не коммутирует с оператором импульса. Будет ли энергия частицы интегралом движения?
\begin{choices}
\choice да     
\choice нет    
\choice это зависит от состояния    
\choice мало информации для ответа
\end{choices}

\question Оператор Гамильтона частицы коммутирует с оператором импульса. Будет ли энергия частицы интегралом движения?
\begin{choices}
\choice да     
\choice нет    
\choice это зависит от состояния    
\choice мало информации для ответа
\end{choices}

\question Оператор Гамильтона частицы коммутирует с оператором импульса. Будет ли импульс частицы интегралом движения?
\begin{choices}
\choice да     
\choice нет    
\choice это зависит от состояния    
\choice мало информации для ответа
\end{choices}

\question Оператор Гамильтона частицы коммутирует с оператором четности. Будет ли энергия частицы интегралом движения?
\begin{choices}
\choice да     
\choice нет    
\choice это зависит от состояния    
\choice мало информации для ответа
\end{choices}

\question Частица движется в одномерном потенциале $U(x)$. В каком случае четность является интегралом движения?
\begin{choices}
\choice если потенциальная энергия обладает определенной четностью
\choice если потенциальная энергия – четная функция
\choice если потенциальная энергия – нечетная функция
\choice четность всегда интеграл движения
\end{choices}

\question Энергия квантовой системы является интегралом движения, если
\begin{choices}
\choice если оператор Гамильтона коммутирует с оператором четности
\choice если оператор Гамильтона коммутирует с оператором координаты
\choice если оператор Гамильтона коммутирует сам с собой
\choice если оператор Гамильтона не зависит от времени
\end{choices}

\question Средний импульс частицы в некотором состоянии не зависит от времени. Будет ли опе-ратор импульса коммутировать с оператором Гамильтона?
\begin{choices}
\choice да     
\choice нет    
\choice это зависит от импульса     
\choice мало информации для ответа
\end{choices}

\question Средний импульс частицы в любом состоянии квантовой системы не зависит от време-ни. Какие из нижеперечисленных утверждений обязатольно имеют место?
\begin{choices}
\choice частица является свободной
\choice все состояния системы стационарны
\choice оператор координаты коммутирует с оператором Гамильтона
\choice ни одно и перечисленных
\end{choices}

\question Оператор Гамильтона частицы не зависит от времени. Оператор некоторой физической величины коммутирует с оператором Гамильтона. Будут ли волновые функции стационарных состояний собственными функциями этого оператора?
\begin{choices}
\choice да     
\choice нет    
\choice вообще говоря нет, но могут быть выбраны так, чтобы были  
\choice это зависит от оператора
\end{choices}

\question Потенциальная энергия частицы не зависит от времени. Частица находится в состоянии с волновой функцией $\Psi (x,t)$. В каких из нижеперечисленных случаев интеграл $\overline {P(t)}  = \int {{\Psi ^*}(x,t)\hat P\Psi (x,t)dx} $ ($\hat P$ - оператор четности) не зависит от време-ни?
(1) состояние $\Psi (x,t)$ стационарно, 
(2) потенциальная четная функция, 
(3) потенциальная нечетная функция
\begin{choices}
\choice (1) и (2)    
\choice (1) или (2)     
\choice (1) и (3)    
\choice (1) или (3)  
\end{choices}

\question Потенциальная энергия частицы не равна нулю и не зависит от времени. Частица нахо-дится в состоянии с волновой функцией $\Psi (x,t)$. В каких из нижеперечисленных случаев интеграл $\overline {x(t)}  = \int {{\Psi ^*}(x,t)\hat x\Psi (x,t)dx} $ ($\hat x$ - оператор координаты) зависит от времени?
\begin{choices}
\choice этот интеграл всегда зависит от времени
\choice если состояние $\Psi (x,t)$ является стационарным
\choice если состояние является нестационарным
\choice если оператор координаты коммутирует с оператором Гамильтона
\end{choices}

\question Потенциальная энергия частицы не равна нулю и не зависит от времени. Частица нахо-дится в состоянии с волновой функцией $\Psi (x,t)$. В каких из нижеперечисленных случаев интеграл $\overline {p(t)}  = \int {{\Psi ^*}(x,t)\hat p\Psi (x,t)dx} $ ($\hat p$- оператор импульса) зависит от времени?
\begin{choices}
\choice этот интеграл всегда зависит от времени
\choice если состояние $\Psi (x,t)$ является стационарным
\choice если состояние является нестационарным
\choice если оператор координаты коммутирует с оператором Гамильтона
\end{choices}

\question Потенциальная энергия частицы не зависит от времени. Частица находится в состоянии с волновой функцией $\Psi (x,t)$. В каких из нижеперечисленных случаев интеграл $\int {{\Psi ^*}(x,t)\hat H\Psi (x,t)dx} $ ($\hat H$- оператор Гамильтона) не зависит от времени?
\begin{choices}
\choice этот интеграл всегда зависит от времени
\choice если состояние является нестационарным
\choice если потенциальная энергия - четная функция координаты в любой момент времени
\choice этот интеграл не зависит от времени, если не зависит от времени потенциальная энергия
\end{choices}

\question Частица движется в потенциале $U(x)$, который является четной функцией координаты. Волновая функция частицы в начальный момент времени $\Psi (x,t = 0)$ является нечетной функцией координат. Волновая функция частицы в последующем будет
\begin{choices}
\choice нечетной функцией           
\choice четной функцией 
\choice обладать неопределенной четностью    
\choice мало информации для ответа
\end{choices}

\question Частица движется в потенциале $U(x)$, который является нечетной функцией координа-ты. Волновая функция частицы в начальный момент времени $\Psi (x,t = 0)$ является четной функцией координат. Волновая функция частицы в последующем будет
\begin{choices}
\choice нечетной функцией           
\choice четной функцией
\choice обладать неопределенной четностью    
\choice мало информации для ответа
\end{choices}

\question Частица движется в потенциале $U(x)$, который не зависит от времени и является чет-ной функцией координаты. Средняя четность начального состояния частицы равна $\overline {P(t = 0)}  = 3/4$. Как будет изменяться средняя четность состояния частицы в последующие моменты времени?
\begin{choices}
\choice будет расти           
\choice будет убывать   
\choice будет осциллировать         
\choice не будет меняться
\end{choices}

\question Гамильтониан частицы не зависит от времени. Частица находится в таком состоянии, в котором ее энергия может принимать два значения - ${E_1}$ и ${E_2}$. Для оператора неко-торой физической величина $A$ выполнены условия $\int {f_1^*(x)\hat A{f_1}(x)dx}  = \int {f_2^*(x)\hat A{f_2}(x)dx = 0} $, $\int {f_1^*(x)\hat A{f_2}(x)dx}  \ne 0$, где ${f_1}(x)$ и ${f_2}(x)$ - собственные функции гамильтониана, отвечающие собственным значениям ${E_1}$ и ${E_2}$. Как зависит от времени среднее значение величины $A$ в рассматривае-мом состоянии?
\begin{choices}
\choice не зависит            
\choice как $\cos \left( {({E_1} - {E_2})t/\hbar } \right)$
\choice как $\arccos \left( {({E_1} - {E_2})t/\hbar } \right)$    
\choice как ${\rm{tg}}\left( {({E_1} - {E_2})t/\hbar } \right)$
\end{choices}

\question Гамильтониан частицы не зависит от времени. Частица находится в таком состоянии, в котором ее энергия может принимать два значения - ${E_1}$ и ${E_2}$. Для оператора неко-торой физической величина $A$ выполнено условие $\int {f_1^*(x)\hat A{f_2}(x)dx}  = 0$, где ${f_1}(x)$ и ${f_2}(x)$ - собственные функции гамильтониана, отвечающие собственным зна-чениям ${E_1}$ и ${E_2}$. Как зависит от времени среднее значение величины $A$ в рас-сматриваемом состоянии?
\begin{choices}
\choice не зависит            
\choice как $\cos \left( {({E_1} - {E_2})t/\hbar } \right)$
\choice как $\arccos \left( {({E_1} - {E_2})t/\hbar } \right)$    
\choice как ${\rm{tg}}\left( {({E_1} - {E_2})t/\hbar } \right)$
\end{choices}

\question Гамильтониан частицы зависит от времени. Волновая функция частицы в начальный момент времени нормирована на 2. Что можно сказать о нормировочном интеграле в после-дующие моменты времени?
\begin{choices}
\choice в любой момент времени волновая функция нормирована на 2 независимо от гамильтониана
\choice в любой момент, кроме начального, волновая функция нормирована на единицу
\choice нормировочный интеграл с течением времени возрастает
\choice нормировочный интеграл с течением времени убывает
\end{choices}

\question Закон сохранения вероятности есть следствие того, что
\begin{choices}
\choice волновая функция не зависит от времени
\choice оператор координаты не зависит от времени
\choice нормировка волновой функции не зависит от времени
\choice оператор Гамильтона не зависит от времени
\end{choices}

\question Закон сохранения вероятности утверждает, что 
\begin{choices}
\choice волновая функция не зависит от времени
\choice увеличение вероятности обнаружить частицу в одной области пространства сопровождается уменьшением вероятности обнаружить ее в другом
\choice оператор вероятности коммутирует с оператором Гамильтона
\choice вероятность обнаружить частицу в разных точках пространства не зависит от времени
\end{choices}

\question Какая формула есть математическое выражение закона сохранения вероятности?
\begin{choices}
\choice $i\hbar \frac{{\partial {\kern 1pt} \Psi }}{{\partial {\kern 1pt} t}} = \left( { - \frac{{{\hbar ^2}}}{{2m}}\Delta  + U(\vec r,t){\kern 1pt} } \right){\kern 1pt} \Psi $       
\choice $\frac{\partial }{{\partial {\kern 1pt} t}}{\left| {\Psi (\vec r,t)} \right|^2} + div{\kern 1pt} \vec J(\vec r,t) = 0$
\choice $i\hbar \frac{{\partial {\kern 1pt} \Psi }}{{\partial {\kern 1pt} t}} =  - \frac{{{\hbar ^2}}}{{2m}}\Delta {\kern 1pt} \Psi $              
\choice $\hat H{\kern 1pt} \Psi  = \,E{\kern 1pt} \Psi $ 
\end{choices}

\question Состояние частицы описывается волновой функцией $\Psi (\vec r,t)$. Какой формулой определяется вектор плотности потока вероятности $\vec J(\vec r,t)$ (с точностью до множите-ля)?
\begin{choices}
\choice ${\kern 1pt} \vec J(\vec r,t) \sim \Psi (\vec r,t)\nabla {\Psi ^*}(\vec r,t) - {\Psi ^*}(\vec r,t)\nabla \Psi (\vec r,t)$
\choice ${\kern 1pt} \vec J(\vec r,t) \sim \Psi (\vec r,t)\Delta {\Psi ^*}(\vec r,t) - {\Psi ^*}(\vec r,t)\Delta \Psi (\vec r,t)$
\choice ${\kern 1pt} \vec J(\vec r,t) \sim \Delta \Psi (\vec r,t)\nabla {\Psi ^*}(\vec r,t) - \Delta {\Psi ^*}(\vec r,t)\nabla \Psi (\vec r,t)$
\choice ${\kern 1pt} \vec J(\vec r,t) \sim \Psi (\vec r,t){\Psi ^*}(\vec r,t) - {\Psi ^*}(\vec r,t)\Psi (\vec r,t)$
\end{choices}

\question Состояние частицы описывается волновой функцией $\Psi (x,t)$, для которой выполне-ны условия: ${\left| {\Psi (x,t)} \right|^2}$ не зависит от времени при любом $x$, плотность потока $J(x,t)$ тождественно не равна нулю. Какие утверждения относительно $\Psi (x,t)$ справедливы?
\begin{choices}
\choice такого состояния не существует          
\choice функция $\Psi (x,t)$ действительна
\choice интеграл $\int {{{\left| {\Psi (x)} \right|}^2}dx} $ расходится       
\choice интеграл $\int {{{\left| {\Psi (x)} \right|}^2}dx} $ сходится
\end{choices}
\end{questions}



\section{ ГЛАВА 2. ОДНОМЕРНОЕ ДВИЖЕНИЕ }



\subsection{ Общие свойства одномерного движения }
\begin{questions}

\question Частица движется в некотором потенциале $U(x)$. Какое из перечисленных уравнений являет-ся стационарным уравнением Шредингера для этой частицы?
\begin{choices}
\choice $i\hbar \frac{{\partial \Psi (x,t)}}{{\partial t}} = \left( { - \frac{{{\hbar ^2}}}{{2m}}\frac{{{d^2}}}{{d{x^2}}} + U(x)} \right)\Psi (x,t)$       
\choice $\left( { - \frac{{{\hbar ^2}}}{{2m}}\frac{{{d^2}}}{{d{x^2}}} + U(x)} \right){f_n}(x) = {E_n}{f_n}(x)$
\choice $\frac{{\partial |\Psi (x,t){|^2}}}{{\partial t}} + div\vec J(x,t) = 0$          
\choice $ - i\hbar \frac{d}{{dx}}{f_p}(x) = p{f_p}(x)$
\end{choices}

\question Какие состояния называются связанными?
\begin{choices}
\choice состояния, волновые функции которых не затухают при $x \to  \pm \infty $ 
\choice  собственные состояния оператора импульса
\choice состояния, волновые функции которых затухают при $x \to  \pm \infty $
\choice собственные состояния оператора координаты
\end{choices}

\question Собственные состояния гамильтониана, отвечающие непрерывному спектру описывают
\begin{choices}
\choice финитное движение частицы с определенной энергией в некотором потенциале
\choice инфинитное движение частицы с определенной энергией
\choice состояния частицы с определенной координатой
\choice все перечисленное неверно 
\end{choices}

\question Что такое уровни энергии? 
\begin{choices}
\choice Собственные состояния оператора координаты
\choice Собственные состояния оператора импульса
\choice Собственные состояния гамильтониана, отвечающие непрерывному спектру
\choice Собственные состояния гамильтониана, отвечающие дискретному спектру
\end{choices}

\question Какое состояние называется основным?
\begin{choices}
\choice собственное состояние гамильтониана, отвечающее непрерывному спектру, и имеющее минимальную энергию   
\choice собственное состояние гамильтониана, отвечающее дискретному спектру, и имеющее минималь-ную энергию
\choice собственное состояние гамильтониана, отвечающее непрерывному спектру, и имеющее макси-мальную энергию
\choice собственное состояние гамильтониана, отвечающее дискретному спектру, и имеющее максималь-ную энергию
\end{choices}

\question Какое состояние называется первым возбужденным?
\begin{choices}
\choice собственное состояние гамильтониана, отвечающее дискретному спектру, и имеющее минималь-ную энергию
\choice собственное состояние гамильтониана, отвечающее дискретному спектру и имеющее вторую по счету (в порядке возрастания) энергию 
\choice собственное состояние гамильтониана, отвечающее дискретному спектру и имеющее третью по счету (в порядке возрастания) энергию 
\choice собственное состояние гамильтониана, отвечающее непрерывному спектру и имеющее вторую по счету (в порядке возрастания) энергию
\end{choices}

\question Какова максимальная кратность вырождения собственных состояний гамильтониана в одно-мерной задаче?
\begin{choices}
\choice 2     
\choice 3     в. 4     г. 5
\end{choices}

\question Уровням энергии в одномерной задаче отвечают
\begin{choices}
\choice двукратно вырожденные стационарные состояния дискретного спектра
\choice невырожденные стационарные состояния непрерывного спектра
\choice двукратно вырожденные стационарные состояния непрерывного спектра
\choice невырожденные стационарные состояния дискретного спектра 
\end{choices}

\question Потенциальная энергия стремится к $ + \infty $ при $x \to  \pm \infty $ («потенциальная яма», см. рисунок). Все собственные состояния гамильтониана в этом случае 
\begin{choices}
\choice не вырождены
\choice двукратно вырождены 
\choice часть уровней не вырождена, часть двукратно вырождена 
\choice это зависит от конкретного вида потенциала
\end{choices}

\question Частица движется в некотором потенциале $U(x)$. Известно, что $U(x) \to  + \infty $ при $x \to  \pm \infty $ (см. рисунок к предыдущей задаче). Существуют ли среди стационарных состояний час-тицы состояния, относящиеся к непрерывному спектру?
\begin{choices}
\choice да
\choice нет
\choice в некоторых случаях да, в некоторых случаях нет
\choice это зависит от конкретного вида потенциала
\end{choices}

\question Дан график зависимости потенциальной энергии $U(x)$ от координаты $x$ (см. рисунок). Указать области, в которых могут существовать стационарные состояния дискретного спектра
\begin{choices}
\choice $E < {U_0}$
\choice ${U_0} < E < {U_1}$
\choice ${a_1} < x < {a_2}$
\choice $x < {a_1}$  и $x > {a_2}$
\end{choices}

\question Дан график зависимости потенциальной энергии $U(x)$ от координаты $x$ (см. рисунок к задаче 2.1.11). При каких энергиях могут существовать стационарные состояния дискретного спек-тра?
\begin{choices}
\choice $E < {U_0}$
\choice ${U_0} < E < {U_1}$
\choice ${U_0} < E < {U_1}$
\choice ${U_0} < E < {U_1}$
\end{choices}

\question Дан график зависимости потенциальной энергии от координаты $x$ (см. рисунок к задаче 2.1.11). При каких энергиях заведомо не существует стационарных состояний?
\begin{choices}
\choice $E < {U_0}$ 
\choice ${U_0} < E < {U_1}$  
\choice $E > {U_2}$ 
\choice ${U_1} < E < {U_2}$ 
\end{choices}

\question Дан график зависимости потенциальной энергии $U(x)$ от координаты $x$ (см. рисунок к задаче 2.1.11). Указать области, в которых могут существовать невырожденные стационарные со-стояния непрерывного спектра
\begin{choices}
\choice $E > {U_1}$          
\choice ${U_0} < E < {U_1}$
\choice ${a_1} < x < {a_2}$           
\choice $x < {a_1}$ и $x > {a_2}$
\end{choices}

\question Дан график зависимости потенциальной энергии $U(x)$ от координаты $x$ (см. рисунок к задаче 2.1.11). При каких энергиях существуют невырожденные стационарные состояния непрерыв-ного спектра?
\begin{choices}
\choice при $E > {U_2}$         
\choice при ${U_1} < E < {U_2}$
\choice при ${U_0} < E < {U_1}$       
\choice при $E < {U_0}$
\end{choices}

\question Дан график зависимости потенциальной энергии $U(x)$ от координаты $x$ (см. рисунок к задаче 2.1.11). Указать области, в которых могут существовать вырожденные стационарные состоя-ния непрерывного спектра
\begin{choices}
\choice $E > {U_1}$          
\choice ${U_0} < E < {U_1}$
\choice ${a_1} < x < {a_2}$           
\choice $x < {a_1}$ и $x > {a_2}$
\end{choices}

\question Дан график зависимости потенциальной энергии $U(x)$ от координаты $x$ (см. рисунок к задаче 2.1.11). При каких энергиях существуют вырожденные стационарные состояния непрерывного спектра?
\begin{choices}
\choice при $E > {U_2}$         
\choice при ${U_1} < E < {U_2}$
\choice при ${U_0} < E < {U_1}$       
\choice при $E < {U_0}$
\end{choices}

\question Потенциальная энергия обращается в нуль при $x \to  \pm \infty $ (см. рисунок). Какова кратность вырождения собственных значений гамильтониана, относящихся к непрерывному спектру?
\begin{choices}
\choice не вырождены
\choice двукратно вырождены
\choice часть собственных значений не вырождена, часть двукратно вырождена
\choice зависит от конкретного вида потенциала
\end{choices}

\question Дан график зависимости потенциальной энергии $U(x)$ от координаты $x$ (см. рисунок). При каких энергиях существуют стационарные состояния дискретного спектра?
\begin{choices}
\choice при $E > 0$       
\choice при $E < 0$
\choice при $E < {U_0}$         
\choice состояний дискретного спектра в таком потенциале нет (покороче ось)
\end{choices}

\question Частица движется в некотором потенциале $U(x)$. Известно, что $U(x) \to  + \infty $ при $x \to  + \infty $, и $U(x) \to  - \infty $ при $x \to  - \infty $ (см. рисунок). Существуют ли среди стационарных состояний частицы состояния, относящиеся к дискретному спектру?
\begin{choices}
\choice да                
\choice нет
\choice это зависит от потенциала     
\choice бессмысленный вопрос
\end{choices}

\question Частица движется в некотором потенциале $U(x)$. Известно, что $U(x) \to  + \infty $ при $x \to  + \infty $, и $U(x) \to  - \infty $ при $x \to  - \infty $ (см. рисунок к задаче 2.1.21). Существуют ли среди стационарных состояний частицы двукратно вырожденные состояния?
\begin{choices}
\choice да                
\choice нет
\choice это зависит от потенциала     
\choice бессмысленный вопрос
\end{choices}

\question Частица движется в некотором потенциале $U(x)$. Известно, что $U(x) \to  + \infty $ при $x \to  + \infty $, и $U(x) \to  - \infty $ при $x \to  - \infty $ (см. рисунок к задаче 2.1.21). Все собственные со-стояния гамильтониана в этом случае являются 
\begin{choices}
\choice невырожденными состояниями непрерывного спектра       
\choice вырожденными состояниями непрерывного спектра
\choice невырожденными состояниями дискретного спектра        
\choice вырожденными состояниями дискретного спектра
\end{choices}

\question Частица движется в потенциале $U(x)$, график которого представлен на рисунке. Какой формулой описывается асимптотика собственной функции оператора Гамильтона при энергии $E$ (показана на рисунке) при $x \to  - \infty $?
(и таких сделать 8 задач во всех областях энергии и на плюс-минус бесконечности)
\begin{choices}
\choice $\exp ( - kx)$, $k = \sqrt {2mE/{\hbar ^2}} $      
\choice $\exp (kx)$, $k = \sqrt {2m({U_2} - E)/{\hbar ^2}} $
\choice $\exp (kx)$, $k = \sqrt {2mE/{\hbar ^2}} $      
\choice $\exp (kx)$, $k = \sqrt {2m(E - {U_2})/{\hbar ^2}} $
\end{choices}

\question Частица движется в потенциале $U(x)$, график которого представлен на рисунке. Какой формулой описывается асимптотика собственной функции оператора Гамильтона при энергии $E$ (показана на рисунке) при $x \to \infty $?
\begin{choices}
\choice $\exp ( - kx)$, $k = \sqrt {2m(E - {U_1})/{\hbar ^2}} $     
\choice $\exp (ikx)$, $k = \sqrt {2mE/{\hbar ^2}} $ 
\choice линейная комбинация $\exp (ikx)$ и $\exp ( - ikx)$, где $k = \sqrt {2m(E - {U_1})/{\hbar ^2}} $
\choice линейная комбинация $\exp (kx)$ и $\exp ( - kx)$, где $k = \sqrt {2m(E - {U_1})/{\hbar ^2}} $
\end{choices}

\question Частица движется в потенциале $U(x)$, график которого представлен на рисунке. Пусть неко-торая энергия $E$ (показана на рисунке) является собственным значением гамильтониана. Какими формулами описываются асимптотики соответствующей собственной функции при $x \to  - \infty $ и $x \to \infty $?
\begin{choices}
\choice $\exp ({k_2}x)$, ${k_2} = \sqrt {2m({U_2} - E)/{\hbar ^2}} $, и $\exp ( - {k_1}x)$, ${k_1} = \sqrt {2m({U_1} - E)/{\hbar ^2}} $
\choice $\exp (i{k_2}x)$, ${k_2} = \sqrt {2m({U_2} - E)/{\hbar ^2}} $, и $\exp (i{k_1}x)$, ${k_1} = \sqrt {2m({U_1} - E)/{\hbar ^2}} $ 
\choice $\exp ( - {k_2}x)$, ${k_2} = \sqrt {2m({U_2} - E)/{\hbar ^2}} $, и $\exp ({k_1}x)$, ${k_1} = \sqrt {2m({U_1} - E)/{\hbar ^2}} $
\choice другими
2.1.26., 2.1.27., 2.1.28., 2.1.29., 2.1.30.
\end{choices}

\question Частица движется в потенциале $U(x)$, который стремится к некоторым постоянным при $x \to  \pm \infty $ (см. рисунок к задаче 2.1.11). Как ведут себя волновые функции двукратно вырож-денных стационарных состояний при $x \to  \pm \infty $?
\begin{choices}
\choice растут         
\choice затухают       
\choice осциллируют
\choice на одной бесконечности затухают, на другой осциллируют
\end{choices}

\question Частица движется в потенциале $U(x)$, который стремится к некоторым постоянным при $x \to  \pm \infty $ (см. рисунок к задаче 2.1.11). Как ведут себя волновые функции невырожденных со-стояний непрерывного спектра при $x \to  \pm \infty $?
\begin{choices}
\choice растут         
\choice затухают       
\choice осциллируют
\choice на одной бесконечности затухают, на другой осциллируют
\end{choices}

\question Частица движется в потенциале $U(x)$, который стремится к некоторым постоянным при $x \to  \pm \infty $ (см. рисунок к задаче 2.1.31). Как ведут себя волновые функции состояний дискретно-го спектра при $x \to  \pm \infty $?
\begin{choices}
\choice растут         
\choice затухают       
\choice осциллируют
\choice на одной бесконечности затухают, на другой осциллируют
\end{choices}

\question Осцилляционная теорема утверждает, что
\begin{choices}
\choice решения стационарного уравнения Шредингера, отвечающие дискретному спектру, осциллируют
\choice решения стационарного уравнения Шредингера, отвечающие непрерывному спектру, осциллиру-ют
\choice число нулей (узлов) $n$-го по счету в порядке возрастания энергии решения стационарного урав-нения Шредингера, отвечающего дискретному спектру, равно $n$
\choice число нулей (узлов) $n$-го по счету в порядке возрастания энергии решения стационарного урав-нения Шредингера, отвечающего дискретному спектру, равно $n$
\end{choices}

\question Частица движется в некотором потенциале $U(x)$, который стремится к нулю при $x \to  \pm \infty $. Волновая функция третьего возбужденного состояния дискретного спектра (четвертого по счету состояния в порядке возрастания энергии) имеет
\begin{choices}
\choice один узел         
\choice два узла    
\choice три узла       
\choice четыре узла
\end{choices}

\question На рисунке сплошной и пунктирной линией показаны графики двух собственных функций одномерного оператора Гамильтона. Какая из этих функция отвечает большему собственному значе-нию?
\begin{choices}
\choice «сплошная»
\choice «пунктирная»
\choice эти функции отвечают вырожденным по энергии состояниям
\choice информации для ответа недостаточно
\end{choices}

\question Собственная функция одномерного оператора Гамильтона имеет вид (см. рисунок). Что можно сказать о соответствующем собственном значении?
\begin{choices}
\choice относится к дискретному спектру
\choice относится к непрерывному спектру
\choice двукратно вырождено
\choice информации для ответа недостаточно
\end{choices}

\question Собственная функция одномерного оператора Гамильтона имеет вид (см. рисунок). Какому собственному состоянию отвечает эта функция?
\begin{choices}
\choice второму по счету в порядке возрастания энергии состоянию дискретного спектра  
\choice третьему по счету в порядке возрастания энергии состоянию дискретного спектра
\choice четвертому по счету в порядке возрастания энергии состоянию дискретного спектра
\choice пятому по счету в порядке возрастания энергии состоянию дискретного спектра
\end{choices}

\question Собственная функция одномерного оператора Гамильтона затухает при $x \to  - \infty $ и осциллирует при $x \to \infty $ (см. рисунок). Какое утверждение относительно свойств этой функции справедливо?
\begin{choices}
\choice эта функция отвечает дискретному спектру
\choice эта функция отвечает невырожденному состоянию непрерывного спектра
\choice эта функция отвечает двукратно вырожденному состоянию непрерывного спектра
\choice все перечисленное – неверно
\end{choices}

\question Существует ли у волновых функций стационарных состояний нули второго порядка (т.е. точ-ки, в которых и сама волновая функция, и ее первая производная обращаются в нуль)?
\begin{choices}
\choice да    
\choice нет      
\choice это зависит от потенциала     
\choice бессмысленный вопос
\end{choices}

\question Потенциальная энергия частицы $U(x)$ – четная функция координаты. Что можно сказать о коммутаторе операторов Гамильтона и четности $\left[ {\hat H,\hat P} \right]$ для такой частицы?
\begin{choices}
\choice равен нулю
\choice не равен нулю
\choice это зависит от конкретного вида потенциала
\choice четность потенциала и коммутатор операторов $\hat H$ и $\hat P$ никак не связаны друг с другом
\end{choices}

\question Частица движется в потенциале $U(x)$. Коммутатор операторов Гамильтона и четности $\hat P$ для такой частицы равен
\begin{choices}
\choice нулю     
\choice $U( - x)$      
\choice $\left( {U(x) - U( - x)} \right)\hat P$      
\choice $\left( {U(x) - U( - x)} \right)$
\end{choices}

\question Потенциальная энергия частицы – четная функция координаты. Волновая функция третьего возбужденного стационарного состояния дискретного спектра (четвертого по счету состояния в по-рядке возрастания энергии) является
\begin{choices}
\choice четной                  
\choice нечетной
\choice обладает неопределенной четностью      
\choice четность зависит от конкретного вида потенциала
\end{choices}

\question Потенциальная энергия частицы $U(x) = \alpha {x^4}$, где $\alpha  > 0$ - некоторое число. Волновая функция четвертого возбужденного состояния дискретного спектра (пятого по счету со-стояния в порядке возрастания энергии)
\begin{choices}
\choice четная      
\choice нечетная    
\choice неопределенной четности 
\choice это зависит от $\alpha $
\end{choices}

\question Потенциальная энергия частицы $U(x)$ – четная функция координаты. Что можно сказать о волновых функциях стационарных состояний дискретного спектра?
\begin{choices}
\choice все четные  
\choice все нечетные
\choice не обладают определенной четностью
\choice четность чередуется (четная-нечетная-четная и т.д.) с увеличением энергии состояния
\end{choices}

\question Потенциальная энергия частицы $U(x)$ – нечетная функция координаты. Что можно сказать о волновых функциях стационарных состояний дискретного спектра?
\begin{choices}
\choice все четные  
\choice все нечетные
\choice не обладают определенной четностью
\choice четность чередуется (четная-нечетная-четная и т.д.) с увеличением энергии состояния
\end{choices}

\question Потенциальная энергия частицы $U(x)$ – четная функция координаты. Что можно сказать о волновых функциях стационарных состояний непрерывного спектра?
\begin{choices}
\choice все четные  
\choice все нечетные
\choice можно выбрать так, чтобы одна из двух функций, отвечающих каждому собственному значению, была четной, вторая - нечетной
\choice четность чередуется (четная-нечетная-четная и т.д.) с увеличением энергии состояния
\end{choices}

\question Потенциальная энергия частицы $U(x)$ – нечетная функция координаты. Что можно сказать о волновых функциях стационарных состояний непрерывного спектра?
\begin{choices}
\choice все четные  
\choice все нечетные
\choice можно выбрать так, чтобы одна из двух функций, отвечающих каждому собственному значению, была четной, вторая - нечетной
\choice обладают неопределенной четностью
\end{choices}

\question Потенциальная энергия частицы $U(x)$ – четная функция координаты. Какова кратность вы-рождения стационарных состояний непрерывного спектра?
\begin{choices}
\choice только 1
\choice только 2
\choice либо 2, либо 1
\choice только 3
\end{choices}

\question Что можно сказать об интеграле …, где … линейно независимые функции стационарных со-стояний непрерывного спектра
\begin{choices}
\choice всегда равен нулю
\choice никогда не равен нулю
\choice это зависит от потенциала
\choice это зависит от выбора функций
\end{choices}

\question Что можно сказать об интеграле …, где … линейно независимые функции стационарных со-стояний дискретного спектра … и непрерывного спектра …
\begin{choices}
\choice всегда равен нулю
\choice никогда не равен нулю
\choice это зависит от потенциала
\choice это зависит от выбора функций
\end{choices}

\question Что можно сказать об интеграле …, где … линейно независимые функции стационарных со-стояний дискретного спектра
\begin{choices}
\choice всегда равен нулю
\choice никогда не равен нулю
\choice это зависит от потенциала
\choice это зависит от выбора функций
\end{choices}

\question Потенциальная энергия частицы показана на рисунке. Существуют ли связанные состояния частицы в таком потенциале?
\begin{choices}
\choice да             
\choice нет
\choice зависит от потенциала      
\choice зависит от массы частицы
\end{choices}

\question Потенциальная энергия показана на рисунке. Существуют ли связанные состояния частицы в таком потенциале?
\begin{choices}
\choice да             
\choice нет
\choice зависит от потенциала      
\choice зависит от массы частицы
\end{choices}

\question Потенциальная энергия показана на рисунке. Существуют ли связанные состояния частицы в таком потенциале?
\begin{choices}
\choice да             
\choice нет
\choice зависит от потенциала      
\choice зависит от массы частицы
\end{choices}

\question Частица движется в потенциале $U(x)$, который стремится к нулю при $x \to  \pm \infty $ и имеет вид, показанный на рисунке. При каких значениях энергии существуют двукратно вырожден-ные состояния дискретного спектра?
\begin{choices}
\choice при $E > {U_1}$         
\choice при $E > {U_2}$         
\choice при $E > {U_3}$   
\choice двукратно вырожденных состояний дискретного спектра в одномерной задаче не существует
\end{choices}

\question Что произойдет с энергетическими интервалами между уровнями энергии частицы в некото-рой одномерной потенциальной яме, если массу частицы (не меняя ширину ямы) увеличить?
\begin{choices}
\choice увеличится     
\choice уменьшится
\choice не изменится      
\choice мало информации для ответа
\end{choices}

\question Что произойдет с энергетическими интервалами между уровнями энергии частицы в некото-рой одномерной потенциальной яме, если ширину ямы (не меняя массу частицы) увеличить?
\begin{choices}
\choice увеличится     
\choice уменьшится
\choice не изменится      
\choice мало информации для ответа
\end{choices}

\question Потенциальная энергия частицы является четной функцией координаты. Частица находится в связанном нестационарном состоянии. При измерении энергии частицы могут быть обнаружены два значения. Будет ли волновая функция частицы обладать определенной четностью?
\begin{choices}
\choice да
\choice нет
\choice может быть выбрана так, чтобы обладала
\choice мало информации для ответа
\end{choices}

\question Потенциальная энергия частицы является четной функцией координаты. Частица находится в связанном нестационарном состоянии. При измерении энергии частицы могут быть обнаружены три значения. Будет ли волновая функция частицы обладать определенной четностью?
\begin{choices}
\choice да
\choice нет
\choice может быть выбрана так, чтобы обладала
\choice мало информации для ответа
\end{choices}

\question Потенциальная энергия частицы является четной функцией координаты. Частица находится в связанном состоянии, волновая функция которого является четной. Будет ли это состояние стацио-нарным? 
\begin{choices}
\choice да
\choice нет
\choice может быть выбрана так, чтобы обладала
\choice мало информации для ответа
\end{choices}

\question Потенциальная энергия частицы является нечетной функцией координаты. Частица находится состоянии, средняя четность которого в некоторый момент времени равна -1. Будет ли это состояние стационарным?
\begin{choices}
\choice да
\choice нет
\choice это несвязанные вещи
\choice такого состояния быть не может
\end{choices}

\end{questions}




\subsection{ Бесконечно глубокая прямоугольная потенциальная яма }
Если это не оговорено в условии, в этом параграфе предполагается, что бесконечно глубокой прямоугольной потенциальной яме отвечает следующая потенциальная энергия
$U(x) = \left\{ {\begin{array}{*{20}{c}}
{\infty ,\quad x < 0,\;x > a;}\\
{0,\quad \quad 0 < x < a}
\end{array}} \right.$.
где $a$ - ширина ямы. Если в условии задачи сказано, что яма расположена между точками $x =  - a/2$ и $x = a/2$, подразумевается, что ей отвечает следующая потенциальная энергия
$U(x) = \left\{ {\begin{array}{*{20}{c}}
{\infty ,\quad x <  - a/2,\;x > a/2;}\\
{0,\quad \quad  - a/2 < x < a/2}
\end{array}} \right.$.

\begin{questions}

\question Частица находится в бесконечно глубокой прямоугольной потенциальной яме. Что можно сказать о собственных значениях оператора Гамильтона?
\begin{choices}
\choice все относятся к непрерывному спектру
\choice все относятся к дискретному спектру
\choice часть относится к непрерывному спектру, часть – к дискретному
\choice это зависит от ширины ямы
\end{choices}

\question Какой формулой определяются энергии стационарных состояний частицы массой $m$ в бесконечно глубокой прямоугольной потенциальной яме шириной $a$?
\begin{choices}
\choice $\frac{{{\pi ^2}{\hbar ^2}n}}{{2m{a^2}}}$, $n = 1,2,3, \ldots $      
\choice $\frac{{{\pi ^2}{\hbar ^2}{n^2}}}{{2m{a^2}}}$, $n = 1,2,3, \ldots $
\choice $\frac{{{\pi ^2}{\hbar ^2}{n^2}}}{{2m{a^2}}}$, $n = 0,1,2, \ldots $     
\choice $\frac{{{\pi ^2}{\hbar ^2}n}}{{2m{a^2}}}$, $n = 0,1,2, \ldots $
\end{choices}

\question Все уровни энергии частицы в одномерной бесконечно глубокой прямоугольной потенциальной яме
\begin{choices}
\choice не вырождены         
\choice двукратно вырождены
\choice трехкратно вырождены    
\choice часть уровней не вырождена, часть двукратно вырождена
\end{choices}

\question Какова кратность вырождения 100-го уровня энергии частицы в одномерной бесконечно глубокой яме?
\begin{choices}
\choice 1        
\choice 2        
\choice 3        
\choice 100
\end{choices}

\question Как энергетические интервалы между энергиями соседних стационарных состояний час-тицы в бесконечно глубокой яме зависят от номера состояния $n$ для больших $n$? 
\begin{choices}
\choice $\Delta {E_n} - \Delta {E_{n - 1}} \sim {n^2}$        
\choice $\Delta {E_n} - \Delta {E_{n - 1}} \sim 1/n$
\choice $\Delta {E_n} - \Delta {E_{n - 1}} \sim n$         
\choice по-другому
\end{choices}

\question Даны волновые функции нескольких нестационарных состояний частицы в бесконечно глубокой прямоугольной потенциальной яме. В определенный момент времени. В каком из них средняя энергия частицы меньше нуля?
\begin{choices}
\choice $Ax(x - a)$           
\choice $A{x^2}{(x - a)^2}$   
\choice $A{x^3}{(x - a)^3}$         
\choice такого состояния не существует
\end{choices}

\question Волновая функция частицы массой $m$ в бесконечно глубокой прямоугольной потенци-альной яме в некоторый момент времени имеет вид $Ax(x - a)$. В каком из нижеперечислен-ных интервалов лежит средняя энергия частицы в этом состоянии?
\begin{choices}
\choice $\overline E  < 0$    
\choice $0 < \overline E  < \frac{{{\pi ^2}{\hbar ^2}}}{{4m{a^2}}}$     
\choice $\frac{{{\pi ^2}{\hbar ^2}}}{{4m{a^2}}} < \overline E  < \frac{{{\pi ^2}{\hbar ^2}}}{{2m{a^2}}}$    
\choice $\frac{{{\pi ^2}{\hbar ^2}}}{{2m{a^2}}} < \overline E $
\end{choices}

\question Волновая функция частицы массой $m$ в бесконечно глубокой прямоугольной потенци-альной яме в некоторый момент времени имеет вид $Ax(x - a)(x - a/2)$. В каком из нижепере-численных интервалов лежит средняя энергия частицы в этом состоянии?
\begin{choices}
\choice $0 < \overline E  < \frac{{{\pi ^2}{\hbar ^2}}}{{4m{a^2}}}$  
\choice $\frac{{{\pi ^2}{\hbar ^2}}}{{4m{a^2}}} < \overline E  < \frac{{{\pi ^2}{\hbar ^2}}}{{2m{a^2}}}$    
\choice $\frac{{{\pi ^2}{\hbar ^2}}}{{2m{a^2}}} < \overline E  < \frac{{2{\pi ^2}{\hbar ^2}}}{{m{a^2}}}$ 
\choice $\frac{{2{\pi ^2}{\hbar ^2}}}{{m{a^2}}} < \overline E $
\end{choices}

\question Волновая функция частицы массой $m$ в бесконечно глубокой прямоугольной потенци-альной яме в некоторый момент времени имеет вид $A{x^2}(x - a)$. Измеряют энергию части-цы. Можно ли обнаружить при этом значение $\frac{{7{\pi ^2}{\hbar ^2}}}{{2m{a^2}}}$?
\begin{choices}
\choice да    
\choice нет      
\choice зависит от способа измерения     
\choice зависит от ширины ямы
\end{choices}

\question Волновая функция частицы массой $m$ в бесконечно глубокой прямоугольной потенци-альной яме в некоторый момент времени имеет вид $Ax(x - a)$ где $A$ - постоянная. Измеряют энергию частицы. Можно ли обнаружить при этом значение $\frac{{{\pi ^2}{\hbar ^2}}}{{2m{a^2}}}$?
\begin{choices}
\choice да                
\choice нет
\choice это зависит от способа измерения 
\choice это зависит от волновой функции
\end{choices}

\question Волновая функция частицы массой $m$ в бесконечно глубокой прямоугольной потенци-альной яме в некоторый момент времени имеет вид $Ax(x - a)$ где $A$ - постоянная. Измеряют энергию частицы. Можно ли обнаружить при этом значение $\frac{{2{\pi ^2}{\hbar ^2}}}{{m{a^2}}}$?
\begin{choices}
\choice да                
\choice нет
\choice это зависит от способа измерения 
\choice это зависит от волновой функции
\end{choices}

\question Волновая функция частицы массой $m$ в бесконечно глубокой прямоугольной потенци-альной яме в некоторый момент времени имеет вид $Ax(x - a)$ где $A$ - постоянная. Измеряют энергию частицы. Можно ли обнаружить при этом значение $\frac{{9{\pi ^2}{\hbar ^2}}}{{2m{a^2}}}$?
\begin{choices}
\choice да                
\choice нет
\choice это зависит от способа измерения 
\choice зависит от волновой функции
\end{choices}

\question Волновая функция частицы массой $m$ в бесконечно глубокой прямоугольной потенци-альной яме в некоторый момент времени имеет вид $Ax(x - a)(x - a/2)$ где $A$ - постоянная. Измеряют энергию частицы. Можно ли обнаружить при этом значение $\frac{{{\pi ^2}{\hbar ^2}}}{{2m{a^2}}}$?
\begin{choices}
\choice да                
\choice нет
\choice это зависит от способа измерения 
\choice это зависит от волновой функции
\end{choices}

\question Волновая функция частицы массой $m$ в бесконечно глубокой прямоугольной потенци-альной яме в некоторый момент времени имеет вид $Ax(x - a)(x - a/2)$ где $A$ - постоянная. Измеряют энергию частицы. Можно ли обнаружить при этом значение $\frac{{2{\pi ^2}{\hbar ^2}}}{{m{a^2}}}$?
\begin{choices}
\choice да                
\choice нет
\choice это зависит от способа измерения 
\choice это зависит от волновой функции
\end{choices}

\question Волновая функция частицы массой $m$ в бесконечно глубокой прямоугольной потенци-альной яме в некоторый момент времени имеет вид $Ax(x - a)(x - a/2)$ где $A$ - постоянная. Измеряют энергию частицы. Можно ли обнаружить при этом значение $\frac{{9{\pi ^2}{\hbar ^2}}}{{2m{a^2}}}$?
\begin{choices}
\choice да                
\choice нет
\choice это зависит от способа измерения 
\choice зависит от волновой функции
\end{choices}

\question Волновая функция частицы массой $m$ в бесконечно глубокой прямоугольной потенци-альной яме в некоторый момент времени имеет вид $Ax(x - a)(x - a/2)$ где $A$ - постоянная. Измеряют энергию частицы. Можно ли обнаружить при этом значение $\frac{{8{\pi ^2}{\hbar ^2}}}{{m{a^2}}}$?
\begin{choices}
\choice да                
\choice нет
\choice это зависит от способа измерения 
\choice это зависит от волновой функции
2.2.17 Какой формулой определяются собственные функции гамильтониана частицы в беско-нечно глубокой прямоугольной потенциальной яме (здесь $n = 1,2,3, \ldots $, $A$ - постоян-ная)?
\choice $A\sin \frac{{\pi nx}}{a}$    
\choice $A\cos \frac{{\pi nx}}{a}$    
\choice $A\exp \left( {i\frac{{\pi nx}}{a}} \right)$    
\choice $A\exp \left( { - i\frac{{\pi nx}}{a}} \right)$ 
2.2.18 Какой формулой определяются собственные функции гамильтониана частицы в беско-нечно глубокой прямоугольной потенциальной яме, расположенной между точками $x =  - a/2$ и $x = a/2$ (здесь $n = 1,2,3, \ldots $, $A$ - постоянная).
\choice $A\sin \left( {\frac{{\pi nx}}{a} + \pi n} \right)$         
\choice $A\sin \left( {\frac{{\pi nx}}{a} + \frac{{\pi n}}{2}} \right)$
\choice $A\cos \left( {\frac{{\pi nx}}{a} + \pi n} \right)$         
\choice $A\cos \left( {\frac{{\pi nx}}{a} + \frac{{\pi n}}{2}} \right)$
\end{choices}

\question Частица находится в бесконечно глубокой прямоугольной потенциальной яме. Каким нужно выбрать множитель перед волновой функцией стационарного состояния частицы $\psi (x) \sim \sin \left( {\pi nx/a} \right)$, чтобы эта функция была нормирована на единицу?
\begin{choices}
\choice $\sqrt {\frac{a}{2}} $        
\choice $\sqrt {\frac{{na}}{2}} $     
\choice $\sqrt {\frac{2}{a}} $        
\choice $\sqrt {\frac{2}{{na}}} $
\end{choices}

\question Волновые функции стационарных состояний частицы, находящейся в бесконечно глубо-кой прямоугольной потенциальной яме
\begin{choices}
\choice непрерывны вместе со всеми своими производными
\choice непрерывны, но имеют разрывную в некоторых точках первую производную
\choice непрерывны но имеют разрывную в некоторых точках вторую производную
\choice непрерывны, но имеют разрывную в некоторых точках третью производную
\end{choices}

\question Как ведут себя волновые функции стационарных состояний частицы, находящейся в бесконечно глубокой потенциальной яме, при $x \to  \pm \infty $?
\begin{choices}
\choice экспоненциально затухают
\choice осциллируют
\choice равны нулю
\choice бессмысленный вопрос (волновые функции частицы при $x \to  \pm \infty $ не определены)
\end{choices}

\question Частица находится в 2012 стационарном состоянии в бесконечно глубокой потенциаль-ной яме (основное состояние – первое). Сколько нулей во внутренней области ямы (исключая нули на границах) имеет волновая функция частицы?
\begin{choices}
\choice 2010     
\choice 2011     
\choice 2012     
\choice 2013
\end{choices}

\question На рисунке приведен график волновой функции некоторого стационарного состояния частицы в бесконечно глубокой прямоугольной потенциальной яме. Какова энергия этого со-стояния?
\begin{choices}
\choice $\frac{{{\pi ^2}{\hbar ^2}}}{{2m{a^2}}}$  
\choice $\frac{{2{\pi ^2}{\hbar ^2}}}{{m{a^2}}}$  
\choice $\frac{{9{\pi ^2}{\hbar ^2}}}{{2m{a^2}}}$ 
\choice $\frac{{8{\pi ^2}{\hbar ^2}}}{{m{a^2}}}$
\end{choices}

\question Будут ли волновые функции стационарных состояний частицы в бесконечно глубокой прямоугольной потенциальной яме обладать определенной четностью по отношению к центру ямы?
\begin{choices}
\choice да    
\choice нет         
\choice зависит от состояния    
\choice бессмысленный вопрос
\end{choices}

\question Частица находится в 2012 стационарном состоянии в бесконечно глубокой прямоуголь-ной потенциальной яме (основное состояние – первое). Какова четность волновой функция час-тицы относительно центра ямы?
\begin{choices}
\choice четная            
\choice нечетная
\choice неопределенной четности 
\choice зависит от расположения ямы
\end{choices}

\question Частица находится в 2012 стационарном состоянии в бесконечно глубокой прямоуголь-ной потенциальной яме (основное состояние – первое). Чему равно значение волновой функция частицы в центре ямы (с точностью до фазового множителя)?
\begin{choices}
\choice $\psi (x = a/2) = \sqrt {2/a} $        
\choice $\psi (x = a/2) = \sqrt {a/2} $
\choice $\psi (x = a/2) = 0$          
\choice $\psi (x = a/2) = 2/a$ 
\end{choices}

\question Частица находится в 2013 стационарном состоянии в бесконечно глубокой прямоуголь-ной потенциальной яме (основное состояние – первое). Чему равно значение волновой функция частицы в центре ямы (с точностью до фазового множителя)?
\begin{choices}
\choice $\psi (x = a/2) = \sqrt {2/a} $        
\choice $\psi (x = a/2) = \sqrt {a/2} $
\choice $\psi (x = a/2) = 0$          
\choice $\psi (x = a/2) = 2/a$ 
\end{choices}

\question Частица находится в $n$-ом стационарном состоянии в бесконечно глубокой прямо-угольной потенциальной яме. Как среднее значение координаты частицы зависит от $n$?
\begin{choices}
\choice возрастает с ростом $n$    
\choice убывает с ростом $n$
\choice не зависит от $n$       
\choice при малых $n$ возрастает, при больших убывает с ростом $n$
\end{choices}

\question Частица находится в $n$-ом стационарном состоянии в бесконечно глубокой прямо-угольной потенциальной яме, расположенной между точками $x =  - a/2$ и $x = a/2$. Как сред-ний квадрат координаты зависит от квантового числа $n$ этого состояния?
\begin{choices}
\choice растет с ростом $n$        
\choice убывает с ростом $n$
\choice не зависит от $n$       
\choice это зависит от ширины ямы
\end{choices}

\question Волновая функция частицы в бесконечно глубокой прямоугольной потенциальной яме в некоторый момент времени имеет вид $A\left( {\sin \frac{{\pi x}}{a} + \sin \frac{{2\pi x}}{a}} \right)$, где $A$ - постоянная. Что можно сказать о средней координате частицы в этом состоя-нии в этот момент времени?
\begin{choices}
\choice $ > a/2$          
\choice $ < a/2$
\choice $ = a/2$          
\choice среднюю координату определить нельзя
\end{choices}

\question Частица находится в $n$-ом стационарном состоянии в бесконечно глубокой прямо-угольной потенциальной яме. Как средний импульс частицы в этом состоянии зависит от $n$?
\begin{choices}
\choice возрастает с ростом $n$    
\choice убывает с ростом $n$
\choice не зависит от $n$       
\choice при малых $n$ возрастает, при больших убывает с ростом $n$
\end{choices}

\question Частица находится в $n$-ом стационарном состоянии в бесконечно глубокой прямо-угольной потенциальной яме. Как средний квадрат импульса частицы в этом состоянии зависит от $n$?
\begin{choices}
\choice возрастает с ростом $n$    
\choice убывает с ростом $n$
\choice не зависит от $n$       
\choice при малых $n$ возрастает, при больших убывает с ростом $n$
\end{choices}

\question Волновая функция частицы в бесконечно глубокой прямоугольной потенциальной яме в некоторый момент времени имеет вид $A\left( {\sin \frac{{\pi x}}{a} + \sin \frac{{2\pi x}}{a}} \right)$, где $A$ - постоянная. Чему равен средний импульс в этом состоянии в этот момент времени?
\begin{choices}
\choice $\overline p  = 0$   
\choice $\overline p  = \hbar /a$  
\choice $\overline p  =  - \hbar /a$  
\choice средний импульс посчитать нельзя
\end{choices}

\question В каком из нижеперечисленных состояний частицы в бесконечно глубокой потенциаль-ной ямы энергия частицы имеет определенное значение?
\begin{choices}
\choice $A\cos \frac{{\pi nx}}{a}$    
\choice $A\sin \frac{{\pi nx}}{a}$    
\choice $A{\rm{tg}}\frac{{\pi nx}}{a}$      
\choice $A{\rm{ctg}}\frac{{\pi nx}}{a}$
\end{choices}

\question Частица находится в некотором стационарном состоянии в бесконечно глубокой потен-циальной яме. Будет ли энергия частицы иметь определенное значение?
\begin{choices}
\choice да             
\choice нет
\choice зависит от состояния    
\choice мало информации для ответа
\end{choices}

\question Частица находится в нестационарном состоянии в бесконечно глубокой яме. Будет ли энергия частицы иметь определенное значение?
\begin{choices}
\choice да             
\choice нет 
\choice зависит от состояния    
\choice мало информации для ответа 
\end{choices}

\question Волновая функция частицы в бесконечно глубокой прямоугольной потенциальной яме в некоторый момент времени имеет вид $A\left( {\sin \frac{{\pi x}}{a} + \sin \frac{{2\pi x}}{a}} \right)$, где $a$ - ширина ямы, $A$ - постоянная (яма расположена между точками $x = 0$ и $x = a$). Будет ли это состояние стационарным?
\begin{choices}
\choice да    
\choice нет      
\choice зависит от $A$    
\choice мало информации для ответа
\end{choices}

\question Волновая функция частицы в бесконечно глубокой прямоугольной потенциальной яме в некоторый момент времени имеет вид $A\left( {\sin \frac{{\pi nx}}{a} + \sin \frac{{\pi kx}}{a}} \right)$, где $A$ - постоянная, $n$ и $k$ - целые числа. Какой должна быть постоянная $A$, чтобы эта функция была нормирована на единицу?
\begin{choices}
\choice $A = \sqrt {\frac{1}{a}} $    
\choice $A = \sqrt {\frac{2}{a}} $    
\choice $A = \sqrt {\frac{{kn}}{a}} $    
\choice $A = \sqrt {\frac{1}{{kna}}} $
\end{choices}

\question Волновая функция частицы в бесконечно глубокой прямоугольной потенциальной яме в некоторый момент времени имеет вид $A\left( {\sin \frac{{\pi nx}}{a} + 2\sin \frac{{\pi kx}}{a}} \right)$, где $A$ - постоянная, $n$ и $k$ - целые числа. Какой должна быть постоянная, чтобы эта функция была нормирована на единицу?
\begin{choices}
\choice $A = \sqrt {\frac{1}{{3a}}} $    
\choice $A = \sqrt {\frac{2}{{5a}}} $    
\choice $A = \sqrt {\frac{2}{{5a}}} $    
\choice $A = \sqrt {\frac{1}{{5a}}} $
\end{choices}

\question Волновая функция частицы в бесконечно глубокой прямоугольной потенциальной яме в некоторый момент времени имеет вид $A\left( {\sin \frac{{\pi x}}{a} + \frac{1}{2}\sin \frac{{2\pi x}}{a}} \right)$, где $A$ - постоянная. Чему равна средняя энергия частицы в этом состоянии?
\begin{choices}
\choice $\overline E  = \frac{{{\pi ^2}{\hbar ^2}}}{{5m{a^2}}}$     
\choice $\overline E  = \frac{{2{\pi ^2}{\hbar ^2}}}{{5m{a^2}}}$    
\choice $\overline E  = \frac{{3{\pi ^2}{\hbar ^2}}}{{5m{a^2}}}$    Г.$\overline E  = \frac{{4{\pi ^2}{\hbar ^2}}}{{5m{a^2}}}$
\end{choices}

\question Волновая функция частицы в бесконечно глубокой прямоугольной потенциальной яме в некоторый момент времени имеет вид $A\left( {\sin \frac{{\pi x}}{a} + \frac{1}{3}\sin \frac{{2\pi x}}{a}} \right)$, где $A$ - постоянная. Измеряют энергию. Какие значения можно получить и с какими вероятностями? 
\begin{choices}
\choice $w\left( {{E_1}} \right) = \frac{3}{4},\quad w\left( {{E_2}} \right) = \frac{1}{4}$    
\choice $w\left( {{E_1}} \right) = \frac{5}{6},\quad w\left( {{E_2}} \right) = \frac{1}{6}$
\choice $w\left( {{E_1}} \right) = \frac{7}{8},\quad w\left( {{E_2}} \right) = \frac{1}{8}$    
\choice $w\left( {{E_1}} \right) = \frac{9}{{10}},\quad w\left( {{E_2}} \right) = \frac{1}{{10}}$
(здесь ${E_1}$ и ${E_2}$ - энергии первого и второго стационарных состояний частицы в яме; остальные вероятности равны нулю).
\end{choices}

\question Волновая функция частицы массой $m$ в бесконечно глубокой прямоугольной потенци-альной яме в некоторый момент времени имеет вид $A\sin (\pi x/a)\cos (2\pi x/a)$, где $A$ - по-стоянная. Какие значения энергии частицы могут быть обнаружены при измерениях и с какими вероятностями?
\begin{choices}
\choice определенное значение $\frac{{4{\pi ^2}{\hbar ^2}}}{{2m{a^2}}}$         
\choice $\frac{{{\pi ^2}{\hbar ^2}}}{{2m{a^2}}}$ и $\frac{{9{\pi ^2}{\hbar ^2}}}{{2m{a^2}}}$ с вероят-ностями 1/2
\choice $\frac{{9{\pi ^2}{\hbar ^2}}}{{2m{a^2}}}$ и $\frac{{16{\pi ^2}{\hbar ^2}}}{{2m{a^2}}}$ с ве-роятностями 1/2      
\choice определенное значение $\frac{{9{\pi ^2}{\hbar ^2}}}{{2m{a^2}}}$
\end{choices}

\question Волновая функция частицы, находящейся в бесконечно глубокой прямоугольной потен-циальной яме шириной $a$ имеет вид $A\sin \frac{{3\pi x}}{a}\cos \frac{{7\pi x}}{a}$, где $A$ - постоянная. Средняя энергия частицы равна
\begin{choices}
\choice $\overline E  = \frac{{27{\pi ^2}{\hbar ^2}}}{{m{a^2}}}$ 
\choice $\overline E  = \frac{{28{\pi ^2}{\hbar ^2}}}{{m{a^2}}}$ 
\choice $\overline E  = \frac{{29{\pi ^2}{\hbar ^2}}}{{m{a^2}}}$ 
\choice $\overline E  = \frac{{30{\pi ^2}{\hbar ^2}}}{{m{a^2}}}$
\end{choices}

\question Волновая функция частицы массой $m$ в бесконечно глубокой прямоугольной потенци-альной яме в некоторый момент времени имеет вид $A\sin (\pi x/a)\cos (2\pi x/a)$, где $A$ - по-стоянная. Средний квадрат энергии в этом состоянии равен
\begin{choices}
\choice $\overline {{E^2}}  = \frac{{41{\pi ^4}{\hbar ^4}}}{{4{m^2}{a^4}}}$  
\choice $\overline {{E^2}}  = \frac{{43{\pi ^4}{\hbar ^4}}}{{4{m^2}{a^4}}}$  
\choice $\overline {{E^2}}  = \frac{{45{\pi ^4}{\hbar ^4}}}{{4{m^2}{a^4}}}$  
\choice $\overline {{E^2}}  = \frac{{47{\pi ^4}{\hbar ^4}}}{{4{m^2}{a^4}}}$
\end{choices}

\question Частица массой $m$ находится в бесконечно глубокой прямоугольной потенциальной яме в таком состоянии, в котором ее энергия имеет определенное значение $\frac{{4{\pi ^2}{\hbar ^2}}}{{2m{a^2}}}$. Какой из нижеприведенных формул определяется волновая функ-ция частицы (с точностью до множителя)?
\begin{choices}
\choice $\sin \left( {\frac{{2\pi x}}{a}} \right)\exp \left( {i\frac{{2{\pi ^2}\hbar t}}{{m{a^2}}}} \right)$           
\choice $\exp \left( {i\frac{{2\pi x}}{a}} \right)\sin \left( {\frac{{2{\pi ^2}\hbar t}}{{m{a^2}}}} \right)$
\choice $\sin \left( {\frac{{2\pi x}}{a}} \right)\exp \left( { - i\frac{{2{\pi ^2}\hbar t}}{{m{a^2}}}} \right)$           
\choice $\exp \left( { - i\frac{{2\pi x}}{a}} \right)\sin \left( {\frac{{2{\pi ^2}\hbar t}}{{m{a^2}}}} \right)$
\end{choices}

\question Какой из нижеприведенных формул может описываться волновая функция частицы в бесконечно глубокой прямоугольной потенциальной яме ($A$ - постоянная, яма расположена между точками $x = 0$ и $x = a$)?
\begin{choices}
\choice $\Psi (x,t) = A\sin \left( {\frac{{2\pi x}}{a}} \right)\exp \left( { - i\frac{{{\pi ^2}\hbar t}}{{2m{a^2}}}} \right)$      
\choice $\Psi (x,t) = A\sin \left( {\frac{{\pi x}}{a}} \right)\exp \left( {i\frac{{{\pi ^2}\hbar t}}{{2m{a^2}}}} \right)$ 
\choice $\Psi (x,t) = A\sin \left( {\frac{{\pi x}}{a}} \right)\exp \left( { - i\frac{{4{\pi ^2}\hbar t}}{{2m{a^2}}}} \right)$      
\choice $\Psi (x,t) = A\sin \left( {\frac{{\pi x}}{a}} \right)\exp \left( { - i\frac{{{\pi ^2}\hbar t}}{{2m{a^2}}}} \right)$
\end{choices}

\question Какой из нижеприведенных формул не может описываться волновая функция частицы в бесконечно глубокой прямоугольной потенциальной яме ($A$ и $B$- постоянные)? 
\begin{choices}
\choice $A\sin \left( {\frac{{2\pi x}}{a}} \right){e^{ - i\frac{{4{\pi ^2}\hbar t}}{{2m{a^2}}}}} + B\sin \left( {\frac{{3\pi x}}{a}} \right){e^{ - i\frac{{9{\pi ^2}\hbar t}}{{2m{a^2}}}}}$    
\choice $A\sin \left( {\frac{{2\pi x}}{a}} \right){e^{ - i\frac{{4{\pi ^2}\hbar t}}{{2m{a^2}}}}} - iB\sin \left( {\frac{{5\pi x}}{a}} \right){e^{ - i\frac{{25{\pi ^2}\hbar t}}{{2m{a^2}}}}}$
\choice $A\sin \left( {\frac{{2\pi x}}{a}} \right){e^{ - i\frac{{4{\pi ^2}\hbar t}}{{2m{a^2}}}}} - {e^{2i}}B\sin \left( {\frac{{3\pi x}}{a}} \right){e^{ - i\frac{{{\pi ^2}\hbar t}}{{2m{a^2}}}}}$   
\choice $A\sin \left( {\frac{{4\pi x}}{a}} \right){e^{ - i\frac{{16{\pi ^2}\hbar t}}{{2m{a^2}}}}} - {e^{3i}}B\sin \left( {\frac{{\pi x}}{a}} \right){e^{ - i\frac{{{\pi ^2}\hbar t}}{{2m{a^2}}}}}$
\end{choices}

\question Частица находится в бесконечно глубокой потенциальной яме в таком состоянии, в котором ее энергия имеет определенное значение. Как средняя координата частицы зависит от времени в этом состоянии?
\begin{choices}
\choice убывает     
\choice растет      
\choice не меняется    
\choice осциллирует
\end{choices}

\question Частица находится в некотором состоянии с определенной энергией в бесконечно глубо-кой потенциальной яме. Средние значения каких физических величин не зависят от времени в этом состоянии?
\begin{choices}
\choice только координаты
\choice только координаты и энергии
\choice только импульса, координаты и энергии
\choice любых
\end{choices}

\question Частица находится в бесконечно глубокой потенциальной яме в таком состоянии, в котором ее энергия может принимать два значения. Как средняя координата частицы зависит от времени в этом состоянии?
\begin{choices}
\choice убывает           
\choice осциллирует
\choice не меняется       
\choice это зависит от состояния
\end{choices}

\question Волновая функция частицы в бесконечно глубокой прямоугольной потенциальной яме в некоторый момент времени имеет вид $A\left( {\sin \frac{{\pi x}}{a} + \sin \frac{{2\pi x}}{a}} \right)$, где $A$ - постоянная. С какой частотой осциллирует средняя координата частицы?
\begin{choices}
\choice $\frac{{3{\pi ^2}\hbar }}{{2m{a^2}}}$  
\choice $\frac{{{\pi ^2}\hbar }}{{2m{a^2}}}$   
\choice $\frac{{2{\pi ^2}\hbar }}{{m{a^2}}}$   
\choice средняя координата не зависит от времени
\end{choices}

\question Волновая функция частицы в бесконечно глубокой прямоугольной потенциальной яме в некоторый момент времени имеет вид $A\left( {\sin \frac{{\pi x}}{a} + \sin \frac{{2\pi x}}{a}} \right)$, где $A$ - постоянная. Что можно сказать о потоке вероятности при $x = a/2$ как функ-ции времени?
\begin{choices}
\choice положителен в любой момент времени     
\choice отрицателен в любой момент времени
\choice равен нулю в любой момент времени      
\choice осциллирует как функция времени
\end{choices}

\question Волновая функция частицы в бесконечно глубокой прямоугольной потенциальной яме в некоторый момент времени имеет вид $A\left( {\sin \frac{{\pi x}}{a} + \sin \frac{{3\pi x}}{a}} \right)$, где $A$ - постоянная. Как средняя координата частицы в этом состоянии зависит от времени?
\begin{choices}
\choice всегда растет        
\choice всегда убывает
\choice осциллирует       
\choice не зависит от времени
\end{choices}

\question Волновая функция частицы в бесконечно глубокой прямоугольной потенциальной яме в некоторый момент времени имеет вид $A\left( {\sin \frac{{\pi x}}{a} + \sin \frac{{3\pi x}}{a}} \right)$, где $A$ - постоянная. Что можно сказать о средней координате частицы в этом состоя-нии?
\begin{choices}
\choice $ > a/2$ 
\choice $ < a/2$ 
\choice $ = a/2$ 
\choice среднюю координату посчитать нельзя
\end{choices}

\question Волновая функция частицы в бесконечно глубокой прямоугольной потенциальной яме в некоторый момент времени имеет вид $A\left( {\sin \frac{{\pi x}}{a} + \sin \frac{{3\pi x}}{a}} \right)$, где $A$ - постоянная. Что можно сказать о потоке вероятности при $x = a/2$ как функ-ции времени?
\begin{choices}
\choice положителен в любой момент времени     
\choice отрицателен в любой момент времени
\choice равен нулю в любой момент времени      
\choice осциллирует как функция времени
\end{choices}

\question Волновая функция частицы в бесконечно глубокой прямоугольной потенциальной яме в некоторый момент времени имеет вид $A\left( {\sin \frac{{\pi x}}{a} + \sin \frac{{2\pi x}}{a}} \right)$, где $A$ - постоянная (яма расположена между точками $x = 0$ и $x = a$). Как средний импульс в этом состоянии зависит от времени?
\begin{choices}
\choice всегда растет  
\choice всегда убывает 
\choice не зависит от времени   
\choice осциллирует
\end{choices}

\question Волновая функция частицы в бесконечно глубокой прямоугольной потенциальной яме в некоторый момент времени имеет вид $A\sin \frac{{\pi x}}{a} + B\sin \frac{{3\pi x}}{a}$, где $A$ и $B$- постоянные (яма расположена между точками $x = 0$ и $x = a$). Как средний импульс частицы в этом состоянии зависит от времени?
\begin{choices}
\choice всегда растет  
\choice всегда убывает 
\choice не зависит от времени   
\choice осциллирует
\end{choices}

\question Частица в бесконечно глубокой прямоугольной потенциальной яме, расположенной ме-жду точками $x =  - a/2$ и $x = a/2$, находится в таком состоянии, в котором ее энергия может принимать два значения: $n$-ое и $n + 1$-ое. Как средняя четность этого состояния зависит от времени?
\begin{choices}
\choice всегда растет  
\choice всегда убывает 
\choice осциллирует 
\choice не зависит от времени
\end{choices}

\question Частица в бесконечно глубокой прямоугольной потенциальной яме, расположенной ме-жду точками $x =  - a/2$ и $x = a/2$, находится в таком состоянии, в котором ее энергия может принимать два значения: $n$-ое и $n + 2$-ое. Как средняя четность этого состояния зависит от времени?
\begin{choices}
\choice всегда растет  
\choice всегда убывает 
\choice осциллирует 
\choice не зависит от времени
\end{choices}

\question Частица в бесконечно глубокой прямоугольной потенциальной яме, расположенной ме-жду точками $x =  - a/2$ и $x = a/2$, находится в таком состоянии, в котором ее энергия может принимать три значения: $10$-ое с вероятностью 1/5 и 13-ое с вероятностью 2/5 и 16-ое с веро-ятностью 2/5 (основное состояние – первое). Найти среднюю четность этого состояния.
\begin{choices}
\choice $\overline P  =  - 1/5$     
\choice $\overline P  = 1/5$     
\choice $\overline P  =  - 2/5$     
\choice $\overline P  = 2/5$
\end{choices}

\question Частица находится в бесконечно глубокой потенциальной яме. Какие из перечисленных величин являются интегралами движения?
\begin{choices}
\choice координата     
\choice импульс     
\choice энергия     
\choice потенциальная энергия
\end{choices}

\question Частица находится в бесконечно глубокой прямоугольной потенциальной яме, располо-женной между точками $x =  - a/2$ и $x = a/2$. Какие из нижеперечисленных величин являются интегралами движения?
\begin{choices}
\choice четность    
\choice импульс     
\choice координата     
\choice потенциальная энергия
\end{choices}

\question Частица в бесконечно глубокой прямоугольной потенциальной яме находится в таком состоянии, в котором ее энергия может принимать два значения: первое и второе с вероятностя-ми 3/4 и 1/4-ое соответственно. С какой частотой осциллирует средняя энергия частицы в этом состоянии?
\begin{choices}
\choice $\frac{{3{\pi ^2}\hbar }}{{2m{a^2}}}$  
\choice $\frac{{5{\pi ^2}\hbar }}{{2m{a^2}}}$  
\choice $\frac{{7{\pi ^2}\hbar }}{{2m{a^2}}}$  
\choice средняя энергия не зависит от времени
\end{choices}

\question Частица в бесконечно глубокой прямоугольной потенциальной яме находится в таком состоянии, в котором ее энергия может принимать два значения: первое и третье с вероятностя-ми 3/4 и 1/4-ое соответственно. С какой частотой осциллирует средняя энергия частицы в этом состоянии?
\begin{choices}
\choice $\frac{{8{\pi ^2}\hbar }}{{2m{a^2}}}$  
\choice $\frac{{10{\pi ^2}\hbar }}{{2m{a^2}}}$ 
\choice $\frac{{12{\pi ^2}\hbar }}{{2m{a^2}}}$ 
\choice средняя энергия не зависит от времени
\end{choices}
\end{questions}




\subsection{ Одномерный гармонический осциллятор }
В этом параграфе, под $x$ везде подразумевается безразмерная координата осцилля-тора $x=x/a$, где $a=\sqrt{\hbar /m\omega }$ - параметр длины для осциллятора.

\begin{questions}

\question Какой формулой определяются энергии собственных состояний одномерного гармонического осциллятора с частотой $\omega $?
\begin{choices}
\choice $\hbar \omega ({{n}^{2}}+1/2)$, $n=0,1,2,3,\ldots $      
\choice $\hbar \omega ({{n}^{2}}+1/2)$, $n=1,2,3,\ldots $
\choice $\overline{P}=3/4$, $n=0,1,2,3,\ldots $      
\choice $\overline{P}=3/4$, $n=1,2,3,\ldots $ 
\end{choices}

\question Какой формулой определяются собственные функции гамильтониана гармонического осциллятора (здесь $n=0,1,2,3,\ldots $)?
\begin{choices}
\choice ${{P}_{n}}(x)\exp \left( -{{x}^{2}}/2 \right)$ (${{P}_{n}}$ - полиномы Лежандра)
\choice ${{L}_{n}}(x)\exp \left( -{{x}^{2}}/2 \right)$ (${{L}_{n}}$ - полиномы Лагерра)
\choice ${{P}_{n}}^{\left| m \right|}(x)\exp \left( -{{x}^{2}}/2 \right)$ (${{P}_{n}}^{\left| m \right|}$ - присоединенные полиномы Лежандра)
\choice ${{H}_{n}}(x)\exp \left( -{{x}^{2}}/2 \right)$ (${{H}_{n}}$ - полиномы Эрмита).
\end{choices}

\question Какой формулой определяются собственные функции гамильтониана гармонического осциллятора (здесь ${{H}_{n}}$ - полиномы Эрмита, $n=0,1,2,3,\ldots $)?
\begin{choices}
\choice ${{H}_{n}}(x)\exp \left( -{{x}^{2}}/2 \right)$  
\choice ${{H}_{n}}(x)\exp \left( -{{x}^{4}}/2 \right)$
\choice ${{H}_{n}}(x)\exp \left( -{{x}^{2}} \right)$    
\choice ${{H}_{n}}(x)\exp \left( -{{x}^{4}}/2 \right)$ 
\end{choices}

\question Чему равен коэффициент перед ${{x}^{99}}$ в сотом полиноме Эрмита ${{H}_{100}}(x)$?
\begin{choices}
\choice 1     
\choice 1/2     
\choice 0     
\choice –1
\end{choices}

\question Чему равен коэффициент перед ${{x}^{2}}$ в восемьдесят седьмом полиноме Эрмита ${{H}_{87}}(x)$?
\begin{choices}
\choice 1     
\choice 1/2     
\choice 0     
\choice –1
\end{choices}

\question Какой формулой определяется условие ортогональности полиномов Эрмита?
\begin{choices}
\choice $\int\limits_{-\infty }^{+\infty }{{{H}_{n}}(x){{H}_{m}}(x)dx\sim {{\delta }_{nm}}}$      
\choice $\int\limits_{-\infty }^{+\infty }{{{H}_{n}}(x){{H}_{m}}(x){{e}^{-{{x}^{2}}}}dx\sim {{\delta }_{nm}}}$
\choice $\int\limits_{-\infty }^{+\infty }{{{H}_{n}}(x){{H}_{m}}(x){{x}^{2}}dx\sim {{\delta }_{nm}}}$      
\choice $\int\limits_{-\infty }^{+\infty }{{{H}_{n}}(x){{H}_{m}}(x)\sin x\,dx\sim {{\delta }_{nm}}}$
\end{choices}

\question Все уровни энергии одномерного гармонического осциллятора 
\begin{choices}
\choice не вырождены
\choice двукратно вырождены
\choice часть уровней не вырождена, часть двукратно вырождена
\choice кратность вырождения $n$-го уровня энергии равна $n$
\end{choices}

\question Как действует оператор четности на 77-ой полином Эрмита $\hat{P}{{H}_{77}}(x)$?
\begin{choices}
\choice $\hat{P}{{H}_{77}}(x)=-{{H}_{77}}(x)$     
\choice $\hat{P}{{H}_{77}}(x)={{H}_{77}}(x)$
\choice $\hat{P}{{H}_{77}}(x)=-{{H}_{76}}(x)$     
\choice $\hat{P}{{H}_{77}}(x)={{H}_{76}}(x)$
\end{choices}

\question Как действует оператор четности на 78-ой полином Эрмита $\hat{P}{{H}_{77}}(x)$?
\begin{choices}
\choice $\hat{P}{{H}_{78}}(x)=-{{H}_{78}}(x)$     
\choice $\hat{P}{{H}_{78}}(x)={{H}_{78}}(x)$
\choice $\hat{P}{{H}_{78}}(x)=-{{H}_{77}}(x)$     
\choice $\hat{P}{{H}_{78}}(x)={{H}_{77}}(x)$
\end{choices}

\question Какая величина, составленная из параметров гармонического осциллятора с массой $m$ и частотой $\omega $, имеет размерность длины (то есть является параметром длины для осциллятора)?
\begin{choices}
\choice $\sqrt{\,\frac{m}{\hbar \omega }}$     
\choice $\sqrt{\,\frac{\hbar }{m\omega }}$     
\choice $\sqrt{\,\frac{m\hbar }{\omega }}$     
\choice $\sqrt{\,\frac{m\omega }{\hbar }}$
\end{choices}

\question Какая величина, составленная из параметров гармонического осциллятора с массой $m$ и частотой $\omega $, имеет размерность энергии (то есть является параметром энергии для осциллятора)?
\begin{choices}
\choice $\hbar \omega $         
\choice $\hbar /\omega $     
\choice $m\hbar \omega $     
\choice $m\hbar /\omega $
\end{choices}

\question Какая величина, составленная из параметров гармонического осциллятора с массой $m$ и частотой $\omega $, имеет размерность импульса (то есть является параметром им-пульса для осциллятора)?
\begin{choices}
\choice $\frac{1}{\sqrt{\hbar m\omega }}$      
\choice $\sqrt{\hbar m\omega }$       
\choice $m\hbar \omega $     
\choice $\frac{1}{\hbar m\omega }$
\end{choices}

\question На рисунке приведена волновая функция некоторого собственного состояния гамильтониана гармонического осциллятора. Какому собственному значению она отвечает?
\begin{choices}
\choice $(11/2)\hbar \omega $         
\choice $(13/2)\hbar \omega $
\choice $(15/2)\hbar \omega $         
\choice Такой собственная функция быть не может
\end{choices}

\question На рисунке приведена волновая функция некоторого собственного состояния гамильтониана гармонического осциллятора. Какому собственному значению она отвечает?
\begin{choices}
\choice $(11/2)\hbar \omega $         
\choice $(13/2)\hbar \omega $
\choice $(15/2)\hbar \omega $         
\choice Такой собственная функция быть не может
\end{choices}

\question Осциллятор находится в 2013 стационарном состоянии (основному состоянию отве-чает $n=0$). Чему равна вероятность обнаружить осциллятор в бесконечно малом интервале $dx$ вблизи точки $x=0$?
\begin{choices}
\choice $dw=\frac{1}{2013}dx$      
\choice $dw=\frac{1}{{{2013}^{2}}}dx$    
\choice $2013\,dx$     
\choice 0
\end{choices}

\question Осциллятор находится в $n$-ом  стационарном состоянии (основному состоянию от-вечает $n=0$), причем число $n$ - четное. Как вероятность обнаружить осциллятор в беско-нечно малом интервале $dx$ вблизи точки $x=0$ зависит от $n$ при больших $n$?
\begin{choices}
\choice как $1/\sqrt{n}$        
\choice как $1/n$
\choice как $1/{{n}^{2}}$       
\choice эта вероятность строго рана нулю
\end{choices}

\question Рассмотрим уравнение ${{H}_{n}}(x)=0$ (${{H}_{n}}(x)$ - полином Эрмита). Как интервалы между соседними корнями этого уравнения зависят от $n$ при больших $n$?
\begin{choices}
\choice как $1/n$      
\choice как $1/\sqrt{n}$     
\choice как $\sqrt{n}$    
\choice не зависят от $n$
\end{choices}

\question Осциллятор находится в $n$-ом стационарном состоянии. Чему равна средняя коор-дината осциллятора в этом состоянии?
\begin{choices}
\choice $\sqrt{\hbar /m\omega }$      
\choice $n\sqrt{\hbar /m\omega }$     
\choice $\sqrt{n\hbar /m\omega }$     
\choice 0
\end{choices}

\question Осциллятор находится в ${{L}_{n}}(x)\exp \left( -{{x}^{2}}/2 \right)$-ом стационар-ном состоянии. Как средний импульс осциллятора зависит от $n$?
\begin{choices}
\choice растет         
\choice убывает
\choice не зависит от $n$    
\choice при малых $n$ растет, при больших – убывает.
\end{choices}

\question Осциллятор находится в ${{L}_{n}}(x)\exp \left( -{{x}^{2}}/2 \right)$-ом стационар-ном состоянии. Как средний квадрат координаты осциллятора зависит от $n$ при больших $n$?
\begin{choices}
\choice не зависит     
\choice как $\sqrt{n}$    
\choice как $n$  
\choice как ${{n}^{2}}$
\end{choices}

\question Осциллятор находится в ${{L}_{n}}(x)\exp \left( -{{x}^{2}}/2 \right)$-ом стационар-ном состоянии. Как средний квадрат импульса осциллятора зависит от $n$ при больших $n$?
\begin{choices}
\choice не зависит     
\choice как $\sqrt{n}$    
\choice как $n$  
\choice как ${{n}^{2}}$
\end{choices}

\question Волновая функция гармонического осциллятора в некоторый момент времени имеет вид $(1+2x)\exp (-{{x}^{2}}/2)$. Какие значения энергии осциллятора могут быть обнару-жены при измерениях?
\begin{choices}
\choice только $\hbar \omega /2$ и $3\hbar \omega /2$         
\choice только $3\hbar \omega /2$ и $5\hbar \omega /2$
\choice только $3\hbar \omega /2$           
\choice только $\hbar \omega /2$
\end{choices}

\question Волновая функция гармонического осциллятора в некоторый момент времени равна $A\exp (-{{x}^{4}}/2)$, где $A$ - постоянная. Можно ли при измерениях энергии осцилля-тора в этом состоянии обнаружить значение $5\hbar \omega /2$?
\begin{choices}
\choice да                   
\choice нет
\choice это зависит от способа измерения    
\choice бессмысленный вопрос
\end{choices}

\question Волновая функция осциллятора в момент времени $t=0$ имеет вид $\Psi (x,t=0)=Ax\exp (-{{x}^{4}}/2)$, где $A$ - постоянная. Какие значения энергии осциллятора можно обнаружить при измерениях (собственные значения нумеруются, начиная с $n=0$)?
\begin{choices}
\choice только $3\hbar \omega /2$           
\choice все нечетные собственные значе-ния
\choice все четные собственные значения  
\choice все собственные значения
\end{choices}

\question Волновая функция гармонического осциллятора в некоторый момент времени имеет неопределенную четность. Можно ли при измерениях энергии осциллятора в этом состоянии обнаружить значение $3\hbar \omega /4$?
\begin{choices}
\choice да    
\choice нет      
\choice это зависит от состояния      
\choice бессмысленный вопрос
\end{choices}

\question Волновая функция осциллятора имеет вид $\Psi (x)=(1/2){{f}_{0}}(x)+(\sqrt{3}/2){{f}_{1}}$, где ${{f}_{0}}(x)$ и ${{f}_{1}}(x)$ - волно-вые функции нулевого и первого стационарных состояний осциллятора. Чему равна средняя энергия осциллятора в этом состоянии?
\begin{choices}
\choice $\overline{E}=5\hbar \omega /2$  
\choice $\overline{E}=5\hbar \omega /6$     
\choice $\overline{E}=5\hbar \omega /3$  
\choice $\overline{E}=5\hbar \omega /4$
\end{choices}

\question Волновая функция осциллятора имеет вид $\Psi (x)=(1/3){{f}_{0}}(x)+(2/3){{f}_{1}}$, где ${{f}_{0}}(x)$ и ${{f}_{1}}(x)$ - волновые функции нулевого и первого стационарных состояний осциллятора. Чему равна средняя энергия осциллятора в этом состоянии?
\begin{choices}
\choice $\overline{E}=11\hbar \omega /10$   
\choice $\overline{E}=12\hbar \omega /10$   
\choice $\overline{E}=13\hbar \omega /10$   
\choice $\overline{E}=14\hbar \omega /10$
\end{choices}

\question При измерениях энергии осциллятора обнаружены нулевое и второе собственные значения с вероятностями ${{w}_{0}}=3/4$ и ${{w}_{2}}=1/4$. Чему равен средний квадрат энергии осциллятора?
\begin{choices}
\choice $\overline{{{E}^{2}}}=(7/4){{\hbar }^{2}}{{\omega }^{2}}$      
\choice $\overline{{{E}^{2}}}=(9/4){{\hbar }^{2}}{{\omega }^{2}}$
\choice $\overline{{{E}^{2}}}=(11/4){{\hbar }^{2}}{{\omega }^{2}}$     
\choice $\overline{{{E}^{2}}}=(13/4){{\hbar }^{2}}{{\omega }^{2}}$
\end{choices}

\question Волновая функция гармонического осциллятора в некоторый момент времени имеет вид $(1-10{{x}^{2}})\exp (-{{x}^{2}}/2)$. Какие значения энергии осциллятора могут быть обнаружены при измерениях?
\begin{choices}
\choice только $\hbar \omega /2$, $3\hbar \omega /2$ и $5\hbar \omega /2$       
\choice только $3\hbar \omega /2$ и $5\hbar \omega /2$
\choice только $3\hbar \omega /2$              
\choice  только $\hbar \omega /2$ и $5\hbar \omega /2$
\end{choices}

\question Волновая функция гармонического осциллятора в некоторый момент времени имеет вид $(1-10{{x}^{6}})\exp (-{{x}^{2}}/2)$. Какие значения энергии осциллятора могут быть обнаружены при измерениях?
\begin{choices}
\choice только $\hbar \omega /2$ и $13\hbar \omega /2$        
\choice только $13\hbar \omega /2$
В $\hbar \omega /2$, $5\hbar \omega /2$ и $13\hbar \omega /2$        
\choice только $\hbar \omega /2$, $5\hbar \omega /2$, $9\hbar \omega /2$ и $13\hbar \omega /2$
\end{choices}

\question Волновая функция гармонического осциллятора в некоторый момент времени имеет вид $C(-1+2{{x}^{2}})\exp (-{{x}^{2}}/2)$ ($C$ - постоянная). Какие значения энергии ос-циллятора могут быть обнаружены при измерениях?
\begin{choices}
\choice только $\hbar \omega /2$, $3\hbar \omega /2$ и $5\hbar \omega /2$       
\choice только $3\hbar \omega /2$ и $5\hbar \omega /2$
\choice только $5\hbar \omega /2$              
\choice  только $\hbar \omega /2$ и $5\hbar \omega /2$
\end{choices}

\question Осциллятор находится в таком состоянии, в котором его средняя четность равна $\overline{P}=3/4$. Какие значения энергии осциллятора можно обнаружить при измерениях в этом состоянии?
\begin{choices}
\choice $\hbar \omega /2$ и $3\hbar \omega /2$       
\choice такой средняя четность быть не может
\choice $5\hbar \omega /2$ и $7\hbar \omega /2$         
\choice мало информации, чтобы ответить
\end{choices}

\question Осциллятор находится в состоянии, в котором его энергия имеет неопределенное зна-чение. Будет ли четность осциллятора иметь определенное значение в этом состоянии?
\begin{choices}
\choice да    
\choice нет      
\choice зависит от состояния    
\choice зависит от средней энергии
\end{choices}

\question При измерении энергии осциллятора были обнаружены два значения $\hbar \omega /2$ с вероятностью 1/4 и $3\hbar \omega /2$ с вероятностью 3/4. Чему равна средняя четность состояния осциллятора?
\begin{choices}
\choice $\overline{P}=-1/2$     
\choice $\overline{P}=-3/4$     
\choice $\overline{P}=+1/2$     
\choice $\overline{P}=3/4$
\end{choices}

\question При измерении энергии осциллятора были обнаружены два значения $99\hbar \omega /2$ с вероятностью 1/4 и $135\hbar \omega /2$ с вероятностью 3/4. Чему равна средняя четность состояния осциллятора?
\begin{choices}
\choice $\overline{P}=-1/2$     
\choice $\overline{P}=-1$    
\choice $\overline{P}=+1/2$     
\choice $\overline{P}=1$
\end{choices}

\question Осциллятор находится в состоянии, в котором его энергия имеет определенное значе-ние. Будет ли координата осциллятора иметь в этом состоянии определенное значение?
\begin{choices}
\choice да    
\choice нет      
\choice зависит от состояния    
\choice зависит от энергии
\end{choices}

\question Осциллятор находится в состоянии, в котором его энергия имеет определенное значе-ние. Будет ли импульс осциллятора иметь в этом состоянии определенное значение?
\begin{choices}
\choice да    
\choice нет      
\choice зависит от состояния    
\choice зависит от энергии
\end{choices}

\question Осциллятор находится в состоянии, в котором его энергия имеет определенное значе-ние. Будет ли четность осциллятора иметь в этом состоянии определенное значение?
\begin{choices}
\choice да    
\choice нет      
\choice зависит от состояния    
\choice зависит от энергии
\end{choices}

\question Для каких значений индексов $n$ и $m$ отличен от нуля интеграл $\int\limits_{-\infty }^{+\infty }{{{f}_{n}}(x)\,\hat{x}\,{{f}_{m}}(x)dx}$, где ${{f}_{n}}(x)$ и ${{f}_{m}}(x)$ собственные функции гамильтониана осциллятора, $\hat{x}$ - оператор координаты?
\begin{choices}
\choice если индексы $n$ и $m$ совпадают
\choice если индексы $n$ и $m$ - числа одной четности
\choice если индексы $n$ и $m$ отличаются на 2
\choice если индексы $n$ и $m$ отличаются на 1
\end{choices}

\question Для каких значений индексов $n$ и $m$ отличен от нуля интеграл $\int\limits_{-\infty }^{+\infty }{{{f}_{n}}(x)\,{{{\hat{x}}}^{2}}\,{{f}_{m}}(x)dx}$, где ${{f}_{n}}(x)$ и ${{f}_{m}}(x)$ собственные функции осциллятора, $\hat{x}$ - оператор координаты?
\begin{choices}
\choice только если индексы $n$ и $m$ совпадают
\choice если индексы $n$ и $m$ - числа одной четности
\choice если индексы $n$ и $m$ совпадают или отличаются на 2
\choice только если индексы $n$ и $m$ отличаются на 2
\end{choices}

\question Какой формулой (с точностью до безразмерного множителя) определяется интеграл $\int\limits_{-\infty }^{+\infty }{{{f}_{n}}(x)\,\hat{x}\,{{f}_{n+1}}(x)dx}$, где ${{f}_{n}}(x)$ и ${{f}_{n+1}}(x)$ нормированные собственные функции осциллятора, $\hat{x}$ - оператор координаты?
\begin{choices}
\choice $\sim \sqrt{\hbar /m\omega }$    
\choice $\sim \sqrt{m\omega /\hbar }$    
\choice $\sim \sqrt{\omega /m\hbar }$    
\choice $\sim \sqrt{\hbar \omega /m}$
\end{choices}

\question Для каких значений индексов $n$ и $m$ отличен от нуля интеграл $\int\limits_{-\infty }^{+\infty }{{{f}_{n}}(x)\,{{{\hat{p}}}_{x}}\,{{f}_{m}}(x)dx}$, где ${{f}_{n}}(x)$ и ${{f}_{m}}(x)$ собственные функции осциллятора, ${{\hat{p}}_{x}}$ - оператор импульса?
\begin{choices}
\choice если индексы $n$ и ${{H}_{n}}(x)\exp \left( -{{x}^{2}}/2 \right)$ совпадают
\choice если индексы $n$ и $m$ отличаются на 1
\choice если индексы $n$ и $m$ отличаются на 2
\choice если индексы $n$ и $m$ - числа одной четности
\end{choices}

\question Для каких значений индексов $n$ и $m$ отличен от нуля интеграл $\int\limits_{-\infty }^{+\infty }{{{f}_{n}}(x)\,\hat{p}_{x}^{2}\,{{f}_{m}}(x)dx}$, где ${{f}_{n}}(x)$ и ${{f}_{m}}(x)$ собственные функции осциллятора, ${{\hat{p}}_{x}}$ - оператор импульса?
\begin{choices}
\choice только если индексы $n$ и ${{H}_{n}}(x)\exp \left( -{{x}^{2}}/2 \right)$ совпадают
\choice только если индексы $n$ и $m$ отличаются на 2
\choice если индексы $n$ и $m$ совпадают или отличаются на 2
\choice если индексы $n$ и $m$ - числа одной четности
\end{choices}

\question Какой формулой (с точностью до безразмерного множителя) определяется интеграл $\int\limits_{-\infty }^{+\infty }{{{f}_{n}}(x)\,\hat{p}\,{{f}_{n+1}}(x)dx}$, где ${{f}_{n}}(x)$ и ${{f}_{n+1}}(x)$ нормированные собственные функции осциллятора, $\hat{p}$ - оператор импульса?
\begin{choices}
\choice $\sim \sqrt{\hbar m\omega }$        
\choice $\sim \sqrt{\hbar m/\omega }$       
\choice $\sim 1/\sqrt{\hbar m\omega }$           
\choice $\sim \sqrt{\hbar \omega /m}$
\end{choices}

\question К какой функции сводится результат действия оператора координаты на полином Эр-мита $\hat{x}{{H}_{n}}$? 
\begin{choices}
\choice к ${{H}_{n+1}}$ (с точностью до множителя)
\choice к линейной комбинации ${{H}_{n+1}}$ и ${{H}_{n-1}}$
\choice к линейной комбинации всех полиномов Эрмита той же четности, что и ${{H}_{n+1}}$ до ${{H}_{n+1}}$-го включительно
\choice к линейной комбинации всех полиномов Эрмита до ${{H}_{n+1}}$-го включительно
\end{choices}

\question К какой функции сводится результат действия оператора импульса на полином Эрми-та $\hat{p}{{H}_{n}}$? 
\begin{choices}
\choice к ${{H}_{n-1}}$ (с точностью до множителя)
\choice к линейной комбинации ${{H}_{n+1}}$ и ${{H}_{n-1}}$
\choice к линейной комбинации всех полиномов Эрмита той же четности, что и ${{H}_{n-1}}$ до ${{H}_{n-1}}$-го включительно
\choice к линейной комбинации всех полиномов Эрмита до ${{H}_{n+1}}$-го включительно
\end{choices}

\question Волновая функция осциллятора в момент времени $t=0$ имеет вид $\Psi (x,t=0)=A\exp (-{{x}^{2}}/2)$, где $A$ - постоянная. Как зависит от времени средний импульс осциллятора в этом состоянии? 
\begin{choices}
\choice не меняется    
\choice растет   
\choice убывает  
\choice осциллирует
\end{choices}

\question Волновая функция осциллятора в момент времени $t=0$ имеет вид $\Psi (x,t=0)=A\exp (-{{x}^{4}}/2)$, где $A$ - постоянная. Как зависит от времени средняя координата осциллятора в этом состоянии? 
\begin{choices}
\choice растет      
\choice убывает     
\choice осциллирует 
\choice не меняется
\end{choices}

\question Волновая функция осциллятора в момент времени $t=0$ имеет вид $\Psi (x,t=0)=Ax\exp (-{{x}^{2}}/2)$, где $A$ - постоянная. Как зависит от времени средняя коор-дината осциллятора в этом состоянии? 
\begin{choices}
\choice не меняется    
\choice растет   
\choice убывает  
\choice осциллирует
\end{choices}

\question Осциллятор находится в стационарном состоянии. Какие из нижеперечисленных средних не зависят от времени в этом состоянии?
\begin{choices}
\choice только энергия             
\choice только энергия и координата
\choice только импульс и координата      
\choice любые
\end{choices}

\question Осциллятор находится в состоянии, в котором его энергия имеет определенное значе-ние. Как средняя координата осциллятора зависит от времени в этом состоянии?
\begin{choices}
\choice растет      
\choice убывает     
\choice не зависит от времени      
\choice осциллирует
\end{choices}

\question Осциллятор находится в нестационарном состоянии. Какие из нижеперечисленных средних не зависят от времени независимо от того, какое это состояние?
\begin{choices}
\choice средняя энергия      
\choice средняя координата
\choice средний импульс      
\choice любые
\end{choices}

\question Какие из нижеперечисленных величин будут интегралами движения для гармониче-ского осциллятора?
\begin{choices}
\choice координата     
\choice импульс
\choice четность       
\choice никакие из перечисленных
\end{choices}

\question Волновая функция осциллятора в момент времени $t=0$ имеет вид $\Psi (x,t=0)=A(1+{{x}^{3}})\exp (\text{-}{{x}^{2}}/2)$, где $A$ - постоянная. Как будет зависеть от времени средняя энергия осциллятора в этом состоянии?
\begin{choices}
\choice не будет меняться       
\choice будет возрастать
\choice будет убывать        
\choice Будет осциллировать
\end{choices}

\question Осциллятор находится в состоянии, в котором его энергия может принимать три зна-чения – нулевое, первое и второе с определенными вероятностями. Как средняя четность ос-циллятора зависит от времени в этом состоянии?
\begin{choices}
\choice растет         
\choice убывает
\choice осциллирует    
\choice не зависит от времени
\end{choices}

\question Осциллятор находится в нестационарном состоянии. Будет ли средняя координата осциллятора зависеть от времени?
\begin{choices}
\choice да             
\choice нет
\choice это зависит от состояния      
\choice среднюю координату вычислить нельзя
\end{choices}

\question Осциллятор в момент времени   находится в состоянии, волновая функция кото-рого имеет вид $\Psi (x,t=0)\sim {{f}_{100}}(x)+{{f}_{200}}(x)+{{f}_{300}}(x)$, где ${{f}_{i}}(x)$ - функции соответствующих стационарных состояний осциллятора. Что мож-но сказать о средней координате осциллятора в этом состоянии в момент времени $t=0$? 
\begin{choices}
\choice $\overline{x}<0$   
\choice $\overline{x}>0$   
\choice $\overline{x}=0$   
\choice мало информации для отве-та
\end{choices}

\question Даны три варианта волновой функции осциллятора в момент времени $t=0$: (1) $(1+x)\exp (-{{x}^{2}}/2)$; (2) $(1+{{x}^{2}})\exp (-{{x}^{2}}/2)$, (3) $(1+{{x}^{3}})\exp (-{{x}^{2}}/2)$. В каком из них средняя координата осциллятора зависит от времени?
\begin{choices}
\choice только (1)     
\choice только (2)     
\choice только (3)     
\choice ни в одном 
\end{choices}

\question Волновая функция в момент времени $t=0$ имеет вид: $(1+2x)\exp (-{{x}^{2}}/2)$. Как зависит от времени средняя координата осциллятора?
\begin{choices}
\choice растет      
\choice убывает     
\choice осциллирует 
\choice не зависит от времени 
\end{choices}

\question Волновая функция в момент времени $t=0$ имеет вид: $({{x}^{3}}+2{{x}^{4}})\exp (-{{x}^{2}}/2)$. С какой частотой осциллирует средняя координата осциллятора как функ-ция времени?
\begin{choices}
\choice $\omega /2$     
\choice $\omega $    
\choice $2\omega $      
\choice не зависит от времени 
\end{choices}

\question Даны три варианта волновой функции осциллятора в момент времени $t=0$: (1) $(1+x)\exp (-{{x}^{2}}/2)$; (2) $(1+{{x}^{2}})\exp (-{{x}^{2}}/2)$, (3) $(1+{{x}^{3}})\exp (-{{x}^{2}}/2)$. В каком из них средний импульс осциллятора зависит от времени?
\begin{choices}
\choice только (1)     
\choice только (2)     
\choice только (3)     
\choice ни в одном 
\end{choices}

\question Осциллятор, имеющий частоту $\omega $, в момент времени $t=0$ находится в со-стоянии с волновой функцией $(1+3x)\exp (-{{x}^{2}}/2)$. С какой частотой осциллирует средний импульс осциллятора как функция времени?
\begin{choices}
\choice $\omega /2$     
\choice $\omega $    
\choice $2\omega $      
\choice не зависит от времени 
\end{choices}

\question Энергия осциллятора может принимать нулевое и второе собственные значения. Как средний импульс осциллятора зависит от времени в рассматриваемом состоянии?
\begin{choices}
\choice растет      
\choice убывает     
\choice осциллирует    
\choice не зависит
\end{choices}

\question Осциллятор находится в таком состоянии, в котором его средняя четность равна $\overline{P}=+1/2$. Будет ли средняя координата осциллятора зависеть от времени в этом состоянии?
\begin{choices}
\choice да    
\choice нет      
\choice зависит от состояния    
\choice бессмысленный вопрос
\end{choices}

\question Волновая функция осциллятора в момент времени $t=0$ имеет вид $\Psi (x,t=0)=(1+x)\exp (-{{x}^{2}}/2)$. Как зависит от времени поток вероятности при $x=0$ в этом состоянии?
\begin{choices}
\choice растет   
\choice убывает  
\choice не меняется 
\choice осциллирует
\end{choices}

\question Волновая функция осциллятора в момент времени $t=0$ имеет вид $\Psi (x,t=0)=(1+{{x}^{2}})\exp (-{{x}^{2}}/2)$. Как зависит от времени поток вероятности при $x=0$ в этом состоянии?
\begin{choices}
\choice растет   
\choice убывает     
\choice не меняется 
\choice осциллирует
\end{choices}

\question Даны три варианта волновой функции осциллятора в момент времени $t=0$: (1) $(1+x)\exp (-{{x}^{2}}/2)$; (2) $(1+{{x}^{2}})\exp (-{{x}^{2}}/2)$, (3) $(1+{{x}^{3}})\exp (-{{x}^{2}}/2)$. В каком из них средний квадрат координаты зависит от времени?
\begin{choices}
\choice только (1)     
\choice только (2)     
\choice только (3)     
\choice ни в одном 
\end{choices}

\question Осциллятор находится в нестационарном состоянии с неопределенной четностью. Зависит ли средний квадрат импульса осциллятора от времени в этом состоянии?
\begin{choices}
\choice да    
\choice нет      
\choice зависит от состояния       
\choice бессмысленный вопрос
\end{choices}

\question Осциллятор, имеющий частоту $\omega $, в момент времени $t=0$ находится в со-стоянии с волновой функцией $\Psi (x,t=0)=(1+{{x}^{3}})\exp (-{{x}^{2}}/2)$. С какой час-тотой осциллирует средний квадрат импульса осциллятора как функция времени?
\begin{choices}
\choice $\omega /2$     
\choice $\omega $    
\choice $2\omega $      
\choice не зависит от времени 
\end{choices}

\question Осциллятор, имеющий частоту $\omega $, в момент времени $t=0$ находится в со-стоянии с волновой функцией $(1+4x)\exp (-{{x}^{2}}/2)$. С какой частотой осциллирует средняя четность осциллятора как функция времени?
\begin{choices}
\choice $\omega /2$     
\choice $\omega $    
\choice $2\omega $      
\choice не зависит от времени 
\end{choices}

\question Какой формулой может описываться волновая функция осциллятора с частотой $\omega $ ($x=x/\sqrt{\hbar /m\omega }$ - безразмерная координата осциллятора, $A$ - посто-янная)?
\begin{choices}
\choice $\Psi (x,t)=A{{e}^{-{{x}^{2}}/2}}{{e}^{-3i\omega t/2}}$         
\choice $\Psi (x,t)=A{{e}^{-{{x}^{2}}/2}}{{e}^{i\omega t/2}}$  
\choice $\Psi (x,t)=Ax{{e}^{-{{x}^{2}}/2}}{{e}^{-3i\omega t/2}}$        
\choice $\Psi (x,t)=A{{x}^{2}}{{e}^{-{{x}^{2}}/2}}{{e}^{-5i\omega t/2}}$
\end{choices}

\question Какой формулой может описываться волновая функция осциллятора с частотой $\omega $?
\begin{choices}
\choice $\Psi (x,t)=x{{e}^{-{{x}^{2}}/2}}{{e}^{-3i\omega t/2}}$         
\choice $\Psi (x,t)={{x}^{2}}{{e}^{-{{x}^{2}}/2}}{{e}^{-5i\omega t/2}}$
\choice $\Psi (x,t)={{x}^{3}}{{e}^{-{{x}^{2}}/2}}{{e}^{-7i\omega t/2}}$       
\choice $\Psi (x,t)={{x}^{4}}{{e}^{-{{x}^{2}}/2}}{{e}^{-9i\omega t/2}}$
\end{choices}

\question Какой формулой может описываться волновая функция осциллятора с частотой $\omega $?
\begin{choices}
\choice $\Psi (x,t)=\left( {{e}^{-3i\omega t/2}}+x{{e}^{-i\omega t/2}} \right){{e}^{-{{x}^{2}}/2}}$      
\choice $\Psi (x,t)=\left( {{e}^{-i\omega t/2}}+x{{e}^{-3i\omega t/2}} \right){{e}^{-{{x}^{2}}/2}}$
\choice $\Psi (x,t)=\left( {{e}^{3i\omega t/2}}+x{{e}^{i\omega t/2}} \right){{e}^{-{{x}^{2}}/2}}$        
\choice $\Psi (x,t)=\left( {{e}^{i\omega t/2}}+x{{e}^{3i\omega t/2}} \right){{e}^{-{{x}^{2}}/2}}$
\end{choices}

\question Какая из перечисленных функций является волновой функцией $\Psi (x,t)$ стационар-ного состояния осциллятора?
\begin{choices}
\choice $\Psi (x,t)=A{{e}^{-{{x}^{2}}/2}}$                 
\choice $\Psi (x,t)=Ax{{e}^{-{{x}^{2}}/2}}{{e}^{3i\omega t/2}}$
\choice $\Psi (x,t)=A\left( {{e}^{-{{x}^{2}}/2}}{{e}^{-i\omega t/2}}+x{{e}^{-{{x}^{2}}/2}}{{e}^{-3i\omega t/2}} \right)$        
\choice $\Psi (x,t)=Ax{{e}^{-{{x}^{2}}/2}}{{e}^{-3i\omega t/2}}$
\end{choices}

\end{questions}





\subsection{ Непрерывный спектр. Прохождение через барьеры }


\begin{questions}

\question Частица находится в состоянии с волновой функцией ${e^{ikx}} - {e^{ - ikx}}$, где $k$ - некоторое число. Сравнить вероятность обнаружить частицу в малом интервале координат око-ло точки с координатой $x = \pi /4k$ -$w(x = \pi /4k)$ и вероятность обнаружить частицу в том же интервале координат около точки с координатой $x = \pi /3k$ -$w(x = \pi /3k)$.
\begin{choices}
\choice $w(x = \pi /4k) > w(x = \pi /3k)$         
\choice $w(x = \pi /4k) < w(x = \pi /3k)$
\choice $w(x = \pi /4k) = w(x = \pi /3k)$         
\choice информации для ответа недостаточно
\end{choices}

\question Частица находится в состоянии с волновой функцией ${e^{ikx}}$, где $k$ - некоторое число. Сравнить вероятность обнаружить частицу в малом интервале координат около точки с координатой $x = \pi /4k$ -$w(x = \pi /4k)$ и вероятность обнаружить частицу в том же интерва-ле координат около точки с координатой $x = \pi /3k$ -$w(x = \pi /3k)$.
\begin{choices}
\choice $w(x = \pi /4k) > w(x = \pi /3k)$            
\choice $w(x = \pi /4k) < w(x = \pi /3k)$   
\choice $w(x = \pi /4k) = w(x = \pi /3k)$            
\choice информации для ответа недостаточно
\end{choices}

\question Частица находится в состоянии с волновой функцией ${e^{ikx}} + {e^{ - ikx}}$, где $k$ - некоторое число. Сравнить вероятность обнаружить частицу в малом интервале координат око-ло точки с координатой $x = \pi /6k$ -$w(x = \pi /6k)$ и вероятность обнаружить частицу во вдвое большем интервале координат около точки с координатой $x = \pi /3k$ -$w(x = \pi /3k)$
\begin{choices}
\choice $w(x = \pi /4k) > w(x = \pi /3k)$            
\choice $w(x = \pi /4k) < w(x = \pi /3k)$
\choice $w(x = \pi /4k) = w(x = \pi /3k)$            
\choice информации для ответа недостаточно
\end{choices}

\question Волновая функция ${e^{ikx}}$, где $k$ - некоторое положительное число, описывает по-ток частиц, распространяющихся
\begin{choices}
\choice в положительном направлении оси $x$
\choice в отрицательном направлении оси $x$
\choice покоящихся частиц, так как вероятность обнаружить их в разных точках пространства одина-кова
\choice информации для ответа недостаточно
\end{choices}

\question Частица движется в некотором потенциале притяжения. Какое движение описывает вол-новая функция стационарного состояния непрерывного спектра?
\begin{choices}
\choice финитное    
\choice инфинитное        
\choice периодическое     
\choice стохасти-ческое
\end{choices}

\question Потенциальная энергия частицы равна нулю при $x > a$ и бесконечности при $x < a$ (см. рисунок). Какие значения энергии являются собственными значениями гамильтониана
\begin{choices}
\choice любые          
\choice любые положительные
\choice большие, чем $a$        
\choice целые положительные
\end{choices}

\question Потенциальная энергия частицы равна нулю при $x > 0$ и бесконечности при $x < 0$ (см. рисунок). Какой из нижеследующих формул определяются собственные функции гамильтониана, отвечающие собственному значению $E$ ($A$ - постоянная, $k = \sqrt {2mE/{\hbar ^2}} $, $m$ - масса частицы):
\begin{choices}
\choice $f(x) = A\exp ( - ikx)$ 
\choice $f(x) = A\cos kx$
\choice $f(x) = A\exp (ikx)$    
\choice $f(x) = A\sin kx$
\end{choices}

\question Потенциальная энергия частицы равна нулю при $x > 0$ и бесконечности при $x < 0$ (см. рисунок к предыдущей задаче). Какова кратность вырождения собственных значений га-мильтониана частицы?
\begin{choices}
\choice 1     
\choice 2     
\choice 3     
\choice зависит от состояния
\end{choices}

\question Потенциальная энергия частицы $U(x)$ – четная функция координат и отлична от нуля в конечной области. Будут ли волновые функции стационарных состояний непрерывного спектра обладать определенной четностью?
\begin{choices}
\choice да
\choice нет
\choice вообще говоря, нет, но могут быть выбраны так, чтобы обладали
\choice зависит от потенциала
\end{choices}

\question В каком из нижеперечисленных состояний свободной частицы и энергия, и импульс час-тицы имеют определенные значения?
\begin{choices}
\choice ${e^{ikx}} + {e^{ - ikx}}$    
\choice ${e^{ikx}} - {e^{ - ikx}}$    
\choice таких состояний не су-ществует      
\choice ${e^{ - ikx}}$
\end{choices}

\question В каком из перечисленных состояний свободной частицы энергия частицы имеет определенное значение, а импульс нет?
\begin{choices}
\choice ${e^{ikx}}$       
\choice ${e^{ikx}} - {e^{ - ikx}}$    
\choice таких состояний не су-ществует      
\choice ${e^{ - ikx}}$
\end{choices}

\question В каком из перечисленных состояний свободной частицы импульс частицы имеет опре-деленное значение, а энергия нет?
\begin{choices}
\choice ${e^{ikx}}$       
\choice ${e^{ikx}} - {e^{ - ikx}}$    
\choice таких состояний не су-ществует      
\choice $\cos kx$
\end{choices}

\question Будет ли импульс свободной частицы иметь определенное значение в состояниях с опре-деленной энергией? 
\begin{choices}
\choice да    
\choice нет      
\choice зависит от состояния    
\choice зависит от энергии
\end{choices}

\question Потенциальная энергия частицы является четной функцией координаты. Частица нахо-дится в невырожденном стационарном состоянии. Будет ли это состояние иметь определенную четность?
\begin{choices}
\choice да    
\choice нет      
\choice зависит от состояния    
\choice такого состояния быть не мо-жет
\end{choices}

\question Свободная частица находится в состоянии с определенной четностью. Будет ли это со-стояние стационарным? 
\begin{choices}
\choice да    
\choice нет      
\choice зависит от состояния    
\choice зависит от энергии
\end{choices}

\question Будет ли энергия свободной частицы иметь определенное значение в состояниях с опре-деленным импульсом? 
\begin{choices}
\choice да    
\choice нет      
\choice зависит от состояния    
\choice зависит от импульса
\end{choices}

\question Потенциальная энергия частицы $U(x)$ отлична от нуля в конечной области. Частица находится в стационарном состоянии непрерывного спектра. Что можно сказать об энергии частицы?
\begin{choices}
\choice имеет определенное отрицательное значение
\choice имеет определенное положительное значение
\choice имеет неопределенное отрицательное значение
\choice имеет неопределенное положительное значение
\end{choices}

\question Потенциальная энергия частицы отлична от нуля в конечной области. Волновая функция при $x \to  + \infty $ в некоторый момент времени имеет вид ${e^{ikx}} + 2{e^{ - ikx}}$, где $k$ - некоторое число. Измеряют энергию частицы. Какие значения можно получить и с какими вероятностями ($m$ - масса частицы)?
\begin{choices}
\choice $E = {\hbar ^2}{k^2}/2m$ с единичной вероятностью
\choice $E = {\hbar ^2}{k^2}/2m$ и $E =  - {\hbar ^2}{k^2}/2m$ с одинаковыми вероятностями
\choice $E = {\hbar ^2}{k^2}/2m$  с вероятностью 1/5 и $E =  - {\hbar ^2}{k^2}/2m$ с вероятностью 4/5
\choice мало информации для ответа
\end{choices}

\question Потенциальная энергия частицы отлична от нуля в конечной области. Волновая функция при $x \to  + \infty $ в некоторый момент времени имеет вид ${e^{ikx}} + 2{e^{ - 2ikx}}$, где $k$ - некоторое число. Измеряют энергию частицы. Какие значения можно получить и с какими вероятностями ($m$ - масса частицы)?
\begin{choices}
\choice $E = {\hbar ^2}{k^2}/2m$ с единичной вероятностью
\choice $E = 4{\hbar ^2}{k^2}/2m$ и $E = {\hbar ^2}{k^2}/2m$ с одинаковыми вероятностями
\choice $E =  - {\hbar ^2}{k^2}/2m$ с вероятностью 4/5 и $E = {\hbar ^2}{k^2}/2m$ с вероятностью 1/5
\choice $E = 4{\hbar ^2}{k^2}/2m$ с вероятностью 4/5 и $E = {\hbar ^2}{k^2}/2m$ с вероятностью 1/5 
\end{choices}

\question Волновая функция частицы в некоторый момент времени имеет вид ${e^{ikx}} + 2{e^{ - ikx}}$, где $k$ - некоторое число. Измеряют импульс частицы. Какие значения можно получить и с какими вероятностями?
\begin{choices}
\choice $p = \hbar k$ с единичной вероятностью
\choice $p = \hbar k$ и $p =  - \hbar k$ с одинаковыми вероятностями
\choice $p = \hbar k$ с вероятностью 1/5 и $p =  - \hbar k$ с вероятностью 4/5
\choice мало информации для ответа
\end{choices}

\question Волновая функция частицы в некоторый момент времени имеет вид $a{e^{ibx}} + c{e^{idx}}$, где  - некоторые положительные действительные числа. Измеряют модуль импуль-са частицы. Какие значения можно получить и с какими вероятностями?
\begin{choices}
\choice $\hbar a$ и $\hbar c$ с вероятностями ${b^2}$ и ${d^2}$ соответственно
\choice $\hbar a$ и $\hbar c$ с вероятностями ${b^2}/({b^2} + {d^2})$ и ${d^2}/({b^2} + {d^2})$ со-ответственно
\choice $\hbar b$ и $\hbar d$ с вероятностями ${a^2}$ и ${c^2}$ соответственно
\choice $\hbar b$ и $\hbar d$ с вероятностями ${a^2}/({a^2} + {c^2})$ и ${c^2}/({a^2} + {c^2})$ со-ответственно
\end{choices}

\question Волновая функция свободной частицы в некоторый момент времени имеет вид $a{e^{ibx}} + c{e^{ - ibx}}$, где  - некоторые действительные числа. Измеряют энергию части-цы. Какие значения будут получены в результате измерений и с какими вероятностями ($m$ - масса частицы)?
\begin{choices}
\choice $b$ и $ - b$ с вероятностями ${a^2}/({a^2} + {c^2})$ и ${c^2}/({a^2} + {c^2})$ соответственно
\choice $b$ с единичной вероятностью
\choice ${\hbar ^2}{b^2}/2m$ и $ - {\hbar ^2}{b^2}/2m$ с вероятностями ${a^2}/({a^2} + {c^2})$ и ${c^2}/({a^2} + {c^2})$ соответственно
\choice ${\hbar ^2}{b^2}/2m$ с единичной вероятностью 
\end{choices}

\question Волновая функция свободной частицы в некоторый момент времени имеет вид $A{e^{iax}} + {B_1}{e^{ibx}} + {B_2}{e^{ - ibx}}$, где  - некоторые действительные числа, $A,\;{B_1},\;{B_2}$ - некоторые комплексные числа, сумма квадратов модулей которых равна единице. Какие значения может принимать энергия частицы в этом состоянии и с какими веро-ятностями ($m$ - масса частицы)?
\begin{choices}
\choice ${\hbar ^2}{a^2}/2m$ с вероятностью ${\left| A \right|^2}$, ${\hbar ^2}{b^2}/2m$ с вероят-ностью ${\left| {{B_1}} \right|^2}$ и $ - {\hbar ^2}{b^2}/2m$ с вероятностью ${\left| {{B_2}} \right|^2}$
\choice ${\hbar ^2}{a^2}/2m$ с вероятностью ${\left| A \right|^2}$ и ${\hbar ^2}{b^2}/2m$ с вероят-ностью ${\left| {{B_1}} \right|^2} + {\left| {{B_2}} \right|^2}$
\choice ${\hbar ^2}{a^2}/2m$ с вероятностью ${\left| A \right|^2}$ и $ - {\hbar ^2}{b^2}/2m$ с веро-ятностью ${\left| {{B_1}} \right|^2} + {\left| {{B_2}} \right|^2}$
\choice ${\hbar ^2}{a^2}/2m$ с вероятностью ${\left| A \right|^2}$ и ${\hbar ^2}{b^2}/2m$ с вероят-ностью ${\left| {{B_1} + {B_2}} \right|^2}$ 
\end{choices}

\question Волновая функция свободной частицы в некоторый момент времени имеет вид $A{e^{iax}} + {B_1}{e^{ibx}} + {B_2}{e^{ - ibx}}$, где  - некоторые действительные числа, $A,\;{B_1},\;{B_2}$ - некоторые комплексные числа, сумма квадратов модулей которых равна единице. Измеряют модуль импульса частицы. Какие значения будут получены и с какими ве-роятностями?
\begin{choices}
\choice $\hbar a$ с вероятностью ${\left| A \right|^2}$, $\hbar b$ с вероятностью ${\left| {{B_1}} \right|^2}$ и $ - \hbar b$ с вероятностью ${\left| {{B_2}} \right|^2}$
\choice $\hbar a$ с вероятностью ${\left| A \right|^2}$ и $\hbar b$ с вероятностью ${\left| {{B_1}} \right|^2} + {\left| {{B_2}} \right|^2}$
\choice $\hbar a$ с вероятностью ${\left| A \right|^2}$ и $\hbar b$ с вероятностью ${\left| {{B_1}} \right|^2} - {\left| {{B_2}} \right|^2}$
\choice $\hbar a$ с вероятностью ${\left| A \right|^2}$ и $\hbar b$ с вероятностью ${\left| {{B_1} + {B_2}} \right|^2}$ 
\end{choices}

\question Частица движется в некотором потенциале, отличном от нуля в конечной области. На асимптотике (при $x \to \infty $) волновая функция частицы в некоторый момент времени име-ет вид ${e^{ikx}} + 2{e^{ - 2ikx}}$, где $k$ - некоторое действительное число. Будет ли это состояние стационарным?
\begin{choices}
\choice да    
\choice нет      
\choice зависит о гамильтониана    
\choice бессмысленный вопрос
\end{choices}

\question Частица движется в некотором потенциале, отличном от нуля в конечной области. На асимптотике (при $x \to \infty $) волновая функция частицы в некоторый момент времени име-ет вид ${e^{ikx}} + 2{e^{ - 2ikx}}$, где $k$ - некоторое действительное число. Найти среднюю энергию частицы.
\begin{choices}
\choice $\overline E  = \frac{{13{\hbar ^2}{k^2}}}{{10m}}$ 
\choice $\overline E  = \frac{{15{\hbar ^2}{k^2}}}{{10m}}$    
\choice $\overline E  = \frac{{17{\hbar ^2}{k^2}}}{{10m}}$    
\choice $\overline E  = \frac{{19{\hbar ^2}{k^2}}}{{10m}}$
\end{choices}

\question Частица движется в некотором потенциале, отличном от нуля в конечной области. На асимптотике (при $x \to \infty $) волновая функция частицы в некоторый момент времени име-ет вид ${e^{ikx}} + 2{e^{ - 2ikx}}$, где $k$ - некоторое действительное число. Найти средний квадрат энергии частицы.
\begin{choices}
\choice $\overline {{E^2}}  = \frac{{13{\hbar ^4}{k^4}}}{{4{m^2}}}$ 
\choice $\overline {{E^2}}  = \frac{{15{\hbar ^4}{k^4}}}{{4{m^2}}}$    
\choice $\overline {{E^2}}  = \frac{{17{\hbar ^4}{k^4}}}{{4{m^2}}}$    
\choice $\overline {{E^2}}  = \frac{{19{\hbar ^4}{k^4}}}{{4{m^2}}}$
\end{choices}

\question Волновая функция свободной частицы в некоторый момент времени имеет вид ${e^{ikx}} + 2{e^{ - 2ikx}}$, где $k$ - некоторое действительное число. Найти среднюю проек-цию импульса частицы.
\begin{choices}
\choice $\overline p  =  - (3/5)\hbar k$ 
\choice $\overline p  =  - (7/5)\hbar k$    
\choice $\overline p  =  - (11/5)\hbar k$      
\choice $\overline p  =  - (9/5)\hbar k$
\end{choices}

\question Волновая функция свободной частицы в некоторый момент времени имеет вид ${e^{ikx}} + 2{e^{ - 2ikx}}$, где $k$ - некоторое действительное число. Найти средний квадрат проекции импульса частицы.
\begin{choices}
\choice $\overline {{p^2}}  = \frac{{13}}{5}{\hbar ^2}{k^2}$  
\choice $\overline {{p^2}}  = \frac{{15}}{5}{\hbar ^2}{k^2}$     
\choice $\overline {{p^2}}  = \frac{{17}}{5}{\hbar ^2}{k^2}$     
\choice $\overline {{p^2}}  = \frac{{19}}{5}{\hbar ^2}{k^2}$
\end{choices}

\question Волновая функция свободной частицы в некоторый момент времени имеет вид ${e^{ikx}} + 2{e^{ - 2ikx}}$, где $k$ - некоторое действительное число. Как средний импульс в этом состоянии зависит от времени?
\begin{choices}
\choice растет      
\choice убывает     
\choice осциллирует    
\choice не зависит от времени
\end{choices}

\question В некоторый момент времени свободная частица находится в состоянии с определенным значением координаты $x = a$. Является ли это состояние стационарным?
\begin{choices}
\choice да       
\choice нет         
\choice это зависит от $a$      
\choice это зависит от энергии
\end{choices}

\question В некоторый момент времени свободная частица находится в состоянии с определенным значением импульса $p = {p_0}$. Является ли это состояние стационарным?
\begin{choices}
\choice да       
\choice нет         
\choice это зависит от ${p_0}$     
\choice это зависит от энергии
\end{choices}

\question Потенциальная энергия частицы $U(x)$ равна нулю. Какой из нижеприведенных фор-мул может описываться волновая функция стационарного состояния частицы при энергии $E$ ($k = \sqrt {2mE/{\hbar ^2}} $, $m$ - масса частицы)?
\begin{choices}
\choice $\sin (kx)\exp (iEt/\hbar )$           
\choice $\exp (ikx)\cos (Et/\hbar )$
\choice $\cos (kx)\exp ( - iEt/\hbar )$        
\choice $\exp ( - ikx)\sin (Et/\hbar )$
\end{choices}

\question Потенциальная энергия частицы $U(x)$ равна нулю. Какой из нижеприведенных фор-мул не может описываться волновая функция стационарного состояние частицы при энергии $E$ ($E > 0$, $k = \sqrt {2mE/{\hbar ^2}} $, $m$ - масса частицы)?
\begin{choices}
\choice $\exp ( - ikx)\exp ( - iEt/\hbar ) + \exp (ikx)\exp ( - iEt/\hbar )$    
\choice $\cos (kx)\exp ( - iEt/\hbar )$
\choice $\exp ( - ikx)\exp ( - iEt/\hbar ) + \exp ( - ikx)\exp (iEt/\hbar )$     
\choice $\exp ( - ikx)\exp ( - iEt/\hbar )$
\end{choices}

\question Частица, движущаяся в некотором потенциале, находится в стационарном состоянии не-прерывного спектра. Как зависит от времени средняя координата частицы?
\begin{choices}
\choice растет      
\choice убывает     
\choice осциллирует    
\choice не зависит от времени
\end{choices}

\question Частица, движущаяся в некотором потенциале, который отличен от нуля в конечной области (см. рисунок), находится в стационарном состоянии непрерывного спектра. Как меняется с течением времени вероятность обнаружить частицу между точками с координатой ${x_1}$ и ${x_2}$?
\begin{choices}
\choice растет   
\choice убывает  
\choice осциллирует 
\choice не зависит от времени
\end{choices}

\question Частица, движущаяся в некотором потенциале (см. рисунок), находится в двукратно вырожденном стационарном состоянии непрерывного спектра. В какой области координат самый большой поток вероятности? 
\begin{choices}
\choice при $x < {x_1}$         
\choice при ${x_1} < x < {x_2}$
\choice при $x > {x_1}$         
\choice одинаков
\end{choices}

\question Частица, движущаяся в некотором потенциале (см. рисунок к предыдущей задаче), нахо-дится в невырожденном стационарном состоянии непрерывного спектра. В какой области координат поток вероятности равен нулю? 
\begin{choices}
\choice при $x < {x_1}$         
\choice при ${x_1} < x < {x_2}$
\choice при $x > {x_1}$         
\choice при любых значениях координат
\end{choices}

\question Какое утверждение связано с сохранением числа частиц?
\begin{choices}
\choice Средняя энергия частиц в стационарном состоянии непрерывного спектра не зависит от вре-мени
\choice Поток вероятности в стационарном состоянии непрерывного спектра не зависит от времени
\choice Поток вероятности в стационарном состоянии непрерывного спектра не зависит от координат
\choice Средняя энергия частиц в стационарном состоянии непрерывного спектра не зависит от координат
\end{choices}

\question Частица, движущаяся в некотором потенциале, находится в стационарном состоянии не-прерывного спектра. Какое утверждение справедливо?
(1) Поток вероятности в стационарном состоянии непрерывного спектра не зависит от времени
(2) Поток вероятности в стационарном состоянии непрерывного спектра не зависит от коорди-нат
\begin{choices}
\choice только (1)     
\choice только (2)     
\choice и (1), и (2)      
\choice ни (1), ни (2)
\end{choices}

\question Волновая функция свободных частиц имеет вид $5{e^{ikx}} + 3{e^{ - ikx}}$, где $k$ - некоторое число. Чему равна плотность потока вероятности в этом состоянии (в единицах $\hbar k/m$, где $m$ - масса частиц)?
\begin{choices}
\choice $|5 - 3{|^2} = 4$    
\choice ${5^2} - {3^2} = 16$    
\choice $|5 + 3{|^2} = 64$         
\choice ${5^2} + {3^2} = 34$ 
\end{choices}

\question Потенциальная энергия частицы $U(x)$ отлична от нуля в конечной области. Может ли волновая функция стационарного состояния частицы одновременно иметь следующие асимпто-тики: при $x \to  - \infty $ - ${e^{ikx}}$, и при $x \to \infty $ - ${e^{ - ikx}}$?
\begin{choices}
\choice да             
\choice нет      
\choice это зависит от энергии     
\choice бессмысленный вопрос
\end{choices}

\question Потенциальная энергия частицы отлична от нуля в конечной области. Волновая функция при $x \to  + \infty $ имеет вид $2{e^{ikx}}$, где $k$ - некоторое число. Может ли волновая функция при $x \to  - \infty $ быть равной $5{e^{ikx}} + 3{e^{ - ikx}}$?
\begin{choices}
\choice да             
\choice нет
\choice это зависит от энергии     
\choice бессмысленный вопрос
\end{choices}

\question Потенциальная энергия частицы отлична от нуля в конечной области. Волновая функция при $x \to  + \infty $ имеет вид $4i{e^{ikx}}$, где $k$ - некоторое число. Может ли волновая функция при $x \to  - \infty $ быть равной $3{e^{ikx}} + 5{e^{ - ikx}}$?
\begin{choices}
\choice да             
\choice нет
\choice это зависит от энергии     
\choice бессмысленный вопрос
\end{choices}

\question Потенциальная энергия частицы отлична от нуля в конечной области. Волновая функция при $x \to  + \infty $ имеет вид $4i{e^{ikx}}$, где $k$ - некоторое число. Может ли волновая функция при $x \to  - \infty $ быть равной $5{e^{ikx}} - {e^{ - ikx}}$?
\begin{choices}
\choice да             
\choice нет
\choice это зависит от энергии     
\choice бессмысленный вопрос
\end{choices}

\question Какова размерность коэффициентов отражения и прохождения?
\begin{choices}
\choice       
\choice       
\choice       
\choice 
\end{choices}

\question Могут ли коэффициенты прохождения и отражения частиц от некоторого потенциала быть равными $R = 0.125,\quad T = 0.885$?
\begin{choices}
\choice да             
\choice нет
\choice это зависит от энергии     
\choice бессмысленный вопрос
\end{choices}

\question Потенциальная энергия частицы $U(x)$ отлична от нуля в конечной области. Волновая функция стационарного состояния частицы в некоторый момент времени имеет следующие асимптотики ($k = \sqrt {2mE/{\hbar ^2}} $,$m$ - масса частицы): при $x \to  - \infty $ - ${e^{ikx}}$, при $x \to \infty $ - $\frac{1}{2}{e^{ - ikx}} - \frac{{i\sqrt 3 }}{2}{e^{ikx}}$. Чему равны коэффициенты отражения $R$ и прохождения $T$?
\begin{choices}
\choice $R = 1/4,\quad T = 3/4$       
\choice $R = 3/4,\quad T = 1/4$
\choice $R = 1/3,\quad T = 2/3$       
\choice такие асимптотики волновая функция иметь не мо-жет
\end{choices}

\question Потенциальная энергия частицы $U(x)$ отлична от нуля в конечной области. Волновая функция стационарного состояния частицы в некоторый момент времени имеет следующие асимптотики ($k = \sqrt {2mE/{\hbar ^2}} $,$m$ - масса частицы): при $x \to  - \infty $ - $\frac{1}{{\sqrt 2 }}{e^{ - ikx}}$, при $x \to \infty $ - $\frac{1}{2}{e^{ikx}} - \frac{{i\sqrt 3 }}{2}{e^{ - ikx}}$. Чему равны коэффициенты отражения $R$ и прохождения $T$?
\begin{choices}
\choice $R = 1/4,\quad T = 3/4$       
\choice $R = 3/4,\quad T = 1/4$
\choice $R = 1/3,\quad T = 2/3$       
\choice такие асимптотики волновая функция иметь не мо-жет
\end{choices}

\question Потенциальная энергия частицы $U(x)$ отлична от нуля в конечной области. Волновая функция стационарного состояния частицы в некоторый момент времени имеет следующие асимптотики ($k = \sqrt {2mE/{\hbar ^2}} $, $m$ - масса частицы): при $x \to  - \infty $ - $\frac{1}{3}{e^{ - ikx}} - \frac{{i\sqrt {10} }}{3}{e^{ikx}}$, при $x \to \infty $ - ${e^{ikx}}$. Чему равны коэффициенты отражения $R$ и прохождения $T$?
\begin{choices}
\choice $R = 1/10,\quad T = 9/10$     
\choice $R = 1/9,\quad T = 8/9$
\choice $R = 9/10,\quad T = 1/10$     
\choice такие асимптотики волновая функция иметь не мо-жет
\end{choices}

\question Потенциальная энергия частицы $U(x)$ отлична от нуля в конечной области. Волновая функция стационарного состояния частицы в некоторый момент времени имеет следующие асимптотики ($k = \sqrt {2mE/{\hbar ^2}} $, $m$ - масса частицы): при $x \to  - \infty $ - $4{e^{ikx}}$, при $x \to \infty $ - $5{e^{ikx}} - 3{e^{ - ikx}}$. Чему равны коэффициенты отра-жения $R$ и прохождения $T$?
\begin{choices}
\choice $R = 16/25,\quad T = 9/25$    
\choice по данной функции определить $R$ и $T$ нельзя
\choice $R = 9/25,\quad T = 16/25$    
\choice такие асимптотики волновая функция иметь не мо-жет
\end{choices}

\question Потенциальная энергия частицы $U(x)$ отлична от нуля в конечной области. Волновая функция стационарного состояния частицы в некоторый момент времени имеет следующие асимптотики ($k = \sqrt {2mE/{\hbar ^2}} $, $m$ - масса частицы): при $x \to  - \infty $ - $4{e^{ - ikx}}$, при $x \to \infty $ - $5{e^{ - ikx}} - 3{e^{ikx}}$. Чему равны коэффициенты отражения $R$ и прохождения $T$?
\begin{choices}
\choice $R = 16/25,\quad T = 9/25$    
\choice по данной функции определить $R$ и $T$ нельзя
\choice $R = 9/25,\quad T = 16/25$    
\choice такие асимптотики волновая функция иметь не мо-жет
\end{choices}

\question Потенциальная энергия частицы $U(x)$ отлична от нуля в конечной области. Волновая функция стационарного состояния частицы имеет следующую асимптотику при $x \to  - \infty $ - $\frac{1}{2}{e^{ikx}} - \frac{{i\sqrt 3 }}{2}{e^{ - ikx}}$. Измеряют поток частиц в этом состоя-нии в области действия потенциала при $x = 0$. Какое значение будет получено?
\begin{choices}
\choice $ - \frac{{\hbar k}}{{2m}}$      
\choice $\frac{{\hbar k}}{{2m}}$      
\choice $ - \frac{{\hbar k}}{{4m}}$      
\choice мало информации для ответа
\end{choices}

\question Потенциальная энергия частицы $U(x)$ отлична от нуля в конечной области. Волновая функция стационарного состояния частицы в некоторый момент времени имеет следующие асимптотики: при $x \to  - \infty $ - $\frac{1}{3}{e^{ikx}} - \frac{{i\sqrt {10} }}{3}{e^{ - ikx}}$, при $x \to \infty $ - ${e^{ - ikx}}$. Где расположены источники частиц?
\begin{choices}
\choice только на $ + \infty $        
\choice только на $ - \infty $
\choice и на $ + \infty $ и на $ - \infty $       
\choice такие асимптотики волновая функция иметь не может
\end{choices}

\question Потенциальная энергия частицы $U(x)$ отлична от нуля в конечной области. Волновая функция стационарного состояния частицы в некоторый момент времени имеет следующие асимптотики: при $x \to  - \infty $ - $\frac{1}{3}{e^{ikx}} - \frac{{i\sqrt {10} }}{3}{e^{ - ikx}}$, при $x \to \infty $ - ${e^{ikx}}$. Где расположены источники частиц?
\begin{choices}
\choice только на $ + \infty $        
\choice только на $ - \infty $
\choice и на $ + \infty $ и на $ - \infty $       
\choice такие асимптотики волновая функция иметь не может
\end{choices}

\question Потенциальная энергия частицы $U(x)$ отлична от нуля в конечной области. Волновая функция стационарного состояния частицы в некоторый момент времени имеет следующие асимптотики: при $x \to  - \infty $ - $\frac{1}{3}{e^{ - ikx}} - \frac{{i\sqrt {10} }}{3}{e^{ikx}}$, при $x \to \infty $ - ${e^{ikx}}$. Где расположены источники частиц?
\begin{choices}
\choice только на $ + \infty $        
\choice только на $ - \infty $
\choice и на $ + \infty $ и на $ - \infty $       
\choice такие асимптотики волновая функция иметь не может
\end{choices}

\question Два источника частиц, расположенные при $x \to  + \infty $и при $x \to  - \infty $ направ-ляют стационарные потоки частиц с одинаковой энергией на область действия некоторого по-тенциала. Может ли волновая функция этих частиц в некоторый момент времени иметь асим-птотический вид ${e^{ikx}}$ при $x \to  - \infty $, и $\sqrt 2 {e^{ikx}} - {e^{ - ikx}}$ при $x \to  + \infty $?
\begin{choices}
\choice да    
\choice нет      
\choice это зависит от энергии     
\choice бессмысленный вопрос
\end{choices}

\question Два источника частиц, расположенные при $x \to  + \infty $и при $x \to  - \infty $ направ-ляют стационарные потоки частиц с одинаковой энергией на область действия некоторого по-тенциала. Может ли волновая функция этих частиц в некоторый момент времени иметь асим-птотический вид ${e^{ - ikx}}$ при $x \to  - \infty $, и $\sqrt 2 {e^{ - ikx}} - {e^{ikx}}$ при $x \to  + \infty $?
\begin{choices}
\choice да    
\choice нет      
\choice зависит от энергии      
\choice зависит от потенциала
\end{choices}

\question Два источника частиц, расположенные при $x \to  + \infty $и при $x \to  - \infty $ направ-ляют стационарные потоки частиц с одинаковой энергией на область действия некоторого по-тенциала. Известно, что волновая функция этих частиц в некоторый момент времени имеет асимптотики: $2{e^{ikx}} + 3{e^{ - ikx}}$ при $x \to  - \infty $, и $4{e^{ikx}} + \sqrt {21} {e^{ - ikx}}$ при $x \to  + \infty $. Чему равно отношение интенсивности ${W_ - }$ источника, распо-ложенного на $ - \infty $, к интенсивности ${W_ + }$ источника, расположенного на $ + \infty $?
\begin{choices}
\choice $\frac{{{W_ - }}}{{{W_ + }}} = \frac{4}{{16}}$     
\choice $\frac{{{W_ - }}}{{{W_ + }}} = \frac{9}{{16}}$     
\choice $\frac{{{W_ - }}}{{{W_ + }}} = \frac{4}{{21}}$     
\choice $\frac{{{W_ - }}}{{{W_ + }}} = \frac{9}{{21}}$
\end{choices}

\question Сравнить коэффициенты отражения и прохождения для падения частиц на потенциал (см. рисунок) из $ - \infty $ (${R_ - },\quad {T_ - }$) и для падения частиц на этот же потенциал из $ + \infty $ (${R_ + },\quad {T_ + }$)?
\begin{choices}
\choice ${R_ - } = {R_ + },\quad {T_ - } = {T_ + }$        
\choice ${R_ - } > {R_ + },\quad {T_ - } < {T_ + }$
\choice ${R_ - } < {R_ + },\quad {T_ - } > {T_ + }$        
\choice ${R_ - } > {R_ + },\quad {T_ - } > {T_ + }$
\end{choices}

\question Частицы с определенной энергией падают на потенциальную ступеньку (см. рисунок) из $ - \infty $. Высота ступеньки больше энергии частиц. Чему равен поток частиц при $x \to  + \infty $?
\begin{choices}
\choice нулю           
\choice потоку падающих частиц
\choice потоку отраженных частиц      
\choice этот поток найти нельзя
\end{choices}

\question Частицы с определенной энергией падают на потенциальную ступеньку (см. рисунок к предыдущей задаче) из $ - \infty $. Высота ступеньки больше энергии частиц. Чему равны ко-эффициенты отражения $R$ и прохождения $T$?
\begin{choices}
\choice $R = 0,\quad T = 1$         
\choice $R = 1,\quad T = 0$
\choice $R = 0,5,\quad T = 0,5$        
\choice в этом случае определить $R$ и $T$ нельзя
\end{choices}

\end{questions}





\section{ ГЛАВА 3. МОМЕНТ ИМПУЛЬСА }
\subsection{ Общие свойства операторов момента импульса и их собственных функций }
(в этой главе $\vartheta $ и $\varphi $ - полярный и азимутальный углы в сферических коорди-натах)

\begin{questions}

\question Размерность момента импульса совпадает
\begin{choices}
\choice с размерностью $\hbar $       
\choice с размерностью ${\hbar ^2}$ 
\choice с размерностью $\sqrt \hbar  $         
\choice с размерностью $1/\hbar $
\end{choices}

\question Какой формулой определяется выражение для оператора проекции орбитального момента на ось $x$ в декартовых координатах?
\begin{choices}
\choice $i\hbar z\frac{\partial }{{\partial y}}$     
\choice $ - i\hbar y\frac{\partial }{{\partial z}}$     
\choice $i\hbar \left( {z\frac{\partial }{{\partial y}} - y\frac{\partial }{{\partial z}}} \right)$     
\choice $i\hbar \left( {y\frac{\partial }{{\partial z}} - z\frac{\partial }{{\partial y}}} \right)$
\end{choices}

\question Какой формулой определяется выражение для оператора проекции орбитального момента на ось $y$ в декартовых координатах?
\begin{choices}
\choice $i\hbar \left( {y\frac{\partial }{{\partial z}} - z\frac{\partial }{{\partial y}}} \right)$  
\choice $i\hbar \left( {z\frac{\partial }{{\partial x}} - x\frac{\partial }{{\partial z}}} \right)$  
\choice $i\hbar \left( {z\frac{\partial }{{\partial y}} - y\frac{\partial }{{\partial z}}} \right)$  
\choice $i\hbar \left( {x\frac{\partial }{{\partial z}} - z\frac{\partial }{{\partial x}}} \right)$
\end{choices}

\question Какой формулой определяется выражение для оператора проекции орбитального момента на ось $z$ в сферических координатах?
\begin{choices}
\choice $ - i\hbar \frac{\partial }{{\partial \varphi }}$     
\choice $ - i\hbar \frac{\partial }{{\partial \vartheta }}$      
\choice $i\hbar \left( {\frac{\partial }{{\partial \varphi }} - \frac{\partial }{{\partial \vartheta }}} \right)$      
\choice $i\hbar \left( {\frac{\partial }{{\partial \vartheta }} - \frac{\partial }{{\partial \varphi }}} \right)$
\end{choices}

\question От каких переменных зависят в декартовых координатах собственные функции оператора проекции орбитального момента на ось $z$ ${\hat L_z}$?
\begin{choices}
\choice только от $z$
\choice только от $x$ и $y$
\choice их можно выбрать так, чтобы они зависели только от $z$
\choice их можно выбрать так, чтобы они зависели только от $x$ и $y$
\end{choices}

\question От каких переменных зависят в сферических координатах собственные функции операто-ра проекции орбитального момента на ось $z$ ${\hat L_z}$?
\begin{choices}
\choice только от $\varphi $
\choice только от $\varphi $ и $\vartheta $
\choice их можно выбрать так, чтобы они зависели только от $\varphi $
\choice их можно выбрать так, чтобы они зависели только от $\varphi $ и $\vartheta $
\end{choices}

\question Какое из перечисленных равенств верное?
\begin{choices}
\choice $\left[ {{{\hat L}_x},{{\hat L}_z}} \right] =  - i\hbar {\hat L_y}$     
\choice $\left[ {{{\hat L}_z},{{\hat L}_y}} \right] = i\hbar {\hat L_z}$     
\choice $\left[ {{{\hat L}_x},{{\hat L}_z}} \right] = i\hbar {\hat L_x}$  
\choice $\left[ {{{\hat L}_x},{{\hat L}_z}} \right] = i\hbar {L_y}$
\end{choices}

\question Какой из перечисленных коммутаторов равен нулю?
\begin{choices}
\choice $\left[ {{{\hat L}_x},{{\hat p}_z}} \right]$    
\choice $\left[ {{{\hat L}_x},\hat y} \right]$    
\choice $\left[ {{{\hat L}_x},\hat z} \right]$    
\choice $\left[ {{{\hat L}_x},{{\hat p}_x}} \right]$
\end{choices}

\question Операторы ${\hat L_ + } = {\hat L_x} + i{\hat L_y}$ и ${\hat L_ - } = {\hat L_x} - i{\hat L_y}$ являются
\begin{choices}
\choice эрмитовыми  
\choice самосопряженными  
\choice унитарными  
\choice линейными
\end{choices}

\question Оператор ${\hat L_ + } = {\hat L_x} + i{\hat L_y}$ по отношению к оператору ${\hat L_ - } = {\hat L_x} - i{\hat L_y}$ является 
\begin{choices}
\choice обратным       
\choice эрмитово самосопряженными
\choice транспонированным 
\choice комплексно сопряженным
\end{choices}

\question Какой из перечисленных коммутаторов равен нулю?
\begin{choices}
\choice $\left[ {{{\hat L}_x},{{\hat L}_z}} \right]$    
\choice $\left[ {{{\hat L}^2},{{\hat L}_ + }} \right]$     
\choice $\left[ {{{\hat L}_ + },{{\hat L}_ - }} \right]$      
\choice $\left[ {{{\hat L}_ + },{{\hat L}_z}} \right]$
\end{choices}

\question Какое из перечисленных равенств верное?
\begin{choices}
\choice $\left[ {{{\hat L}^2},{{\hat L}_x}} \right] = {\hat L_y}^3$ 
\choice $\left[ {{{\hat L}^2},{{\hat L}_x}} \right] =  - i\hbar {\hat L_y}^2$   
\choice $\left[ {{{\hat L}^2},{{\hat L}_x}} \right] =  - {\hbar ^2}{\hat L_y}$  
\choice $\left[ {{{\hat L}^2},{{\hat L}_x}} \right] = 0$
\end{choices}

\question Какое из перечисленных равенств верное (${\hat L^4}$ - оператор четвертой степени момента)?
\begin{choices}
\choice $\left[ {{{\hat L}^4},{{\hat L}_x}} \right] = {\hat L_y}^5$ 
\choice $\left[ {{{\hat L}^4},{{\hat L}_x}} \right] =  - i\hbar {\hat L_y}^4$      
\choice $\left[ {{{\hat L}^4},{{\hat L}_x}} \right] =  - {\hbar ^2}{\hat L_y}^3$      
\choice $\left[ {{{\hat L}^4},{{\hat L}_x}} \right] = 0$
\end{choices}

\question Какой оператор повышает проекцию момента импульса частицы на ось $z$?
\begin{choices}
\choice ${\hat L_x} + {\hat L_y}$  
\choice ${\hat L_x} - i{\hat L_y}$ 
\choice ${\hat L_x} + i{\hat L_y}$ 
\choice ${\hat L_x} + {\hat L_y}$
\end{choices}

\question Какой оператор понижает проекцию момента импульса частицы на ось $x$?
\begin{choices}
\choice ${\hat L_z} - i{\hat L_y}$ 
\choice ${\hat L_y} - i{\hat L_z}$ 
\choice ${\hat L_y} + i{\hat L_z}$ 
\choice такого оператора не существует
\end{choices}

\question Оператор ${\hat L_ + }{\hat L_ - }$ является
\begin{choices}
\choice эрмитовым   
\choice нелинейным  
\choice унитарными  
\choice никаким из перечесленного
\end{choices}

\question Выбрать верное операторное равенство для оператора ${\hat L_ + }{\hat L_ - }$
\begin{choices}
\choice ${\hat L_ + }{\hat L_ - } = {\hat L^2} - \hat L_z^2$  
\choice ${\hat L_ + }{\hat L_ - } = {\hat L^2} - \hat L_z^2 + \hbar {\hat L_z}$ 
\choice ${\hat L_ + }{\hat L_ - } = {\hat L^2} - \hat L_z^2 - \hbar {\hat L_z}$ 
\choice ${\hat L_ + }{\hat L_ - } = \hat L_x^2 + \hat L_y^2$
\end{choices}

\question Выбрать верное операторное равенство для оператора ${\hat L_ - }{\hat L_ + }$
\begin{choices}
\choice ${\hat L_ - }{\hat L_ + } = {\hat L^2} - \hat L_z^2$  
\choice ${\hat L_ - }{\hat L_ + } = {\hat L^2} - \hat L_z^2 + \hbar {\hat L_z}$ 
\choice ${\hat L_ - }{\hat L_ + } = {\hat L^2} - \hat L_z^2 - \hbar {\hat L_z}$ 
\choice ${\hat L_ - }{\hat L_ + } = \hat L_x^2 + \hat L_y^2$
\end{choices}

\question Частица находится в состоянии с определенной проекцией импульса на ось $z$ (${p_z} \ne 0$). Имеет ли в этом состоянии проекция орбитального момента на ось $z$ определенное значение?
\begin{choices}
\choice да    
\choice нет      
\choice зависит от состояния    
\choice зависит от ${p_z}$
\end{choices}

\question Частица находится в состоянии с определенной проекцией импульса на ось $x$ (${p_x} \ne 0$).. Имеет ли в этом состоянии проекция орбитального момента на ось $z$ определенное значение?
\begin{choices}
\choice да    
\choice нет      
\choice зависит от состояния    
\choice зависит от ${p_x}$
\end{choices}

\question Операторы ${\hat L_z}$ и ${\hat p_x}$ не коммутируют. Имеют ли эти операторы общие собственные функции?
\begin{choices}
\choice да
\choice нет
\choice это зависит от оператора Гамильтона
\choice бессмысленный вопрос
\end{choices}

\question Операторы ${\hat L_z}$ и ${\hat p_x}$ не коммутируют. Имеют ли эти операторы пол-ную систему общих собственных функций?
\begin{choices}
\choice да
\choice нет
\choice это зависит от оператора Гамильтона
\choice бессмысленный вопрос
\end{choices}

\question Операторы ${\hat L_z}$ и ${\hat p_z}$ коммутируют. Будет ли любая собственная функ-ция одного из них собственной функцией другого?
\begin{choices}
\choice да       
\choice нет         
\choice это зависит от оператора Гамильтона
\choice бессмысленный вопрос
\end{choices}

\question Частица находится в состоянии с определенной проекцией импульса на ось $z$ (${p_z} \ne 0$). Какие из нижеперечисленных величин могут в этом состоянии также иметь определен-ное значение?
\begin{choices}
\choice ${L_x}$     
\choice ${L_y}$     
\choice ${L_z}$     
\choice ${L^2}$
\end{choices}

\question Частица находится в состоянии, в котором квадрат орбитального момента имеет опреде-ленное значение. Какие из нижеперечисленных величин имеют в этом состоянии определенное значение?
\begin{choices}
\choice только ${L_x}$    
\choice только ${L_y}$    
\choice только ${L_z}$    
\choice это зави-сит от состояния
\end{choices}

\question Пусть $f$ - общая собственная функция операторов ${\hat L_x}$ и ${\hat L_y}$. Какие утверждения относительно функции $f$ справедливы?
\begin{choices}
\choice она также собственная для оператора ${\hat L_z}$   
\choice она также собственная для оператора $\hat y$
\choice она также собственная для оператора $\hat x$ 
\choice такой функции не существует
\end{choices}

\question Пусть $f$ - общая собственная функция операторов ${\hat L_x}$ и ${\hat L_y}$. Каким собственным значениям она отвечает?
\begin{choices}
\choice ${l_x} = 0$ и ${l_y} = 0$        
\choice ${l_x} = \hbar $ и ${l_y} = 0$
\choice ${l_x} = \hbar $ и ${l_y} = \hbar $       
\choice такой функции не существует
\end{choices}

\question Какие из ниже перечисленных пар операторов имеют полную систему общих собствен-ных функций?
\begin{choices}
\choice ${\hat L^2}$ и ${\hat L_x}$      
\choice ${\hat L_x}$ и ${\hat L_y}$      
\choice ${\hat L_x}$ и ${\hat p_z}$      
\choice никакие из перечисленных
\end{choices}

\question Какие утверждения относительно любой собственной функции оператора квадрата орбитального момента импульса частицы справедливы?
\begin{choices}
\choice она также обязательно является собственной функцией оператора ${\hat L_x}$
\choice она также обязательно является собственной функцией оператора ${\hat L_y}$
\choice она также обязательно является собственной функцией оператора ${\hat L_z}$
\choice мало информации чтобы выбрать между А., 
\choice и В.
\end{choices}

\question Частица находится в состоянии, в котором проекция орбитального момента на ось $z$ имеет определенное значение. Будет ли эта волновая функция частицы собственной функцией оператора квадрата момента?
\begin{choices}
\choice да                
\choice нет
\choice это зависит от состояния      
\choice это зависит от оператора Гамильтона
\end{choices}

\question Вырождены или нет собственные значения оператора квадрата момента?
\begin{choices}
\choice да
\choice нет
\choice это зависит от собственного значения
\choice это зависит от оператора Гамильтона
\end{choices}

\question Вырождены или нет не равные нулю собственные значения оператора квадрата момен-та?
\begin{choices}
\choice да
\choice нет
\choice это зависит от собственного значения
\choice это зависит от оператора Гамильтона
\end{choices}

\question Какое из нижеперечисленных утверждений справедливо?
\begin{choices}
\choice любая собственная функция оператора ${\hat L^2}$ будет собственной функцией оператора ${\hat L_z}$
\choice любая собственная функция оператора ${\hat L_z}$ будет собственной функцией  оператора ${\hat L^2}$
\choice собственная функция оператора ${\hat L^2}$ не будет, вообще говоря, собственной функци-ей оператора ${\hat L_z}$, но можно построить такие линейные комбинации собственных функций оператора ${\hat L^2}$, которые будут собственными функциями ${\hat L_z}$
\choice все перечисленное неверно
\end{choices}

\question Частица находится в состоянии, в котором квадрат момента импульса имеет определен-ное значение, а проекция момента на ось $z$ может принимать два значения. Волновая функ-ция этого состояния
\begin{choices}
\choice будет собственной функцией операторов ${\hat L^2}$ и ${\hat L_z}$
\choice будет собственной для оператора ${\hat L^2}$ и не будет собственной для оператора ${\hat L_z}$
\choice будет собственной для оператора ${\hat L_z}$ и не будет собственной для оператора ${\hat L^2}$
\choice информации для выбора между А, Б и В недостаточно
\end{choices}

\question Частица находится в состоянии, в котором проекция момента на ось $z$ имеет опреде-ленное значение, а квадрат момента может принимать два значения. Волновая функция этого состояния
\begin{choices}
\choice будет собственной функцией операторов ${\hat L^2}$ и ${\hat L_z}$
\choice будет собственной функцией оператора ${\hat L^2}$ и не будет собственной функцией опе-ратора ${\hat L_z}$
\choice будет собственной функцией оператора ${\hat L_z}$ и не будет собственной функцией опе-ратора ${\hat L^2}$
\choice информации для выбора между А, Б и В недостаточно
\end{choices}

\question Пусть ${l_z}$ - собственное значение оператора ${\hat L_z}$. Какие утверждения отно-сительно ${l_z}$ справедливы?
\begin{choices}
\choice это число будет также собственным значением оператора ${\hat L^2}$
\choice это число будет также собственным значением оператора ${\hat L_y}$
\choice это число чисто мнимо
\choice это число имеет размерность квадрата постоянной Планка
\end{choices}

\question Частица находится в состоянии, в котором проекция орбитального момента импульса на ось $z$ может принимать значение ${l_{z,1}}$ с вероятностью 1/4 и с вероятностью 3/4 – значение ${l_{z,2}}$. Какие из нижеследующих утверждений относительно чисел ${l_{z,1}}$ и ${l_{z,2}}$ справедливы?
\begin{choices}
\choice они будут собственными значениями оператора $x$
\choice они будут собственными значениями оператора ${\hat p_z}$
\choice они будут собственными значениями оператора ${\hat L^2}$
\choice они будут собственными значениями оператора ${\hat L_y}$
\end{choices}

\question Пусть $f$ - общая собственная функция операторов ${\hat L^2}$ и ${\hat L_z}$, отве-чающая ненулевым собственным значениям. Какое из нижеследующих утверждений относи-тельно этой функции справедливо?
\begin{choices}
\choice эта функция является также собственной функцией оператора ${\hat L_x}^2$
\choice эта функция является также собственной функцией оператора ${\hat L_y}^2$
\choice эта функция является также собственной функцией оператора ${\hat L_x}^2 + {\hat L_y}^2$
\choice эта функция является также собственной функцией оператора ${\hat L_x}^2 - {\hat L_y}^2$
\end{choices}

\question Пусть ${f_{{l^2},{l_z}}}$ - общая собственная функция операторов ${\hat L^2}$ и ${\hat L_z}$, отвечающая собственным значениям ${l^2}$ и ${l_z}$. Будет ли эта функция собствен-ной функцией оператора ${\hat L_x}^2 + {\hat L_y}^2$ и если да, то какому собственному зна-чению она отвечает?
\begin{choices}
\choice да, собственному значению ${l^2} - {l_z}$
\choice да, собственному значению ${l^2} - l_z^2$
\choice да, собственному значению ${l^2} + l_z^2$
\choice нет
\end{choices}

\question Волновая функция частицы имеет вид $\alpha {f_{{l^2},{l_{z1}}}} + \beta {f_{{l^2},{l_{z2}}}}$, где ${f_{{l^2},{l_z}}}$ - общая собственная функция операторов ${\hat L^2}$ и ${\hat L_z}$, отвечающая собственным значениям ${l^2}$ и ${l_z}$. Какие значения квадрата момента ${L^2}$ могут быть получены при измерениях в этом состоянии и с какими вероятностями?
\begin{choices}
\choice определенное значение ${L^2} = {l^2}$
\choice ${L^2} = l_{z1}^2$ с вероятностью ${\alpha ^2}$ и ${L^2} = l_{z2}^2$ с вероятностью ${\beta ^2}$ 
\choice ${L^2} = l_{z1}^2$ с вероятностью ${\alpha ^2}/({\alpha ^2} + {\beta ^2})$ и ${L^2} = l_{z2}^2$ с вероятностью ${\beta ^2}/({\alpha ^2} + {\beta ^2})$
\choice все возможные в интервале $l_{z1}^2 < {L^2} < l_{z2}^2$ с одинаковыми вероятностями
\end{choices}

\question Волновая функция частицы имеет вид $\alpha {f_{{l^2},{l_{z1}}}} + \beta {f_{{l^2},{l_{z2}}}}$, где ${f_{{l^2},{l_z}}}$ - общая собственная функция операторов ${\hat L^2}$ и ${\hat L_z}$, отвечающая собственным значениям ${l^2}$ и ${l_z}$. Какие значения проекции момента на ось $z$ ${L_z}$могут быть получены при измерениях в этом состоянии и с какими вероятностями?
\begin{choices}
\choice определенное значение ${L_z} = \sqrt {{l^2}} $
\choice ${L_z} = l_{z1}^{}$ с вероятностью ${\alpha ^2}$ и ${L_z} = l_{z2}^{}$ с вероятностью ${\beta ^2}$
\choice ${L_z} = l_{z1}^{}$ с вероятностью ${\alpha ^2}/({\alpha ^2} + {\beta ^2})$ и ${L_z} = l_{z2}^{}$ с вероятностью ${\beta ^2}/({\alpha ^2} + {\beta ^2})$
\choice все возможные в интервале $ - \sqrt {{l^2}}  < {L_z} < \sqrt {{l^2}} $ с одинаковыми вероят-ностями
\end{choices}

\question Волновая функция частицы имеет вид $\alpha {f_{l_1^2,{l_z}}} + \beta {f_{l_2^2,{l_z}}}$, где ${f_{{l^2},{l_z}}}$ - общая собственная функция операторов ${\hat L^2}$ и ${\hat L_z}$, от-вечающая собственным значениям ${l^2}$ и ${l_z}$. Какие значения квадрата момента ${L^2}$ могут быть получены при измерениях в этом состоянии и с какими вероятностями?
\begin{choices}
\choice определенное значение ${L^2} = l_z^2$
\choice ${L^2} = l_1^2$ с вероятностью ${\alpha ^2}$ и ${L^2} = l_2^2$ с вероятностью ${\beta ^2}$
\choice ${L^2} = l_1^2$ с вероятностью ${\alpha ^2}/({\alpha ^2} + {\beta ^2})$ и ${L^2} = l_2^2$ с вероятностью ${\beta ^2}/({\alpha ^2} + {\beta ^2})$
\choice все возможные в интервале $l_1^2 \le {L^2} \le l_2^2$ с одинаковыми вероятностями
\end{choices}

\question Волновая функция частицы имеет вид $\alpha {f_{l_1^2,{l_z}}} + \beta {f_{l_2^2,{l_z}}}$, где ${f_{{l^2},{l_z}}}$ - общая собственная функция операторов ${\hat L^2}$ и ${\hat L_z}$, от-вечающая собственным значениям ${l^2}$ и ${l_z}$. Какие значения проекции момента на ось $z$ ${L_z}$ могут быть получены при измерениях в этом состоянии и с какими вероятностями?
\begin{choices}
\choice определенное значение ${L_z} = {l_z}$
\choice ${L_z} = \sqrt {l_1^2} $ с вероятностью ${\alpha ^2}$ и ${L_z} = \sqrt {l_2^2} $ с вероятно-стью ${\beta ^2}$
\choice ${L_z} = \sqrt {l_1^2} $ с вероятностью ${\alpha ^2}/({\alpha ^2} + {\beta ^2})$ и ${L_z} = \sqrt {l_2^2} $ с вероятностью ${\beta ^2}/({\alpha ^2} + {\beta ^2})$
\choice все возможные в интервале $\sqrt {l_1^2}  < {l_z} < \sqrt {l_2^2} $ с одинаковыми вероятно-стями
\end{choices}

\question Частица находится в состоянии с определенным значением проекции момента на ось $z$. Будет ли квадрат момента иметь определенное значение в этом состоянии?
\begin{choices}
\choice да    
\choice нет      
\choice зависит от состояния       
\choice бессмысленный вопрос
\end{choices}

\question Частица находится в состоянии с определенным (и не равным нулю) значением проек-ции момента на ось $z$. Будет ли проекция момента на ось $x$ иметь определенное значение в этом состоянии?
\begin{choices}
\choice да             
\choice нет
\choice зависит от состояния       
\choice бессмысленный вопрос
\end{choices}

\question Частица находится в состоянии с определенным значением квадрата момента. Будет ли проекция момента на ось $x$ иметь определенное значение в этом состоянии?
\begin{choices}
\choice да             
\choice нет
\choice зависит от состояния    
\choice бессмысленный вопрос
\end{choices}

\question Частица находится в состоянии ${\psi _{{l_z}}}$, волновая функция которого является собственной для оператора ${\hat L_z}$, отвечающей собственному значению ${l_z}$. Проводят измерения квадрата момента ${L^2}$. Какие значения при этом могут быть получены?
\begin{choices}
\choice Собственные значения оператора ${\hat L^2}$ из интервала $0 \le {L^2} \le l_z^2$ с некото-рыми вероятностями
\choice Собственные значения оператора ${\hat L^2}$ из интервала $ - l_z^2 \le {L^2} \le l_z^2$ с не-которыми вероятностями
\choice Собственные значения оператора ${\hat L^2}$ из интервала $l_z^2 < {L^2}$ с некоторыми вероятностями 
\choice Собственные значения оператора ${\hat L^2}$ из интервала ${L^2} <  - l_z^2$ с некоторыми вероятностями 
\end{choices}

\question Частица находится в состоянии ${\psi _{{l^2}}}$, волновая функция которого является собственной для оператора ${\hat L^2}$, отвечающей собственному значению ${l^2}$. Прово-дят измерения проекции момента на ось $z$ ${L_z}$. Какие значения при этом могут быть по-лучены?
\begin{choices}
\choice Собственные значения оператора ${\hat L_z}$ из интервала $0 \le {l_z} \le \sqrt {{l^2}} $ с не-которыми вероятностями
\choice Собственные значения оператора ${\hat L_z}$ из интервала $ - \sqrt {{l^2}}  \le {l_z} \le \sqrt {{l^2}} $ с некоторыми вероятностями
\choice Собственные значения оператора ${\hat L_z}$ из интервала $\sqrt {{l^2}}  < {l_z}$ с некото-рыми вероятностями
\choice Собственные значения оператора ${\hat L_z}$ из интервала ${l_z} <  - \sqrt {{l^2}} $ с неко-торыми вероятностями
\end{choices}

\question Частица находится в состоянии, в котором проекция орбитального момента на ось $z$ имеет определенное значение ${l_z}$. Чему равно среднее значение величины ${L_y}$ в этом состоянии?
\begin{choices}
\choice $\overline {{L_y}}  = {l_z}$     
\choice $\overline {{L_y}}  = {l_z}/2$      
\choice $\overline {{L_y}}  = 0$      Г.$\overline {{L_y}}  =  - {l_z}/2$
\end{choices}

\question Частица находится в состоянии, в котором проекция орбитального момента на ось $x$ имеет определенное значение ${l_x}$. Чему равно среднее значение величины ${L_y}$ в этом состоянии?
\begin{choices}
\choice $\overline {{L_y}}  = {l_x}$     
\choice $\overline {{L_y}}  = {l_x}/2$      
\choice $\overline {{L_y}}  = 0$      
\choice $\overline {{L_y}}  =  - {l_x}/2$
\end{choices}

\question Частица находится в состоянии, в котором квадрат момента и его проекция на ось $z$ имеют определенные значения ${l^2}$ и ${l_z}$. Измеряют проекцию на ось $x$. Какие значе-ния могут быть получены?
\begin{choices}
\choice Определенное значение ${L_x} = {l_z}$
\choice определенное значение ${L_x} = \sqrt {{l^2}} $
\choice любое собственное значение оператора ${\hat L_x}$ из интервала $ - \sqrt {{l^2}}  < {L_x} < \sqrt {{l^2}} $
\choice любое собственное значение оператора ${\hat L_x}$ из интервала $ - {l_z} < {L_x} < {l_z}$
\end{choices}

\question Волновая функция частицы имеет вид $\alpha {f_{l_1^2,{l_z}}} + \beta {f_{l_2^2,{l_z}}}$, где ${f_{{l^2},{l_z}}}$ - общая собственная функция операторов ${\hat L^2}$ и ${\hat L_z}$, от-вечающая собственным значениям ${l^2}$ и ${l_z}$ (${l_1}^2 > {l_2}^2$). Измеряют проекцию на ось $x$. Какие значения могут быть получены?
\begin{choices}
\choice Определенное значение ${L_x} = {l_z}$
\choice любое собственное значение оператора ${\hat L_x}$ из интервала $ - {l_z} < {L_x} < {l_z}$
\choice любое собственное значение оператора ${\hat L_x}$ из интервала $ - \sqrt {{l_1}^2}  < {L_x} < \sqrt {{l_1}^2} $
\choice любое собственное значение оператора ${\hat L_x}$ из интервала $ - \sqrt {{l_2}^2}  < {L_x} < \sqrt {{l_2}^2} $ 
\end{choices}

\question Волновая функция частицы имеет вид $\alpha {f_{l_{}^2,{l_z}}} + \beta {g_{l_{}^2,{l_x}}}$, где ${f_{{l^2},{l_z}}}$ - общая собственная функция операторов ${\hat L^2}$ и ${\hat L_z}$, ${g_{l_{}^2,{l_x}}}$ - общая собственная функция операторов ${\hat L^2}$ и ${\hat L_x}$,. Измеряют проекцию на ось $y$. Какие значения могут быть получены?
\begin{choices}
\choice Определенное значение ${L_y} = \sqrt {{l^2}} $
\choice любое собственное значение оператора ${\hat L_y}$ из интервала $ - \sqrt {{l^2}}  < {L_y} < \sqrt {{l^2}} $
\choice определенное значение ${L_y} = \sqrt {{l^2} - l_x^2 - l_z^2} $ 
\choice любое собственное значение оператора ${\hat L_y}$ из интервала ${l_x} < {L_y} < {l_z}$
\end{choices}

\question Коммутируют ли операторы квадрата орбитального момента и Гамильтона частицы?
\begin{choices}
\choice да             
\choice нет
\choice это зависит от оператора ${\hat L^2}$  
\choice это зависит от потенциальной энергии
\end{choices}

\question В каком случае операторы квадрата орбитального момента и Гамильтона частицы ком-мутируют?
\begin{choices}
\choice если потенциальная энергия зависит только от $r$
\choice если потенциальная энергия не зависит от $\vartheta $
\choice это зависит от оператора ${\hat L^2}$  
\choice это зависит от потенциальной энергии
\end{choices}

\question Частица находится в состоянии, в котором проекция орбитального момента на ось $z$ имеет определенное значение. Будет ли это состояние стационарным?
\begin{choices}
\choice да                   
\choice нет
\choice эти утверждения никак не связаны    
\choice зависит от величины проекции
\end{choices}

\question Частица находится в состоянии, в котором квадрат орбитального момента имеет опреде-ленное значение. Будет ли это состояние стационарным?
\begin{choices}
\choice да                   
\choice нет
\choice эти утверждения никак не связаны    
\choice зависит от величины квадрата момента
\end{choices}

\question Пусть ${f_{{l^2},{l_z}}}$ - собственная функция операторов ${\hat L^2}$ и ${\hat L_z}$, отвечающая собственным значениям ${l^2}$ и ${l_z}$. Какая функция получится при действии на нее оператора ${\hat L_ + }$? 
\begin{choices}
\choice ${\hat L_ + }{f_{{l^2},{l_z}}} \sim {f_{{l^2} + {\hbar ^2},{l_z}}}$ или тождественно равная ну-лю функция
\choice ${\hat L_ + }{f_{{l^2},{l_z}}} \sim {f_{{l^2} + {\hbar ^2},{l_z} + \hbar }}$ или тождественно равная нулю функция
\choice ${\hat L_ + }{f_{{l^2},{l_z}}} \sim {f_{{l^2},{l_z} + \hbar }}$ или тождественно равная нулю функция
\choice ${\hat L_ + }{f_{{l^2},{l_z}}} \sim {f_{{l^2},{l_z} + \hbar }}$ или тождественно равная нулю функция
\end{choices}

\question Пусть $f$ - общая собственная функция операторов ${\hat L^2}$ и ${\hat L_z}$. Будет ли она собственной для оператора ${\hat L_ + }{\hat L_ - }$ и если да, то какому собственному зна-чению будет отвечать?
\begin{choices}
\choice да, ${l^2} - l_z^2$           
\choice нет
\choice да, ${l^2} - l_z^2 - \hbar {l_z}$         
\choice да, ${l^2} - l_z^2 + \hbar {l_z}$
\end{choices}

\question Пусть ${f_{{l^2},{l_z}}}$ - общая собственная функция операторов ${\hat L^2}$ и ${\hat L_z}$, отвечающая соответствующим собственным значениям. Будет ли она собственной для оператора ${\hat L_ - }{\hat L_ + }$, и если да, то какому собственному значению будет отве-чать?
\begin{choices}
\choice да, ${l^2} - l_z^2$           
\choice нет
\choice да, ${l^2} - l_z^2 - \hbar {l_z}$         
\choice да, ${l^2} - l_z^2 + \hbar {l_z}$
\end{choices}

\question Операторы квадрата орбитального момента ${\hat L^2}$ и четности $\hat P$ 
\begin{choices}
\choice коммутируют
\choice не коммутируют
\choice в некоторых случаях коммутируют, в некоторых – нет
\choice это зависит от оператора Гамильтона
(оператор четности следующим образом действует в пространстве функций трех переменных: $\hat P\psi (x,y,z) = \psi ( - x, - y, - z)$)
\end{choices}

\question Операторы проекции орбитального момента на ось${\hat L_i}$ и четности $\hat P$ 
\begin{choices}
\choice коммутируют
\choice не коммутируют
\choice в некоторых случаях коммутируют, в некоторых – нет
\choice это зависит от оператора Гамильтона
\end{choices}

\question Операторы, повышающий и понижающий проекцию момента импульса частицы на ось ${\hat L_ \pm }$, и четности $\hat P$ 
\begin{choices}
\choice коммутируют
\choice не коммутируют
\choice в некоторых случаях коммутируют, в некоторых – нет
\choice это зависит от оператора Гамильтона
\end{choices}

\question Будут ли общие собственные функции операторов ${\hat L^2}$ и ${\hat L_z}$ обладать определенной четностью, и если да, то какой? 
\begin{choices}
\choice да, четные или нечетные в зависимости от собственного значения оператора ${\hat L^2}$  
\choice да, четные или нечетные в зависимости от собственного значения оператора ${\hat L_z}$
\choice нет
\choice это зависит от оператора Гамильтона
\end{choices}

\question Будут ли собственные функции оператора ${\hat L^2}$ обладать определенной четно-стью, и если да, то какой? 
\begin{choices}
\choice да, четные
\choice да, нечетные
\choice да, четные или нечетные в зависимости от собственного значения оператора ${\hat L^2}$
\choice вообще говоря, нет
\end{choices}

\question Будут ли собственные функции оператора ${\hat L_z}$ обладать определенной четно-стью, и если да, то какой? 
\begin{choices}
\choice да, четные
\choice да, нечетные
\choice да, четные или нечетные в зависимости от собственного значения оператора ${\hat L_z}$
\choice вообще говоря, нет 
\end{choices}

\end{questions}





\subsection{ Собственные значения и собственные функции операторов момента импульса }

\begin{questions}

\question Какие из нижеследующих чисел исчерпывают все собственные значения оператора квад-рата орбитального момента импульса ($l$ - целое неотрицательное число)?
\begin{choices}
\choice ${\hbar ^2}{l^2}$ 
\choice ${\hbar ^2}l(l + 1)$    
\choice ${\hbar ^2}l(l + 2)$    
\choice ${\hbar ^2}l(l - 1)$
\end{choices}

\question Собственными значениями оператора квадрата орбитального момента импульса являются числа вида ${\hbar ^2}l(l + 1)$, где $l$
\begin{choices}
\choice полуцелое неотрицательное число (число вида 1/2, 3/2, …)
\choice целое неотрицательное число
\choice целое неположительное число
\choice полуцелое неположительное число
\end{choices}

\question В некотором состоянии квадрат орбитального момента импульса частицы принимает оп-ределенное значение: ${L^2} = 12{\hbar ^2}$. Какую из перечисленных величин называют мо-ментом этого состояния?
\begin{choices}
\choice $3$      
\choice $4$      
\choice $\sqrt {12} $  
\choice таким квадрат момента быть не может
\end{choices}

\question В некотором состоянии квадрат орбитального момента импульса частицы принимает оп-ределенное значение: ${L^2} = 16{\hbar ^2}$. Какую из перечисленных величин называют мо-ментом этого состояния?
\begin{choices}
\choice $ - 4$      
\choice $4$      
\choice $16$     
\choice таким квадрат момента быть не может
\end{choices}

\question Какие из нижеследующих чисел исчерпывают все собственные значения оператора проекции орбитального момента импульса на ось $z$?
\begin{choices}
\choice $\hbar m$ ($m$ - любое целое или полуцелое число)
\choice $\hbar m$ ($m$ - любое целое положительное число)
\choice $\hbar m$ ($m$ - любое целое число)
\choice $\hbar m$ ($m$ - любое целое отрицательное число)
\end{choices}

\question Какие из нижеследующих чисел исчерпывают все собственные значения оператора проекции орбитального момента импульса на ось $x$?
\begin{choices}
\choice $\hbar m$ ($m$ - любое целое или полуцелое число)
\choice проекция момента на ось $x$ определенного значения принимать не может
\choice $\hbar m$ ($m$ - любое целое положительное число)
\choice $\hbar m$ ($m$ - любое целое число)
\end{choices}

\question Частица находится в состоянии с определенной проекцией орбитального момента на ось $y$: ${L_y} = 4\hbar $. Измеряют квадрат момента. Какое из перечисленных значений не могло быть получено?
\begin{choices}
\choice $24{\hbar ^2}$    
\choice $30{\hbar ^2}$    
\choice $42{\hbar ^2}$    
\choice $20{\hbar ^2}$
\end{choices}

\question Частица находится в состоянии с определенной проекцией орбитального момента импульса на ось $y$: ${L_y} = 4\hbar $. Измеряют квадрат момента. Какое из перечисленных значений не могло быть получено?
\begin{choices}
\choice $49{\hbar ^2}$    
\choice $12{\hbar ^2}$    
\choice $25{\hbar ^2}$    
\choice $72{\hbar ^2}$
\end{choices}

\question Частица находится в состоянии с определенным квадратом орбитального момента: ${L^2} = 30{\hbar ^2}$. Измеряют проекцию момента на ось $x$. Какое из перечисленных значений не могло быть получено?
\begin{choices}
\choice $ - 5\hbar $      
\choice $ - 4\hbar $         
\choice $6\hbar $         
\choice $3\hbar $
\end{choices}

\question Какие из перечисленных функций являются собственными для оператора ${\hat L_z}$ ($m$ - целое число)?
\begin{choices}
\choice $\sin m\varphi $     
\choice $\exp (m\varphi )$      
\choice $\exp \left( {im\varphi } \right)$     
\choice ${\rm{cos}}\,m\varphi $
\end{choices}

\question Какие из нижеследующих функций являются собственными для оператора ${\hat L_z}$ ($m$ - целое число)?
\begin{choices}
\choice $\sin m\vartheta $      
\choice $\sin m\varphi $     
\choice $\sin \left( {m\vartheta  + m\varphi } \right)$ 
\choice никакие из них
\end{choices}

\question Какие из нижеследующих функций являются собственными для оператора ${\hat L_y}$?
\begin{choices}
\choice $f(x,y,z) = x$ 
\choice $f(x,y,z) = y$ 
\choice $f(x,y,z) = z$ 
\choice никакие из перечисленных
\end{choices}

\question Какие из нижеследующих функций являются собственными для оператора ${\hat L_y}$?
\begin{choices}
\choice $xy$        
\choice $yz$        
\choice $zx$        
\choice никакие из перечисленных
\end{choices}

\question Частица находится в состоянии с одной из нижеследующих волновых функций:
\begin{choices}
\choice $\sin \varphi $      
\choice $\exp ( - i\varphi )$      
\choice $\exp ( - i\vartheta )$    
\choice $\sin \vartheta $
В каком из этих состояний проекция орбитального момента на ось $z$ не имеет определенного значения?
\end{choices}

\question Частица находится в состоянии с волновой функцией $\sin m\varphi $, где $m$ - некото-рое целое число. Измеряют проекцию орбитального момента импульса частицы на ось $z$. Ка-кие значения можно получить и с какими вероятностями?
\begin{choices}
\choice единственное значение ${L_z} = \hbar m$         
\choice ${L_z} = \hbar m$ и ${L_z} =  - \hbar m$ с вероятностями 1/2
\choice единственное значение ${L_z} =  - \hbar m$      
\choice единственное значение ${L_z} = 0$ 
\end{choices}

\question Частица находится в состоянии с волновой функцией $1 + \cos \varphi $, где $m$ - неко-торое целое число. Измеряют проекцию орбитального момента импульса частицы на ось $z$. Какие значения можно получить и с какими вероятностями?
\begin{choices}
\choice единственное значение ${L_z} = \hbar $
\choice ${L_z} = \hbar $ и ${L_z} =  - \hbar $ с вероятностями 1/2
\choice  с вероятностями 1/2, 1/4 и 1/4
Г.с вероятностями 1/3 
\end{choices}

\question Частица находится в состоянии с волновой функцией ${\cos ^2}\varphi $. Измеряют про-екцию момента на ось $z$. Какие значения проекции орбитального момента на ось $z$ могут быть получены при измерениях в этом состоянии и с какими вероятностями?
\begin{choices}
\choice ${L_z} = \hbar ,\quad 2\hbar ,\quad 3\hbar $ с вероятностями 1/4, 1/4 и 1/2
\choice ${L_z} =  - \hbar ,\quad 0,\quad \hbar $ с вероятностями 1/4, 1/2 и 1/4
\choice ${L_z} = 0,\quad \hbar ,\quad 2\hbar $ с вероятностями 1/3
\choice ${L_z} =  - 2\hbar ,\quad 0,\quad 2\hbar $ с вероятностями 1/6, 2/3 и 1/6
\end{choices}

\question Частица находится в состоянии с волновой функцией $\sin m\vartheta $, где $m$ - неко-торое целое число. Измеряют проекцию орбитального момента импульса частицы на ось $z$. Какие значения можно получить и с какими вероятностями?
\begin{choices}
\choice единственное значение ${L_z} = \hbar m$
\choice ${L_z} = \hbar m$ и ${L_z} =  - \hbar m$ с вероятностями 1/2
\choice единственное значение ${L_z} =  - \hbar m$
\choice единственное значение ${L_z} = 0$ 
\end{choices}

\question Какая из нижеследующих функций является общей собственной функцией операторов ${\hat L_z}$ и ${\hat p_y}$ (здесь $k$ - некоторое действительное число)?
\begin{choices}
\choice ${e^{i\varphi }}{e^{iky}}$    
\choice ${e^{i\varphi }}{e^{ - iky}}$    
\choice ${e^{ - i\varphi }}{e^{ - iky}}$    
\choice такой функции не существует
\end{choices}

\question Какая из нижеследующих функций является общей собственной функцией операторов ${\hat p_z}$ и ${\hat L_z}$ (здесь $k$ - некоторое действительное число)?
\begin{choices}
\choice ${e^{ - i\varphi }}{e^{ - ikz}}$    
\choice ${e^{ - i\varphi }}{e^{ - ky}}$     
\choice ${e^{ - \varphi }}{e^{ - iky}}$     
\choice такой функции не существует
\end{choices}

\question Какая из нижеследующих функций является собственной функцией оператора ${\hat L_z}$?
\begin{choices}
\choice $x + iy$    
\choice $x + 2iy$
\choice $x + 3iy$      
\choice в декартовых координатах записать эту функцию нельзя
\end{choices}

\question Сферические функции – это
\begin{choices}
\choice общие собственные функции операторов квадрата момента и его проекции на ось $z$
\choice общие собственные функции операторов квадрата момента и его проекции на ось $x$
\choice общие собственные функции операторов квадрата момента и его проекции на ось $y$
\choice все перечисленные
\end{choices}

\question Существует ли среди сферических функций такая функция (или функции), которая не зависит от полярного и азимутального углов $\vartheta $ и $\varphi $?
\begin{choices}
\choice да
\choice нет
\choice существуют функции не зависящие от $\vartheta $, но зависящие $\varphi $
\choice все сферические функции не зависят от углов, так как они обладают сферической сим-метрией
\end{choices}

\question Частица находится в состоянии с волновой функцией $\sin \vartheta \cos \varphi $. Будут ли орбитальный момент и его проекция на ось $z$ иметь определенные значения в этом состоя-нии?
\begin{choices}
\choice момент – нет, проекция – да         
\choice момент – да, проекция – нет 
\choice и момент, и проекция          
\choice ни момент, ни проекция.
\end{choices}

\question Частица находится в состоянии с волновой функцией $\cos \vartheta \sin \varphi $. Будут ли орбитальный момент и его проекция на ось $z$ иметь определенные значения в этом состоя-нии?
\begin{choices}
\choice момент – нет, проекция – да         
\choice момент – да, проекция – нет 
\choice и момент, и проекция          
\choice ни момент, ни проекция.
\end{choices}

\question Частица находится в состоянии с волновой функцией ${Y_{54}}(\vartheta ,\varphi )$. Ка-кие значения квадрата орбитального момента могут быть получены при измерениях?
\begin{choices}
\choice $0,\quad 2{\hbar ^2},\quad 6{\hbar ^2},\quad 12{\hbar ^2},\quad 20{\hbar ^2},\quad 30{\hbar ^2}$      Б.$20{\hbar ^2}$
\choice $30{\hbar ^2}$                
\choice $25{\hbar ^2}$
\end{choices}

\question Частица находится в состоянии с волновой функцией ${Y_{54}}(\vartheta ,\varphi )$. Измеряют проекцию орбитального момента на ось $z$. Какие значения могут быть получены при измерениях?
\begin{choices}
\choice $m = 5$        
\choice любое целое число из интервала $ - 4 \le m \le 4$
\choice $m = 4$        
\choice любое целое число из интервала $ - 5 \le m \le 5$
\end{choices}

\question Частица находится в состоянии с волновой функцией ${Y_{lm}}(\vartheta ,\varphi )$. Чему равно среднее значение проекции момента на ось $x$ в этом состоянии?
\begin{choices}
\choice $\overline {{L_x}^2}  = l$    
\choice $\overline {{L_x}}  = m$      
\choice $\overline {{L_x}}  = 0$      
\choice $\overline {{L_x}}  = l - m$
\end{choices}

\question Частица находится в состоянии с волновой функцией ${Y_{lm}}(\vartheta ,\varphi )$. Чему равно среднее значение квадрата проекции момента на ось $x$ в этом состоянии?
\begin{choices}
\choice $\overline {{L_x}^2}  = {l^2}$      
\choice $\overline {{L_x}^2}  = {m^2}$      
\choice $\overline {{L_x}^2}  = 0$    
\choice $\overline {{L_x}^2}  = \frac{{l(l + 1) - {m^2}}}{2}$
\end{choices}

\question Частица находится в состоянии с волновой функцией ${Y_{54}}(\vartheta ,\varphi )$. Измеряют проекцию орбитального момента на ось $x$. Какие значения могут быть получены при измерениях?
\begin{choices}
\choice ${l_x} = 5$       
\choice любое целое число из интервала $ - 4 \le {l_x} \le 4$
\choice ${l_x} = 4$       
\choice любое целое число из интервала $ - 5 \le {l_x} \le 5$
\end{choices}

\question Частица находится в состоянии с волновой функцией ${Y_{54}}(\vartheta ,\varphi )$. Ка-кие значения орбитального момента и его проекции на ось $z$могут быть получены при изме-рениях?
\begin{choices}
\choice $l = 4$, $m = 5$  
\choice $l = 5$, $m = 4$  
\choice $l = 4$, $m =  - 5$  
\choice никакие из перечисленных
\end{choices}

\question Частица находится в состоянии с нормированной волновой функцией ${C_1}{Y_{54}}(\vartheta ,\varphi ) + {C_2}{Y_{53}}(\vartheta ,\varphi )$, где ${C_1}$ и ${C_2}$ - числа. Какие значения орбитального момента можно обнаружить при измерениях и с какими вероятностями?
\begin{choices}
\choice $l = 5$ с единичной вероятностью
\choice $l = 4$ с вероятностью $|{C_1}{|^2}$ и $l = 3$ с вероятностью $|{C_2}{|^2}$ 
\choice $l = 5$ с вероятностью $|{C_1}{|^2} + |{C_2}{|^2}$, $l = 3$ с вероятностью $|{C_2}{|^2}$ и $l = 4$ с вероятностью $|{C_1}{|^2}$
\choice $l = 3$ и $l = 4$ с одинаковыми вероятностями
\end{choices}

\question Частица находится в состоянии с нормированной волновой функцией ${C_1}{Y_{54}}(\vartheta ,\varphi ) + {C_2}{Y_{53}}(\vartheta ,\varphi )$, где ${C_1}$ и ${C_2}$ - числа. Какие значения проекции орбитального момента на ось $z$ можно обнаружить в этом состоянии и с какими вероятностями?
\begin{choices}
\choice $m = 5$ с единичной вероятностью
\choice $m = 4$ с вероятностью $|{C_1}{|^2}$ и $m = 3$ с вероятностью $|{C_2}{|^2}$ 
\choice $m = 5$ с вероятностью $|{C_1}{|^2} + |{C_2}{|^2}$, $m = 3$ с вероятностью $|{C_2}{|^2}$, $m = 4$ с вероятностью $|{C_1}{|^2}$
\choice $m = 3$ и $m = 4$ с одинаковыми вероятностями
\end{choices}

\question Частица находится в состоянии с волновой функцией ${\cos ^2}\vartheta  - 3\cos \vartheta  + 1$. Измеряют проекцию орбитального момента на ось $z$. Какие значения можно при этом получить?
\begin{choices}
\choice определенное значение $m = 0$
\choice определенное значение $m = 2$
\choice любое значение из $m = 0,\quad 1,\quad 2$
\choice любое целое значение из интервала $ - 2 \le m \le 2$
\end{choices}

\question Частица находится в состоянии с волновой функцией ${\cos ^2}\vartheta  - 3\cos \vartheta  + 1$. Какие значения орбитального момента можно получить при измерениях в этом состоя-нии?
\begin{choices}
\choice одно из чисел $0,\quad 1,\quad 2$
\choice одно из чисел $0,\quad 2$
\choice одно из чисел $ - 2,\quad 0,\quad 2$
\choice одно из чисел $0,\quad 1,\quad 2,\quad 3$
\end{choices}

\question Частица находится в состоянии с нормированной волновой функцией ${C_1}{Y_{63}}(\vartheta ,\varphi ) + {C_2}{Y_{53}}(\vartheta ,\varphi )$, где ${C_1}$ и ${C_2}$ - числа. Какие значения орбитального момента можно получить при измерениях в этом состоянии и с какими вероятностями?
\begin{choices}
\choice $l = 3$ с единичной вероятностью
\choice $l = 6$ с вероятностью $|{C_1}{|^2}$ и $l = 5$ с вероятностью $|{C_2}{|^2}$
\choice $l = 3$ с вероятностью $|{C_1}{|^2} + |{C_2}{|^2}$, $l = 6$ с вероятностью $|{C_1}{|^2}$, $l = 5$ с вероятностью $|{C_2}{|^2}$
\choice $l = 5$ и $l = 6$ с одинаковыми вероятностями
\end{choices}

\question Частица находится в состоянии с нормированной волновой функцией ${C_1}{Y_{63}}(\vartheta ,\varphi ) + {C_2}{Y_{53}}(\vartheta ,\varphi )$, где ${C_1}$ и ${C_2}$ - числа. Какие значения проекции орбитального момента на ось $z$ можно получить при измерениях в этом состоянии и с какими вероятностями?
\begin{choices}
\choice $m = 3$ с единичной вероятностью
\choice $m = 6$ с вероятностью $|{C_1}{|^2}$ и $m = 5$ с вероятностью $|{C_2}{|^2}$
\choice $m = 3$ с вероятностью $|{C_1}{|^2} + |{C_2}{|^2}$, $m = 6$ с вероятностью $|{C_1}{|^2}$ и $m = 5$ с вероятностью $|{C_2}{|^2}$
\choice $m = 5$ и $m = 6$ с одинаковыми вероятностями
\end{choices}

\question Частица находится в состоянии, в котором ее орбитальный момент импульса и его про-екция на ось $z$ имеют определенные значения: $l = 1$, $m =  - 1$. Сравнить вероятности раз-личных значений проекции момента на ось $y$ в этом состоянии: $w({l_y} = 1)$ и $w({l_y} =  - 1)$
\begin{choices}
\choice $w({l_y} = 1) > w({l_y} =  - 1)$
\choice $w({l_y} = 1) < w({l_y} =  - 1)$
\choice $w({l_y} = 1) = w({l_y} =  - 1)$
\choice такие значения проекции момента на ось $y$ в данном состоянии не могут быть обнаружены
\end{choices}

\question Частица находится в состоянии, в котором ее орбитальный момент импульса и его про-екция на ось $z$ имеют определенные значения: $l = 1$, $m =  - 1$. Сравнить вероятности раз-личных значений проекции момента на ось $y$ в этом состоянии: $w({l_y} = 2)$ и $w({l_y} =  - 2)$
\begin{choices}
\choice $w({l_y} = 2) > w({l_y} =  - 2)$
\choice $w({l_y} = 2) < w({l_y} =  - 2)$
\choice $w({l_y} = 2) = w({l_y} =  - 2)$
\choice такие значения проекции момента на ось $y$ в данном состоянии не могут быть обнаружены
\end{choices}

\question Частица находится в состоянии, в котором проекция ее момента импульса на ось $z$ имеет определенное значение ${l_z} = m$. Измеряют проекцию момента на ось $y$. Какие зна-чения можно обнаружить?
\begin{choices}
\choice ${l_y} = m$
\choice любое целое число из интервала $ - m \le {l_y} \le m$
\choice любое собственное значение оператора ${\hat L_y}$
\choice это зависит от состояния частицы
\end{choices}

\question Как сферическая функция ${Y_{lm}}(\vartheta ,\varphi )$ зависит от азимутального угла $\varphi $?
\begin{choices}
\choice как $\sin m\varphi $       
\choice как $\cos m\varphi $       
\choice как ${e^{im\varphi }}$     
\choice как ${e^{ - im\varphi }}$
\end{choices}

\question Как сферическая функция ${Y_{lm}}(\vartheta ,\varphi )$ зависит от азимутального угла $\varphi $?
\begin{choices}
\choice как ${P_l}^{|m|}(\cos \varphi )$       
\choice как ${P_m}^{|l|}(\cos \varphi )$
\choice как ${P_l}^{|m|}(\sin \varphi )$       
\choice по-другому
\end{choices}

\question Частица находится в состоянии с волновой функцией ${Y_{21}}(\vartheta ,\varphi )$. Проводят измерения проекции орбитального момента импульса на ось $y$. Какие значения можно получить?
\begin{choices}
\choice определенное значение ${l_y} = 2$      
\choice ${l_y} =  - 1,\;0,\;1$
\choice ${l_y} =  - 2,\; - 1,\;0,\;1,\;2$            
\choice определенное значение ${l_y} = 1$
\end{choices}

\question Частица находится в состоянии с определенными значениями орбитального момента $l = 3$ и его проекции на ось $x$ ${l_x} = 1$. Чему равно среднее значение проекции орбитального момента на ось $z$?
\begin{choices}
\choice $\overline {{l_z}}  = 1$      
\choice $\overline {{l_z}}  = 3$      
\choice $\overline {{l_z}}  = 3/2$    
\choice $\overline {{l_z}}  = 0$
\end{choices}

\question Частица находится в состоянии с волновой функцией $\psi  \sim {Y_{54}} + 2{Y_{53}}$. Какие из нижеследующих величин имеют в этом состоянии определенные значения?
\begin{choices}
\choice проекция орбитального момента на ось $z$
\choice проекция орбитального момента на ось $y$
\choice проекция орбитального момента на ось $x$
\choice квадрат орбитального момента 
\end{choices}

\question Частица находится в состоянии с волновой функцией $\psi  \sim {Y_{54}} + 2{Y_{64}}$. Какие из нижеследующих величин имеют в этом состоянии определенные значения?
\begin{choices}
\choice проекция орбитального момента на ось $z$
\choice проекция орбитального момента на ось $y$
\choice проекция орбитального момента на ось $x$
\choice квадрат орбитального момента 
\end{choices}

\question Частица находится в состоянии, в котором ее момент имеет определенное значение $l = 3$, а проекция с вероятностью $1/4$ принимает значение $m = 1$, а с вероятностью $3/4$ - зна-чение $m = 2$. Какой формулой может описываться волновая функция частицы?
\begin{choices}
\choice $\psi  = \frac{1}{4}{Y_{31}} - \frac{{3i}}{4}{Y_{32}}$         
\choice $\psi  = \frac{1}{2}{Y_{31}} - \frac{{i\sqrt 3 }}{2}{Y_{21}}$
\choice $\psi  = \frac{i}{2}{Y_{31}} + \frac{{\sqrt 3 }}{2}{Y_{32}}$         
\choice $\psi  = \frac{1}{4}{Y_{31}} - \frac{3}{4}{Y_{21}}$
\end{choices}

\question Частица находится в состоянии с волновой функцией $\psi  \sim 2{Y_{64}} + {Y_{54}}$. Какие значения проекции момента на ось $y$ можно получить при измерениях в этом состоя-нии?
\begin{choices}
\choice $ - 5 \le {l_y} \le 5$     
\choice $ - 6 \le {l_y} \le 6$     
\choice $ - 4 \le {l_y} \le 4$     
\choice ${l_y} = 4$
\end{choices}

\question Частица находится в состоянии с нормированной волновой функцией $\psi  = \frac{1}{2}{Y_{64}} - \frac{{\sqrt 3 i}}{2}{Y_{42}}$. Найти $\overline {{L_z}} $ в этом состоянии?
\begin{choices}
\choice $\overline {{L_z}}  = 0$      
\choice $\overline {{L_z}}  = 5/2$    
\choice $\overline {{L_z}}  = 3$      
\choice $\overline {{L_z}}  = 7/2$
\end{choices}

\question Частица находится в состоянии с нормированной волновой функцией $\psi  = \frac{1}{2}{Y_{64}} - \frac{{\sqrt 3 i}}{2}{Y_{42}}$. Найти $\overline {{L_z}^2} $ в этом состоя-нии?
\begin{choices}
\choice $\overline {{L_z}^2}  = 0$    
\choice $\overline {{L_z}^2}  = 6$    
\choice $\overline {{L_z}^2}  = 7$    
\choice $\overline {{L_z}^2}  = 8$
\end{choices}

\question Частица находится в состоянии с нормированной волновой функцией $\psi  = \frac{1}{2}{Y_{64}} - \frac{{\sqrt 3 i}}{2}{Y_{42}}$. Найти $\overline {{L_x}} $ в этом состоянии?
\begin{choices}
\choice $\overline {{L_x}}  = 0$      
\choice $\overline {{L_x}}  = 5/2$    
\choice $\overline {{L_x}}  = 3$      
\choice $\overline {{L_x}}  = 7/2$
\end{choices}

\question Частица находится в состоянии с нормированной волновой функцией $\psi  = \frac{1}{2}{Y_{64}} - \frac{{\sqrt 3 i}}{2}{Y_{42}}$. Найти $\overline {{L^2}} $ в этом состоянии?
\begin{choices}
\choice $\overline {{L^2}}  = 0$      
\choice $\overline {{L^2}}  = 51/2$      
\choice $\overline {{L^2}}  = 53/2$      
\choice $\overline {{L^2}}  = 55/2$
\end{choices}

\question Какая функция получится в результате действия оператора ${\hat L_ + }$ на функцию ${Y_{55}}(\vartheta ,\varphi )$ (с точностью до множителя):
\begin{choices}
\choice ${\hat L_ + }{Y_{55}}(\vartheta ,\varphi ) \sim {Y_{56}}(\vartheta ,\varphi )$         
\choice ${\hat L_ + }{Y_{55}}(\vartheta ,\varphi ) \equiv 0$
\choice ${\hat L_ + }{Y_{55}}(\vartheta ,\varphi ) \sim {Y_{65}}(\vartheta ,\varphi )$         
\choice ${\hat L_ + }{Y_{55}}(\vartheta ,\varphi ) \sim {Y_{66}}(\vartheta ,\varphi )$
\end{choices}

\question Какое из нижеследующих равенств является верным (с точностью до множителя)?
\begin{choices}
\choice ${\hat L_ + }{Y_{52}}(\vartheta ,\varphi ) \sim {Y_{62}}(\vartheta ,\varphi )$         
\choice ${\hat L_ + }{Y_{52}}(\vartheta ,\varphi ) \equiv 0$
\choice ${\hat L_ + }{Y_{52}}(\vartheta ,\varphi ) \sim {Y_{53}}(\vartheta ,\varphi )$         
\choice ${\hat L_ + }{Y_{52}}(\vartheta ,\varphi ) \sim {Y_{51}}(\vartheta ,\varphi )$
\end{choices}

\question Какое из нижеследующих утверждений правильно описывает «структуру» сферических функций ${Y_{lm}}(\vartheta ,\varphi )$ ($m$ - целое число)?
\begin{choices}
\choice некоторая функция от $\varphi $, умноженная на $\exp (im\vartheta )$
\choice некоторая функция от $\varphi $, умноженная на $\cos m\vartheta $
\choice некоторая функция от $\vartheta $, умноженная на $\exp (im\varphi )$
\choice некоторая функция от $\vartheta $, умноженная на $\cos m\varphi $ 
\end{choices}

\question Какова «структура» сферических функций ${Y_{lm}}(\vartheta ,\varphi )$ (для произ-вольных $l$ и $m$)?
\begin{choices}
\choice некоторый многочлен от $\cos \vartheta $, умноженный $\exp (im\varphi )$
\choice некоторый многочлен от $\sin \vartheta $, умноженный на $\exp (im\varphi )$
\choice некоторый многочлен от $\cos \vartheta $, умноженный $\exp (im\varphi )$ и (для нечетных $m$) на $\sin \vartheta $
\choice некоторый многочлен от $\cos \vartheta $, умноженный $\exp (im\varphi )$ и (для четных $m$) на $\sin \vartheta $
\end{choices}

\question Какой формулой определяется сферическая функция ${Y_{12}}(\vartheta ,\varphi )$ (с точностью до множителя)?
\begin{choices}
\choice $\cos \vartheta \;{e^{2i\varphi }}$    
\choice $\cos \vartheta \;{e^{ - 2i\varphi }}$    
\choice $\sin \vartheta \;{e^{2i\varphi }}$    
\choice такой функции не существует
\end{choices}

\question От каких переменных зависит функция ${Y_{l0}}$?
\begin{choices}
\choice только от $\varphi $    
\choice только от $\vartheta $
\choice только от $r$     
\choice от $\varphi $ и от $\vartheta $
\end{choices}

\question Какая из нижеследующих функций не является сферической (с точностью до множите-ля)?
\begin{choices}
\choice $f(\vartheta ,\varphi ) = 1$     
\choice $f(\vartheta ,\varphi ) = \cos \vartheta $   
\choice $f(\vartheta ,\varphi ) = \sin \vartheta {e^{i\varphi }}$      
\choice $f(\vartheta ,\varphi ) = \cos \vartheta {e^{ - i\varphi }}$
\end{choices}

\question Какая из нижеследующих функций является сферической (с точностью до множителя)
\begin{choices}
\choice $f(\vartheta ,\varphi ) = \sin \vartheta \sin \varphi $     
\choice $f(\vartheta ,\varphi ) = \cos \vartheta {e^{ - i\varphi }}$
\choice $f(\vartheta ,\varphi ) = \sin \vartheta {e^{i\varphi }}$         
\choice $f(\vartheta ,\varphi ) = \cos \vartheta \sin \varphi $
\end{choices}

\question Какая из нижеследующих функций является сферической (с точностью до множителя)?
\begin{choices}
\choice $\frac{z}{{\sqrt {{x^2} + {y^2} + {z^2}} }}$    
\choice $\frac{x}{{\sqrt {{x^2} + {y^2} + {z^2}} }}$    
\choice $\frac{y}{{\sqrt {{x^2} + {y^2} + {z^2}} }}$
\choice в декартовых координатах записать эту функцию нельзя
\end{choices}

\question Частица находится в состоянии с волновой функцией $\sin \vartheta \sin \varphi $. Какие значения орбитального момента и его проекции на ось $z$ можно обнаружить при измерениях?
\begin{choices}
\choice только $l = 1$ и $m = 1$
\choice только $l = 1$, для проекции два значения $m = 1$ и $m =  - 1$
\choice для момента два значения $l = 0$ и $l = 1$, для проекции – только $m = 1$
\choice для момента два значения $l = 0$ и $l = 1$ и для проекции два значения $m = 1$ и $m =  - 1$
\end{choices}

\question Какие функции от $\cos \vartheta $ входят в сферические функции?
\begin{choices}
\choice $\Gamma $-функция             
\choice присоединенные полиномы Лагерра
\choice присоединенные полиномы Лежандра       
\choice присоединенные полиномы Эрмита
\end{choices}

\question К каким функциям сводятся присоединенные полиномы Лежандра ${P_l}^{|m|}(x)$ при $m = 0$?
\begin{choices}
\choice к полиномам Лежандра       
\choice к полиномам Лагерра
\choice к полиномам Эрмита         
\choice ни к каким с самостоятельным именем соб-ственным
\end{choices}

\question К каким функциям сводятся присоединенные полиномы Лежандра ${P_l}^{|m|}(x)$ при $m = 1$?
\begin{choices}
\choice к полиномам Лежандра       
\choice к полиномам Лагерра
\choice к полиномам Эрмита         
\choice ни к каким с самостоятельным именем соб-ственным
\end{choices}

\question Входящие в сферические функции присоединенные полиномы Лежандра ${P_l}^{|m|}$являются 
\begin{choices}
\choice некоторыми многочленами от $\vartheta $
\choice некоторыми многочленами от $\cos \vartheta $
\choice некоторыми многочленами от $\cos \vartheta $ для четных $m$, и некоторыми многочленами от $\cos \vartheta $, умноженными на $\sin \vartheta $ для нечетных $m$
\choice некоторыми многочленами от $\cos \vartheta $ для четных $l$, и некоторыми многочленами от $\cos \vartheta $, умноженными на $\sin \vartheta $ для нечетных $l$
\end{choices}

\question Какая формула правильно описывает условие ортогональности сферических функций?
\begin{choices}
\choice $\int\limits_0^\pi  {d\vartheta \int\limits_0^{2\pi } {d\varphi \sin \varphi {Y_{lm}}^*(\vartheta ,\varphi )} } {Y_{\lambda \mu }}(\vartheta ,\varphi ) = {\delta _{l\lambda }}{\delta _{m\mu }}$  
\choice $\int\limits_0^\pi  {d\vartheta \sin \vartheta \int\limits_0^{2\pi } {d\varphi {Y_{lm}}^*(\vartheta ,\varphi )} } {Y_{\lambda \mu }}(\vartheta ,\varphi ) = {\delta _{l\lambda }}{\delta _{m\mu }}$
\choice $\int\limits_0^\pi  {d\vartheta \int\limits_0^{2\pi } {d\varphi \cos \varphi {Y_{lm}}^*(\vartheta ,\varphi )} } {Y_{\lambda \mu }}(\vartheta ,\varphi ) = {\delta _{l\lambda }}{\delta _{m\mu }}$  
\choice $\int\limits_0^\pi  {d\vartheta \cos \vartheta \int\limits_0^{2\pi } {d\varphi {Y^*}_{lm}(\vartheta ,\varphi )} } {Y_{\lambda \mu }}(\vartheta ,\varphi ) = {\delta _{l\lambda }}{\delta _{m\mu }}$
\end{choices}

\question Вычислить $\left( {{Y_{lm}}(\vartheta ,\varphi ),{{\hat L}_ + }{Y_{lm}}(\vartheta ,\varphi )} \right)$, где $\left( {...\;,\;...} \right)$ - интеграл по полному телесному углу.
\begin{choices}
\choice $l + 1$     
\choice $m + 1$     
\choice $l + m$     
\choice 0
\end{choices}

\question Вычислить $\left( {{Y_{lm}}(\vartheta ,\varphi ),\hat L_ + ^2{Y_{lm}}(\vartheta ,\varphi )} \right)$, где $\left( {...\;,\;...} \right)$ - интеграл по полному телесному углу.
\begin{choices}
\choice ${\left( {l + 1} \right)^2}$        
\choice ${\left( {m + 1} \right)^2}$     
\choice ${\left( {l + m} \right)^2}$     
\choice 0
\end{choices}

\question Какое из нижеперечисленных выражений не равно нулю ($\left( {...\;,\;...} \right)$ - инте-грал по полному телесному углу)?
\begin{choices}
\choice $\left( {\left( {{Y_{51}} + {Y_{52}}} \right),{{\hat L}_ + }\left( {{Y_{53}} + {Y_{54}}} \right)} \right)$        
\choice $\left( {\left( {{Y_{51}} + {Y_{52}}} \right),{{\hat L}_ + }\left( {{Y_{41}} + {Y_{42}}} \right)} \right)$
\choice $\left( {\left( {{Y_{51}} + {Y_{52}}} \right),{{\hat L}_ + }\left( {{Y_{5 - 1}} + {Y_{50}}} \right)} \right)$        
\choice $\left( {\left( {{Y_{51}} + {Y_{52}}} \right),{{\hat L}_ + }\left( {{Y_{61}} + {Y_{62}}} \right)} \right)$
\end{choices}

\question Какой формулой определяется условие ортогональности полиномов Лежандра (${\delta _{nm}}$ - дельта-символ Кронекера)?
\begin{choices}
\choice $\int\limits_{ - 1}^{ + 1} {{P_n}(x){P_m}(x){x^2}dx \sim {\delta _{nm}}} $       
\choice $\int\limits_{ - 1}^{ + 1} {{P_n}(x){P_m}(x)dx \sim {\delta _{nm}}} $
\choice $\int\limits_{ - 1}^{ + 1} {{P_n}(x){P_m}(x){x^3}dx \sim {\delta _{nm}}} $       
\choice $\int\limits_{ - 1}^{ + 1} {{P_n}(x){P_m}(x)xdx \sim {\delta _{nm}}} $ 
\end{choices}

\question Какие значения может принимать индекс $l$ в сферической функции ${Y_{l - 5}}(\vartheta ,\varphi )$?
\begin{choices}
\choice только $l = 5$             
\choice целые значения в интервале $ - 5 \le l \le 5$
\choice целые значения в интервале $0 \le l \le 5$      
\choice целые значения в интервале $5 \le l$
\end{choices}

\question Какие значения может принимать индекс $m$ в сферической функции ${Y_{5m}}(\vartheta ,\varphi )$?
\begin{choices}
\choice только $m = 5$             
\choice целые значения в интервале $ - 5 \le m \le 5$
\choice целые значения в интервале $0 \le m \le 5$      
\choice целые значения в интервале $5 \le m$
\end{choices}

\question Какой формулой определяется условие ортогональности присоединенных полиномов Лежандра?
\begin{choices}
\choice $\int\limits_{ - 1}^{ + 1} {{P_{{l_1}}}^{|m|}(x){P_{{l_2}}}^{|m|}(x)dx \sim {\delta _{{l_1}{l_2}}}} $       
\choice $\int\limits_{ - 1}^{ + 1} {{P_l}^{|{m_1}|}(x){P_l}^{|{m_2}|}(x)dx}  \sim {\delta _{{m_1}{m_2}}}$
\choice $\int\limits_{ - 1}^{ + 1} {{P_{{l_1}}}^{|{m_1}|}(x){P_{{l_2}}}^{|{m_2}|}(x)dx}  \sim {\delta _{{l_1}{l_2}}}{\delta _{{m_1}{m_2}}}$    
\choice никакой из них (${\delta _{nm}}$ - дельта-символ Кронекера)
\end{choices}

\question Каким будет результат действия оператора ${\hat L_x}^2 + {\hat L_y}^2$ на сфериче-скую функцию ${Y_{lm}}(\vartheta ,\varphi )$?
\begin{choices}
\choice ${\hbar ^2}\left( {l(l + 1) - {m^2}} \right){Y_{lm}}(\vartheta ,\varphi )$       
\choice ${\hbar ^2}\left( {l(l + 1) - {m^2}} \right){Y_{lm + 1}}(\vartheta ,\varphi )$
\choice ${\hbar ^2}\left( {{l^2} - {m^2}} \right){Y_{lm}}(\vartheta ,\varphi )$       
\choice ${\hbar ^2}\left( {{l^2} - m(m + 1)} \right){Y_{lm + 1}}(\vartheta ,\varphi )$
\end{choices}

\question Каким будет результат действия оператора ${\hat L_ + }{\hat L_ - }$ на сферическую функцию ${Y_{55}}(\vartheta ,\varphi )$ (с точностью до множителя, не равного нулю)?
\begin{choices}
\choice ${Y_{55}}(\vartheta ,\varphi )$     
\choice ${Y_{54}}(\vartheta ,\varphi )$     
\choice ${Y_{45}}(\vartheta ,\varphi )$     
\choice ${\hat L_ + }{\hat L_ - }{Y_{55}} \equiv 0$
\end{choices}

\question Каким будет результат действия оператора ${\hat L_ - }{\hat L_ + }$ на сферическую функцию ${Y_{55}}(\vartheta ,\varphi )$ (с точностью до множителя, не равного нулю)?
\begin{choices}
\choice ${Y_{55}}(\vartheta ,\varphi )$     
\choice ${Y_{54}}(\vartheta ,\varphi )$     
\choice ${Y_{45}}(\vartheta ,\varphi )$     
\choice ${\hat L_ - }{\hat L_ + }{Y_{55}} \equiv 0$
\end{choices}

\question Какое утверждение относительно четности сферических функции ${Y_{lm}}(\vartheta ,\varphi )$ является верным?
\begin{choices}
\choice эта функция не обладает определенной четностью ни при каких $l$ и $m$
\choice эта функция – четная для четных $l$ и нечетная для нечетных $l$ независимо от $m$
\choice эта функция – четная для четных $m$ и нечетная для нечетных $m$ независимо от $l$
\choice эта функция – четная для четных $l + m$ и нечетная для нечетных $l + m$
\end{choices}

\question Какое утверждение относительно четности функции ${Y_{81}}(\vartheta ,\varphi ) + {Y_{82}}(\vartheta ,\varphi )$ является справедливым?
\begin{choices}
\choice четная            
\choice нечетная
\choice неопределенной четности 
\choice четность этой функции зависит от оператора четности
\end{choices}

\question Что можно сказать о четности функции ${Y_{72}}(\vartheta ,\varphi ) + {Y_{82}}(\vartheta ,\varphi )$
\begin{choices}
\choice четная            
\choice нечетная
\choice неопределенной четности 
\choice четность этой функции зависит от оператора четности
\end{choices}

\question Найти среднюю четность состояния $\frac{1}{2}{Y_{81}}(\vartheta ,\varphi ) + \frac{{\sqrt 3 }}{2}{Y_{72}}(\vartheta ,\varphi )$
\begin{choices}
\choice $\overline P  =  - \frac{1}{2}$     
\choice $\overline P  = \frac{1}{2}$     
\choice $\overline P  =  - \frac{1}{4}$     
\choice $\overline P  = \frac{1}{4}$
\end{choices}

\end{questions}




\section{ ТРЕХМЕРНОЕ ДВИЖЕНИЕ }
\subsection{ Общие свойства трехмерного движения }

\begin{questions}

\question Какой из нижеследующих формул определяется оператор Гамильтона частицы в цен-трально-симметричном поле (${\hat L^2}$ - оператор квадрата момента, ${\hat L_z}$ - оператор проекции момента на ось $z$)?
\begin{choices}
\choice $ - \frac{{{\hbar ^2}}}{{2m}}{r^2}\frac{\partial }{{\partial r}}\left( {\frac{1}{{{r^2}}}\frac{\partial }{{\partial r}}} \right) - \frac{1}{{2m{r^2}}}{\hat L^2} + U(r)$      
\choice $ - \frac{{{\hbar ^2}}}{{2m}}\frac{1}{{{r^2}}}\frac{\partial }{{\partial r}}\left( {{r^2}\frac{\partial }{{\partial r}}} \right) - \frac{1}{{2m{r^2}}}{\hat L^2} + U(r)$   
\choice $ - \frac{{{\hbar ^2}}}{{2m}}{r^2}\frac{\partial }{{\partial r}}\left( {\frac{1}{{{r^2}}}\frac{\partial }{{\partial r}}} \right) - \frac{1}{{2m{r^2}}}{\hat L_z}^2 + U(r)$    
\choice $ - \frac{{{\hbar ^2}}}{{2m}}\frac{1}{{{r^2}}}\frac{\partial }{{\partial r}}\left( {{r^2}\frac{\partial }{{\partial r}}} \right) - \frac{1}{{2m{r^2}}}{\hat L_z}^2 + U(r)$
\end{choices}

\question Частица движется в центральном поле. Будет ли гамильтониан частицы коммутировать с оператором проекции момента на ось $z$?
\begin{choices}
\choice да    
\choice нет      
\choice зависит от поля      
\choice это зависит от состояния частицы
\end{choices}

\question Будет ли гамильтониан частицы, движущейся в центральном поле, коммутировать с опе-ратором проекции момента на ось $x$?
\begin{choices}
\choice да    
\choice нет      
\choice зависит от поля      
\choice это зависит от состояния частицы 
\end{choices}

\question Найти коммутатор $\left[ {\hat H,{{\hat L}_ \pm }} \right]$, где $\hat H$ - оператор Гамиль-тона частицы, ${\hat L_ \pm }$ - операторы, повышающий и понижающий проекцию момента.
\begin{choices}
\choice $\left[ {\hat H,{{\hat L}_ \pm }} \right] = {\hat L_ \pm }$ 
\choice $\left[ {\hat H,{{\hat L}_ \pm }} \right] = \hbar \hat H$   
\choice $\left[ {\hat H,{{\hat L}_ \pm }} \right] =  - \hbar \hat H$   
\choice $\left[ {\hat H,{{\hat L}_ \pm }} \right] = 0$
\end{choices}

\question Будет ли гамильтониан частицы, движущейся в центральном поле коммутировать с оператором квадрата момента?
\begin{choices}
\choice да    
\choice нет      
\choice зависит от поля      
\choice это зависит от состояния частицы
\end{choices}

\question Частица движется в центрально-симметричном поле. Какой из четырех нижеприведенных коммутаторов не равен нулю ($\hat H,\;{\hat p_x},\;{\hat L_x},\;{\hat L^2}$ и $\hat P$ - операторы Га-мильтона, проекции импульса на ось $x$, проекции момента на ось $x$, квадрата момента и чет-ности)?
\begin{choices}
\choice $\left[ {\hat H,{{\hat p}_x}} \right]$       
\choice $\left[ {\hat H,{{\hat L}_x}} \right]$       
\choice $\left[ {\hat H,{{\hat L}^2}} \right]$       
\choice $\left[ {\hat H,\hat P} \right]$
\end{choices}

\question Что можно сказать о зависимости волновых функций стационарных состояний дискретного спектра частицы в некотором центрально-симметричном поле от полярного $\vartheta $ и азимутального $\varphi $ углов?
\begin{choices}
\choice не зависят от углов, поскольку поле центрально
\choice зависимость от углов всегда сводится к одной из сферических функции
\choice их можно выбрать так, что зависимость от углов сводится к одной из сферических функций
\choice зависимость от углов определяется видом поля
\end{choices}

\question Частица движется в центральном поле. Какими квантовыми числами можно классифици-ровать волновые функции стационарных состояний дискретного спектра?
\begin{choices}
\choice ${n_r}$        
\choice ${n_r}$ и $l$     
\choice ${n_r}$, $l$, $m$    
\choice ${n_r}$ и $m$
(здесь ${n_r}$ - радиальное квантовое число, $l$ - момент, $m$ - проекция момента на ось $z$)
\end{choices}

\question Частица движется в центральном поле. Какими квантовыми числами можно классифици-ровать энергии стационарных состояний дискретного спектра?
\begin{choices}
\choice ${n_r}$        
\choice ${n_r}$ и $l$     
\choice ${n_r}$, $l$, $m$    
\choice ${n_r}$ и $m$
(здесь ${n_r}$ - радиальное квантовое число, $l$ - момент, $m$ - проекция момента на ось $z$)
\end{choices}

\question Частица движется в центральном поле. От каких квантовых чисел зависят радиальные части волновых функций стационарных состояний дискретного спектра?
\begin{choices}
\choice только от ${n_r}$       
\choice только от ${n_r}$ и $l$
\choice и от ${n_r}$, и от $l$, и от $m$    
\choice только от ${n_r}$ и $m$ 
(здесь ${n_r}$ - радиальное квантовое число, $l$ - момент, $m$ - проекция момента на ось $z$)
\end{choices}

\question Какая формула правильно определяет условие ортогональности радиальных волновых функций стационарных состояний дискретного спектра?
\begin{choices}
\choice $\int\limits_0^\infty  {{\chi _{{n_r}{l_1}}}(r)} {\chi _{{n_r}{l_2}}}(r)dr = {\delta _{{l_1}{l_2}}}$        
\choice $\int\limits_0^\infty  {{\chi _{{n_{r1}},l}}(r)} {\chi _{{n_{r2}}l}}(r)dr = {\delta _{{n_{r1}}{n_{r2}}}}$
\choice $\int\limits_0^\infty  {{\chi _{{n_{r1}},{l_1}}}(r)} {\chi _{{n_{r2}}{l_2}}}(r)dr = {\delta _{{n_{r1}}{n_{r2}}}}{\delta _{{l_1}{l_2}}}$      
\choice разные радиальные функции не ортогональны
\end{choices}

\question Какая формула правильно определяет условие ортогональности радиальных волновых функций стационарных состояний дискретного спектра?
\begin{choices}
\choice $\int\limits_0^\infty  {{\chi _{{n_{r1}},l}}(r)} {\chi _{{n_{r2}}l}}(r)dr = {\delta _{{n_{r1}}{n_{r2}}}}$         
\choice $\int\limits_0^\infty  {{\chi _{{n_{r1}},l}}(r)} {\chi _{{n_{r2}}l}}(r)rdr = {\delta _{{n_{r1}}{n_{r2}}}}$
\choice $\int\limits_0^\infty  {{\chi _{{n_{r1}},l}}(r)} {\chi _{{n_{r2}}l}}(r){r^2}dr = {\delta _{{n_{r1}}{n_{r2}}}}$    
\choice разные радиальные функции не ортогональны
\end{choices}

\question Что такое вырождение уровней энергии частицы в центрально-симметричном поле по проекции момента?
\begin{choices}
\choice совпадение проекций момента у состояний с разными энергиями
\choice совпадение проекций у состояний с разными моментами
\choice совпадение моментов у состояний с разными проекциями
\choice совпадение энергий у состояний с разными проекциями момента
\end{choices}

\question Пусть $\psi (r,\vartheta ,\varphi )$ - волновая функция стационарного состояния дискретного спектра частицы в центральном поле с определенной проекцией момента на ось $z$. Будет ли функция ${\hat L_ + }\psi (r,\vartheta ,\varphi )$, где ${\hat L_ + }$ - оператор, повышающий проекцию момента на ось $z$, собственной функцией оператора Гамильтона?
\begin{choices}
\choice всегда будет
\choice никогда не будет
\choice будет, если $\psi $ не отвечает максимальной проекции момента
\choice будет, если $\psi $ не отвечает минимальной проекции момента
\end{choices}

\question Пусть $\psi (r,\vartheta ,\varphi )$ - волновая функция стационарного состояния дискретного спектра частицы в центральном поле с определенной проекцией момента на ось $z$. Будет ли функция ${\hat L_ - }\psi (r,\vartheta ,\varphi )$, где ${\hat L_ - }$ - оператор, понижающий проекцию мо-мента на ось $z$, собственной функцией оператора Гамильтона?
\begin{choices}
\choice всегда будет
\choice никогда не будет
\choice будет, если $\psi $ не отвечает максимальной проекции момента
\choice будет, если $\psi $ не отвечает минимальной проекции момента
\end{choices}

\question Что такое «вырождение по моменту» в центрально-симметричном поле?
\begin{choices}
\choice совпадение моментов у состояний с разными энергиями
\choice совпадение энергий у состояний с разными моментами
\choice совпадение моментов у состояний с разными проекциями
\choice совпадение проекций у состояний с разными моментами
\end{choices}

\question Что такое «случайное» вырождение уровней энергии в центрально-симметричном поле?
\begin{choices}
\choice совпадение энергий у частиц, движущихся в разных потенциалах
\choice совпадение энергий у состояний с разными моментами
\choice случайное столкновение частиц, имеющих одинаковые энергии
\choice совпадение моментов у состояний с разными энергиями
\end{choices}

\question «Случайное вырождение» уровней энергии в центральном поле и «вырождение по мо-менту» – это одно и то же, или нет?
\begin{choices}
\choice да                
\choice нет
\choice да, при совпадении энергий    
\choice это зависит от потенциала
\end{choices}

\question Частица движется в центральном поле, в котором отсутствует «случайное» вырождение. Какова кратность вырождения уровня энергии частицы с моментом $l = 5$?
\begin{choices}
\choice 1        
\choice 5        
\choice 10       
\choice 11 
\end{choices}

\question Частица движется в центральном поле, в котором отсутствует «случайное» вырождение. Какова кратность вырождения стационарных состояний непрерывного спектра?
\begin{choices}
\choice 1        
\choice $2l + 1$    
\choice $\infty $         
\choice 2 
\end{choices}

\question Частица движется в центральном поле, в котором имеет место «случайное» вырождение. Какова кратность вырождения стационарных состояний непрерывного спектра?
\begin{choices}
\choice 1        
\choice $2l + 1$    
\choice $\infty $         
\choice другая 
\end{choices}

\question Частица движется в центральном поле, в котором имеет место «случайное» вырождение. Частица находится в стационарном состоянии и имеет определенный момент $l = 5$. Какова кратность вырождения уровня энергии, на котором находится частица?
\begin{choices}
\choice 1     
\choice 5     
\choice 11    
\choice мало информации, чтобы ответить 
\end{choices}

\question Частица движется в центральном поле, в котором имеет место «случайное» вырождение состояний с моментом $l = 1$ и $l = 3$. Какова кратность вырождения такого уровня энергии?
\begin{choices}
\choice 9     
\choice 10    
\choice 11    
\choice 12 
\end{choices}

\question Уровень энергии частицы в центральном поле вырожден. Существует ли «случайное» вырождение в этом поле?
\begin{choices}
\choice да    
\choice нет      
\choice мало информации для ответа    
\choice зависит от состояния
\end{choices}

\question Кратность вырождения уровня энергии частицы в центральном поле, равна 6. Существу-ет ли «случайное» вырождение уровней энергии частицы в этом поле?
\begin{choices}
\choice да, существует       
\choice нет, не существует
\choice мало информации      
\choice это зависит от поля
\end{choices}

\question Кратность вырождения некоторого уровня энергии частицы в центральном поле, равна 9. Существует ли «случайное» вырождение уровней энергии частицы в этом поле?
\begin{choices}
\choice да, существует       
\choice нет, не существует
\choice мало информации      
\choice это зависит от поля
\end{choices}

\question Радиальное квантовое число нумерует
\begin{choices}
\choice Состояния с определенным моментом в порядке возрастания энергии
\choice Все уровни в порядке возрастания энергии
\choice Состояния с определенной проекцией момента в порядке возрастания энергии
\choice Состояния с определенной энергией в порядке возрастания момента
\end{choices}

\question Орбитальное квантовое число – это
\begin{choices}
\choice значение проекции момента стационарного состояния
\choice значение энергии стационарного состояния
\choice значение момента стационарного состояния
\choice номер орбитали
\end{choices}

\question Магнитное квантовое число – это
\begin{choices}
\choice значение проекции момента стационарного состояния
\choice значение энергии стационарного состояния
\choice значение момента стационарного состояния
\choice магнитный момент системы
\end{choices}

\question Радиальное квантовое число и момент стационарного состояния частицы в центральном поле фиксированы. Как изменяется энергия при увеличении проекции момента на ось $z$?
\begin{choices}
\choice растет      
\choice убывает     
\choice не меняется    
\choice это зависит от поля
\end{choices}

\question Момент стационарного состояния частицы в центральном поле фиксирован. Как изменя-ется энергия при увеличении радиального квантового числа?
\begin{choices}
\choice растет      
\choice убывает     
\choice не меняется    
\choice это зависит от поля
\end{choices}

\question Радиальное квантовое число стационарного состояния частицы в центральном поле фик-сированы. Как изменяется энергия при увеличении момента?
\begin{choices}
\choice растет      
\choice убывает     
\choice не меняется    
\choice это зависит от поля
\end{choices}

\question Каково число узлов радиальной волновой функции стационарного состояния с кванто-выми числами ${n_r} = 3$, $l = 5$, $m = 4$ частицы в центральном поле (исключая узел при $r = 0$, состоянию с минимальной энергией при фиксированном моменте отвечает ${n_r} = 0$)?
\begin{choices}
\choice 2        
\choice 3
\choice 5        
\choice такие значения квантовые числа состояния принимать не могут
\end{choices}

\question Каково число узлов радиальной волновой функции стационарного состояния с кванто-выми числами ${n_r} = 3$, $l = 4$, $m = 5$ частицы в центральном поле (исключая узел при $r = 0$, состоянию с минимальной энергией при фиксированном моменте отвечает ${n_r} = 0$)?
\begin{choices}
\choice 2        
\choice 3
\choice 4        
\choice такие значения квантовые числа состояния принимать не могут
\end{choices}

\question Каково число узлов радиальной волновой функции стационарного состояния с кванто-выми числами ${n_r} = 3$, $l = 5$, $m = 4$ частицы в центральном поле (исключая узел при $r = 0$, состоянию с минимальной энергией при фиксированном моменте отвечает ${n_r} = 1$)?
\begin{choices}
\choice 2        
\choice 3  
\choice 5        
\choice такие значения квантовые числа состояния принимать не могут
\end{choices}

\question Частица находится в центральном поле, которое не имеет особенности при $r \to 0$. Как ведет себя радиальная волновая функция $R(r)$ стационарного состояния частицы с квантовыми числами ${n_r}$, $l$, $m$ при $r \to 0$ ($\psi (r,\vartheta ,\varphi ) = R(r){Y_{lm}}(\vartheta ,\varphi )$)?
\begin{choices}
\choice как ${r^l}$    
\choice как ${r^{{n_r}}}$    
\choice ${r^m}$     
\choice ${r^{l + {n_r}}}$
\end{choices}

\question Частица находится в центральном поле, которое затухает при $r \to \infty $. Как ведут себя радиальные волновые функции стационарных состояний дискретного спектра при $r \to \infty $?
\begin{choices}
\choice возрастают     
\choice затухают
\choice осциллируют    
\choice тождественно равны нулю
\end{choices}

\question Частица находится в центральном поле, которое затухает при $r \to \infty $. Как ведут себя радиальные волновые функции стационарных состояний непрерывного спектра при $r \to \infty $?
\begin{choices}
\choice возрастают     
\choice затухают
\choice осциллируют    
\choice тождественно равны нулю
\end{choices}

\question Частица движется в центральном поле $U(r)$. Известно, что $U(r \to \infty ) \to \infty $. Какое утверждение правильно определяет соотношение между радиальным и орбитальным квантовы-ми числами частицы?
\begin{choices}
\choice ${n_r} \le l$  
\choice ${n_r} \ge l$  
\choice ${n_r} < l$ 
\choice квантовые числа ${n_r}$ и $l$ независимы
\end{choices}

\question Какова кратность вырождения основного состояния частицы в центральном поле
\begin{choices}
\choice 1        
\choice 2        
\choice 3        
\choice это зависит от поля
\end{choices}

\question Может ли первый возбужденный уровень энергии частицы в центрально-симметричном поле иметь кратность вырождения, равную 4?
\begin{choices}
\choice да    
\choice нет      
\choice мало информации, чтобы ответить  
\choice это зависит от энергии
\end{choices}

\question Может ли первый возбужденный уровень энергии частицы в центрально-симметричном поле иметь кратность вырождения, равную 1?
\begin{choices}
\choice да    
\choice нет      
\choice мало информации, чтобы ответить  
\choice это зависит от энергии
\end{choices}

\question Какой не может быть кратность вырождения первого возбужденного уровня энергии частицы в центрально-симметричном поле?
\begin{choices}
\choice 1     
\choice 2     
\choice 3     
\choice 4
\end{choices}

\question Какой не может быть кратность вырождения второго возбужденного уровня энергии частицы в центрально-симметричном поле?
\begin{choices}
\choice 1     
\choice 3     
\choice 6     
\choice 7
\end{choices}

\question В каком из нижеперечисленных центральных полей кратность вырождения первого воз-бужденного уровня энергии равна 4?
\begin{choices}
\choice $U(r) \sim  - \frac{1}{r}$    
\choice $U(r) \sim  - \frac{1}{{{r^2}}}$ 
\choice $U(r) \sim r$     
\choice $U(r) \sim {r^2}$
\end{choices}

\question Может ли кратность вырождения уровней энергии частицы в центральном поле равнять-ся 2?
\begin{choices}
\choice да    
\choice нет      
\choice зависит от поля      
\choice зависит от энергии
\end{choices}

\question Уровень энергии частицы в центральном поле невырожден. Какие значения орбитально-го момента можно обнаружить при измерениях над частицей, находящейся на этом уровне и с какими вероятностями?
\begin{choices}
\choice $l = 1$ с единичной вероятностью       
\choice $l = 0$ с единичной вероятностью
\choice $l = 0$ и $l = 1$ с равными вероятностями    
\choice мало информации для ответа 
\end{choices}

\question Кратность вырождения уровня энергии частицы в центральном поле равна 9. Будет ли момент частицы, находящейся на этом уровне, иметь определенное значение?
\begin{choices}
\choice да             
\choice нет
\choice мало информации для ответа 
\choice такой кратность вырождения быть не может 
\end{choices}

\question Кратность вырождения уровня энергии частицы в центральном поле равна 7. Будет ли момент частицы, находящейся на этом уровне, иметь определенное значение?
\begin{choices}
\choice да             
\choice нет
\choice мало информации для ответа 
\choice такой кратность вырождения быть не может
\end{choices}

\question $s$-состояние частицы в центральном поле, это 
\begin{choices}
\choice состояние с определенным моментом $l$, равным нулю
\choice стационарное состояние
\choice состояние с моментом $l$, равным семи  
\choice седьмое стационарной состояние
\end{choices}

\question Как зависит от углов волновая функция $s$-состояния частицы в центральном поле?
\begin{choices}
\choice как $\cos \vartheta $
\choice как $\cos \vartheta {e^{i\varphi }}$
\choice не зависит  
\choice как $\sin \vartheta {e^{i\varphi }}$
\end{choices}

\question $p$-состояние частицы в центральном поле, это 
\begin{choices}
\choice состояние с определенным моментом $l$, равным нулю
\choice состояние с определенным импульсом
\choice состояние с моментом $l$, равным единице  
\choice пятое стационарной состояние
\end{choices}

\question $d$-состояние частицы в центральном поле, это 
\begin{choices}
\choice четвертый уровень энергии
\choice состояние с определенной проекцией момента на ось $z$ ${l_z} = 4$
\choice девятое стационарное состояние
\choice состояние с определенным моментом $l = 2$
\end{choices}

\question Какова кратность вырождения $f$-уровня энергии частицы в центральном поле?
\begin{choices}
\choice 5        
\choice 7        
\choice 9        
\choice 11
\end{choices}

\question Частица находится в некотором стационарном состоянии с моментом $l \ne 0$ в централь-но-симметричном поле. Будет ли проекция момента импульса частицы на ось $z$ иметь опреде-ленное значение? 
\begin{choices}
\choice да    
\choice нет      
\choice это зависит от состояния      
\choice если $l = 1$, то да
\end{choices}

\question Частица находится в некотором стационарном состоянии с моментом $l \ne 0$ в централь-но-симметричном поле. Будет ли проекция момента импульса частицы на ось $x$ иметь опреде-ленное значение?
\begin{choices}
\choice да    
\choice нет      
\choice это зависит от состояния      
\choice если $l = 1$, то да
\end{choices}

\question Будут ли волновые функции стационарных состояний частицы с моментом $l \ne 0$ в не-котором центрально-симметричном поле собственными функциями оператора ${\hat L_z}$?
\begin{choices}
\choice да
\choice нет
\choice можно выбрать так, чтобы были
\choice если $l = 1$, то да 
\end{choices}

\question Частица находится в некотором невырожденном стационарном состоянии в центрально-симметричном поле, в котором имеет место «случайное» вырождение. Будет ли момент частицы иметь определенное значение, и если да, то какое?
\begin{choices}
\choice да, $l = 1$    
\choice да, $l = 0$    
\choice это зависит от состояния      
\choice да, $l = 2$
\end{choices}

\question Частица движется в некотором центральном поле и находится на уровне энергии с крат-ностью вырождения $g = 6$. Измеряют проекцию момента. Какое из нижеперечисленных значе-ний не могло быть при этом получено?
\begin{choices}
\choice ${l_z} = 0$    
\choice ${l_z} = 1$    
\choice ${l_z} = 2$    
\choice ${l_z} = 3$
\end{choices}

\question Частица находится в некотором стационарном состоянии в центрально-симметричном поле, в котором имеет место «случайное» вырождение. Будет ли момент импульса частицы иметь определенное значение?
\begin{choices}
\choice да    
\choice нет      
\choice это зависит от состояния      
\choice если ${l_z} = 1$, то да
\end{choices}

\question Частица находится в стационарном состоянии в некотором центральном поле, в котором отсутствует «случайное» вырождение. Будет ли момент импульса частицы иметь определенное значение? 
\begin{choices}
\choice да    
\choice нет      
\choice это зависит от состояния      
\choice зависит от энергии
\end{choices}

\question Частица движется в центрально-симметричном поле, в котором имеет место «случайное» вырождение. Будут ли собственные функции оператора Гамильтона частицы обладать опреде-ленной четностью?
\begin{choices}
\choice да
\choice нет
\choice вообще говоря, нет, но можно выбрать так, чтобы обладали
\choice это зависит от энергии
\end{choices}

\question Частица движется в центрально-симметричном поле, в котором отсутствует «случайное» вырождение. Будут ли собственные функции гамильтониана частицы обладать определенной четностью?
\begin{choices}
\choice да
\choice нет
\choice вообще говоря, нет, но можно выбрать так, чтобы обладали
\choice это зависит от потенциала
\end{choices}

\question Частица движется в центрально-симметричном поле, в котором имеет место «случайное» вырождение состояний с четными или нечетными значениями момента. Будут ли собственные функции оператора Гамильтона частицы обладать определенной четностью?
\begin{choices}
\choice да
\choice нет
\choice вообще говоря, нет, но можно выбрать так, чтобы обладали
\choice это зависит от потенциала
\end{choices}

\question Частица движется в центрально-симметричном поле и находится на уровне энергии с кратностью вырождения, равной 4. Будет ли волновая функция частицы обладать определенной четностью?
\begin{choices}
\choice да    
\choice нет      
\choice это зависит от состояния      
\choice это зависит от потенциала
\end{choices}

\question Частица движется в центрально-симметричном поле и находится на уровне энергии с кратностью вырождения, равной 3. Будет ли волновая функция частицы обладать определенной четностью?
\begin{choices}
\choice да    
\choice нет      
\choice это зависит от состояния      
\choice это зависит от потенциала
\end{choices}

\question Частица движется в центрально-симметричном поле и находится на уровне энергии с кратностью вырождения, равной 6. Будет ли волновая функция частицы обладать определенной четностью?
\begin{choices}
\choice да    
\choice нет      
\choice это зависит от состояния      
\choice это зависит от потенциала
\end{choices}

\question Для частицы в центрально-симметричном поле выбрать правильный знак сравнения ${E_{{n_r} = 2l = 3}} \vee {E_{{n_r} = 1l = 4}}$ 
\begin{choices}
\choice $ > $       
\choice $ < $       
\choice $ = $       
\choice это зависит от потен-циала
\end{choices}

\question Для частицы в центрально-симметричном поле выбрать правильный знак сравнения ${E_{{n_r} = 2l = 3}}$ и ${E_{{n_r} = 1l = 2}}$
\begin{choices}
\choice $ > $       
\choice $ < $       
\choice $ = $       
\choice это зависит от потен-циала.
\end{choices}

\question Частица находится на втором возбужденном уровне энергии в некотором центрально-симметричном поле. Может ли проекция орбитального момента импульса частицы на ось $y$ принимать значение $ - 3\hbar $?
\begin{choices}
\choice да    
\choice нет      
\choice это зависит от состояния      
\choice это зависит от поля
\end{choices}

\question Частица находится на втором возбужденном уровне энергии в некотором центрально-симметричном поле и имеет определенную проекцию орбитального момента импульса ось $y$, равную $2\hbar $. Существует ли «случайное» вырождение в этом поле?
\begin{choices}
\choice да    
\choice нет      
\choice это зависит от состояния   
\choice мало информации для ответа 
\end{choices}

\question Частица находится на втором возбужденном уровне энергии в некотором центрально-симметричном поле и имеет проекцию орбитального момента импульса на ось $y$, равную $\hbar $. Существует ли случайное вырождение в этом поле?
\begin{choices}
\choice да    
\choice нет      
\choice это зависит от состояния   
\choice мало информации для ответа 
\end{choices}

\question Частица движется в центрально-симметричном поле. Какие из нижеследующих величин являются интегралами движения?
\begin{choices}
\choice координата $x$
\choice проекция импульса на ось $x$
\choice проекция момента на ось $y$
\choice таких величин нет
\end{choices}

\question Частица находится в некотором стационарном состоянии в центрально-симметричном поле. Средние значения каких величин будут зависеть от времени в этом состоянии?
\begin{choices}
\choice координаты $x$
\choice проекции импульса на ось $y$
\choice проекции орбитального момента на ось $z$
\choice таких величин нет
\end{choices}

\question Частица находится в состоянии с определенным моментом в некотором центрально-симметричном поле. Будет ли это состояние стационарным?
\begin{choices}
\choice да
\choice нет
\choice да, если отсутствует «случайное» вырождение
\choice мало информации для ответа 
\end{choices}

\question Частица движется в центрально-симметричном поле со «случайным» вырождением и находится на некотором уровне энергии в состоянии с неопределенным моментом. Будет ли это состояние стационарным?
\begin{choices}
\choice да    
\choice нет      
\choice это зависит от поля     
\choice это зависит от состояния
\end{choices}

\question Каким является минимально возможный (в зависимости от вида центрального потенциа-ла) порядковый номер (в порядке возрастания энергии) уровня, который содержит состояние с моментом $l = 2$ (основное состояние – первый уровень энергии)?
\begin{choices}
\choice 3     
\choice 4     
\choice 5     
\choice 6
\end{choices}

\end{questions}

\subsection{ Сферический осциллятор, кулоновский потенциал и бесконечно глубокая сферическая прямоугольная потенциальная яма }

\begin{questions}

\question В каких из перечисленных ниже четырех центральных потенциалов существует вырождение стационарных состояний дискретного спектра по моменту ($\alpha  > 0$): 
\begin{choices}
(1) $U(r) = \alpha |r|$, (2) $U(r) = \alpha {r^2}$, (3) $U(r) =  - \frac{\alpha }{r}$, (4) $U(r) =  - \frac{\alpha }{{{r^2}}}$
\choice только (1)     
\choice только (3)     
\choice (1) и (3)      
\choice (2) и (3)
\end{choices}

\question Какой формулой – А, Б, В или Г – исчерпываются энергии всех стационарных состояний сферического осциллятора ($N = 0,1,\,2,...$)?
\begin{choices}
\choice $\hbar \omega \left( {N + 1/2} \right)$   
\choice $\hbar \omega \left( {N + 3/2} \right)$   
\choice $\hbar \omega \left( {N + 5/2} \right)$   
\choice $\hbar \omega \left( {N + 7/2} \right)$
\end{choices}

\question Для сферического осциллятора имеет место специфическое вырождение по моменту: вырождаются состояния только с четными или только нечетными значениями момента, причем к каждому следующему уровню «подключается» состояние со «следующим» по величине моментом: для основного состояния $l = 0$, для первого возбужденного $l = 1$, для второго возбужденного $l = 0$ и $l = 2$, для третьего возбужденного  $l = 1$ и $l = 3$, для четвертого возбужденного $l = 0$, $l = 2$ и $l = 4$ и т.д. Какие значения момента отвечают 2012 возбужденному уровню энергии сферического осциллятора?
\begin{choices}
\choice $l = 0,2,4,...,2012$    
\choice $l = 0,2,4,...,1006$ 
\choice $l = 1,3,5,...,2011$    
\choice $l = 1,3,5,...,1005$
\end{choices}

\question Чему равна энергия основного состояния сферического осциллятора $U(r) = \frac{{m{\omega ^2}{r^2}}}{2}$?
\begin{choices}
\choice $\frac{1}{2}\hbar \omega $    
\choice $\hbar \omega $      
\choice $\frac{3}{2}\hbar \omega $    
\choice $2\hbar \omega $
\end{choices}

\question Первому возбужденному уровню энергии сферического осциллятора отвечает единственное значение момента $l = 1$. Какова кратность вырождения этого уровня?
\begin{choices}
\choice 1        
\choice 2        
\choice 3     
\choice 4
\end{choices}

\question Второму возбужденному уровню энергии сферического осциллятора отвечают два значения момента $l = 0$ и $l = 2$. Какова кратность вырождения этого уровня энергии?
\begin{choices}
\choice 3        
\choice 4        
\choice 5     
\choice 6
\end{choices}

\question Пятому возбужденному уровню энергии сферического осциллятора отвечают значения момента $l = 1$, $l = 3$ и $l = 5$ и т.д. Какова кратность вырождения этого уровня?
\begin{choices}
\choice 6        
\choice 10       
\choice 21    
\choice 36
\end{choices}

\question  Кратность вырождения второго возбужденного состояния сферического осциллятора равна 6. На основе этих данных сравнить энергии второго $s$-состояния ${E_{{n_r} = 2l = 0}}$ и первого $d$-состояния ${E_{{n_r} = 1l = 2}}$
\begin{choices}
\choice $ > $
\choice $ < $
\choice $ = $
\choice данные условия и соотношение указанных энергий никак не связаны между собой
\end{choices}

\question Кратность вырождения второго возбужденного состояния сферического осциллятора равна 6. На основе этих данных определить знак сравнения ${E_{{n_r} = 1l = 1}} \vee {E_{{n_r} = 2l = 0}}$ 
\begin{choices}
\choice $ > $
\choice $ < $
\choice $ = $
\choice данные условия и соотношение указанных энергий никак не связаны между собой
\end{choices}

\question Кратность вырождения третьего возбужденного состояния сферического осциллятора равна 10. На основе этих данных определить знак сравнения ${E_{{n_r} = 2l = 1}} \vee {E_{{n_r} = 3l = 0}}$ 
\begin{choices}
\choice $ > $
\choice $ < $
\choice $ = $
\choice данные условия и соотношение указанных энергий никак не связаны между собой
\end{choices}

\question Сферический осциллятор имеет «декартовые» квантовые числа ${n_x} = 2,\;{n_y} = 3,\;{n_z} = 4$. Чему равна энергия осциллятора?
\begin{choices}
\choice $\frac{{19}}{2}\hbar \omega $    
\choice $\frac{{21}}{2}\hbar \omega $    
\choice $\frac{{23}}{2}\hbar \omega $    
\choice $\frac{{25}}{2}\hbar \omega $
\end{choices}

\question Волновая функция основного состояния сферического осциллятора, это (здесь $C$ - нормировочная постоянная, $a = \sqrt {\hbar /m\omega } $).
\begin{choices}
\choice $C\exp ( - {r^2}/2{a^2})$     
\choice $C\exp ( - r/a)$  
\choice $C\exp ( - r/2a)$ 
\choice $C\exp ( - {r^2}/{a^2})$ 
\end{choices}

\question  Сферический осциллятор находится в стационарном состоянии с волновой функцией $xy{e^{ - ({x^2} + {y^2} + {z^2})/2}}$ ($x,y,z$ - безразмерные декартовы координаты осциллятора). Каковы «декартовы» квантовые числа этого состояния осциллятора?
\begin{choices}
\choice ${n_x} = 0,\;{n_y} = 0,\;{n_z} = 2$    
\choice ${n_x} = 1,\;{n_y} = 1,\;{n_z} = 0$
\choice ${n_x} = 2,\;{n_y} = 2,\;{n_z} = 0$    
\choice ${n_x} = 1,\;{n_y} = 1,\;{n_z} = 3$
\end{choices}

\question Сферический осциллятор находится на втором возбужденном уровне энергии. Перечислить все значения момента импульса, которые можно обнаружить при измерениях
\begin{choices}
\choice $l = 1$ и $l = 2$       
\choice $l = 0$ и $l = 1$       
\choice $l = 0$ и $l = 2$       
\choice $l = 2$ 
Указание. Кратность вырождения второго возбужденного уровня энергии сферического гармонического осциллятора равна 6. 
\end{choices}

\question Сферический осциллятор находится на первом возбужденном уровне энергии. Какова вероятность того, что проекция момента осциллятора на ось $z$ равна $m =  - 1$?
\begin{choices}
\choice 1/3            
\choice 2/3         
\choice 0     
\choice это зависит от состояния
\end{choices}

\question Второму возбужденному уровню энергии сферического осциллятора отвечает сумма трех «декартовыми» квантовых чисел, равная 2. Какова кратность вырождения этого уровня энергии? 
\begin{choices}
\choice $1$      
\choice $3$      
\choice $6$      
\choice $10$ 
\end{choices}

\question Третьему возбужденному уровню энергии сферического осциллятора отвечает сумма трех «декартовыми» квантовых чисел, равная 3. Какова кратность вырождения этого уровня энергии? 
\begin{choices}
\choice $1$      
\choice $3$      
\choice $6$      
\choice $10$ 
\end{choices}

\question Сферический осциллятор находится в состоянии с «декартовыми» квантовыми числами ${n_x} = 1,\;{n_y} = 0,\;{n_z} = 0$. Какие значения момента импульса можно получит при измерениях? 
\begin{choices}
\choice $l = 1$     
\choice $l = 0$     
\choice $l = 1$ и $l = 2$    
\choice $l = 2$ и $l = 3$ 
\end{choices}

\question Сферический осциллятор находится в состоянии с «декартовыми» квантовыми числами ${n_x} = 2,\;{n_y} = 0,\;{n_z} = 0$. Какие значения момента импульса можно получит при измерениях? 
\begin{choices}
\choice $l = 1$     
\choice $l = 0$     
\choice $l = 0$ и $l = 2$    
\choice $l = 0$ и $l = 1$ 
\end{choices}

\question Сферический осциллятор находится в состоянии с «декартовыми» квантовыми числами ${n_x} = 0,\;{n_y} = 1,\;{n_z} = 0$. Какие значения проекции момента импульса на ось $z$ можно получить при измерениях? 
\begin{choices}
\choice $m = 0$     
\choice $m = 1$     
\choice $m = 1$ и $m =  - 1$       
\choice $m = 2$ и $m =  - 2$ 
\end{choices}

\question Сферический осциллятор находится в состоянии с «декартовыми» квантовыми числами ${n_x} = 0,\;{n_y} = 0,\;{n_z} = 1$. Какие значения проекции момента импульса на ось $y$ можно получить при измерениях? 
\begin{choices}
\choice $m = 0$     
\choice $m = 1$     
\choice $m = 1$ и $m =  - 1$.      
\choice $m = 2$ и $m =  - 2$ 
\end{choices}

\question Сферический осциллятор находится в состоянии с «декартовыми» квантовыми числами ${n_x} = 0,\;{n_y} = 0,\;{n_z} = 2012$. Какие значения проекции момента импульса на ось $z$ можно получить при измерениях? 
\begin{choices}
\choice $m = 0$           
\choice $m = 2012$
\choice $m =  - 2012$ и $m = 2012$    
\choice все значения от $m = 0$ до $m = 2012$ 
\end{choices}

\question Сферический осциллятор находится на первом возбужденном уровне энергии. Какой формулой не может описываться зависимость его волновой функции от полярного и азимутального углов $\vartheta $ и $\varphi $?
\begin{choices}
\choice $\sin \theta \sin \varphi $      
\choice $\cos \theta $    
\choice $\sin \theta $$\exp ( - i\varphi )$    
\choice $\cos \theta \cos \varphi $
Указание. Кратность вырождения первого возбужденного уровня энергии сферического гармонического осциллятора равна 3. 
\end{choices}

\question Сферический осциллятор находится в стационарном состоянии с волновой функцией $\left( {x + iy} \right){e^{ - ({x^2} + {y^2} + {z^2})/2}}$ ($x,y,z$ - безразмерные декартовы координаты осциллятора). Измеряют проекцию момента импульса осциллятора на ось $z$. Какие значения можно при этом обнаружить? 
\begin{choices}
\choice $m = 0$     
\choice $m =  - 1$
\choice $m = 1$     
\choice $m =  - 1$ и $m = 1$
\end{choices}

\question  Сферический осциллятор находится на втором возбужденном уровне энергии в состоянии с моментом $l = 2$. Сколько узлов имеет радиальная волновая функция осциллятора $\chi (r)$, включая узел при $r = 0$ ($\Psi (r,\vartheta ,\varphi ) = R(r){Y_{lm}}(\vartheta ,\varphi )$, $R(r) = \chi (r)/r$)?
\begin{choices}
\choice один  
\choice два      
\choice три      
\choice четыре
\end{choices}

\question  Сферический осциллятор находится на втором возбужденном уровне энергии в состоянии с моментом $l = 0$. Сколько узлов имеет радиальная волновая функция осциллятора $\chi (r)$, включая узел при $r = 0$ ($\Psi (r,\vartheta ,\varphi ) = R(r){Y_{lm}}(\vartheta ,\varphi )$, $R(r) = \chi (r)/r$)?
\begin{choices}
\choice один  
\choice два      
\choice три      
\choice четыре
\end{choices}

\question Сферический осциллятор находится на втором возбужденном уровне энергии в состоянии с определенной проекцией $m = 1$. Будет ли момент осциллятора иметь определенной значение?
\begin{choices}
\choice да    
\choice нет      
\choice это зависит от состояния      
\choice бессмысленный вопрос
Указание. Кратность вырождения второго возбужденного уровня энергии сферического гармонического осциллятора равна 6. 
\end{choices}

\question Сферический осциллятор находится на втором возбужденном уровне энергии в состоянии с определенной проекцией $m = 0$. Будет ли момент осциллятора иметь определенной значение?
\begin{choices}
\choice да    
\choice нет      
\choice это зависит от состояния      
\choice бессмысленный вопрос
Указание. Кратность вырождения второго возбужденного уровня энергии сферического гармонического осциллятора равна 6. 
\end{choices}

\question Сферический осциллятор находится в стационарном состоянии. Будет ли четность осциллятора иметь определенное значение?
\begin{choices}
\choice да          
\choice нет      
\choice зависит от состояния 
\choice только для основного или первого возбужденного состояния
Указание. Для сферического осциллятора имеет место вырождение только четных или только нечетных значений момента.
\end{choices}

\question Сферический осциллятор находится в состоянии, средняя четность которого равна $\overline P  =  - 1/2$. Будет ли это состояние стационарным? 
\begin{choices}
\choice да          
\choice нет      
\choice зависит от состояния 
\choice только для основного или первого возбужденного состояния
Указание. Для сферического осциллятора имеет место вырождение только четных или только нечетных значений момента.
\end{choices}

\question Сферический осциллятор находится в состоянии, средняя четность которого равна $\overline P  = 1$. Будет ли это состояние стационарным? 
\begin{choices}
\choice да          
\choice нет      
\choice зависит от состояния 
\choice только для основного или первого возбужденного состояния
Указание. Для сферического осциллятора имеет место вырождение только четных или только нечетных значений момента.
\end{choices}

\question Боровский радиус – это (здесь $\mu $ - масса электрона, $e$ - его заряд, $c$ - скорость света)
\begin{choices}
\choice $\frac{{{\hbar ^2}}}{{\mu {c^2}}}$     
\choice $\frac{{{e^2}}}{{\mu {\hbar ^2}}}$     
\choice $\frac{{{\hbar ^2}}}{{{e^2}{c^2}}}$    
\choice $\frac{{{\hbar ^2}}}{{\mu {e^2}}}$
\end{choices}

\question Как боровский радиус водородоподобного иона с зарядом ядра $Z$ зависит от $Z$?
\begin{choices}
\choice как $\frac{1}{{\sqrt Z }}$    
\choice как $\frac{1}{{{Z^2}}}$    
\choice как $\frac{1}{Z}$    
\choice как $\frac{1}{{{Z^{3/2}}}}$
\end{choices}

\question Какая из четырех нижеперечисленных величин является атомной единицей энергии ($\mu $ - масса электрона, $e$ - его заряд)
\begin{choices}
\choice $\frac{{\mu {e^2}}}{{{\hbar ^2}}}$     
\choice $\frac{{{\hbar ^2}}}{{\mu {e^2}}}$        
\choice $\frac{{\mu {e^4}}}{{{\hbar ^2}}}$     
\choice $\frac{{{\hbar ^2}}}{{\mu {e^4}}}$
\end{choices}

\question Какая из четырех нижеперечисленных величин является атомной единицей импульса ($\mu $ - масса электрона, $e$ - его заряд)
\begin{choices}
\choice $\frac{{\mu {e^2}}}{\hbar }$     
\choice $\frac{{{\hbar ^2}}}{{\mu {e^2}}}$        
\choice $\frac{{\mu {e^4}}}{{{\hbar ^2}}}$     
\choice $\frac{{{\hbar ^2}}}{{\mu {e^4}}}$
\end{choices}

\question Какой формулой – А, Б, В или Г – исчерпываются энергии всех стационарных состояний электрона в атоме водорода ($N = 1,\,2,\,3,...$,$\mu $ - масса электрона, $e$ - его заряд)?
\begin{choices}
\choice $ - \frac{{\mu {e^4}}}{{{N^2}{\hbar ^2}}}$      
\choice $ - \frac{{\mu {e^4}}}{{N{\hbar ^2}}}$    
\choice $ - \frac{{\mu {e^4}}}{{2{N^2}{\hbar ^2}}}$     
\choice $ - \frac{{\mu {e^4}}}{{2N{\hbar ^2}}}$
\end{choices}

\question Волновая функция основного состояния электрона в атоме водорода, это 
\begin{choices}
\choice $C\exp ( - r/a)$  
\choice $Cr\exp ( - r/a)$ 
\choice $C\exp ( - r/2a)$ 
\choice $Cr\exp ( - r/2a)$
(здесь $C$ - нормировочная постоянная, $a$ - боровский радиус).
\end{choices}

\question  Электрон находится в основном состоянии атома водорода, то есть его состояние описывается волновой функцией $C\exp ( - r/a)$ ($C$ - нормировочная постоянная, $a$ - боровский радиус). Каково наиболее вероятное расстояние от электрона до ядра?
\begin{choices}
\choice       
\choice       
\choice       
\choice 
\end{choices}

\question Для частицы, движущейся в кулоновском потенциале (электрон в атоме водорода), имеет место специфическое вырождение по моменту: вырождаются состояния со всеми значениями момента,, меньшими некоторого, причем к каждому следующему уровню «подключается» состояние со «следующим» по величине моментом: для основного состояния $l = 0$, для первого возбужденного $l = 0$ и $l = 1$, для второго возбужденного $l = 0$, $l = 1$ и $l = 2$, для третьего возбужденного $l = 0$, $l = 1$, $l = 2$ и $l = 3$ и т.д. Какова кратность вырождения второго возбужденного уровня энергии электрона в атоме водорода? 
\begin{choices}
\choice 8     
\choice 9     
\choice 10    
\choice 11
\end{choices}

\question Какова кратность вырождения четвертого возбужденного уровня энергии электрона в атоме водорода? 
\begin{choices}
\choice 25    
\choice 26    
\choice 27    
\choice 36
Указание. См. условие предыдущей задачи.
\end{choices}

\question Электрон в атоме водорода находится в стационарном состоянии. Будет ли момент импульса электрона иметь определенное значение?
\begin{choices}
\choice да                
\choice нет
\choice зависит от состояния       
\choice зависит от гамильтониана электрона
\end{choices}

\question Для электрона в атоме водорода сравнить ${E_{{n_r} = 2l = 0}}$ и ${E_{{n_r} = 1l = 1}}$.
\begin{choices}
\choice $ > $       
\choice $ < $       
\choice $ = $    
\choice зависит от $m$
\end{choices}

\question Для электрона в атоме водорода сравнить ${E_{{n_r} = 4l = 3}}$ и ${E_{{n_r} = 5l = 2}}$.
\begin{choices}
\choice $ > $       
\choice $ < $       
\choice $ = $       
\choice зависит от $m$
\end{choices}

\question Для электрона в атоме водорода сравнить ${E_{{n_r} = 0l = 3}}$ и ${E_{{n_r} = 1l = 4}}$.
\begin{choices}
\choice $ > $       
\choice $ < $       
\choice $ = $       
\choice зависит от $m$ 
\end{choices}

\question Какие ортогональные многочлены определяют радиальные волновые функции стационарных состояний электрона в атоме водорода
\begin{choices}
\choice полиномы Лежандра          
\choice полиномы Лагерра
\choice присоединенные полиномы Лежандра       
\choice полиномы Эрмита
\end{choices}

\question Электрон в атоме водорода находится в стационарном состоянии с квантовыми числами ${n_r}$, $l$ и $m$ (${n_r}$ - радиальное квантовое число, $l$ - момент и $m$-проекция момента). От каких квантовых чисел зависит радиальная волновая функция $R(r)$ электрона ($\Psi (r,\vartheta ,\varphi ) = R(r){Y_{lm}}(\vartheta ,\varphi )$)?
\begin{choices}
\choice функция $R(r)$ зависит от всех квантовых чисел ${n_r}$, $l$ и $m$
\choice функция $R(r)$ зависит от ${n_r}$, но не зависит от $l$ и $m$
\choice функция $R(r)$ зависит от ${n_r}$ и $l$, но не зависит от $m$
\choice функция $R(r)$ зависит от $l$ и $m$, но не зависит от ${n_r}$
\end{choices}

\question  Электрон в атоме водорода находится в стационарном состоянии. Будет ли четность этого состояния иметь определенное значение?
\begin{choices}
\choice да             
\choice нет
\choice зависит от состояния    
\choice только если электрон в основном или первом возбужденном состоянии
\end{choices}

\question  В каких пределах может меняться момент $l$ стационарного состояния электрона в атоме водорода  при фиксированном радиальном квантовом числе ${n_r}$ этого состояния?
\begin{choices}
\choice любое целое число из интервала значений $0 \le l \le {n_r} - 1$
\choice любое целое число из интервала значений $0 \le l \le {n_r}$
\choice любое целое число из интервала значений $0 \le l \le {n_r} + 1$
\choice любое целое положительное число
\end{choices}

\question Какой формулой определяется энергия стационарного состояния с квантовыми числами ${n_r}$, $l$ и $m$ электрона в атоме водорода (${n_r}$ - радиальное квантовое число, $l$ - момент и $m$-проекция момента; нумерация ${n_r}$ начинается с единицы)?
\begin{choices}
\choice $ - \frac{{{e^2}}}{{2a{{({n_r} + l)}^2}}}$      
\choice $ - \frac{{{e^2}}}{{2a{{({n_r} + l + 1)}^2}}}$     
\choice $ - \frac{{{e^2}}}{{2a{{({n_r} + l + m)}^2}}}$     
\choice $ - \frac{{{e^2}}}{{2a{{({n_r} + l + m + 1)}^2}}}$
\end{choices}

\question  Электрон в атоме водорода находится в состоянии с квантовыми числами ${n_r} = 4,\,l = 6,\,m = 2$ (нумерация ${n_r}$ начинается с единицы). Сколько узлов имеет радиальная волновая функция $\chi (r)$ (с учетом узла при $r = 0$)
\begin{choices}
\choice 4        
\choice 5        
\choice 6        
\choice 7
\end{choices}

\question  Радиальная волновая функция электрона, находящегося в стационарном состоянии в атоме водорода имеет пять узлов. Чему равен момент импульса электрона?
\begin{choices}
\choice $l = 4$     
\choice $l = 5$     
\choice $l = 6$     
\choice это не связанные величины
\end{choices}

\question Радиальная волновая функция электрона, находящегося в стационарном состоянии в атоме водорода имеет пять узлов. Чему равна проекция момента импульса электрона на ось $z$?
\begin{choices}
\choice $l = 4$     
\choice $l = 5$     
\choice $l = 6$     
\choice это не связанные величины
\end{choices}

\question Какова кратность вырождения уровня энергии $ - \frac{{m{e^4}}}{{32{\hbar ^2}}}$ электрона в атоме водорода 
\begin{choices}
\choice 4     
\choice 9     
\choice 16    
\choice 25
\end{choices}

\question  Электрон находится на первом возбужденном уровне энергии атома водорода. Измеряют момент импульса электрона. С какой вероятностью будет получено значение $l = 1$?
\begin{choices}
\choice $w = 1/4$         
\choice $w = 3/4$      
\choice $w = 0$     
\choice зависит от состояния
\end{choices}

\question  Электрон находится на первом возбужденном уровне энергии атома водорода. Измеряют момент импульса электрона. С какой вероятностью будет получено значение $l = 2$?
\begin{choices}
\choice $w = 1/4$         
\choice $w = 3/4$      
\choice $w = 0$     
\choice зависит от состояния
\end{choices}

\question Электрон находится на первом возбужденном уровне энергии атома водорода. Какой функцией не может описываться зависимость его волновой функции от углов?
\begin{choices}
\choice $\sin \vartheta \cos \varphi $      
\choice $\cos \vartheta  + \sin \vartheta {e^{ - i\varphi }}$    
\choice $\cos \vartheta \cos \varphi $      
\choice $1 + \sin \vartheta (\cos \varphi  - 2\sin \varphi )$
\end{choices}

\question Электрон находится на втором возбужденном уровне энергии атома водорода. Перечислить все значения момента импульса, которые можно обнаружить при измерениях
\begin{choices}
\choice $l = 0$, $l = 1$ и $l = 2$    
\choice $l = 0$, $l = 2$ и $l = 4$       
\choice $l = 1$ и $l = 3$    
\choice $l = 0$ и $l = 2$
\end{choices}

\question Электрон находится на третьем возбужденном уровне энергии атома водорода. Перечислить все значения проекции момента импульса на ось $z$, которые можно обнаружить при измерениях.
\begin{choices}
\choice ${l_z} =  - \hbar ,\quad 0,\quad \hbar $           
\choice только значение ${l_z} = 3\hbar $
\choice ${l_z} =  - 2\hbar ,\quad \hbar ,\quad 0,\quad \hbar ,\quad 2\hbar $       
\choice ${l_z} =  - 3\hbar ,\quad  - 2\hbar ,\quad \hbar ,\quad 0,\quad \hbar ,\quad 2\hbar ,\quad 3\hbar $
\end{choices}

\question  Какие значения момента импульса отвечают 10-му возбужденному уровню энергии электрона в атоме водорода?
\begin{choices}
\choice все четные от $l = 0$ до $l = 10$         
\choice все нечетные от $l = 1$ до $l = 11$
\choice все целые от $l = 0$ до $l = 10$       
\choice все целые от $l = 0$ до $l = 11$
\end{choices}

\question Электрон в атоме водорода с вероятностью 1/4 находится в состоянии с квантовыми числами ${n_r} = 3$, $l = 0$, а с вероятностью 3/4 – в состоянии с квантовыми числами ${n_r} = 2$, $l = 1$. Будет ли состояние электрона стационарным?
\begin{choices}
\choice да                         
\choice нет         
\choice зависит от значений магнитного квантового числа    
\choice в некоторых случаях будет, в некоторых нет
\end{choices}

\question  Электрон в атоме водорода с вероятностью 1/4 находится в состоянии с квантовыми числами ${n_r} = 3$, $l = 0$, а с вероятностью 3/4 – в состоянии с квантовыми числами ${n_r} = 1$, $l = 1$. Будет ли состояние электрона стационарным?
\begin{choices}
\choice да                         
\choice нет
\choice зависит от значений магнитного квантового числа    
\choice в некоторых случаях будет, в некоторых нет
\end{choices}

\question Волновая функция электрона в атоме водорода в момент времени $t = 0$ равна $C\exp ( - r/a)$ ($C$ - нормировочная постоянная, $a$ - боровский радиус). Какой формулой описывается волновая функция электрона в любой момент времени?
\begin{choices}
\choice $C\exp ( - r/a){e^{ - \frac{{im{e^4}t}}{{2{\hbar ^3}}}}}$         
\choice $C\exp ( - r/a){e^{\frac{{im{e^4}t}}{{2{\hbar ^3}}}}}$         
\choice $C\exp ( - r/a){e^{\frac{{im{e^4}t}}{{{\hbar ^3}}}}}$ 
\choice аналитически найти решение временного уравнения Шредингера для такого начального условия не удается
\end{choices}

\question Какой формулой определяются энергии $s$-состояний частицы массой $\mu $ в сферической бесконечно глубокой прямоугольной потенциальной яме радиуса $a$ ($n = 1,2,3...$)?
\begin{choices}
\choice $\frac{{{\pi ^2}{\hbar ^2}{n^2}}}{{2\mu {{(2a)}^2}}}$       
\choice $\frac{{{\pi ^2}{\hbar ^2}{n^2}}}{{2\mu {a^2}}}$         
\choice $\frac{{{\pi ^2}{\hbar ^2}n}}{{2\mu {a^2}}}$       
\choice $\frac{{{\pi ^2}{\hbar ^2}n}}{{2\mu {{(2a)}^2}}}$
\end{choices}

\question Какой формулой – А, Б, В или г – определяются волновые функции стационарных состояний с $l = 0$ для частицы, находящейся в сферической бесконечно глубокой прямоугольной потенциальной яме ($k$ - некоторое число)?
\begin{choices}
\choice $\sin kr/r$ внутри ямы и нуль снаружи        
\choice $\cos kr/r$ внутри ямы и нуль снаружи
\choice $\exp ( - kr)/r$ внутри ямы и нуль снаружи         
\choice $\exp (ikr)/r$ внутри ямы и нуль снаружи
\end{choices}

\question  Чему равна энергия второго стационарного $p$-состояния частицы массой $\mu $ в сферической бесконечно глубокой прямоугольной потенциальной яме радиуса $a$?
\begin{choices}
\choice $\frac{{12{\pi ^2}{\hbar ^2}}}{{2\mu {a^2}}}$         
\choice $\frac{{9{\pi ^2}{\hbar ^2}}}{{2\mu {a^2}}}$       
\choice $\frac{{16{\pi ^2}{\hbar ^2}}}{{2\mu {a^2}}}$
\choice аналитическое выражение для энергий $p$-состояния частицы в сферической яме получить не удается
\end{choices}

\question находится в состоянии с квантовыми числами ${n_r} = 2$, $l = 3$, $m =  - 2$. Какова кратность вырождения уровня энергии частицы в сферической бесконечно глубокой прямоугольной потенциальной яме, которому принадлежит состояние с квантовыми числами ${n_r} = 2$, $l = 3$, $m =  - 2$?
\begin{choices}
\choice 5        
\choice 7        
\choice 9        
\choice 11
\end{choices}

\question Волновая функция частицы в бесконечно глубокой сферической прямоугольной потенциальной яме представляет собой линейную комбинацию собственных функций с квантовыми числами ${n_r} = 3$, $l = 0$, $m = 0$ и ${n_r} = 2$, $l = 1$, $m = 0$. Будет ли это состояние стационарным?
\begin{choices}
\choice да          
\choice нет            
\choice зависит от размера ямы
\choice в некоторых случаях будет, в некоторых нет
\end{choices}

\end{questions}


\section{ ГЛАВА 5. СПИН }
(во всех задачах этой главы $\hbar  = 1$)

\begin{questions}

\question Спин электрона равен 
\begin{choices}
\choice $s =  - 1/2$
\choice $s = 1/2$
\choice $s = 1/2$ и $s =  - 1/2$ с равными вероятностями
\choice $s = 1/2$ и $s =  - 1/2$, вероятности зависят от состояния
\end{choices}

\question Проекция спина электрона на ось $z$ равна 
\begin{choices}
\choice ${s_z} = 1/2$
\choice ${s_z} =  - 1/2$
\choice ${s_z} = 1/2$ и ${s_z} =  - 1/2$, вероятности зависят от состояния 
\choice ${s_z} = 1/2$ и ${s_z} =  - 1/2$ с равными вероятностями
\end{choices}

\question Проекция спина электрона на ось $y$ равна 
\begin{choices}
\choice ${s_y} = 1/2$
\choice ${s_y} =  - 1/2$
\choice ${s_y} = 1/2$ и ${s_y} =  - 1/2$, вероятности зависят от состояния
\choice ${s_y} = 1/2$ и ${s_y} =  - 1/2$ с равными вероятностями
\end{choices}

\question Спин частицы равен 3/4. Какие значения может принимать проекция ее спина на ось $z$?
\begin{choices}
\choice +3/4 и –3/4       
\choice +3/4, 3/2, -3/2 и –3/4
\choice +3/4, 0 и –3/4       
\choice спин таким быть не может 
\end{choices}

\question Спин частицы равен $s = 5/2$. Какова вероятность того, что проекция спина на ось $z$ прини-мает значение ${s_z} = 5/2$?
\begin{choices}
\choice 1     
\choice 0     
\choice 1/2     
\choice это зависит от состояния
\end{choices}

\question Спин частицы равен $s = 99/2$. Какова размерность линейного пространства спиновых функ-ций частицы?
\begin{choices}
\choice 99    
\choice 100      
\choice 101      
\choice 102
\end{choices}

\question Частица имеет спин $s$. В каких пределах меняется координата ${s_z}$ в списке аргументов этой функции $\psi (\vec r,{s_z})$? 
\begin{choices}
\choice может быть любым числом от 0 до 1
\choice может быть любым числом от $ - s$ до $s$
\choice может быть любым числом от $ - s$ до $s$ через единицу
\choice может быть любым целым числом в интервале от $ - s$ до $s$
\end{choices}

\question В результате измерений были получены следующие значения вероятностей различных значений проекции спина частицы на ось $y$ в некотором состоянии: $w({s_y} =  - 1) = 1/4$, $w({s_y} = 0) = 1/4$, $w({s_y} = 1) = 1/2$. Чему равен спин частицы?
\begin{choices}
\choice 1     
\choice 2     
\choice 3     
\choice информации для ответа недостаточно
\end{choices}

\question Волновая функция частицы со спином $s = 1/2$ $\psi (\vec r,{s_z})$ известна. Вероятность того, что частица находится в малом элементе объема $dV$ вблизи точки с радиусом-вектором ${\vec r_0}$ равна
\begin{choices}
\choice $\left( {|\psi ({{\vec r}_0},{s_z} =  - 1/2){|^2} + |\psi ({{\vec r}_0},{s_z} = 1/2){|^2}} \right)dV$          
\choice $\left( {\int\limits_{ - s}^{ + s} {|\psi ({{\vec r}_0},{s_z}){|^2}d{s_z}} } \right)dV$
\choice $\left( {|\psi ({{\vec r}_0},{s_z} =  - 1){|^2} + |\psi ({{\vec r}_0},{s_z} = 0){|^2} + |\psi ({{\vec r}_0},{s_z} = 1){|^2}} \right)dV$      
\choice $|\psi ({\vec r_0},{s_z} = 1/2){|^2}dV$
\end{choices}

\question Волновая функция частицы со спином $s = 1/2$ $\psi (\vec r,{s_z})$ известна. Каким из ниже перечисленных выражений определяется условие нормировки этой функции?
\begin{choices}
\choice $\left( {|\psi (\vec r,{s_z} =  - 1/2){|^2} + |\psi (\vec r,{s_z} = 1/2){|^2}} \right) = 1$
\choice $\int {\left( {|\psi (\vec r,{s_z} =  - 1/2){|^2} + |\psi (\vec r,{s_z} =  + 1/2){|^2}} \right)dV = 1} $
\choice $\int {\left( {|\psi (\vec r,{s_z} =  - 1/2){|^2} - |\psi (\vec r,{s_z} =  + 1/2){|^2}} \right)dV = 1} $
\choice $\left( {|\psi (\vec r,{s_z} =  - 1/2){|^2} - |\psi (\vec r,{s_z} = 1/2){|^2}} \right) = 1$ 
\end{choices}

\question Волновая функция частицы со спином $s = 1/2$ $\psi (\vec r,{s_z})$ известна. Каков смысл величины $\int {|\psi (\vec r,{s_z} = 1/2){|^2}d\vec r} $ (интегрирование проводится по всем значениям пространственных координат)?
\begin{choices}
\choice это вероятность того, что частица находится в малом элементе объема $d\vec r$ вблизи точки с радиусом-вектором $\vec r$
\choice это вероятность того, что частица имеет проекцию спина ${s_z} = 1/2$ независимо от ее положения в пространстве
\choice это вероятность того, что частица имеет проекцию спина ${s_z} =  - 1/2$ независимо от ее положения в пространстве
\choice это вероятность того, что частица находится в малом элементе объема $d\vec r$ вблизи точки с радиусом-вектором $\vec r$ с проекцией спина на ось $z$, равной ${s_z} = 1/2$.
\end{choices}

\question Состояние частицы описывается следующей спиновой волновой функцией 
$\psi ({s_z}) = \left( {\begin{array}{*{20}{c}}
{ - 1/2}\\
0\\
{i\sqrt 3 /2}\\
0
\end{array}} \right)$.
Чему равен спин такой частицы? 
\begin{choices}
\choice 3/2      
\choice 1     
\choice 5/2      
\choice 2
\end{choices}

\question Состояние частицы описывается следующей спиновой волновой функцией, данной в условии предыдущей задачи. Будут ли спин частицы и его проекция на ось $z$ иметь определенные значения в этом состоянии? 
\begin{choices}
\choice спин - нет, проекция - да     
\choice спин - да, проекция - нет
\choice и спин, и его проекция     
\choice ни спин, ни его проекция
\end{choices}

\question Ниже приведены спиновые волновые функции трех состояний частицы. В каком из них частица имеет определенный спин?
(1) $\left( {\begin{array}{*{20}{c}}
{\sqrt 3 /2}\\
{1/2}
\end{array}} \right)$,  (2) $\left( {\begin{array}{*{20}{c}}
1\\
0
\end{array}} \right)$, (3) $\frac{1}{{\sqrt 2 }}\left( {\begin{array}{*{20}{c}}
1\\
1
\end{array}} \right)$
\begin{choices}
\choice только в состоянии (1)        
\choice только в состоянии (2)
\choice только в состоянии (3)        
\choice во всех перечисленных
\end{choices}

\question Какие из ниже перечисленных функций являются собственными функциями оператора ${\hat s^2}$?
\begin{choices}
\choice только $\left( {\begin{array}{*{20}{c}}
1\\
0
\end{array}} \right)$ и $\left( {\begin{array}{*{20}{c}}
0\\
1
\end{array}} \right)$         
\choice только $\left( {\begin{array}{*{20}{c}}
1\\
1
\end{array}} \right)$ и $\left( {\begin{array}{*{20}{c}}
1\\
{ - 1}
\end{array}} \right)$
\choice только $\left( {\begin{array}{*{20}{c}}
1\\
i
\end{array}} \right)$ и $\left( {\begin{array}{*{20}{c}}
1\\
{ - i}
\end{array}} \right)$      
\choice все двухкомпонентные столбцы
\end{choices}

\question Ниже приведены спиновые волновые функции трех состояний частицы. В каком из них частица имеет определенную проекцию спина на ось $z$?
(1) $\left( {\begin{array}{*{20}{c}}
{\sqrt 3 /2}\\
{1/2}
\end{array}} \right)$,  (2) $\left( {\begin{array}{*{20}{c}}
1\\
0
\end{array}} \right)$, (3) $\frac{1}{{\sqrt 2 }}\left( {\begin{array}{*{20}{c}}
1\\
1
\end{array}} \right)$
\begin{choices}
\choice только в состоянии (1)        
\choice только в состоянии (2)
\choice только в состоянии (3)        
\choice во всех перечисленных
\end{choices}

\question Какая из перечисленных функций является собственной функцией оператора ${\hat s_z}$?
\begin{choices}
\choice $\psi ({s_z}) = \left( {\begin{array}{*{20}{c}}
0\\
1
\end{array}} \right)$   
\choice $\psi ({s_z}) = \frac{1}{{\sqrt 2 }}\left( {\begin{array}{*{20}{c}}
1\\
1
\end{array}} \right)$   
\choice $\psi ({s_z}) = \frac{1}{{\sqrt 2 }}\left( {\begin{array}{*{20}{c}}
1\\
i
\end{array}} \right)$   
\choice все двухстрочные столбцы
\end{choices}

\question Частица находится в состоянии $\psi ({s_z}) = \left( {\begin{array}{*{20}{c}}
{\sqrt 3 /2}\\
{1/2}
\end{array}} \right)$. Что можно сказать о проекции спина на ось $z$ в этом состоянии?
\begin{choices}
\choice может принимать два значения ${s_z} = \sqrt 3 /2$ и ${s_z} = 1/2$ с одинаковыми вероятно-стями
\choice может принимать два значения ${s_z} = 1/2$ и ${s_z} =  - 1/2$ с вероятностями $3/4$ и $1/4$
\choice проекция спина частицы, волновая функция которой - двухкомпонентный столбец, равна $1/2$ независимо от состояния
\choice может принимать два значения ${s_z} = 1/2$ и ${s_z} =  - 1/2$ с одинаковыми вероятно-стями
\end{choices}

\question В каком из четырех состояний, спиновые волновые функции которых приведены ниже, частица имеет определенную проекцию спина на ось $z$?
\begin{choices}
\choice $\psi \left( {{s_z}} \right) = \left( {\begin{array}{*{20}{c}}
{ - 1/\sqrt 2 }\\
0\\
{ - 1/\sqrt 2 }
\end{array}} \right)$      
\choice $\psi \left( {{s_z}} \right) = \left( {\begin{array}{*{20}{c}}
{ - i/\sqrt 3 }\\
{i/\sqrt 3 }\\
{1/\sqrt 3 }
\end{array}} \right)$      
\choice $\psi \left( {{s_z}} \right) = \left( {\begin{array}{*{20}{c}}
0\\
i\\
0
\end{array}} \right)$   
\choice $\psi \left( {{s_z}} \right) = \left( {\begin{array}{*{20}{c}}
{1/\sqrt 3 }\\
{1/\sqrt 3 }\\
{1/\sqrt 3 }
\end{array}} \right)$
\end{choices}

\question Чему равно среднее значение проекции спина частицы на ось $z$ в состоянии $\psi ({s_z}) = \left( {\begin{array}{*{20}{c}}
0\\
1
\end{array}} \right)$?
\begin{choices}
\choice ${\bar s_z} =  - 1/2$      
\choice ${\bar s_z} = 1/2$      
\choice ${\bar s_z} = 1/4$      
\choice ${\bar s_z} = 0$
\end{choices}

\question Чему равно среднее значение проекции спина частицы на ось $z$ в состоянии $\psi ({s_z}) = \left( {\begin{array}{*{20}{c}}
{ - i/2}\\
{\sqrt 3 /2}
\end{array}} \right)$?
\begin{choices}
\choice ${\bar s_z} =  - 1/8$      
\choice ${\bar s_z} = 1/4$      
\choice ${\bar s_z} = 1/2$      
\choice ${\bar s_z} =  - 1/4$
\end{choices}

\question Чему равно среднее значение квадрата проекции спина частицы на ось $z$ в состоянии $\psi ({s_z}) = \left( {\begin{array}{*{20}{c}}
{ - i/2}\\
{\sqrt 3 /2}
\end{array}} \right)$?
\begin{choices}
\choice $\overline {s_z^2}  = 1/4$    
\choice $\overline {s_z^2}  =  - 1/4$    
\choice $\overline {s_z^2}  = 1/8$    
\choice $\overline {s_z^2}  = 3/16$
\end{choices}

\question Какая из четырех функций ортогональна функции $\left( {\begin{array}{*{20}{c}}
{1/2}\\
{i\sqrt 3 /2}
\end{array}} \right)$?
\begin{choices}
\choice $\left( {\begin{array}{*{20}{c}}
{ - 1/2}\\
{i\sqrt 3 /2}
\end{array}} \right)$      
\choice $\left( {\begin{array}{*{20}{c}}
{\sqrt 3 /2}\\
{ - 1/2}
\end{array}} \right)$      
\choice $\left( {\begin{array}{*{20}{c}}
{i\sqrt 3 /2}\\
{ - 1/2}
\end{array}} \right)$      
\choice $\left( {\begin{array}{*{20}{c}}
{i\sqrt 3 /2}\\
{1/2}
\end{array}} \right)$
\end{choices}

\question Чему равно скалярное произведение двух спиновых функций ${\psi _1}({s_z}) = \left( {\begin{array}{*{20}{c}}
1\\
i
\end{array}} \right)$ и ${\psi _2}({s_z}) = \left( {\begin{array}{*{20}{c}}
i\\
{ - 2i}
\end{array}} \right)$?
\begin{choices}
\choice $\left( {{\psi _1},{\psi _2}} \right) =  - i + 2$  
\choice $\left( {{\psi _1},{\psi _2}} \right) = i + 2$  
\choice $\left( {{\psi _1},{\psi _2}} \right) = i - 2$  
\choice $\left( {{\psi _1},{\psi _2}} \right) =  - i - 2$
\end{choices}

\question Дана спиновая волновая функция некоторого состояния частицы
$\psi \left( {{s_z}} \right) = \left( {\begin{array}{*{20}{c}}
{i/3}\\
{2/3 - 2i/3}\\
0
\end{array}} \right)$. Что можно сказать о нормировке этой функции?
\begin{choices}
\choice она нормирована на 1    
\choice она нормирована на 2
\choice она нормирована на 3    
\choice она нормирована на 9
\end{choices}

\question Частица находится в состоянии с волновой функцией $\psi ({s_z}) = \left( {\begin{array}{*{20}{c}}
{ - 3i/\sqrt {11} }\\
{\sqrt {2/11} }
\end{array}} \right)$. Будет ли квадрат проекции спина на ось $z$ иметь определенное значение в этом состоянии?
\begin{choices}
\choice да                
\choice нет
\choice зависит от способа измерений     
\choice недостаточно информации
\end{choices}

\question Частица находится в состоянии с волновой функцией $\psi ({s_z}) = \left( {\begin{array}{*{20}{c}}
{ - 3i/\sqrt {11} }\\
{\sqrt {2/11} }
\end{array}} \right)$. Будет ли квадрат проекции спина на ось $x$ иметь определенное значение в этом состоянии?
\begin{choices}
\choice да                
\choice нет
\choice зависит от способа измерений     
\choice недостаточно информации
\end{choices}

\question Частица находится в состоянии с волновой функцией $\psi \left( {{s_z}} \right) = \left( {\begin{array}{*{20}{c}}
{ - 1/2}\\
0\\
{i\sqrt 3 /2}
\end{array}} \right)$. Будет ли квадрат проекции спина на ось $z$ иметь определенное значение в этом состоянии?
\begin{choices}
\choice да                
\choice нет      
\choice зависит от способа измерений     
\choice недостаточно информации
\end{choices}

\question Частица находится в состоянии с волновой функцией $\psi \left( {{s_z}} \right) = \left( {\begin{array}{*{20}{c}}
{ - 1/\sqrt 2 }\\
{i/\sqrt 2 }\\
0
\end{array}} \right)$. Будет ли квадрат проекции спина на ось $z$ иметь определенное значение в этом состоянии?
\begin{choices}
\choice да                
\choice нет      
\choice зависит от способа измерений     
\choice недостаточно информации
\end{choices}

\question Частица со спином $s = 1$ находится в состоянии со спиновой функцией 
$\left( {\begin{array}{*{20}{c}}
{1/\sqrt 3 }\\
{1/\sqrt 3 }\\
{1/\sqrt 3 }
\end{array}} \right)$. Найти $\overline {{s_z}} $ в этом состоянии.
\begin{choices}
\choice $\overline {{s_z}}  = 1/3$    
\choice $\overline {{s_z}}  = 2/3$    
\choice $\overline {{s_z}}  = 1/6$    
\choice $\overline {{s_z}}  = 0$
\end{choices}

\question Частица со спином $s = 1$ находится в состоянии со спиновой функцией, приведенной в условии предыдущей задачи. Найти $\overline {{s_z}^2} $ в этом состоянии.
\begin{choices}
\choice $\overline {{s_z}^2}  = 1/3$     
\choice $\overline {{s_z}^2}  = 2/3$     
\choice $\overline {{s_z}^2}  = 1/6$     
\choice $\overline {{s_z}^2}  = 5/6$
\end{choices}

\question Спин частицы равен 1/2. Чему равны собственные значения оператора проекции спина на ось $y$?
\begin{choices}
\choice $ + 1$ и $ - 1$         
\choice $ + 1/2$ и $ - 1/2$
\choice $ + 1$, 0 и $ - 1$         
\choice $ + 3/2$, $ + 1/2$, $ - 1/2$ и $ - 3/2$
\end{choices}

\question Частица со спином 1/2 находится в состоянии, в котором проекция ее спина на ось $z$с вероят-ностью 1/4 принимает значение 1/2 и с вероятностью 3/4 - значение –1/2.  Какой функцией не может опи-сываться состояние такой частицы?
\begin{choices}
\choice $\psi ({s_z}) = \left( {\begin{array}{*{20}{c}}
{ - 1/2}\\
{i\sqrt 3 /2}
\end{array}} \right)$      
\choice $\psi ({s_z}) = \left( {\begin{array}{*{20}{c}}
{ - i\sqrt 3 /2}\\
{1/2}
\end{array}} \right)$      
\choice $\psi ({s_z}) = \frac{1}{2}\left( {\begin{array}{*{20}{c}}
1\\
{i\sqrt 3 }
\end{array}} \right)$   
\choice $\psi ({s_z}) = \frac{1}{2}\left( {\begin{array}{*{20}{c}}
1\\
{\sqrt 3 }
\end{array}} \right)$
\end{choices}

\question Какая матрица (матрицы) отвечает эрмитовому оператору?
\begin{choices}
\choice $\left( {\begin{array}{*{20}{c}}
1&0\\
0&0
\end{array}} \right)$   
\choice $\left( {\begin{array}{*{20}{c}}
0&1\\
0&0
\end{array}} \right)$   
\choice $\left( {\begin{array}{*{20}{c}}
0&0\\
0&1
\end{array}} \right)$   
\choice $\left( {\begin{array}{*{20}{c}}
0&0\\
1&0
\end{array}} \right)$
\end{choices}

\question Какая матрица (матрицы) отвечает эрмитовому оператору?
\begin{choices}
\choice $\left( {\begin{array}{*{20}{c}}
i&0\\
0&i
\end{array}} \right)$   
\choice $\left( {\begin{array}{*{20}{c}}
0&i\\
{ - i}&0
\end{array}} \right)$   
\choice $\left( {\begin{array}{*{20}{c}}
i&0\\
0&{ - i}
\end{array}} \right)$   
\choice $\left( {\begin{array}{*{20}{c}}
0&i\\
i&0
\end{array}} \right)$
\end{choices}

\question Коммутатор $\left[ {{{\hat s}_x},\;{{\hat s}_z}} \right]$ равен
\begin{choices}
\choice $i{\hat s_x}$     
\choice $i{\hat s_y}$     
\choice $ - i{\hat s_y}$  
\choice ${\hat s^2}$
\end{choices}

\question Какой из перечисленных коммутаторов равен нулю?
\begin{choices}
\choice только $\left[ {{{\hat s}^2},\;{{\hat s}_ - }} \right]$        
\choice только $\left[ {{{\hat s}^2},\;{{\hat s}_ + }} \right]$
в. только $\left[ {{{\hat s}^2},\;{{\hat s}_z}} \right]$       
\choice все перечисленные
\end{choices}

\question Выбрать верное равенство
\begin{choices}
\choice $\left[ {{{\hat s}_ + },\;{{\hat s}_ - }} \right] = 2{\hat s_x}$  
\choice $\left[ {{{\hat s}_ + },\;{{\hat s}_ - }} \right] = 2{\hat s_y}$  
\choice $\left[ {{{\hat s}_ + },\;{{\hat s}_ - }} \right] = 2{\hat s_z}$  
\choice $\left[ {{{\hat s}_ + },\;{{\hat s}_ - }} \right] = 2{\hat s^2}$
\end{choices}

\question Спин частицы равен 1/2. Матрица оператора ${\hat s^2}$ – это
\begin{choices}
\choice $\frac{3}{4}\left( {\begin{array}{*{20}{c}}
1&0\\
0&1
\end{array}} \right)$      
\choice $\frac{3}{4}\left( {\begin{array}{*{20}{c}}
1&0\\
0&{ - 1}
\end{array}} \right)$      
\choice $\frac{1}{4}\left( {\begin{array}{*{20}{c}}
1&0\\
0&{ - 1}
\end{array}} \right)$      
\choice $\frac{1}{4}\left( {\begin{array}{*{20}{c}}
1&0\\
0&1
\end{array}} \right)$
\end{choices}

\question Спин частицы равен 1/2. Матрица оператора ${\hat s_z}$ в ${s_z}$-представлении – это
\begin{choices}
\choice $\frac{1}{2}\left( {\begin{array}{*{20}{c}}
1&0\\
0&1
\end{array}} \right)$      
\choice $\frac{1}{2}\left( {\begin{array}{*{20}{c}}
1&0\\
0&{ - 1}
\end{array}} \right)$      
\choice $\frac{1}{2}\left( {\begin{array}{*{20}{c}}
0&1\\
1&0
\end{array}} \right)$      
\choice $\frac{1}{2}\left( {\begin{array}{*{20}{c}}
0&{ - i}\\
i&0
\end{array}} \right)$
\end{choices}

\question Спин частицы равен 1. Матрица оператора ${\hat s^2}$ – это
\begin{choices}
\choice $\left( {\begin{array}{*{20}{c}}
2&0&0\\
0&2&0\\
0&0&2
\end{array}} \right)$      
\choice $\left( {\begin{array}{*{20}{c}}
1&0&0\\
0&0&0\\
0&0&1
\end{array}} \right)$      
\choice $\left( {\begin{array}{*{20}{c}}
1&0&0\\
0&0&0\\
0&0&{ - 1}
\end{array}} \right)$   
\choice $\left( {\begin{array}{*{20}{c}}
1&0&0\\
0&1&0\\
0&0&1
\end{array}} \right)$
\end{choices}

\question Спин частицы равен 1. Матрица оператора ${\hat s_z}$ в ${s_z}$-представлении – это
\begin{choices}
\choice $\left( {\begin{array}{*{20}{c}}
1&1&1\\
0&1&1\\
0&0&1
\end{array}} \right)$      
\choice $\left( {\begin{array}{*{20}{c}}
1&0&0\\
0&0&0\\
0&0&1
\end{array}} \right)$      
\choice $\left( {\begin{array}{*{20}{c}}
1&0&0\\
0&0&0\\
0&0&{ - 1}
\end{array}} \right)$   
\choice $\left( {\begin{array}{*{20}{c}}
1&0&0\\
0&1&0\\
0&0&1
\end{array}} \right)$
\end{choices}

\question Спин частицы равен 3/2. Матрица оператора ${\hat s^2}$ – это
\begin{choices}
\choice $\frac{9}{4}\left( {\begin{array}{*{20}{c}}
1&0&0&0\\
0&1&0&0\\
0&0&1&0\\
0&0&0&1
\end{array}} \right)$   
\choice $\frac{{15}}{4}\left( {\begin{array}{*{20}{c}}
3&0&0&0\\
0&1&0&0\\
0&0&{ - 1}&0\\
0&0&0&{ - 3}
\end{array}} \right)$   
\choice $\frac{{15}}{4}\left( {\begin{array}{*{20}{c}}
1&0&0&0\\
0&1&0&0\\
0&0&1&0\\
0&0&0&1
\end{array}} \right)$   
\choice $\frac{9}{4}\left( {\begin{array}{*{20}{c}}
3&0&0&0\\
0&1&0&0\\
0&0&{ - 1}&0\\
0&0&0&{ - 3}
\end{array}} \right)$
\end{choices}

\question Спин частицы равен 3/2. Матрица оператора ${\hat s_z}$ в ${s_z}$-представлении – это
\begin{choices}
\choice $\frac{3}{2}\left( {\begin{array}{*{20}{c}}
1&0&0&0\\
0&1&0&0\\
0&0&1&0\\
0&0&0&1
\end{array}} \right)$   
\choice $\frac{1}{2}\left( {\begin{array}{*{20}{c}}
3&0&0&0\\
0&1&0&0\\
0&0&{ - 1}&0\\
0&0&0&{ - 3}
\end{array}} \right)$      
\choice $\frac{1}{2}\left( {\begin{array}{*{20}{c}}
1&0&0&0\\
0&1&0&0\\
0&0&1&0\\
0&0&0&1
\end{array}} \right)$   
\choice $\frac{1}{2}\left( {\begin{array}{*{20}{c}}
3&0&0&0\\
0&1&0&0\\
0&0&1&0\\
0&0&0&3
\end{array}} \right)$
\end{choices}

\question Спин частицы равен $100$. Какова матрица оператора ${\hat s_z}$ в ${s_z}$-представлении?
\begin{choices}
\choice диагональная, размерности $100 \times 100$, на диагонали числа $\psi ({s_z}) = \left( {\begin{array}{*{20}{c}}
1\\
3
\end{array}} \right)$
\choice диагональная, размерности $101 \times 101$, на диагонали числа $100,\;99,\;...,\;0$
\choice диагональная, размерности $200 \times 200$, на диагонали числа $100,\;99,\;...,\; - 99,\; - 100$   
\choice диагональная, размерности $201 \times 201$, на диагонали числа $100,\;99,\;...,\; - 99,\; - 100$
\end{choices}

\question Спин частицы равен $100$. Какова матрица оператора ${\hat s^2}$?
\begin{choices}
\choice кратная единичной, размерности $200 \times 200$, множитель 10000
\choice кратная единичной, размерности $200 \times 200$, множитель 10100
\choice кратная единичной, размерности $201 \times 201$, множитель 10000  
\choice кратная единичной,  размерности $201 \times 201$, множитель 10100
\end{choices}

\question В результате действия на спиновую волновую функцию $\psi ({s_z}) = \left( {\begin{array}{*{20}{c}}
1\\
2
\end{array}} \right)$ оператора ${\hat s^2}$ получится следующая функция
\begin{choices}
\choice $\psi ({s_z}) = \left( {\begin{array}{*{20}{c}}
{3/4}\\
{ - 6/4}
\end{array}} \right)$   
\choice $\psi ({s_z}) = \left( {\begin{array}{*{20}{c}}
{3/4}\\
{6/4}
\end{array}} \right)$   
\choice $\psi ({s_z}) = \left( {\begin{array}{*{20}{c}}
{ - 3/4}\\
{6/4}
\end{array}} \right)$   
\choice $\psi ({s_z}) = \left( {\begin{array}{*{20}{c}}
{ - 3/4}\\
{ - 6/4}
\end{array}} \right)$
\end{choices}

\question Какая из четырех нижеприведенных матриц отвечает оператору ${\hat s_ + }$?
\begin{choices}
\choice $\left( {\begin{array}{*{20}{c}}
0&1\\
0&0
\end{array}} \right)$   
\choice $\left( {\begin{array}{*{20}{c}}
1&0\\
0&0
\end{array}} \right)$   
\choice $\left( {\begin{array}{*{20}{c}}
0&0\\
0&1
\end{array}} \right)$   
\choice $\left( {\begin{array}{*{20}{c}}
0&0\\
1&0
\end{array}} \right)$
\end{choices}

\question Какая из четырех нижеприведенных матриц отвечает оператору ${\hat s_ - }$?
\begin{choices}
\choice $\left( {\begin{array}{*{20}{c}}
0&1\\
0&0
\end{array}} \right)$      
\choice $\left( {\begin{array}{*{20}{c}}
1&0\\
0&0
\end{array}} \right)$      
\choice $\left( {\begin{array}{*{20}{c}}
0&0\\
0&1
\end{array}} \right)$      
\choice $\left( {\begin{array}{*{20}{c}}
0&0\\
1&0
\end{array}} \right)$
\end{choices}

\question Какая функция поучится в результате действия оператора, повышающего проекцию спина час-тицы на ось $z$, на функцию $\psi ({s_z}) = \left( {\begin{array}{*{20}{c}}
1\\
0
\end{array}} \right)$?
\begin{choices}
\choice $\psi ({s_z}) = \left( {\begin{array}{*{20}{c}}
0\\
0
\end{array}} \right)$   
\choice $\psi ({s_z}) \sim \left( {\begin{array}{*{20}{c}}
0\\
1
\end{array}} \right)$   
\choice $\psi ({s_z}) \sim \left( {\begin{array}{*{20}{c}}
1\\
1
\end{array}} \right)$   
\choice $\psi ({s_z}) \sim \left( {\begin{array}{*{20}{c}}
1\\
0
\end{array}} \right)$
\end{choices}

\question Какая функция получится при действии оператора ${\hat s_ + }$ на функцию $\psi ({s_z}) = \frac{1}{{\sqrt 2 }}\left( {\begin{array}{*{20}{c}}
1\\
1
\end{array}} \right)$?
\begin{choices}
\choice $\frac{1}{{\sqrt 2 }}\left( {\begin{array}{*{20}{c}}
0\\
1
\end{array}} \right)$   
\choice $\frac{1}{{\sqrt 2 }}\left( {\begin{array}{*{20}{c}}
1\\
0
\end{array}} \right)$   
\choice $\frac{1}{{\sqrt 2 }}\left( {\begin{array}{*{20}{c}}
1\\
{ - 1}
\end{array}} \right)$   
\choice на такую функцию оператор ${\hat s_ + }$ не действует
\end{choices}

\question Спин частицы равен 1/2. Какая из нижеприведенных матриц является матрицей оператора ${\hat s_y}$ в ${s_z}$-представлении?
\begin{choices}
\choice $\frac{1}{2}\left( {\begin{array}{*{20}{c}}
0&1\\
{ - 1}&0
\end{array}} \right)$      
\choice $\frac{1}{2}\left( {\begin{array}{*{20}{c}}
1&0\\
0&{ - 1}
\end{array}} \right)$      
\choice $\frac{1}{2}\left( {\begin{array}{*{20}{c}}
0&{ - i}\\
i&0
\end{array}} \right)$      
\choice $\frac{1}{2}\left( {\begin{array}{*{20}{c}}
{ - i}&0\\
0&i
\end{array}} \right)$
\end{choices}

\question Спин частицы равен 1/2. Оператор квадрата проекции спина на ось $y$ равен
\begin{choices}
\choice $\frac{1}{4}\left( {\begin{array}{*{20}{c}}
0&1\\
1&0
\end{array}} \right)$      
\choice $\frac{1}{4}\left( {\begin{array}{*{20}{c}}
1&0\\
0&{ - 1}
\end{array}} \right)$      
\choice $\frac{1}{4}\left( {\begin{array}{*{20}{c}}
0&{ - 1}\\
1&0
\end{array}} \right)$      
\choice $\frac{1}{4}\left( {\begin{array}{*{20}{c}}
1&0\\
0&1
\end{array}} \right)$
\end{choices}

\question Спин частицы равен 1/2. Какие из перечисленных функций являются собственными функциями оператора ${\hat s_y}^2$?
\begin{choices}
\choice $\left( {\begin{array}{*{20}{c}}
1\\
0
\end{array}} \right)$ и $\left( {\begin{array}{*{20}{c}}
0\\
1
\end{array}} \right)$      
\choice $\left( {\begin{array}{*{20}{c}}
1\\
1
\end{array}} \right)$ и $\left( {\begin{array}{*{20}{c}}
1\\
{ - 1}
\end{array}} \right)$      
\choice $\left( {\begin{array}{*{20}{c}}
1\\
i
\end{array}} \right)$ и $\left( {\begin{array}{*{20}{c}}
1\\
{ - i}
\end{array}} \right)$      
\choice любые двухстрочные столбцы
\end{choices}

\question Какая из четырех ниже перечисленных матриц является одной из матриц Паули?
\begin{choices}
\choice $\left( {\begin{array}{*{20}{c}}
0&1\\
{ - 1}&0
\end{array}} \right)$      
\choice $\left( {\begin{array}{*{20}{c}}
i&0\\
0&{ - i}
\end{array}} \right)$      
\choice $\left( {\begin{array}{*{20}{c}}
0&{ - i}\\
i&0
\end{array}} \right)$      
\choice $\left( {\begin{array}{*{20}{c}}
0&i\\
i&0
\end{array}} \right)$
\end{choices}

\question Какая из четырех ниже перечисленных матриц является матрицей Паули ${\sigma _z}$
\begin{choices}
\choice $\left( {\begin{array}{*{20}{c}}
0&1\\
{ - 1}&0
\end{array}} \right)$      
\choice $\left( {\begin{array}{*{20}{c}}
1&0\\
0&{ - 1}
\end{array}} \right)$      
\choice $\left( {\begin{array}{*{20}{c}}
0&{ - i}\\
i&0
\end{array}} \right)$      
\choice $\left( {\begin{array}{*{20}{c}}
0&i\\
i&0
\end{array}} \right)$
\end{choices}

\question Какая матрица отвечает оператору ${\hat s_x}{\hat s_y} - {\hat s_y}{\hat s_x}$
\begin{choices}
\choice $\frac{1}{2}\left( {\begin{array}{*{20}{c}}
i&0\\
0&{ - i}
\end{array}} \right)$      
\choice $\frac{1}{2}\left( {\begin{array}{*{20}{c}}
0&i\\
{ - i}&0
\end{array}} \right)$      
\choice $\frac{1}{2}\left( {\begin{array}{*{20}{c}}
i&0\\
0&i
\end{array}} \right)$      
\choice $\frac{1}{2}\left( {\begin{array}{*{20}{c}}
0&i\\
i&0
\end{array}} \right)$
\end{choices}

\question Какая матрица отвечает оператору ${\hat s_x}{\hat s_y} + {\hat s_y}{\hat s_x}$
\begin{choices}
\choice $\left( {\begin{array}{*{20}{c}}
1&0\\
0&1
\end{array}} \right)$      
\choice $\left( {\begin{array}{*{20}{c}}
0&0\\
0&0
\end{array}} \right)$      
\choice $\left( {\begin{array}{*{20}{c}}
1&1\\
1&1
\end{array}} \right)$      
\choice $\left( {\begin{array}{*{20}{c}}
i&0\\
0&i
\end{array}} \right)$
\end{choices}

\question Из четырех нижеприведенных спиновых функций только одна является собственной функцией оператора ${\hat s_x}$. Какая?
\begin{choices}
\choice $\psi ({s_z}) = \left( {\begin{array}{*{20}{c}}
1\\
2
\end{array}} \right)$   
\choice $\psi ({s_z}) = \left( {\begin{array}{*{20}{c}}
i\\
1
\end{array}} \right)$      
\choice $\psi ({s_z}) = \left( {\begin{array}{*{20}{c}}
1\\
{ - 1}
\end{array}} \right)$   
\choice $\psi ({s_z}) = \left( {\begin{array}{*{20}{c}}
1\\
0
\end{array}} \right)$
\end{choices}

\question Из четырех нижеприведенных состояний только в одном величина ${s_x}$ имеет определен-ное значение. В каком?
\begin{choices}
\choice $\psi ({s_z}) = \left( {\begin{array}{*{20}{c}}
1\\
0
\end{array}} \right)$   
\choice $\psi ({s_z}) = \left( {\begin{array}{*{20}{c}}
1\\
2
\end{array}} \right)$   
\choice $\psi ({s_z}) = \left( {\begin{array}{*{20}{c}}
1\\
3
\end{array}} \right)$   
\choice $\psi ({s_z}) = \left( {\begin{array}{*{20}{c}}
1\\
1
\end{array}} \right)$
\end{choices}

\question Какая из ниже перечисленных функций является собственной функцией оператора ${\hat s_y}$
\begin{choices}
\choice $\psi ({s_z}) = \left( {\begin{array}{*{20}{c}}
1\\
i
\end{array}} \right)$   
\choice $\psi ({s_z}) = \left( {\begin{array}{*{20}{c}}
{ - i}\\
i
\end{array}} \right)$   
\choice $\psi ({s_z}) = \left( {\begin{array}{*{20}{c}}
1\\
{ - 1}
\end{array}} \right)$   
\choice $\psi ({s_z}) = \left( {\begin{array}{*{20}{c}}
1\\
1
\end{array}} \right)$
\end{choices}

\question Из четырех нижеприведенных состояний только в одном величина ${s_y}$ имеет определен-ное значение. В каком?
\begin{choices}
\choice $\psi ({s_z}) = \left( {\begin{array}{*{20}{c}}
i\\
i
\end{array}} \right)$   
\choice $\psi ({s_z}) = \left( {\begin{array}{*{20}{c}}
{ - 1}\\
i
\end{array}} \right)$   
\choice $\psi ({s_z}) = \left( {\begin{array}{*{20}{c}}
1\\
1
\end{array}} \right)$   
\choice $\psi ({s_z}) = \left( {\begin{array}{*{20}{c}}
1\\
0
\end{array}} \right)$
\end{choices}

\question Чему равно среднее значение проекции спина на ось $x$ в состоянии $\psi ({s_z}) = \left( {\begin{array}{*{20}{c}}
0\\
1
\end{array}} \right)$
\begin{choices}
\choice ${\bar s_x} =  - 1/4$      
\choice ${\bar s_x} = 1/4$      
\choice ${\bar s_x} = 1/3$      
\choice ${\bar s_x} = 0$
\end{choices}

\question Чему равно среднее значение проекции спина на ось $x$ в состоянии $\psi ({s_z}) = \left( {\begin{array}{*{20}{c}}
{\sqrt 3 /2}\\
{1/2}
\end{array}} \right)$
\begin{choices}
\choice ${\bar s_x} = \sqrt 3 /8$     
\choice ${\bar s_x} =  - \sqrt 3 /8$     
\choice ${\bar s_x} = \sqrt 3 /4$     
\choice ${\bar s_x} =  - \sqrt 3 /4$
\end{choices}

\question Чему равно среднее значение проекции спина на ось $y$ в состоянии $\psi ({s_z}) = \frac{1}{{\sqrt 2 }}\left( {\begin{array}{*{20}{c}}
1\\
1
\end{array}} \right)$
\begin{choices}
\choice ${\bar s_y} =  - 1/4$      
\choice ${\bar s_y} = 1/4$      
\choice ${\bar s_y} = 1/3$      
\choice ${\bar s_y} = 0$
\end{choices}

\question В результате действия на спиновую функцию $\psi ({s_z}) = \left( {\begin{array}{*{20}{c}}
1\\
2
\end{array}} \right)$ оператора ${\hat s_x}$ получится следующая функция
\begin{choices}
\choice $\psi ({s_z}) = \left( {\begin{array}{*{20}{c}}
1\\
{1/2}
\end{array}} \right)$   
\choice $\psi ({s_z}) = \left( {\begin{array}{*{20}{c}}
1\\
{1/4}
\end{array}} \right)$   
\choice $\psi ({s_z}) = \left( {\begin{array}{*{20}{c}}
1\\
2
\end{array}} \right)$   
\choice $\psi ({s_z}) = \left( {\begin{array}{*{20}{c}}
1\\
4
\end{array}} \right)$
\end{choices}

\question В результате действия на спиновую функцию $\psi ({s_z}) = \left( {\begin{array}{*{20}{c}}
1\\
2
\end{array}} \right)$ оператора ${\hat s_y}$ получится следующая функция
\begin{choices}
\choice $\psi ({s_z}) = \left( {\begin{array}{*{20}{c}}
i\\
{ - i/2}
\end{array}} \right)$   
\choice $\psi ({s_z}) = \left( {\begin{array}{*{20}{c}}
{ - i}\\
{i/2}
\end{array}} \right)$   
\choice $\psi ({s_z}) = \left( {\begin{array}{*{20}{c}}
{ - i/2}\\
i
\end{array}} \right)$   
\choice $\psi ({s_z}) = \left( {\begin{array}{*{20}{c}}
{i/2}\\
{ - i}
\end{array}} \right)$
\end{choices}

\question Какое из нижеприведенных утверждений относительно свойств оператора ${\hat s_x}{\hat s_y}$ является верным?
\begin{choices}
\choice он неэрмитов            
\choice он унитарен 
\choice он совпадает со своим обратным   
\choice он нелинеен
\end{choices}

\question Спиновая функция частицы имеет вид $\psi ({s_z}) = \left( {\begin{array}{*{20}{c}}
1\\
1
\end{array}} \right)$. Будет ли это состояние стационарным?
\begin{choices}
\choice да
\choice нет      
\choice мало информации для ответа
\choice зависит от того, коммутирует ли оператор ${\hat s_y}$ с оператором Гамильтона, или нет
\end{choices}

\question Частица имеет спин 1/2. Гамильтониан частицы не зависит от спиновых переменных. Какая мат-рица – А, Б, В или Г – отвечает гамильтониану (здесь $\hat h(\vec r)$ действует только на пространственные переменные)?
\begin{choices}
\choice $\hat H(\vec r,{s_z}) = \hat h(\vec r)\left( {\begin{array}{*{20}{c}}
1&0\\
0&1
\end{array}} \right)$         
\choice $\hat H(\vec r,{s_z}) = \hat h(\vec r)\left( {\begin{array}{*{20}{c}}
1&1\\
1&1
\end{array}} \right)$
\choice $\hat H(\vec r,{s_z}) = \hat h(\vec r)\left( {\begin{array}{*{20}{c}}
1&0\\
0&{ - 1}
\end{array}} \right)$         
\choice $\hat H(\vec r,{s_z}) = \hat h(\vec r)\left( {\begin{array}{*{20}{c}}
0&1\\
1&0
\end{array}} \right)$ 
\end{choices}

\question Гамильтониан частицы со спином 1/2 не зависит от спиновых переменных. Какой является спи-новая часть собственных функций гамильтониана?
\begin{choices}
\choice только $\left( {\begin{array}{*{20}{c}}
1\\
0
\end{array}} \right)$   
\choice только $\left( {\begin{array}{*{20}{c}}
0\\
1
\end{array}} \right)$      
\choice только $\left( {\begin{array}{*{20}{c}}
1\\
1
\end{array}} \right)$      
\choice любым двухстрочным столбцом
\end{choices}

\end{questions}

\section{ ГЛАВА 6. КВАЗИКЛАССИЧЕСКОЕ ПРИБЛИЖЕНИЕ }
(во всех задачах этой главы: $m$ - масса частицы, $E$ - энергия, при которой решается уравнение Шредингера, $U(x)$-потенциальная энергия частицы, $k(x) = \sqrt {2m\left( {E - U(x)} \right)/{\hbar ^2}} $)

\begin{questions}

\question Квазиклассическое приближение это
\begin{choices}
\choice метод перехода от квантовой механики к механике классической
\choice приближение, в котором оператор импульса заменяется на импульс
\choice метод приближенного решения стационарного уравнения Шредингера, основанный на «плавности» потенциала как функции координаты
\choice метод приближенного решения временного уравнения Шредингера, основанный на «плавно-сти» волновой функции системы как функции времени
\end{choices}

\question Квазиклассическое приближение работает, если 
\begin{choices}
\choice потенциальная энергия является резкой функцией координаты
\choice потенциальная энергия является плавной функцией координаты
\choice потенциальная энергия является маленькой
\choice все перечисленное неверно
\end{choices}

\question Квазиклассическое приближение работает, если 
\begin{choices}
\choice потенциальная энергия является резкой функцией времени
\choice потенциальная энергия является плавной функцией времени
\choice потенциальная энергия является маленькой
\choice все перечисленное неверно
\end{choices}

\question Квазиклассическое приближение работает, если 
\begin{choices}
\choice де-бройлевская длина волны частицы большая по сравнению с областью действия потенциала
\choice де-бройлевская длина волны частицы маленькая по сравнению с областью действия потенциа-ла
\choice де-бройлевская длина волны частицы мало меняется на расстоянии, равном ей самой
\choice все перечисленное неверно
\end{choices}

\question Квазиклассическое приближение работает, когда действие $S$, которое имела бы частица, если бы она двигалась по законам классической механики, удовлетворяло следующему неравенст-ву
\begin{choices}
\choice $S \ll \hbar $       
\choice $S \gg \hbar $    
\choice $S \gg m$      
\choice $S \ll E$
\end{choices}

\question Условие применимости квазиклассического приближения часто записывают в виде $\frac{{d\lambda (x)}}{{dx}} \ll 1$. Что означает в этой формуле $\lambda (x)$?
\begin{choices}
\choice длина волны света, который излучает частица в точке с координатой $x$
\choice длина волны света, который имеет ту же энергию, что и частица в точке с координатой $x$
\choice $\lambda (x) = \frac{\hbar }{{p(x)}} = \frac{\hbar }{{\sqrt {2m(E - U(x))} }}$ - де-бройлевская длина волны частицы, выраженная через классический импульс в точке с координатой $x$
\choice $\lambda (x) = \frac{{p(x)}}{\hbar } = \frac{{\sqrt {2m(E - U(x))} }}{\hbar }$ - де-бройлевская длина волны частицы, выраженная через классический импульс в точке с координатой $x$
\end{choices}

\question Постоянная Планка является маленькой (Л.Д.Ландау, Е.М.Лифшиц «Курс теоретической фи-зики», т. 3, Квантовая механика, М., Наука: 1ююю, стр. …). По сравнению с чем должна быть мала постоянная Планка?
\begin{choices}
\choice 1
\choice 2
\end{choices}

\question Какой из нижеследующих формул – А., Б., 
\begin{choices}
\choice или 
\choice – определяется параметр квазиклассично-сти? 
\choice $\left| {\frac{{k'(x)}}{{{k^2}(x)}}} \right|$      
\choice $\left| {\frac{{{k^2}(x)}}{{k'(x)}}} \right|$      
\choice $\left| {\frac{{k'(x)}}{{{k^3}(x)}}} \right|$      
\choice $\left| {\frac{{{k^3}(x)}}{{k'(x)}}} \right|$
\end{choices}

\question Какова размерность параметра квазиклассичности $|k'(x)/{k^2}(x)|$
\begin{choices}
\choice       
\choice       
\choice       
\choice 
\end{choices}

\question Чему равно значение параметра квазиклассичности при таких значениях координаты, где $E = U(x)$?
\begin{choices}
\choice 0     
\choice 1     
\choice $\infty $      
\choice -1
\end{choices}

\question График зависимости потенциальной энергии от координаты приведен на рисунке. В какой точке - ${x_1}$ или ${x_2}$ лучше работает квазиклассическое приближение?
\begin{choices}
\choice в точке ${x_1}$
\choice в точке ${x_2}$
\choice по рисунку это определить невозможно
\choice это зависит от энергии, при которой решается уравнение Шредингера
\end{choices}

\question График потенциальной энергии частицы имеет вид, показанный на рисунке. Уравнение Шредингера решается при двух энергиях - ${E_1}$ и ${E_2}$ (показаны на рисунке) При какой энергии лучше работает квазиклассическое приближение?
\begin{choices}
\choice лучше при энергии ${E_1}$
\choice лучше при энергии ${E_2}$
\choice безразлично
\choice от энергии точность квазиклассического приближения не зависит
\end{choices}

\question Уравнение Шредингера решается при некоторой энергии $E$ (показана на рисунке) в двух потенциалах, графики которых показаны на рисунке (один потенциал – сплошной линией, вто-рой – пунктиром). Для какого потенциала – «сплошного» или «пунктирного» – лучше работает квазиклассическое приближение?
\begin{choices}
\choice лучше для «сплошного»      
\choice лучше для «пунктирного»
\choice одинаково            
\choice зависит от энергии
\end{choices}

\question Уравнение Шредингера решается в окрестности точки, в которой вторая производная потен-циала равна нулю $V''(x) = 0$. Что можно сказать о возможностях квазиклассического приближе-ния в этой области?
\begin{choices}
\choice хорошо работает, поскольку параметр квазиклассичночсти пропорционален $V''(x)$
\choice плохо работает, поскольку параметр квазиклассичночсти пропорционален $1/V''(x)$
\choice точность квазиклассического приближения не связана со второй производной потенциала
\choice не работает вообще
\end{choices}

\question Уравнение Шредингера решается в окрестности точки, в которой потенциал имеет макси-мум, при энергии, отличающейся от значения потенциала в точке максимума. Что можно сказать о возможностях квазиклассического приближения в этой области?
\begin{choices}
\choice оно хорошо работает
\choice оно плохо работает
\choice точность квазиклассического приближения не связана с максимальностью потенциала  
\choice не работает вообще
\end{choices}

\question Частица движется в потенциале $U(x) = {U_0} = const$. Чему равно значение параметра ква-зиклассичности для этой частицы?
\begin{choices}
\choice 0     
\choice $\frac{{{U_0}}}{E}$     
\choice $\frac{{{U_0}}}{{{U_0} - E}}$    
\choice $\frac{{{U_0} - E}}{{{U_0}}}$ 
\end{choices}

\question Частица движется в потенциале $U(x) = \frac{\alpha }{{{x^2}}}$. Каким является параметр квазиклассичности при нулевой энергии частицы?
\begin{choices}
\choice $\sqrt {\left| {\frac{{{\hbar ^2}}}{{2m\alpha }}} \right|} $      
\choice $\sqrt {\left| {\frac{{2m\alpha }}{{{\hbar ^2}}}} \right|} $      
\choice $\left| {\frac{{{\hbar ^2}}}{{2m\alpha }}} \right|$      
\choice $\left| {\frac{{2m\alpha }}{{{\hbar ^2}}}} \right|$
\end{choices}

\question Частица движется в потенциале $U(x) = \frac{\alpha }{{{x^2}}}$. При каких значениях коор-динаты лучше работает квазиклассическое приближение, если энергия частицы равна нулю?
\begin{choices}
\choice при малых, так как потенциальная энергия резкая функция координаты при малых $x$
\choice при больших, так как потенциальная энергия плавная функция при больших $x$
\choice при любых одинаково, так как параметр квазиклассичости не зависит от координат
\choice это зависит от $\alpha $
\end{choices}

\question Частица движется в потенциале $U(x) = \frac{\alpha }{{{x^2}}}$. Каким является параметр квазиклассичности при энергии ?
\begin{choices}
\choice $\sqrt {\left| {\frac{{{\hbar ^2}}}{{2m\alpha }}} \right|} $      
\choice $\sqrt {\left| {\frac{{2m\alpha }}{{{\hbar ^2}}}} \right|} $      
\choice $\left| {\frac{{{\hbar ^2}}}{{2m\alpha }}} \right|$      
\choice $\left| {\frac{{2m\alpha }}{{{\hbar ^2}}}} \right|$
\end{choices}

\question Уравнение Шредингера решается при энергии $E$ в потенциалах, изображенных на рисун-ках (энергия отложена черточкой на оси ординат). Для какого случая можно ожидать лучшей ра-боты квазиклассического приближения?
\begin{choices}
\choice для левого        
\choice для правого
\choice одинаково         
\choice мало информации для ответа
\end{choices}

\question Какая из функций – А, Б, В или Г – является квазиклассическим решением стационарного уравнения Шредингера в первом порядке по параметру квазиклассичности, когда $E > U(x)$ (здесь $C$ и $a$ - постоянные)?  
\begin{choices}
\choice $C\exp \left( { - ik(x)x} \right)$   
\choice $C\exp \left( { - i{k^2}(x)x} \right)$    
\choice $C\exp \left( { - i\int\limits_a^x {k(t)dt} } \right)$       
\choice $C\exp \left( { - \int\limits_a^x {|k(t)|dt} } \right)$
\end{choices}

\question Какая из функций – А, Б, В или Г – является квазиклассическим решением стационарного уравнения Шредингера во втором порядке по параметру квазиклассичности, когда $E > U(x)$ (здесь $C$ и $a$ - постоянные)? 
\begin{choices}
\choice $\frac{C}{{\sqrt {k(x)} }}\exp \left( { - ik(x)x} \right)$         
\choice $C\sqrt {k(x)} \exp \left( { - i\int\limits_a^x {k(t)dt} } \right)$  
\choice $\frac{C}{{\sqrt {k(x)} }}\exp \left( { - i\int\limits_a^x {k(t)dt} } \right)$          
\choice $\frac{C}{{\sqrt {|k(x)|} }}\exp \left( { - \int\limits_a^x {|k(t)|dt} } \right)$
\end{choices}

\question Какая из функций – А., Б., или – является квазиклассическим решением стационарного уравнения Шредингера в первом порядке по параметру квазиклассичности, когда $E < U(x)$ (здесь $C$ и $a$ - постоянные)? 
\begin{choices} 
\choice $C\exp \left( { - \int\limits_a^x {|k(t)|dt} } \right)$  
\choice $C\exp \left( { - i\int\limits_a^x {k(t)dt} } \right)$      
\choice $C\exp \left( { - {k^2}(x)x} \right)$  
\choice $C\exp \left( { - |k(x)|x} \right)$
\end{choices}

\question Если квазиклассическое решение уравнения Шредингера $C\exp \left( { - i\int\limits_a^x {k(t)dt} } \right)$ ($E > U(x)$, первый порядок по параметру квазиклассичности) умножить на функцию $1/\sqrt {k(x)} $, то  
\begin{choices}
\choice эта функция перестанет быть решением
\choice изменится начало отсчета координаты
\choice эта функция останется решением в том же порядке по параметру квазиклассичности
\choice будет учитываться следующий порядок по параметру квазиклассичности
\end{choices}

\question Если квазиклассическое решение уравнения Шредингера $C\exp \left( { - i\int\limits_a^x {k(t)dt} } \right)$ ($E > U(x)$, первый порядок по параметру квазиклассичности) умножить на функцию $\sqrt {k(x)} $, то  
\begin{choices}
\choice эта функция перестанет быть решением
\choice изменится начало отсчета координаты
\choice эта функция останется решением в том же порядке по параметру квазиклассичности
\choice будет учитываться следующий порядок по параметру квазиклассичности
\end{choices}

\question К чему приведет изменение нижнего предела интегрирования $a \to a'$ в общем квазиклассическом решении уравнения Шредингера $C\sin \left( {\int\limits_a^x {k(t)dt} } \right) + D\cos \left( {\int\limits_a^x {k(t)dt} } \right)$
\begin{choices}
\choice к тому, что эта функция перестанет быть решением
\choice к изменению произвольных постоянных $C$ и $D$
\choice к выходу в область неквазиклассичности
\choice к изменению начала отсчета времени
\end{choices}

\question Частица массой $m$ движется в потенциале $U(x) = \alpha x$. Какой формулой определяется квазиклассическое решение уравнение Шредингера при энергии $E$ (в области, где $E > \alpha x$)?
\begin{choices}
\choice $C\exp \left( { \pm ib{{(E - \alpha x)}^{3/2}}} \right)$    
\choice $C\exp \left( { \pm ib{{(E - \alpha x)}^{5/2}}} \right)$
\choice $C\exp \left( { \pm ib{{(E - \alpha x)}^{7/2}}} \right)$    
\choice $C\exp \left( { \pm ib{{(E - \alpha x)}^{9/2}}} \right)$ 
(здесь $C$ - произвольная постоянная, $b = \sqrt {\frac{{4m}}{{3{\alpha ^2}{\hbar ^2}}}} $)
\end{choices}

\question Частица движется в потенциале   ($\alpha  > 0$) . Какой формулой определяются квазиклассические решения уравнения Шредингера (в первом порядке по параметру квазиклас-сичности) при энергии частицы, равной нулю?
\begin{choices}
\choice $\sin \alpha x$ и $\cos \alpha x$     
\choice ${e^{ - \alpha x}}$ и ${e^{\alpha x}}$      
\choice $|x{|^{i\sqrt \alpha  }}$ и $|x{|^{ - i\sqrt \alpha  }}$      
\choice $|x{|^\alpha }$ и $|x{|^{ - \alpha }}$
\end{choices}

\question Частица движется в потенциале   ($\alpha  > 0$) . Какой формулой определяются квазиклассические решения уравнения Шредингера (во втором порядке по параметру квазиклас-сичности) при энергии частицы, равной нулю?
\begin{choices}
\choice $\sin \alpha x$ и $\cos \alpha x$     
\choice ${e^{ - \alpha x}}$ и ${e^{\alpha x}}$      
\choice $|x{|^{i\sqrt \alpha  }}$ и $|x{|^{ - i\sqrt \alpha  }}$      
\choice $|x{|^\alpha }$ и $|x{|^{ - \alpha }}$
\end{choices}

\question Для каких потенциалов квазиклассическое решение уравнения Шредингера совпадает с точ-ным?
\begin{choices}
\choice только для $U(x) = const$
\choice только для: $U(x) = const$ и $U(x) \sim {x^2}$
\choice только для:$U(x) = const$, $U(x) \sim {x^2}$ и $U(x) \sim x$
\choice ни для какого из этих потенциалов
\end{choices}

\question Уравнение Шредингера решается в потенциале $U(x)$ при энергии $E$. Из какого из ниже-следующих уравнений можно найти такие значения координат, при которых квазиклассическое приближение заведомо не работает?
\begin{choices}
\choice $E = U'(x)$    
\choice $\sqrt {\frac{{2m}}{{{\hbar ^2}}}} {E^{3/2}} = U(x)$     
\choice $E = U(x)$     
\choice $\sqrt {\frac{{{\hbar ^2}}}{{2m}}} {E^{ - 3/2}} = U'(x)$
\end{choices}

\question График потенциальной энергии частицы имеет вид, показанный на рисунке. Уравнение Шредингера решается при такой энергии $E$, которая отложена на рисунке горизонтальной пунктирной прямой в масштабе, принятом для оси ординат. При каких значениях координат можно ожидать, что квазиклассическое приближение будет работать?
\begin{choices}
\choice точность квазиклассического приближения от координат не зависит
\choice только в области 
\choice только в областях $a < x < b,\quad c < x < d$
\choice только в областях $x < a,\quad b < x < c,\quad x > d$
\end{choices}

\question График потенциальной энергии частицы имеет вид, показанный на рисунке («потенциальные стенки» при $x = a$ и $x = b$вертикальны). Будет ли квазиклассическое решение уравнения Шредингера при энергии $E$ (отложена на рисунке горизонтальной пунктирной прямой в масштабе, принятом для оси ординат) работать в малых окрестностях точек поворота $x \ge a$ и $x \le b$?
\begin{choices}
\choice да
\choice нет
\choice в окрестности точки $x = a$ да, в окрестности $x = b$ - нет
\choice в окрестности точки $x = b$ да, в окрестности $x = a$ - нет
\end{choices}

\question График потенциальной энергии частицы имеет вид, показанный на рисунке (в точке $x = a$ потенциал имеет вертикальную стенку конечной высоты). Энергия, для которой решается уравнение, показана на рисунке пунктирной горизонтальной прямой. Как установить условия «сшивки» квазиклассических решений справа и слева от точки поворота $x = a$?
\begin{choices}
\choice обходя точку поворота в комплексной плоскости энергии
\choice заменяя потенциал линейной функцией 
\choice приравнивая значения квазиклассических функций и их производные в самой точке поворота
\choice при таком поведении потенциала «сшить» квазиклассические функции невозможно 
\end{choices}

\question Из квазиклассических решений уравнения Шредингера следует, что решение при $E > U(x)$ является
\begin{choices}
\choice растущей или затухающей функцией    
\choice осциллирующей функцией
\choice постоянной              
\choice это зависит от $E$
\end{choices}

\question Из квазиклассических решений уравнения Шредингера следует, что решение при $E < U(x)$ является
\begin{choices}
\choice растущей или затухающей функцией
\choice осциллирующей функцией
\choice постоянной
\choice это зависит от $E$
\end{choices}

\question Являются ли классические точки остановки особыми точками точных решений уравнения Шредингера? 
\begin{choices}
\choice да, так как в этих точках $k(x) = 0$, а $k$ входит в знаменатель квазииклассических решений
\choice нет, так как в этих точках квазиклассические решения не имеют ничего общего с точными
\choice это зависит от энергии
\choice это зависит от поведения потенциальной энергии в окрестностях точек поворота 
\end{choices}

\question Что позволяют найти «условия сшивки» квазиклассических функций?
\begin{choices}
\choice соотношения между постоянными в квазиклассическом решении справа и слева от точек оста-новки классического движения
\choice связывать квазиклассическое решение вдали от точек остановки с точным решением в окрест-ности точек остановки
\choice связывать решения справа и слева от особых точек потенциала
\choice находить значения квазиклассических функций в точках остановки
\end{choices}

\question Какая формула представляет собой условие сшивки квазиклассических функций слева и справа от точки остановки ?
\begin{choices}
\choice $\frac{{2C}}{{\sqrt {k(x)} }}\cos \left( {\int\limits_a^x {k(t)dt}  + \frac{\pi }{4}} \right) \leftrightarrow \frac{C}{{\sqrt {k(x)} }}\sin \left( {\int\limits_x^a {k(t)dt} } \right)$    
\choice $\frac{{2C}}{{\sqrt {k(x)} }}{e^{\left( { - \int\limits_a^x {k(t)dt} } \right)}} \leftrightarrow \frac{C}{{\sqrt {k(x)} }}{e^{\left( {\int\limits_x^a {k(t)dt} } \right)}}$
\choice $\frac{{2C}}{{\sqrt {k(x)} }}\cos \left( {\int\limits_x^a {k(t)dt - } \frac{\pi }{4}} \right) \leftrightarrow \frac{C}{{\sqrt {|k(x)|} }}{e^{\left( { - \int\limits_a^x {|k(t)|dt} } \right)}}$    
\choice $\frac{{2C}}{{\sqrt {k(x)} }}{e^{\left( { - i\int\limits_a^x {k(t)dt} } \right)}} \leftrightarrow \frac{C}{{\sqrt {k(x)} }}{e^{\left( {i\int\limits_x^a {k(t)dt} } \right)}}$
\end{choices}

\question Квазиклассическое правило квантования дает возможность с помощью квазиклассического решения стационарного уравнения Шредингера найти:
\begin{choices}
\choice собственные функции оператора Гамильтона, отвечающие непрерывному спектру
\choice собственные значения оператора Гамильтона, отвечающие непрерывному спектру
\choice собственные функции оператора Гамильтона, отвечающие дискретному спектру
\choice собственные значения оператора Гамильтона, отвечающие дискретному спектру
\end{choices}

\question Для каких уровней энергии выше точность квазиклассического правила квантования?
\begin{choices}
\choice с маленькими квантовыми числами
\choice с большими квантовыми числами
\choice для уровней, энергия которых много больше постоянной Планка
\choice для уровней, энергия которых много меньше постоянной Планка
\end{choices}

\question С помощью квазиклассического правила квантования найдены энергии стационарных состояний в потенциале, график которого изображен на рисунке. Эти энергии показаны на рисунке с помощью горизонтальных отрезков на оси ординат. Для какого из показанных уровней следует ожидать лучшего совпадения квазиклассического результата с точным?
\begin{choices}
\choice для основного состояния    
\choice для второго уровня
\choice для четвертого       
\choice для шестого
\end{choices}

\question Какое из нижеследующих равенств является правилом квантования Бора-Зоммерфельда
\begin{choices}
\choice $\int\limits_a^b {\frac{1}{{{k^2}(x)}}dx \simeq \pi n} $ 
\choice $\int\limits_a^b {{k^2}(x)dx \simeq \pi n} $ 
\choice $\int\limits_a^b {k(x)dx \simeq \pi n} $  
\choice $\int\limits_a^b {\frac{1}{{k(x)}}dx \simeq \pi n} $
(здесь $a$ и $b$ - классические точки поворота при искомой энергии $E$).
\end{choices}

\question Квазиклассическое правило квантования $\int\limits_a^b {k(x)dx \simeq \pi n} $ (где $a$ и $b$ - классические точки поворота) является уравнением, из которого можно найти собственные зна-чения оператора Гамильтона. В какие из нижеследующих величин входят искомые собственные значения? (считать, что график зависимости потенциальной энергии не имеет «вертикальных» стенок)
\begin{choices}
\choice только в $k(x)$         
\choice только в $k(x)$ и $a$
\choice и в $k(x)$, и в $a$, и в $b$     
\choice ни в одну из перечисленных величин
\end{choices}

\question Уравнение Шредингера решается в следующем потенциале: $U(x) = \infty $ при $x < a$, $U(x)$ - некоторая плавная функция координат при $x > a$ (см. рисунок; энергия $E$, для которой решается уравнение также отложена на рисунке). Какой функцией определяется в области далекой от точки поворота при $x > a$ та квазиклассическая функция, которая является хорошим приближением решения, удовлетворяющего правильному граничному условию при $x = a$?
\begin{choices}
\choice $\frac{1}{{\sqrt {k(x)} }}\cos \left( {\int\limits_a^x {k(t)dt} } \right)$    
\choice $\frac{1}{{\sqrt {k(x)} }}\sin \left( {\int\limits_a^x {k(t)dt} } \right)$
\choice $\frac{1}{{\sqrt {k(x)} }}\exp \left( {i\int\limits_a^x {k(t)dt} } \right)$      
\choice $\frac{1}{{\sqrt {k(x)} }}\sin \left( {\int\limits_0^x {k(t)dt} } \right)$
\end{choices}

\question Каким будет правило квантования в потенциале: $U(x) = \infty $ при $x < a$ и при $x > b$, $U(x)$ - некоторая известная плавная функция координаты при $a < x < b$ (бесконечно глубокая потенциальная яма с «неплоским» дном; см. рисунок)?
\begin{choices}
\choice $\int\limits_a^b {k(x)dx = \pi n} $       
\choice $\int\limits_a^b {k(x)dx = \pi (n + 1/4)} $
\choice $\int\limits_a^b {k(x)dx = \pi (n + 1/2)} $     
\choice $\int\limits_a^b {k(x)dx = \pi (n + 3/4)} $
\end{choices}

\end{questions}




\section{ ГЛАВА 7. ТЕОРИЯ ВОЗМУЩЕНИЙ }

\subsection{ Теория возмущений без вырождения }

\begin{questions}

\question Теория возмущений позволяет вычислить:
\begin{choices}
\choice оператор возмущения, если известно классическое выражение для возмущающего потенциала
\choice поправки к энергиям стационарных состояний непрерывного спектра
\choice поправки к энергиям стационарных состояний дискретного спектра
\choice поправку к гамильтониану системы
\end{choices}

\question Какую размерность имеют матричные элементы оператора возмущения с нормированными на единицу собственными функциями невозмущенного гамильтониана?
\begin{choices}
\choice координаты  
\choice импульса    
\choice энергии     
\choice безразмерны
\end{choices}

\question На некоторую квантовую систему накладывают малое возмущение $\hat V$, причем известно, что диагональный матричный элемент оператора возмущения с невозмущенными функциями $n$-го стационарного состояния не равен нулю и положителен. Увеличится или уменьшится при этом энергия этого состояния? 
\begin{choices}
\choice увеличится  
\choice уменьшится  
\choice не изменится   
\choice мало информации для ответа
\end{choices}

\question На некоторую квантовую систему накладывают малое возмущение $\hat V$, причем известно, что диагональный матричный элемент оператора возмущения с невозмущенными функциями $n$-го стационарного состояния не равен нулю и положителен. Увеличится или уменьшится при этом энергия $k$-го стационарного состояния? 
\begin{choices}
\choice увеличится  
\choice уменьшится  
\choice не изменится   
\choice мало информации для ответа
\end{choices}

\question На некоторую квантовую систему накладывают малое возмущение $\hat V$, причем известно, что диагональный матричный элемент оператора возмущения с невозмущенными функциями основ-ного состояния равен нулю, недиагональные матричные элементы, содержащие волновую функцию основного состояния не равны нулю. Увеличится или уменьшится при этом энергия основного со-стояния системы? 
\begin{choices}
\choice увеличится  
\choice уменьшится  
\choice не изменится   
\choice мало информации для ответа
\end{choices}

\question На некоторую квантовую систему накладывают малое возмущение $\hat V$, причем известно, что диагональный матричный элемент оператора возмущения с невозмущенными функциями $n$-го стационарного состояния равен нулю, недиагональные матричные элементы, содержащие волновую функцию этого состояния не равны нулю. Увеличится или уменьшится при этом энергия основного состояния системы? 
\begin{choices}
\choice увеличится  
\choice уменьшится  
\choice не изменится   
\choice мало информации для ответа
\end{choices}

\question На незаряженную частицу, движущуюся в некотором потенциале, накладывают однородное электрическое поле с напряженностью $E$, направленное вдоль оси $z$. Какой из нижеприведенных формул определяется оператор возмущения?
\begin{choices}
\choice $\hat V = 0$      
\choice $\hat V =  - E\hat z$      
\choice $\hat V =  - E{\hat L_z}$     
\choice $\hat V =  - E{\hat p_z}$
\end{choices}

\question На частицу с зарядом $e$, движущуюся в некотором потенциале, накладывают однородное электрическое поле с напряженностью $E$, направленное вдоль оси $z$. Какой из нижеприведенных формул определяется оператор возмущения?
\begin{choices}
\choice $\hat V = 0$      
\choice $\hat V =  - eE\hat z$     
\choice $\hat V =  - eE{\hat L_z}$    
\choice $\hat V =  - eE{\hat p_z}$
\end{choices}

\question На бесспиновую частицу с зарядом $e$ накладывается однородное магнитное поле с напряжен-ностью $H$, направленное вдоль оси $z$. Каким будет оператор возмущения (${\hat L_z}$, ${\hat L^2}$ и ${\hat p_z}$ - операторы проекции орбитального момента на ось $z$, квадрата орбитального момента и проекции импульса)?
\begin{choices}
\choice $\hat V =  - \frac{{eH{{\hat L}_z}}}{{2mc}}$ 
\choice $\hat V =  - \frac{{eH{{\hat L}^2}}}{{2mc}}$    
\choice $\hat V =  - \frac{{eH{{\hat p}_z}}}{{2mc}}$ 
\choice $\hat V =  - \frac{{eHz}}{{2mc}}$
\end{choices}

\question На незаряженную частицу со спином $s = 1/2$ накладывается однородное магнитное поле с напряженностью $H$, направленное вдоль оси $y$. Каким будет оператор возмущения в ${s_z}$-представлении, если отношение собственного магнитного момента к собственному механическому моменту для этой частицы известно и равно $\mu $ (здесь $V = \mu H\hbar /2$)?
\begin{choices}
\choice $\hat V =  - V\left( {\begin{array}{*{20}{c}}
0&{ - i}\\
i&0
\end{array}} \right)$   
\choice $\hat V =  - V\left( {\begin{array}{*{20}{c}}
1&0\\
0&{ - 1}
\end{array}} \right)$   
\choice $\hat V =  - V\left( {\begin{array}{*{20}{c}}
0&1\\
1&0
\end{array}} \right)$   
\choice $\hat V =  - V\left( {\begin{array}{*{20}{c}}
0&i\\
i&0
\end{array}} \right)$
\end{choices}

\question На незаряженную бесспиновую частицу накладывается однородное магнитное поле с напря-женностью $H$, направленное вдоль оси $y$. Каким будет оператор возмущения (${\hat L_z}$, ${\hat L^2}$ и ${\hat p_z}$ - операторы проекции орбитального момента на ось $z$, квадрата орбитального момента и проекции импульса)?
\begin{choices}
\choice $\hat V =  - \frac{{H{{\hat L}_z}}}{{2mc}}$     
\choice $\hat V =  - \frac{{H{{\hat L}^2}}}{{2mc}}$     
\choice $\hat V =  - \frac{{H{{\hat p}_z}}}{{2mc}}$     
\choice $\hat V = 0$
\end{choices}

\question На частицу, находящуюся в бесконечно глубокой прямоугольной потенциальной яме, наложи-ли возмущение $\alpha \delta (x - a/2)$, где $a$ - размер ямы, $\alpha  > 0$. Как изменятся энергии состояний с нечетными квантовыми числами (основное состояние - $n = 1$) в первом порядке тео-рии возмущений по сравнению с невозмущенной задачей?
\begin{choices}
\choice увеличатся     
\choice уменьшатся  
\choice не изменятся   
\choice мало информации для ответа
\end{choices}

\question На частицу, находящуюся в бесконечно глубокой прямоугольной потенциальной яме, наложи-ли возмущение $\alpha \delta (x - a/2)$, где $a$ - размер ямы, $\alpha  > 0$. Как изменятся энергии состояний с четными квантовыми числами (основное состояние - $n = 1$) в первом порядке теории возмущений по сравнению с невозмущенной задачей?
\begin{choices}
\choice увеличатся     
\choice уменьшатся  
\choice не изменятся   
\choice мало информации для ответа
\end{choices}

\question На частицу, находящуюся в бесконечно глубокой прямоугольной потенциальной яме, наложи-ли возмущение $\alpha \delta (x - a/2)$, где $a$ - размер ямы. Какой формулой определяются поправ-ки первого порядка к энергиям состояний с нечетными квантовыми числами (основное состояние - $n = 1$)
\begin{choices}
\choice $\Delta {E_i}^{(1)} = \frac{\alpha }{{2a}}$     
\choice $\Delta {E_i}^{(1)} =  - \frac{{2\alpha }}{a}$  
\choice $\Delta {E_i}^{(1)} = \frac{{2\alpha }}{a}$  
\choice $\Delta {E_i}^{(1)} =  - \frac{\alpha }{{2a}}$
\end{choices}

\question На частицу, находящуюся в бесконечно глубокой прямоугольной потенциальной яме, наложи-ли возмущение $\alpha \delta (x - a/2)$, где $a$ - размер ямы. Каким должен быть параметр $\alpha $, чтобы для расчета энергий можно было пользоваться теорией возмущений
\begin{choices}
\choice $\alpha  \ll \frac{{{\hbar ^2}}}{{ma}}$      
\choice $\alpha  \ll \frac{{m{\hbar ^2}}}{a}$     
\choice $\alpha  \ll \frac{{ma}}{{{\hbar ^2}}}$      
\choice $\alpha  \ll \frac{m}{{a{\hbar ^2}}}$
\end{choices}

\question На частицу, находящуюся в бесконечно глубокой прямоугольной потенциальной яме, наложи-ли возмущение $\alpha \delta (x - a/2)$, где $a$ - размер ямы. Для каких уровней – с большими или малыми квантовыми числами – лучше работает теория возмущений?
\begin{choices}
\choice с большими     
\choice с малыми    
\choice безразлично    
\choice для такого возмущения пользоваться теорией возмущений нельзя
\end{choices}

\question На частицу, находящуюся в бесконечно глубокой потенциальной яме шириной $a$, наложили малое возмущение $V(x) = {V_0}x(a - x)$ (${V_0} > 0$). Как изменятся энергии стационарных со-стояний в первом порядке теории возмущений по сравнению с невозмущенной задачей?
\begin{choices}
\choice увеличатся     
\choice уменьшатся  
\choice не изменятся   
\choice мало информации для ответа
\end{choices}

\question На частицу массой $m$, находящуюся в бесконечно глубокой прямоугольной потенциальной яме шириной $a$, наложили малое возмущение $V(x) = {V_0}x(x - a)$. При каком условии на вели-чину ${V_0}$ для расчета возмущенных энергий и волновых функций стационарных состояний мож-но пользоваться теорией возмущений?
\begin{choices}
\choice ${V_0} \ll \frac{{{\hbar ^2}}}{{ma}}$     
\choice ${V_0} \ll \frac{{{\hbar ^2}}}{{m{a^2}}}$    
\choice ${V_0} \ll \frac{{{\hbar ^2}}}{{m{a^3}}}$    
\choice ${V_0} \ll \frac{{{\hbar ^2}}}{{m{a^4}}}$
\end{choices}

\question На частицу, находящуюся в бесконечно глубокой прямоугольной потенциальной яме шириной $a$, наложили малое возмущение $V(x) = {V_0}(x - a/2)$ (${V_0} > 0$). Как изменятся энергии стационарных состояний в первом порядке теории возмущений по сравнению с невозму-щенной задачей?
\begin{choices}
\choice увеличатся     
\choice уменьшатся  
\choice не изменятся   
\choice мало информации для ответа
\end{choices}

\question На частицу, находящуюся в бесконечно глубокой прямоугольной потенциальной яме, наложи-ли возмущение $V(x) = {V_0}{\sin ^2}\left( {2\pi x/a} \right)$, где $a$ - размер ямы, ${V_0} > 0$. Как изменятся энергии стационарных состояний в первом порядке теории возмущений?
\begin{choices}
\choice увеличатся     
\choice уменьшатся  
\choice не изменятся   
\choice мало информации для ответа
\end{choices}

\question На частицу, находящуюся в бесконечно глубокой прямоугольной потенциальной яме, наложи-ли возмущение $V(x) = {V_0}{\sin ^2}\left( {2\pi x/a} \right)$, где $a$ - размер ямы. Какой формулой определяется поправка третьего порядка к энергии уровней?
\begin{choices}
\choice $\Delta {E^{(3)}} \sim \frac{{{V_0}^3m{a^2}}}{{{\hbar ^2}}}$   
\choice $\Delta {E^{(3)}} \sim \frac{{{V_0}^3{m^2}{a^4}}}{{{\hbar ^4}}}$  
\choice $\Delta {E^{(3)}} \sim \frac{{{V_0}^3{\hbar ^4}}}{{{m^2}{a^4}}}$  
\choice $\Delta {E^{(3)}} \sim \frac{{{V_0}^3{\hbar ^2}}}{{m{a^2}}}$
\end{choices}

\question На частицу, находящуюся в бесконечно глубокой прямоугольной потенциальной яме наложи-ли возмущение $V(x) = {V_0}\sin \left( {\pi x/a} \right)$, где $a$ - размер ямы, ${V_0} > 0$. Как изме-нятся энергии стационарных состояний в первом порядке теории возмущений?
\begin{choices}
\choice увеличатся     
\choice уменьшатся  
\choice не изменятся   
\choice зависит от размера ямы.
\end{choices}

\question На частицу, находящуюся в бесконечно глубокой прямоугольной потенциальной яме наложи-ли возмущение $V(x) = {V_0}\sin \left( {2\pi x/a} \right)$, где $a$ - размер ямы, ${V_0} > 0$. Как из-менятся энергии стационарных состояний в первом порядке теории возмущений?
\begin{choices}
\choice увеличатся     
\choice уменьшатся  
\choice не изменятся   
\choice зависит от размера ямы.
\end{choices}

\question На частицу, находящуюся в бесконечно глубокой прямоугольной потенциальной яме наложи-ли малое возмущение $V(x) = {V_0}f(x)$, где ${V_0}$ - некоторое число. Как поправки второго по-рядка теории возмущений к энергиям стационарных состояний зависят от ${V_0}$?
\begin{choices}
\choice $\Delta {E^{(2)}} \sim {V_0}$    
\choice $\Delta {E^{(2)}} \sim {V_0}^2$     
\choice $\Delta {E^{(2)}} \sim {V_0}^{ - 2}$   
\choice $\Delta {E^{(2)}} \sim {V_0}^{ - 1}$
\end{choices}

\question На частицу, находящуюся в бесконечно глубокой прямоугольной потенциальной яме наложи-ли возмущение $V(x) = {V_0}\cos \left( {\pi x/a} \right)$, где $a$ - размер ямы. Сколько ненулевых слагаемых входят в формулу для поправки второго порядка к энергии основного состояния?
\begin{choices}
\choice 1     
\choice 2     
\choice 3     
\choice бесконечно много
\end{choices}

\question На частицу, находящуюся в бесконечно глубокой прямоугольной потенциальной яме наложи-ли возмущение $V(x) = {V_0}\cos \left( {\pi x/a} \right)$, где $a$ - размер ямы. Сколько ненулевых слагаемых входят в формулу для поправки второго порядка к энергии 99-го стационарного состоя-ния?
\begin{choices}
\choice 1     
\choice 2     
\choice 99    
\choice бесконечно много
\end{choices}

\question На частицу, находящуюся в бесконечно глубокой прямоугольной потенциальной яме наложи-ли возмущение $V(x) = {V_0}\cos \left( {\pi x/a} \right)$, где $a$ - размер ямы. Чему равна поправка второго порядка к энергии основного состояния?
\begin{choices}
\choice $\Delta {E^{(2)}} =  - \frac{{{V_0}^2m{a^2}}}{{6{\pi ^2}{\hbar ^2}}}$      
\choice $\Delta {E^{(2)}} =  - \frac{{{V_0}^2m{a^2}}}{{12{\pi ^2}{\hbar ^2}}}$     
\choice $\Delta {E^{(2)}} =  - \frac{{{V_0}^2m{a^2}}}{{3{\pi ^2}{\hbar ^2}}}$           
\choice $\Delta {E^{(2)}} =  - \frac{{{V_0}^2m{a^2}}}{{8{\pi ^2}{\hbar ^2}}}$
\end{choices}

\question На частицу, находящуюся в бесконечно глубокой прямоугольной потенциальной яме, наложи-ли произвольное возмущение $V(x)$. Как поправка первого порядка к энергии $n$-го стационарного состояния зависит от квантового числа $n$ при больших значениях $n$?
\begin{choices}
\choice растет с ростом $n$        
\choice убывает с ростом $n$
\choice не зависит от $n$       
\choice это зависит от размера ямы
\end{choices}

\question На частицу, находящуюся в бесконечно глубокой прямоугольной потенциальной яме, наложи-ли произвольное возмущение $V(x)$. Найти поправку первого порядка к энергии $n$-го стационар-ного состояния ($n \to \infty $), если значение интеграла $\int\limits_0^a {V(x)dx}  = A$ известно.
\begin{choices}
\choice растет с ростом $n$        
\choice убывает с ростом $n$
\choice не зависит от $n$       
\choice это зависит от размера ямы
\end{choices}

\question На одномерный гармонический осциллятор массой $m$ и частотой $\omega $ накладывают малое возмущение $\hat V(x) = {V_0}\sin (x/b)$. Каким будет сдвиг энергий стационарных состояний осциллятора в первом порядке теории возмущений?
\begin{choices}
\choice $\Delta E = 0$ 
\choice $\Delta E = {V_0}$   
\choice $\Delta E = \frac{{{V_0}}}{b}\sqrt {\frac{\hbar }{{m\omega }}} $  
\choice $\Delta E = \frac{{{V_0}}}{{{b^2}}}\frac{\hbar }{{m\omega }}$
\end{choices}

\question На одномерный гармонический осциллятор накладывают малое возмущение $\hat V(x) = a\sin (x/b)$ ($a > 0$). Как изменится энергия основного состояния осциллятора? 
\begin{choices}
\choice увеличится  
\choice уменьшится  
\choice не изменится   
\choice мало информации для ответа
\end{choices}

\question На одномерный гармонический осциллятор накладывают малое возмущение $\hat V(x) = a{\sin ^2}(x/b)$ ($a < 0$). Как изменятся энергии стационарных состояний осциллятора? 
\begin{choices}
\choice увеличатся     
\choice уменьшатся  
\choice не изменятся   
\choice мало информации для ответа
\end{choices}

\question На одномерный гармонический осциллятор наложили малое возмущение $\hat V(x) = \alpha \delta (x)$. Как изменятся энергии нечетных уровней осциллятора (уровень с самой маленькой энер-гией – нулевой)?
\begin{choices}
\choice увеличатся     
\choice уменьшатся  
\choice не изменятся   
\choice мало информации для ответа
\end{choices}

\question На одномерный гармонический осциллятор наложили возмущение $\hat V(x) = \alpha \delta (x)$. Как поправка первого порядка теории возмущений к энергии четных стационарных состояний зависит от квантового числа состояния для больших значений квантового числа?
\begin{choices}
\choice как $\frac{1}{{\sqrt n }}$    
\choice как $\frac{1}{n}$    
\choice как $\frac{1}{{{n^2}}}$    
\choice не зависит от $n$
\end{choices}

\question На одномерный гармонический осциллятор наложили возмущение $\hat V(x) = \alpha \delta (x)$. Для каких уровней – с большими или малыми квантовыми числами – лучше работает теория возмущений
\begin{choices}
\choice с малыми    
\choice со средними 
\choice с большими  
\choice не зависит от номера уровня
\end{choices}

\question На одномерный гармонический осциллятор наложили возмущение $\hat V(x) = \alpha \delta (x)$. При каких значениях параметра $\alpha $ это возмущение можно считать малым (здесь $a = \sqrt {\hbar /m\omega } $)?
\begin{choices}
\choice $\alpha  \ll \hbar \omega a$     
\choice $\alpha  \gg \hbar \omega a$     
\choice $\alpha  \ll \frac{{\hbar \omega }}{a}$      
\choice $\alpha  \gg \frac{{\hbar \omega }}{a}$   
\end{choices}

\question На одномерный гармонический осциллятор наложили возмущение $\hat V(x) = \alpha x$. Сколько ненулевых слагаемых будут входить в формулу для поправки второго порядка к энергии $n$-го стационарного состояния ($n \ne 0$)?
\begin{choices}
\choice одно     
\choice два         
\choice $n$         
\choice бесконечно много
\end{choices}

\question На одномерный гармонический осциллятор наложили возмущение $\hat V(x) = \alpha x$. Сколько ненулевых слагаемых будут входить в формулу для поправки второго порядка к энергии основного состояния?
\begin{choices}
\choice одно     
\choice два         
\choice $n$         
\choice бесконечно много
\end{choices}

\question На одномерный гармонический осциллятор наложили малое возмущение $\hat V(x) = \alpha x$. Во втором порядке теории возмущений найти сдвиг энергии основного состояния осциллятора
\begin{choices}
\choice $\Delta E =  - \frac{{{\alpha ^2}\hbar }}{{2{m^2}{\omega ^3}}}$   
\choice $\Delta E =  - \frac{{{\alpha ^2}\hbar }}{{4{m^2}{\omega ^3}}}$   
\choice $\Delta E =  - \frac{{{\alpha ^2}\hbar }}{{8{m^2}{\omega ^3}}}$   
\choice $\Delta E =  - \frac{{{\alpha ^2}\hbar }}{{16{m^2}{\omega ^3}}}$
Указание. Для собственных функций основного ${\varphi _0}(x)$ и первого возбужденного ${\varphi _1}(x)$ стационарных состояний гармонического осциллятора справедливо равенство $\int {{\varphi _0}(x)x{\varphi _1}(x)dx = } \sqrt {\frac{\hbar }{{2m\omega }}} $
\end{choices}

\question На одномерный гармонический осциллятор наложили малое возмущение $\hat V(x) = \alpha x$. В третьем порядке теории возмущений найти сдвиг энергии основного состояния осциллятора (здесь $a = \sqrt {\hbar /m\omega } $)
\begin{choices}
\choice $\Delta {E^{(3)}} =  - \frac{{{\alpha ^3}{a^3}}}{{{\hbar ^2}{\omega ^2}}}$ 
\choice $\Delta {E^{(3)}} =  - \frac{{{\alpha ^3}{a^2}}}{{\hbar \omega }}$   
\choice $\Delta {E^{(3)}} =  - \frac{{{\alpha ^3}{a^4}}}{{{\hbar ^3}{\omega ^3}}}$ 
\choice $\Delta {E^{(3)}} = 0$
\end{choices}

\question На одномерный гармонический осциллятор накладывают малое возмущение $\hat V(x) = a{x^2}$. Как поправки теории возмущений к энергии $n$-го уровня энергии зависят от $n$?
\begin{choices}
\choice $ \sim n$    
\choice $ \sim {n^2}$     
\choice $ \sim \frac{1}{n}$     
\choice $ \sim \frac{1}{{{n^2}}}$
\end{choices}

\question На одномерный гармонический осциллятор наложили малое возмущение $\hat V(x) = \alpha {x^4}$ ($\alpha  > 0$). Как изменятся энергии стационарных состояний осциллятора?
\begin{choices}
\choice увеличится  
\choice уменьшится  
\choice не измениться  
\choice мало информации для ответа
\end{choices}

\question На одномерный гармонический осциллятор наложили малое возмущение $\hat V(x) = \alpha {x^4}$. Какой формулой определяется поправка второго порядка к энергиям стационарных состоя-ний?
\begin{choices}
\choice $\Delta {E^{(4)}} \sim \frac{{{\alpha ^2}\hbar }}{{m{\omega ^2}}}$   
\choice $\Delta {E^{(4)}} \sim \frac{{{\alpha ^2}\hbar }}{{{m^2}{\omega ^3}}}$  
\choice $\Delta {E^{(4)}} \sim \frac{{{\alpha ^2}{\hbar ^2}}}{{{m^3}{\omega ^4}}}$ 
\choice $\Delta {E^{(4)}} \sim \frac{{{\alpha ^2}{\hbar ^3}}}{{{m^4}{\omega ^5}}}$
\end{choices}

\question На одномерный гармонический осциллятор наложили возмущение $\hat V(x) = \alpha {x^4}$. При каких условиях на $\alpha $ для расчета влияния этого возмущения можно использовать теорию возмущений?
\begin{choices}
\choice $\alpha  \ll \frac{{{m^5}{\omega ^6}}}{{{\hbar ^4}}}$    
\choice $\alpha  \ll \frac{{{m^4}{\omega ^6}}}{{{\hbar ^3}}}$    
\choice $\alpha  \ll \frac{{{m^2}{\omega ^3}}}{\hbar }$    
\choice нельзя ни при каких $\alpha $
\end{choices}

\question На одномерный гармонический осциллятор наложили возмущение $\hat V(x) = \alpha {x^3}$. При каких условиях на $\alpha $ для расчета влияния этого возмущения можно использовать теорию возмущений?
\begin{choices}
\choice $\alpha  \ll \frac{{{m^3}{\omega ^4}}}{{{\hbar ^4}}}$    
\choice $\alpha  \ll \frac{{{m^2}{\omega ^3}}}{\hbar }$    
\choice $\alpha  \ll \frac{{{m^3}{\omega ^4}}}{{{\hbar ^2}}}$    
\choice нельзя ни при каких $\alpha $
\end{choices}

\question На трехмерный гармонический осциллятор наложили малое возмущение $\hat V(x,y,z) = \alpha x$. Чему равен сдвиг энергии основного состояния осциллятора в первом порядке теории воз-мущений?
\begin{choices}
\choice $\Delta {E^{(1)}} = \alpha \sqrt {\frac{\hbar }{{2m\omega }}} $   
\choice $\Delta {E^{(1)}} =  - \alpha \sqrt {\frac{\hbar }{{m\omega }}} $ 
\choice $\Delta {E^{(1)}} = \alpha \sqrt {\frac{\hbar }{{m\omega }}} $ 
\choice $\Delta {E^{(1)}} = 0$
\end{choices}

\question На трехмерный гармонический осциллятор наложили малое возмущение $\hat V(x,y,z) = \alpha {x^2}$. Чему равен сдвиг энергии $n$-го стационарного состояния осциллятора (основное со-стояние - нулевое)?
\begin{choices}
\choice $\Delta {E^{(1)}} = \alpha \sqrt {\frac{\hbar }{{2m\omega }}} $   
\choice $\Delta {E^{(1)}} =  - \alpha \sqrt {\frac{\hbar }{{m\omega }}} $ 
\choice $\Delta {E^{(1)}} = \alpha \sqrt {\frac{\hbar }{{m\omega }}} $ 
\choice $\Delta {E^{(1)}} = 0$
\end{choices}

\question На трехмерный гармонический осциллятор наложили малое возмущение $\hat V(x,y,z) = \alpha {r^2}$. Чему равен сдвиг энергии $n$-го стационарного состояния осциллятора (основное со-стояние - нулевое)?
\begin{choices}
\choice $\Delta {E^{(1)}} = \alpha \sqrt {\frac{\hbar }{{2m\omega }}} $   
\choice $\Delta {E^{(1)}} =  - \alpha \sqrt {\frac{\hbar }{{m\omega }}} $ 
\choice $\Delta {E^{(1)}} = \alpha \sqrt {\frac{\hbar }{{m\omega }}} $ 
\choice $\Delta {E^{(1)}} = 0$
\end{choices}

\question На атом водорода накладывают однородное электрическое поле с напряженностью $E$. Чему равен сдвиг энергии основного состояния электрона в первом порядке теории возмущений
\begin{choices}
\choice $\Delta {E^{(1)}} = \frac{{E{\hbar ^2}}}{{m{e^2}}}$   
\choice $\Delta {E^{(1)}} =  - \frac{{E{\hbar ^2}}}{{m{e^2}}}$   
\choice $\Delta {E^{(1)}} = 0$     
\choice $\Delta {E^{(1)}} = \frac{{2E{\hbar ^2}}}{{m{e^2}}}$
\end{choices}

\question На атом водорода накладывают малое возмущение $\hat V = \alpha {r^2}$. Какой формулой определяется поправка первого порядка к энергии основного состояния?
\begin{choices}
\choice $\Delta {E^{(1)}} \sim \frac{{\alpha {\hbar ^4}}}{{{m^2}{e^4}}}$  
\choice $\Delta {E^{(1)}} \sim \frac{{\alpha {\hbar ^6}}}{{{m^3}{e^5}}}$  
\choice $\Delta {E^{(1)}} \sim \frac{{\alpha {\hbar ^8}}}{{{m^5}{e^6}}}$  
\choice $\Delta {E^{(1)}} \sim \frac{{\alpha {\hbar ^2}}}{{m{e^2}}}$
\end{choices}

\question На атом водорода накладывают малое однородное магнитное поле с напряженностью $H$. Чему равен сдвиг энергии основного состояния электрона в первом порядке теории возмущений?
\begin{choices}
\choice $\Delta {E^{(1)}} = 0$     
\choice $\Delta {E^{(1)}} =  - \frac{{H{\hbar ^2}}}{{m{e^2}}}$   
\choice $\Delta {E^{(1)}} = \frac{{H{\hbar ^2}}}{{m{e^2}}}$   
\choice $\Delta {E^{(1)}} = \frac{{2E{\hbar ^2}}}{{m{e^2}}}$
\end{choices}

\question На атом водорода накладывают малое возмущение $\hat V = \alpha \cos \varphi $. Какой фор-мулой определяется поправка первого порядка к энергии основного состояния?
\begin{choices}
\choice $\Delta {E^{(1)}} \sim \frac{{\alpha {\hbar ^4}}}{{{m^2}{e^4}}}$  
\choice $\Delta {E^{(1)}} \sim \frac{{\alpha {\hbar ^6}}}{{{m^3}{e^5}}}$  
\choice $\Delta {E^{(1)}} \sim \frac{{\alpha {\hbar ^8}}}{{{m^5}{e^6}}}$  
\choice $\Delta {E^{(1)}} = 0$
\end{choices}

\question На одномерный гармонический осциллятор, накладывают возмущение $\hat V(x) = a\sin (x/b)$. Как изменится средняя четность 99-го стационарного состояния (основное состояние - нуле-вое)?
\begin{choices}
\choice уменьшится  
\choice увеличится     
\choice не изменится   
\choice мало информации для ответа
\end{choices}

\question На одномерный гармонический осциллятор, накладывают возмущение $\hat V(x) = a\sin (x/b)$. Как изменится средняя четность 100-го стационарного состояния (основное состояние - нуле-вое)?
\begin{choices}
\choice уменьшится  
\choice увеличится     
\choice не изменится   
\choice мало информации для ответа
\end{choices}

\question На осциллятор накладывают возмущение $\hat V(x) = \alpha {x^3} + \beta {x^4}$. Как изме-нится средняя четность стационарных состояний осциллятора (основное состояние - нулевое)?
\begin{choices}
\choice четных состояний – уменьшится, нечетных состояний – увеличится
\choice четных состояний – увеличится, нечетных состояний – уменьшится
\choice для всех состояний – уменьшится
\choice для всех состояний – увеличится
\end{choices}

\question На одномерный гармонический осциллятор, накладывают возмущение $\hat V(x) = a\cos (x/b)$. Как изменится средняя четность 99-го стационарного состояния (основное состояние - нуле-вое)?
\begin{choices}
\choice уменьшится  
\choice увеличится     
\choice не изменится   
\choice мало информации для ответа
\end{choices}

\question На одномерный гармонический осциллятор, накладывают возмущение $\hat V(x) = a\cos (x/b)$. В первом порядке теории возмущений для волновой функции найти среднюю четность основ-ного состояния возмущенного осциллятора
\begin{choices}
\choice $\overline P  =  - 1$      
\choice $\overline P  = 1$      
\choice $\overline P  =  - 1 + \frac{{{a^2}}}{{\hbar \omega }}$  
\choice $\overline P  = 1 - \frac{{{a^2}}}{{\hbar \omega }}$
\end{choices}

\question На одномерный гармонический осциллятор наложили малое возмущение $\hat V(x) = \alpha x$. В первом порядке теории возмущений для волновой функции найти вероятность того, что чет-ность возмущенного основного состояния осциллятора равна $ - 1$
\begin{choices}
\choice $w = \frac{{{\alpha ^2}}}{{2m\hbar {\omega ^3}}}$  
\choice $w = \frac{{{\alpha ^2}}}{{4m\hbar {\omega ^3}}}$  
\choice $w = \frac{{{\alpha ^2}\hbar }}{{2m{\omega ^3}}}$     
\choice $w = \frac{{{\alpha ^2}\hbar }}{{4m{\omega ^3}}}$
Указание. Матричный элемент оператора координаты с волновыми функциями основного ${\varphi _0}(x)$ и первого возбужденного ${\varphi _1}(x)$ стационарных состояний гармонического осцил-лятора равен $\int {{\varphi _0}(x)x{\varphi _1}(x)dx = } \sqrt {\frac{\hbar }{{2m\omega }}} $
\end{choices}

\question На частицу, движущуюся в центральном поле, наложили малое возмущение $\hat V(x,y,z) = \alpha z$. Какие значения момента импульса частицы и его проекции на ось $z$ можно обнаружить в возмущенном основном состоянии частицы? Ответ дать в первом порядке теории возмущений для волновой функции.
\begin{choices}
\choice $l = 0,\;\;1;\quad {l_z} = 0$ 
\choice $l = 0;\quad {l_z} = 0,\;\;1$ 
\choice $l = 0,\quad {l_z} = 0$ 
\choice $l = 0,\;\;1;\quad {l_z} = 0,\;\;1,\;\; - 1$
\end{choices}

\question На частицу, движущуюся в центральном поле, наложили малое возмущение $\hat V(x,y,z) = \alpha y$. Какие значения момента импульса частицы и его проекции на ось $z$ можно обнаружить в возмущенном основном состоянии частицы? Ответ дать в первом порядке теории возмущений для волновой функции.
\begin{choices}
\choice $l = 0,\;\;1;\quad {l_z} = 0$ 
\choice $l = 0;\quad {l_z} = 0,\;\;1$ 
\choice $l = 0,\quad {l_z} = 0$ 
\choice $l = 0,\;\;1;\quad {l_z} = 0,\;\;1,\;\; - 1$
\end{choices}

\question На атом водорода накладывают малое возмущение $\hat V = af(r)$, где $f(r)$ - функция от модуля радиуса-вектора. Какие значения момента импульса электрона и его проекции на ось $z$ можно обнаружить в возмущенном основном состоянии атома? 
\begin{choices}
\choice $l = 0,\;\;1$, ${l_z} = 0,\;\; \pm 1$     
\choice $l = 0$, ${l_z} = 0$ 
\choice $l = 0,\;\;2$, ${l_z} = 0,\;\; \pm 2$     
\choice $l = 0,\;\;1$, ${l_z} = 0$
\end{choices}

\question На атом водорода накладывают малое возмущение $\hat V = a\cos \vartheta $. Какие значения момента импульса электрона и его проекции на ось $z$ можно обнаружить в основном состоянии атома? Ответ дать в первом порядке теории возмущений для волновой функции.
\begin{choices}
\choice $l = 0$, ${l_z} = 0$ 
\choice $l = 0,\;\;1$, ${l_z} = 0,\;\; \pm 1$     
\choice $l = 0,\;\;2$, ${l_z} = 0,\;\; \pm 2$  
\choice $l = 0,\;\;1$, ${l_z} = 0$
\end{choices}

\question На атом водорода накладывают малое возмущение $\hat V = a{\cos ^2}\varphi $. Какие значе-ния проекции орбитального момента импульса электрона на ось $z$ можно обнаружить в возмущен-ном основном состоянии атома? Ответ дать в первом порядке теории возмущений для волновой функции.
\begin{choices}
\choice ${l_z} = 0$ 
\choice ${l_z} = 0,\;\; \pm 1$     
\choice ${l_z} = 0,\;\; \pm 2$     
\choice ${l_z} = 0,\;\; \pm 1,\;\; \pm 2$ 
\end{choices}

\question На частицу, движущуюся в центральном поле, накладывают малое возмущение $\hat V = a{\cos ^2}\vartheta $. Какие значения момента импульса частицы можно обнаружить в возмущенном основном состоянии? Ответ дать в первом порядке теории возмущений для волновой функции.
\begin{choices}
\choice $l = 0$  
\choice $l = 0,\;\;1$     
\choice $l = 0,\;\;1,\;\;2$     
\choice $l = 0,\;\;2$
\end{choices}

\question  На частицу, движущуюся в центральном поле, накладывают малое возмущение $\hat V = a{\cos ^2}\vartheta $. Какие значения проекции момента импульса частицы на ось $z$ можно обна-ружить в возмущенном основном состоянии? Ответ дать в первом порядке теории возмущений для волновой функции.
\begin{choices}
\choice ${l_z} = 0$ 
\choice ${l_z} = 0,\;\; \pm 1$     
\choice ${l_z} = 0,\;\; \pm 2$     
\choice ${l_z} = 0,\;\; \pm 1,\;\; \pm 2$
\end{choices}

\question На атом водорода накладывают малое возмущение, оператор которого зависит только от $z$. Какие значения проекции момента импульса электрона на ось $z$ можно обнаружить в возмущенном основном состоянии? Ответ дать в первом порядке теории возмущений для волновой функции.
\begin{choices}
\choice ${l_z} = 0$ 
\choice ${l_z} = 0,\;\; \pm 1$     
\choice ${l_z} = 0,\;\; \pm 2$     
\choice ${l_z} = 0,\;\; \pm 1,\;\; \pm 2$
\end{choices}

\end{questions}




\subsection{ Теория возмущений при наличии вырождения }

\begin{questions}

\question Правильные функции нулевого приближения – это 
\begin{choices}
\choice собственные функции невозмущенного гамильтониана, отвечающие вырожденному уровню
\choice точные собственные функции возмущенного гамильтониана 
\choice суперпозиции собственных функций невозмущенного гамильтониана, отвечающих вырожденному уровню, для которых недиагональные матричные элементы оператора возмущения равны нулю 
\choice суперпозиции собственных функций невозмущенного гамильтониана, отвечающих вырожденному уровню, для которых диагональные матричные элементы оператора возмущения равны нулю
\end{choices}

\question Правильные функции нулевого приближения являются
\begin{choices}
\choice точными собственными функциями невозмущенного гамильтониана
\choice точными собственными функциями возмущенного гамильтониана
\choice точными решениями возмущенного временного уравнения Шредингера
\choice приближенными собственными функциями невозмущенного гамильтониана
\end{choices}

\question Какое из нижеперечисленных утверждений относительно свойств правильных функций нулево-го приближения является верным?
\begin{choices}
\choice эти функции ортогональны невозмущенным собственным функциям
\choice недиагональные матричные элементы возмущения с правильными функциями равны нулю
\choice диагональные матричные элементы оператора возмущения с правильными функциями равны нулю
\choice эти функции являются точными собственными функциями возмущенного гамильтониана
\end{choices}

\question Квантовая система имеет двукратно вырожденный уровень с энергией $\varepsilon $, которому отвечают невозмущенные волновые функции ${\varphi _1}(x)$ и ${\varphi _2}(x)$. На систему накла-дывают возмущение $\hat V$. Правильные функции нулевого приближения ${f_1}(x)$ и ${f_2}(x)$ (не совпадающие с ${\varphi _1}(x)$ и ${\varphi _2}(x)$) известны. Какие формулы правильно определяют возмущенные энергии в первом порядке теории возмущений (здесь $i = 1,2$)?
\begin{choices}
\choice ${E_i} = \varepsilon  + \int {{\varphi _i}^*(x)\hat V} {\varphi _i}(x)dx$        
\choice ${E_i} = \varepsilon  + \int {{f_i}^*(x)\hat V} {\varphi _i}(x)dx$ 
\choice ${E_i} = \varepsilon  + \int {{\varphi _i}^*(x)\hat V} {f_i}(x)dx$         
\choice ${E_i} = \varepsilon  + \int {{f_i}^*(x)\hat V} {f_i}(x)dx$ 
\end{choices}

\question Квантовая система имеет двукратно вырожденный уровень, которому отвечают невозмущенные волновые функции ${\varphi _1}(x)$ и ${\varphi _2}(x)$. На систему накладывают возмущение $\hat V$. Правильные функции нулевого приближения ${f_1}(x)$ и ${f_2}(x)$ (не совпадающие с ${\varphi _1}(x)$ и ${\varphi _2}(x)$) известны. Какие из нижеследующих интегралов обязательно равны нулю?
\begin{choices}
\choice $\int {{\varphi _1}^*(x)\hat V} {f_2}(x)dx$  
\choice $\int {{f_1}^*(x)\hat V} {f_2}(x)dx$   
\choice $\int {{\varphi _2}^*(x)\hat V} {\varphi _2}(x)dx$ 
\choice $\int {{f_1}^*(x)\hat V} {f_1}(x)dx$
\end{choices}

\question Квантовая система имеет двукратно вырожденный уровень, которому отвечают невозмущенные волновые функции ${\varphi _1}(x)$ и ${\varphi _2}(x)$. На систему накладывают возмущение $\hat V$. Правильные функции нулевого приближения ${f_1}(x)$ и ${f_2}(x)$ (не совпадающие с ${\varphi _1}(x)$ и ${\varphi _2}(x)$) известны. Какие из нижеследующих интегралов обязательно равны нулю?
\begin{choices}
\choice $\int {{\varphi _1}^*(x)\hat V} {f_2}(x)dx$  
\choice $\int {{f_1}^*(x)\hat V} {f_1}(x)dx$   
\choice $\int {{\varphi _2}^*(x)\hat V} {\varphi _2}(x)dx$ 
\choice $\int {{\varphi _1}^*(x)\hat V} {\varphi _2}(x)dx$
\end{choices}

\question Квантовая система имеет двукратно вырожденный уровень, которому отвечают невозмущенные волновые функции ${\varphi _1}(x)$ и ${\varphi _2}(x)$. На систему накладывают возмущение $\hat V$. Известно, что матричные элементы оператора возмущения с правильными функциями нулевого приближения одинаковы. Будет ли сниматься вырождения уровня в первом порядке теории возмуще-ний?
\begin{choices}
\choice да                   
\choice нет
\choice информации для ответа недостаточно     
\choice бессмысленный вопрос
\end{choices}

\question Некоторая квантовая система имеет двукратно вырожденный уровень, которому отвечают невоз-мущенные волновые функции ${\varphi _1}(x)$ и ${\varphi _2}(x)$. На систему накладывают возмуще-ние $\hat V$. Известно, что матричные элементы оператора возмущения с правильными функциями нулевого приближения одинаковы. Будет ли сниматься вырождения уровня во втором порядке теории возмущений?
\begin{choices}
\choice да                   
\choice нет
\choice информации для ответа недостаточно     
\choice бессмысленный вопрос
\end{choices}

\question Квантовая система имеет вырожденный уровень, которому отвечают невозмущенные функции ${\varphi _1}$, ${\varphi _2}$, …, ${\varphi _s}$. На систему накладывают возмущение, недиагональ-ные матричные элементы которого с функциями ${\varphi _i}$ равны нулю, диагональные – все одина-ковы. Какие утверждения относительно свойств правильных функций нулевого приближения будут верными?
\begin{choices}
\choice каждая из них обязательно совпадает с одной из функций ${\varphi _i}$
\choice правильными функциями будут произвольные линейные комбинации функций ${\varphi _i}$
\choice ни одна из правильных функций не будет совпадать ни с одной из функций ${\varphi _i}$
\choice только определенные линейные комбинации функций ${\varphi _i}$ будут правильными функциями 
\end{choices}

\question Квантовая система имеет вырожденный уровень, которому отвечают невозмущенные функции ${\varphi _1}$, ${\varphi _2}$, …, ${\varphi _s}$. На систему накладывают возмущение, недиагональ-ные матричные элементы которого с функциями ${\varphi _i}$ равны нулю, диагональные – все раз-личны. Какие утверждения относительно свойств правильных функций нулевого приближения будут верными?
\begin{choices}
\choice каждая из них будет совпадать с одной из функций ${\varphi _i}$
\choice правильными функциями будут произвольные линейные комбинации функций ${\varphi _i}$
\choice ни одна из правильных функций не будет совпадать ни с одной из функций ${\varphi _i}$
\choice только определенные линейные комбинации функций ${\varphi _i}$ будут правильными функциями
\end{choices}

\question Квантовая система имеет вырожденный уровень, которому отвечают собственные функции ${\varphi _1}$, ${\varphi _2}$, …, ${\varphi _s}$. На систему накладывают возмущение, недиагональ-ные матричные элементы которого с функциями ${\varphi _i}$ равны нулю, диагональные – все раз-личны. Будет ли сниматься вырождение уровня?
\begin{choices}
\choice частично    
\choice полностью      
\choice нет      
\choice информации для ответа недостаточно
\end{choices}

\question Квантовая система имеет вырожденный уровень, которому отвечают собственные функции ${\varphi _1}$, ${\varphi _2}$, …, ${\varphi _s}$. На систему накладывают возмущение, диагональные матричные элементы которого с функциями ${\varphi _i}$ равны нулю, недиагональные – все различ-ны. Какие утверждения относительно свойств правильных функций нулевого приближения будут вер-ными?
\begin{choices}
\choice каждая из них будет совпадать с одной из функций ${\varphi _i}$
\choice ими будут любые линейные комбинации функций ${\varphi _i}$
\choice в этом случае правильные функции найти нельзя
\choice только определенные линейные комбинации функций ${\varphi _i}$ будут правильными функциями
\end{choices}

\question Квантовая система имеет вырожденный уровень, которому отвечают собственные функции ${\varphi _1}$, ${\varphi _2}$, …, ${\varphi _s}$. На систему накладывают возмущение, диагональные матричные элементы которого с функциями ${\varphi _i}$ равны нулю, недиагональные – все различ-ны. Будет ли сниматься вырождение уровня?
\begin{choices}
\choice частично    
\choice полностью      
\choice нет      
\choice информации для ответа недостаточно
\end{choices}

\question Квантовая система имеет вырожденный уровень. На систему накладывают возмущение, диаго-нальные матричные элементы которого с невозмущенными функциями известны и равны ${V_{ii}}$, недиагональные – все равны нулю. Какими будут энергетические интервалы между возмущенными подуровнями (здесь индексы $i$ и $k$ нумеруют вырожденные состояния)?
\begin{choices}
\choice $\Delta E = \left( {{V_{ii}} + {V_{kk}}} \right)/2$   
\choice $\Delta E = {V_{ii}} - {V_{kk}}$ 
\choice $\Delta E = {V_{ii}}$   
\choice $\Delta E =  - {V_{kk}}$ 
\end{choices}

\question Квантовая система имеет вырожденный уровень. На систему накладывают возмущение, которое полностью снимает вырождение этого уровня. Будут ли правильные функции нулевого приближения, отвечающие этому уровню, ортогональны?
\begin{choices}
\choice да
\choice нет
\choice вообще говоря, нет, но их можно выбрать так, чтобы были ортогональны 
\choice информации для ответа недостаточно
\end{choices}

\question Некоторая квантовая система имеет вырожденный уровень. На систему накладывают возмуще-ние, которое снимает вырождение этого уровня только частично. Будут ли правильные функции нуле-вого приближения ортогональны?
\begin{choices}
\choice да
\choice нет
\choice вообще говоря, нет, но их можно выбрать так, чтобы были ортогональны
\choice информации для ответа недостаточно
\end{choices}

\question Некоторая квантовая система имеет вырожденный уровень. На систему накладывают возмуще-ние. Будет ли выбор правильных функции нулевого приближения однозначным?
\begin{choices}
\choice да, всегда
\choice нет, всегда
\choice да, если вырождение снимается полностью
\choice да, если вырождение снимается хотя бы частично
\end{choices}

\question Пятый возбужденный уровень (шестой по счету в порядке возрастания энергии) некоторой трех-мерной квантовой системы является четырехкратно вырожденным. На систему накладывается малое возмущение $\hat V$. Какова степень секулярного уравнения для определения возмущенных энергий системы?
\begin{choices}
\choice 3     
\choice 4     
\choice 5     
\choice 6
\end{choices}

\question Шестой возбужденный уровень (седьмой по счету в порядке возрастания энергии) некоторой трехмерной квантовой системы является пятикратно вырожденным. На систему накладывается малое возмущение $\hat V$. На какое максимальное количество подуровней может расщепиться уровень?
\begin{choices}
\choice на 5     
\choice на 6     
\choice на 7     
\choice на 3
\end{choices}

\question Уровень энергии $\varepsilon $ некоторой квантовой системы двукратно вырожден. На систему накладывается малое возмущение, матричные элементы которого с невозмущенными собственными функциями известны: ${V_{11}} = {V_{22}} = 0$, ${V_{12}} = {V_{21}} = V$. Какими будут энергии возмущенных состояний ${E_1}$ и ${E_2}$ в первом порядке теории возмущений?
\begin{choices}
\choice ${E_1} = {E_2} = \varepsilon $   
\choice ${E_1} = \varepsilon  + V$, ${E_2} = \varepsilon $    
\choice ${E_1} = \varepsilon $, ${E_2} = \varepsilon  - V$    
\choice ${E_{1,2}} = \varepsilon  \pm V$
\end{choices}

\question Уровень энергии $\varepsilon $ некоторой квантовой системы двукратно вырожден. На систему накладывается малое возмущение, матричные элементы которого с невозмущенными собственными функциями известны: ${V_{11}} = {V_{22}} = V$, ${V_{12}} = {V_{21}} = 0$. Какими будут энергии возмущенных состояний ${E_1}$ и ${E_2}$ в первом порядке теории возмущений?
\begin{choices}
\choice ${E_1} = {E_2} = \varepsilon  + V$  
\choice ${E_1} = \varepsilon  + V$, ${E_2} = \varepsilon $    
\choice ${E_1} = \varepsilon $, ${E_2} = \varepsilon  - V$    
\choice ${E_{1,2}} = \varepsilon  \pm V$
\end{choices}

\question Уровень энергии $\varepsilon $ некоторой квантовой системы двукратно вырожден. На систему накладывается малое возмущение, матричные элементы которого с невозмущенными собственными функциями известны: ${V_{11}} = V$, ${V_{22}} = U$ ($U \ne V$), ${V_{12}} = {V_{21}} = 0$. Какими будут энергии возмущенных состояний ${E_1}$ и ${E_2}$ в первом порядке теории возмущений?
\begin{choices}
\choice ${E_1} = {E_2} = \varepsilon  + V + U$    
\choice ${E_1} = \varepsilon  + V$, ${E_2} = \varepsilon  - U$
\choice ${E_1} = \varepsilon  - V$, ${E_2} = \varepsilon  + U$      
\choice ${E_1} = \varepsilon  + V$, ${E_2} = \varepsilon  + U$
\end{choices}

\question Уровень энергии $\varepsilon $ некоторой квантовой системы двукратно вырожден. На систему накладывается малое возмущение, матричные элементы которого с невозмущенными собственными функциями ${\varphi _1}$ и ${\varphi _2}$ известны: ${V_{11}} = {V_{22}} = 0$, ${V_{12}} = {V_{21}} = V$. Какими будут правильные функции нулевого приближения?
\begin{choices}
\choice ${\psi _1} = {\varphi _1}$, ${\psi _2} = {\varphi _2}$,           
\choice ${\psi _1} \sim {\varphi _1} + {\varphi _2}$, ${\psi _2} \sim {\varphi _1} - {\varphi _2}$
\choice ${\psi _1} \sim {\varphi _1} + \left( {V/\varepsilon } \right){\varphi _2}$, ${\psi _2} \sim {\varphi _1} - \left( {V/\varepsilon } \right){\varphi _2}$ 
\choice любые линейные комбинации ${\varphi _1}$ и ${\varphi _2}$
\end{choices}

\question Уровень энергии $\varepsilon $ некоторой квантовой системы двукратно вырожден. На систему накладывается малое возмущение, матричные элементы которого с невозмущенными собственными функциями ${\varphi _1}$ и ${\varphi _2}$ известны: ${V_{11}} = V$, ${V_{22}} = U$ ($U \ne V$), ${V_{12}} = {V_{21}} = 0$. Какими будут правильные функции нулевого приближения?
\begin{choices}
\choice ${\psi _1} = {\varphi _1}$, ${\psi _2} = {\varphi _2}$,           
\choice ${\psi _1} \sim {\varphi _1} + {\varphi _2}$, ${\psi _2} \sim {\varphi _1} - {\varphi _2}$
\choice ${\psi _1} \sim {\varphi _1} + \left( {V/\varepsilon } \right){\varphi _2}$, ${\psi _2} \sim {\varphi _1} - \left( {U/\varepsilon } \right){\varphi _2}$ 
\choice любые линейные комбинации ${\varphi _1}$ и ${\varphi _2}$
\end{choices}

\question Уровень энергии $\varepsilon $ некоторой квантовой системы двукратно вырожден. На систему накладывается малое возмущение, матричные элементы которого с невозмущенными собственными функциями ${\varphi _1}$ и ${\varphi _2}$ известны: ${V_{11}} = {V_{22}} = V$, ${V_{12}} = {V_{21}} = 0$. Какими будут правильные функции нулевого приближения?
\begin{choices}
\choice ${\psi _1} = {\varphi _1}$, ${\psi _2} = {\varphi _2}$,           
\choice ${\psi _1} \sim {\varphi _1} + {\varphi _2}$, ${\psi _2} \sim {\varphi _1} - {\varphi _2}$
\choice ${\psi _1} \sim {\varphi _1} + \left( {V/\varepsilon } \right){\varphi _2}$, ${\psi _2} \sim {\varphi _1} - \left( {V/\varepsilon } \right){\varphi _2}$ 
\choice любые линейные комбинации ${\varphi _1}$ и ${\varphi _2}$
\end{choices}

\question Собственные функции ${\varphi _1},\;\;{\varphi _2},\;\;{\varphi _3}$ невозмущенного гамильто-ниана, относящиеся к трехкратно вырожденному уровню с энергией $\varepsilon $, известны. На кван-товую систему накладывают возмущение $\hat V$, матричные элементы которого с функциями ${\varphi _{i\;}}$ равны: ${V_{11}} = {V_{22}} = v$, ${V_{33}} = u$ $(u \ne v)$, ${V_{ik}} = 0$ при $i \ne k$. Какими будут правильные функции нулевого приближения?
\begin{choices}
\choice только ${\psi _1} = {\varphi _1}$, ${\psi _2} = {\varphi _2}$, ${\psi _3} = {\varphi _3}$
\choice ${\psi _1}$ и ${\psi _2}$ - любые линейные комбинации функций , ${\psi _3} = {\varphi _3}$
\choice ${\psi _1}$ и ${\psi _2}$ - любые линейные комбинации функций , ${\psi _3} = {\varphi _2}$
\choice только ${\psi _1} \sim \left( {{\varphi _1} + {\varphi _2}} \right)$, ${\psi _2} \sim \left( {{\varphi _1} - {\varphi _2}} \right)$, ${\psi _3} = {\varphi _3}$
\end{choices}

\question Собственные функции ${\varphi _1},\;\;{\varphi _2},\;\;{\varphi _3}$ невозмущенного гамильто-ниана, относящиеся к трехкратно вырожденному уровню с энергией $\varepsilon $, известны. На кван-товую систему накладывают возмущение $\hat V$, матричные элементы которого с функциями ${\varphi _{i\;}}$ равны: ${V_{11}} = {V_{22}} = V$, ${V_{33}} = U$, ${V_{ik}} = 0$ при $i \ne k$. В первом порядке теории возмущений найти энергии собственных состояний возмущенного гамильто-ниана
\begin{choices}
\choice ${E_1} = {E_2} = \varepsilon  + \frac{{v + u}}{2}$, ${E_3} = \varepsilon  + \frac{{v - u}}{2}$  
\choice ${E_1} = {E_2} = \varepsilon  + v + u$, ${E_3} = \varepsilon  + v - u$
\choice ${E_1} = {E_2} = \varepsilon  + v$, ${E_3} = \varepsilon  + u$       
\choice ${E_1} = {E_2} = {E_3} = \varepsilon  + v + u$
\end{choices}

\question Собственные функции ${\varphi _1},\;\;{\varphi _2},\;\;{\varphi _3}$ невозмущенного гамильто-ниана, относящиеся к трехкратно вырожденному уровню с энергией $\varepsilon $, известны. На кван-товую систему накладывают возмущение $\hat V$, матричные элементы которого с функциями ${\varphi _{i\;}}$ равны: ${V_{11}} = {V_{22}} = V$, ${V_{33}} = U$ ($U \ne V$), ${V_{ik}} = 0$ при $i \ne k$. Будет ли в первом порядке теории возмущений сниматься вырождение уровня?
\begin{choices}
\choice да, полностью  
\choice да, частично   
\choice нет      
\choice информации для ответа недостаточно
\end{choices}

\question Собственные функции ${\varphi _1},\;\;{\varphi _2},\;\;{\varphi _3}$ невозмущенного гамильто-ниана, относящиеся к трехкратно вырожденному уровню с энергией $\varepsilon $, известны. На кван-товую систему накладывают возмущение $\hat V$, матричные элементы которого с функциями ${\varphi _{i\;}}$ равны: ${V_{11}} = {V_{22}} = {V_{33}} = {V_{13}} = {V_{31}} = {V_{23}} = {V_{32}} = 0$, ${V_{12}} = {V_{21}} = V$. В первом порядке теории возмущений найти энергии собственных со-стояний возмущенного гамильтониана
\begin{choices}
\choice ${E_1} = {E_2} = \varepsilon  + V$, ${E_3} = \varepsilon  - V$    
\choice ${E_1} = \varepsilon  + V$, ${E_2} = \varepsilon  - V$, ${E_3} = \varepsilon $
\choice ${E_1} = {E_2} = \varepsilon  + V$, ${E_3} = \varepsilon $        
\choice ${E_1} = {E_2} = {E_3} = \varepsilon  + V$
\end{choices}

\question Собственные функции ${\varphi _1},\;\;{\varphi _2},\;\;{\varphi _3}$ невозмущенного гамильто-ниана, относящиеся к трехкратно вырожденному уровню с энергией $\varepsilon $, известны. На кван-товую систему накладывают возмущение $\hat V$, матричные элементы которого с функциями ${\varphi _{i\;}}$ равны: ${V_{11}} = {V_{22}} = {V_{33}} = {V_{13}} = {V_{31}} = {V_{23}} = {V_{32}} = 0$, ${V_{12}} = {V_{21}} = V$. Найти правильные функции нулевого приближения
\begin{choices}
\choice ${\psi _1} \sim {\varphi _1} + {\varphi _2}$, ${\psi _2} \sim {\varphi _1} - {\varphi _2}$, ${\psi _3} \sim {\varphi _3}$        
\choice ${\psi _1} \sim {\varphi _1}$, ${\psi _2} \sim {\varphi _2}$, ${\psi _3} \sim {\varphi _3}$
\choice ${\psi _1} \sim {\varphi _1} + \left( {V/\varepsilon } \right){\varphi _2}$, ${\psi _2} \sim {\varphi _1} - \left( {V/\varepsilon } \right){\varphi _2}$, ${\psi _3} \sim {\varphi _3}$      
\choice ${\psi _1} \sim {\varphi _1} + {\varphi _2}$, ${\psi _2} \sim {\varphi _2} - {\varphi _3}$, ${\psi _3} \sim {\varphi _3} - {\varphi _1}$
\end{choices}

\question Собственные функции ${\varphi _1},\;\;{\varphi _2},\;\;{\varphi _3}$ невозмущенного гамильто-ниана, относящиеся к трехкратно вырожденному уровню с энергией $\varepsilon $, известны. На кван-товую систему накладывают возмущение $\hat V$, матричные элементы которого с функциями ${\varphi _{i\;}}$ равны: ${V_{11}} = V$, ${V_{22}} = U$, ${V_{33}} = W$ ($V \ne U \ne W$), ${V_{ik}} = 0$ при $i \ne k$. В первом порядке теории возмущений найти энергии собственных состояний возму-щенного гамильтониана
\begin{choices}
\choice ${E_1} = \varepsilon  + V + U$, ${E_2} = \varepsilon  + V - U$, ${E_3} = \varepsilon  + W$   
\choice ${E_1} = \varepsilon  + V$, ${E_2} = \varepsilon  + U$, ${E_3} = \varepsilon  + W$  
\choice ${E_1} = \varepsilon  + V + U + W$, ${E_2} = \varepsilon  + V + U - W$, ${E_3} = \varepsilon  + V - U - W$
\choice ${E_1} = \varepsilon  + V + U + W$, ${E_2} = \varepsilon  - (V + U + W)$, ${E_3} = \varepsilon $
\end{choices}

\question Собственные функции ${\varphi _1},\;\;{\varphi _2},\;\;{\varphi _3}$ невозмущенного гамильто-ниана, относящиеся к трехкратно вырожденному уровню с энергией $\varepsilon $, известны. На кван-товую систему накладывают возмущение $\hat V$, матричные элементы которого с функциями ${\varphi _{i\;}}$ равны: ${V_{11}} = V$, ${V_{22}} = U$, ${V_{33}} = W$ ($V \ne U \ne W$), ${V_{ik}} = 0$ при $i \ne k$. Найти правильные функции нулевого приближения
\begin{choices}
\choice ${\psi _1} \sim {\varphi _1} + {\varphi _2}$, ${\psi _2} \sim {\varphi _1} - {\varphi _2}$, ${\psi _3} \sim {\varphi _3}$  
\choice только ${\psi _1} \sim {\varphi _1}$, ${\psi _2} \sim {\varphi _2}$, ${\psi _3} \sim {\varphi _3}$
\choice ${\psi _1} \sim {\varphi _1} + {\varphi _3}$, ${\psi _2} \sim {\varphi _1} - {\varphi _3}$, ${\psi _3} \sim {\varphi _2}$  
\choice ${\psi _1}$, ${\psi _2}$ и ${\psi _3}$ - любые комбинации функций ${\varphi _1}\;$, ${\varphi _2}$ и ${\varphi _3}$
\end{choices}

\question На незаряженную частицу со спином $s$ накладывают слабое электрическое поле. На сколько подуровней расщепятся уровни энергии частицы, если частица обладает магнитным моментом?
\begin{choices}
\choice не расщепятся  
\choice на $s$   
\choice на $2s + 1$ 
\choice на $2s + 2$
\end{choices}

\question На заряженную частицу, находящуюся в центральном поле (без случайного вырождения), накла-дывают электрическое поле. На сколько подуровней расщепится уровень энергии с моментом $l \ne 0$ в первом порядке теории возмущений?
\begin{choices}
\choice не расщепится  
\choice на $l$      
\choice на $2l + 1$ 
\choice на $2l + 2$
\end{choices}

\question Заряженная частица находится в центральном поле со случайным вырождением. Имеется уро-вень энергии с вырождением состояний с $l = 0$ и $l = 2$. На частицу накладывают однородное элек-трическое поле. Произойдет ли расщепление этого вырожденного уровня энергии в первом порядке теории возмущений, и если да, то на сколько подуровней он расщепится?
\begin{choices}
\choice да, на 6    
\choice нет      
\choice да, на 3 
\choice да, на 2
\end{choices}

\question Заряженная частица находится в центральном поле со случайным вырождением. Имеется уро-вень энергии с вырождением состояний с $l = 0$ и $l = 1$. На частицу накладывают однородное элек-трическое поле. Произойдет ли расщепление этого вырожденного уровня энергии в первом порядке теории возмущений, и если да, то на сколько подуровней он расщепится?
\begin{choices}
\choice да, на 16      
\choice нет      
\choice да, на 3 
\choice да, на 2
7.2.35 Заряженная частица находится в центральном поле со случайным вырождением. Имеется уро-вень энергии с вырождением состояний с $l = 3$ и $l = 4$. На частицу накладывают однородное электрическое поле. Произойдет ли расщепление этого вырожденного уровня энергии в первом порядке теории возмущений?
\choice да
\choice нет
\choice зависит от величины поля
\choice мало информации для ответа
\end{choices}

\question На бесспиновую заряженную частицу, находящуюся в центральном поле (без случайного выро-ждения), накладывают магнитное поле. На сколько подуровней расщепится уровень энергии с момен-том $l \ne 0$?
\begin{choices}
\choice не расщепится  
\choice на $l$      
\choice на $2l + 1$ 
\choice на $2l + 2$
\end{choices}

\question На бесспиновую заряженную частицу, находящуюся в центральном поле (без случайного выро-ждения), накладывают магнитное поле, направленное вдоль оси $z$. Какими будут правильные функ-ции нулевого приближения, отвечающие расщепленному уровню с моментом $l$?
\begin{choices}
\choice каждая содержит одну сферическую функцию ${Y_{lm}}$
\choice каждая содержит одну комбинацию ${Y_{lm}} + {Y_{l - m}}$
\choice каждая содержит одну комбинацию ${Y_{lm}} + {Y_{l0}}$
\choice правильными функциями будут произвольные комбинации функций ${Y_{lm}}$ 
(здесь $m$ меняется от $ - l$ до $l$, через единицу).
\end{choices}

\question На бесспиновую заряженную частицу, находящуюся в центральном поле (без случайного выро-ждения), накладывают магнитное поле, направленное вдоль оси $y$. Какими будут правильные функ-ции нулевого приближения, отвечающие расщепленному уровню с моментом $l$?
\begin{choices}
\choice каждая содержит одну сферическую функцию ${Y_{lm}}$
\choice каждая содержит такую комбинацию функций ${Y_{lm}}$, которая является собственной для опера-тора ${\hat L_y}$
\choice каждая содержит такую комбинацию функций ${Y_{lm}}$, которая является собственной для опера-тора ${\hat L_x}$
\choice правильными функциями будут произвольные комбинации функций ${Y_{lm}}$ 
(здесь $m$ меняется от $ - l$ до $l$, через единицу).
\end{choices}

\question На бесспиновую заряженную частицу, находящуюся в центральном поле (без случайного выро-ждения), накладывают магнитное поле, направленное вдоль оси $x$. Какая (какие) из нижеперечислен-ных функций верно описывает угловую часть правильной функции нулевого приближения, отве-чающей расщепленному уровню с моментом $l = 1$?
\begin{choices}
\choice $\cos \vartheta $  
\choice $\cos \vartheta \sin \varphi $    
\choice $\sin \vartheta \sin \varphi $    
\choice $\sin \vartheta \cos \varphi $
\end{choices}

\question На бесспиновую заряженную частицу, находящуюся в центральном поле (без случайного выро-ждения), накладывают магнитное поле, направленное вдоль оси $y$. Правильные функции нулевого приближения, отвечающие расщепленному уровню с моментом $l$, будут собственными функциями
\begin{choices}
\choice оператора ${\hat L_x}$  
\choice оператора ${\hat L_y}$  
\choice оператора ${\hat L_z}$  
\choice ни одного из них
\end{choices}

\question На бесспиновую заряженную частицу, находящуюся в центральном поле (без случайного выро-ждения), накладывают слабое магнитное поле с напряженностью $H$. В каком энергетическом интер-вале будут лежать подуровни, на которые расщепится уровень энергии с моментом $l$?
\begin{choices}
\choice $\Delta E = \frac{{e\hbar Hl}}{{mc}}$  
\choice $\Delta E = \frac{{2e\hbar H}}{{mc}}$  
\choice $\Delta E = \frac{{e\hbar Hl}}{{2mc}}$    
\choice $\Delta E = \frac{{2e\hbar Hl}}{{mc}}$
\end{choices}

\question На бесспиновую положительно заряженную частицу, находящуюся в центральном поле (без слу-чайного вырождения), накладывают слабое магнитное поле, направленное вдоль оси $z$. Какой проек-цией орбитального момента импульса на ось $z$ будет обладать подуровень с минимальной энергией?
\begin{choices}
\choice ${l_z} = l$ 
\choice ${l_z} =  - l$ 
\choice ${l_z} = 0$ 
\choice проекция не будет иметь определенного значения
\end{choices}

\question На бесспиновую незаряженную частицу, находящуюся в центральном поле (без случайного вы-рождения), накладывают магнитное поле. На сколько подуровней расщепится уровень энергии с мо-ментом $l$?
\begin{choices}
\choice не расщепится  
\choice на $l$      
\choice на $2l + 1$    
\choice на $2l + 2$
\end{choices}

\question На частицу, находящуюся в центральном поле без случайного вырождения, накладывается воз-мущение, оператор которого зависит только от модуля радиуса-вектора. Будет ли «сниматься» вырож-дение уровней энергии в первом порядке теории возмущений?
\begin{choices}
\choice да             
\choice нет
\choice будет сниматься частично   
\choice его и не было: по условию вырождение отсутствует
\end{choices}

\question На частицу, находящуюся в центральном поле без случайного вырождения, накладывается воз-мущение, оператор которого зависит только от модуля радиуса-вектора. Будет ли «сниматься» вырож-дение уровней энергии во втором порядке теории возмущений?
\begin{choices}
\choice да             
\choice нет
\choice будет сниматься частично   
\choice его и не было: по условию вырождение отсутствует
\end{choices}

\question На частицу, находящуюся в центральном поле со случайным вырождением, накладывается воз-мущение, оператор которого зависит только от модуля радиуса-вектора. Будет ли «сниматься» вырож-дение уровней энергии, и если да, то полностью или частично?
\begin{choices}
\choice да, полностью
\choice нет
\choice да, частично, если зависимость возмущения от $r$ не такая, как невозмущенного центрального поля
\choice да, полностью, если зависимость возмущения от $r$ не такая, как невозмущенного центрального по-ля
\end{choices}

\question На частицу, находящуюся в центрально-симметричном поле, в котором отсутствует случайное вырождение, накладывается возмущение $\hat V = A\cos \vartheta $. На сколько подуровней расщепит-ся уровень с моментом $l$ в первом порядке теории возмущений?
\begin{choices}
\choice на $2l + 1$    
\choice расщепления не будет    
\choice на $l + 1$     
\choice на $l$
\end{choices}

\question На частицу, находящуюся в центрально-симметричном поле, в котором отсутствует случайное вырождение, накладывается возмущение $\hat V = A{\cos ^2}\vartheta $. На сколько подуровней рас-щепится уровень с моментом $l$?
\begin{choices}
\choice на $2l + 1$    
\choice расщепления не будет    
\choice на $l + 1$     
\choice на $l$
\end{choices}

\question На частицу, находящуюся в центрально-симметричном поле, в котором отсутствует случайное вырождение, накладывается возмущение $\hat V = A{\cos ^2}\vartheta $. Какое утверждение относи-тельно правильных функций нулевого приближения для уровня с моментом $l$ является верным?
\begin{choices}
\choice каждая правильная функция содержит одну сферическую функцию ${Y_{lm}}$
\choice каждая правильная функция будет произвольной линейной комбинацией функций ${Y_{lm}}$ и ${Y_{l - m}}$
\choice каждая правильная функция будет произвольной линейной комбинацией функций ${Y_{lm}}$ и ${Y_{lm + 1}}$
\choice каждая правильная функция будет произвольной линейной комбинацией функций ${Y_{lm}}$ и ${Y_{lm - 2}}$
 
\end{choices}

\question На заряженный трехмерный гармонический осциллятор накладывают однородное электриче-ское поле с напряженностью $E$. На какие подуровни расщепится первый возбужденный уровень энергии осциллятора (здесь $q$ - заряд, $a = \sqrt {\hbar /m\omega } $ - параметр длины для осциллятора)? Указание: кратность вырождения первого возбужденного уровня энергии трехмерного гармонического осциллятора равна 3.
\begin{choices}
\choice не расщепится           
\choice ${E_1} = \varepsilon  + qEa$, ${E_2} = \varepsilon  - qEa$, ${E_3} = \varepsilon $
\choice ${E_1} = \varepsilon  + 2qEa$,  ${E_2} = \varepsilon  - 2qEa$, ${E_3} = \varepsilon $  
\choice ${E_1} = \varepsilon  + 3qEa$, ${E_2} = \varepsilon  - 3qEa$, ${E_3} = \varepsilon $
\end{choices}

\question На заряженный трехмерный гармонический осциллятор накладывают однородное электриче-ское поле с напряженностью $E$. На какие подуровни расщепится второй возбужденный уровень энергии осциллятора (здесь $q$ - заряд, $a = \sqrt {\hbar /m\omega } $ - параметр длины для осциллятора)? Указание: кратность вырождения второго возбужденного уровня энергии трехмерного гармонического осциллятора равна 6.
\begin{choices}
\choice не расщепится           
\choice ${E_1} = \varepsilon  + qEa$, ${E_2} = \varepsilon  - qEa$, ${E_3} = \varepsilon $
\choice ${E_1} = \varepsilon  + 2qEa$,  ${E_2} = \varepsilon  - 2qEa$, ${E_3} = \varepsilon $  
\choice ${E_1} = \varepsilon  + 3qEa$, ${E_2} = \varepsilon  - 3qEa$, ${E_3} = \varepsilon $
\end{choices}

\question На заряженный трехмерный гармонический осциллятор накладывают однородное электриче-ское поле. Произойдет ли расщепление каких-нибудь уровней осциллятора в первом порядке теории возмущений? Указание. Все уровни энергии трехмерного гармонического осциллятора обладают определенной четностью.
\begin{choices}
\choice да    
\choice нет      
\choice только четных     
\choice только нечетных
\end{choices}

\question На трехмерный гармонический осциллятор накладывают возмущение $\hat V(r) = \alpha {r^2}$. Будет ли сниматься вырождение уровней энергии в высших порядках теории возмущений?
\begin{choices}
\choice да    
\choice нет      
\choice только четных     
\choice только нечетных
\end{choices}

\question На трехмерный гармонический осциллятор накладывают возмущение $\hat V = \alpha {x^2}$. На сколько подуровней расщепится первый возбужденный уровень энергии? Указание. Кратность вырождения первого возбужденного уровня равна 3, декартовы квантовые числа вырожденных состояний равны: ${n_x} = 1,\;{n_y} = 0,\;{n_z} = 0$; ${n_x} = 0,\;{n_y} = 1,\;{n_z} = 0$, ${n_x} = 0,\;{n_y} = 0,\;{n_z} = 1$.
\begin{choices}
\choice не расщепится  
\choice на два   
\choice на три   
\choice на четыре
\end{choices}

\question На трехмерный гармонический осциллятор накладывают возмущение $\hat V = m{\alpha ^2}{x^2}/2$. Какими будут правильные функции ${\psi _i}$, отвечающие первому возбужденному уровню невозмущенного осциллятора? Указание. Кратность вырождения первого возбужденного уров-ня равна 3, декартовы квантовые числа вырожденных состояний равны: ${n_x} = 1,\;{n_y} = 0,\;{n_z} = 0$; ${n_x} = 0,\;{n_y} = 1,\;{n_z} = 0$, ${n_x} = 0,\;{n_y} = 0,\;{n_z} = 1$.
\begin{choices}
\choice ${\psi _1} \sim x{e^{ - {x^2}/2a_1^2}}{e^{ - ({y^2} + {z^2})/2a_2^2}}$,  ${\psi _2} \sim y{e^{ - {x^2}/2a_1^2}}{e^{ - ({y^2} + {z^2})/2a_2^2}}$, ${\psi _3} \sim z{e^{ - {x^2}/2a_1^2}}{e^{ - ({y^2} + {z^2})/2a_2^2}}$
\choice ${\psi _1} \sim x{e^{ - {x^2}/2a_1^2}}{e^{ - ({y^2} + {z^2})/2a_2^2}}$,  ${\psi _{2,3}} \sim y{e^{ - {x^2}/2a_1^2}}{e^{ - ({y^2} + {z^2})/2a_2^2}} \pm z{e^{ - {x^2}/2a_1^2}}{e^{ - ({y^2} + {z^2})/2a_2^2}}$
\choice ${\psi _1} \sim x{e^{ - {x^2}/2a_1^2}}{e^{ - ({y^2} + {z^2})/2a_2^2}}$, ${\psi _{2,3}}$ - любые комби-нации функций $y{e^{ - {x^2}/2a_1^2}}{e^{ - ({y^2} + {z^2})/2a_2^2}}$, $z{e^{ - {x^2}/2a_1^2}}{e^{ - ({y^2} + {z^2})/2a_2^2}}$
\choice любые комбинации функций $x{e^{ - {x^2}/2a_1^2}}{e^{ - ({y^2} + {z^2})/2a_2^2}}$, $y{e^{ - {x^2}/2a_1^2}}{e^{ - ({y^2} + {z^2})/2a_2^2}}$, $z{e^{ - {x^2}/2a_1^2}}{e^{ - ({y^2} + {z^2})/2a_2^2}}$
(здесь $a_1^2 = \hbar /m\sqrt {{\omega ^2} + {\alpha ^2}} $, $a_2^2 = \hbar /m\omega $).
\end{choices}

\question На двумерный гармонический осциллятор накладывают малое возмущение $\hat V = \alpha xy$. Произойдет ли расщепление первого возбужденного уровня энергии?
\begin{choices}
\choice да    
\choice нет      
\choice зависит от $\alpha $ 
\choice мало информации для ответа
\end{choices}

\question На двумерный гармонический осциллятор накладывают возмущение $\hat V = \alpha xy$. Ка-ким будет расщепление первого возбужденного уровня энергии в первом порядке теории возмущений? Указание. Кратность вырождения первого возбужденного уровня двумерного осциллятора равна 2, де-картовы квантовые числа вырожденных состояний равны: ${n_x} = 1,\;{n_y} = 0$; ${n_x} = 0,\;{n_y} = 1$, для собственных функций основного ${\varphi _0}(x)$ и первого возбужденного ${\varphi _1}(x)$ состояний гармонического осциллятора справедливо равенство $\int {{\varphi _0}(x)x{\varphi _1}(x)dx = } \sqrt {\hbar /2m\omega } $.
\begin{choices}
\choice $\Delta E = \frac{{\alpha \hbar }}{{m\omega }}$    
\choice $\Delta E = \frac{{\alpha \hbar }}{{{m^2}{\omega ^3}}}$     
\choice $\Delta E = 0$    
\choice $\Delta E \sim \alpha \sqrt {\frac{\hbar }{{m\omega }}} $
\end{choices}

\question На трехмерный гармонический осциллятор накладывают возмущение, оператор которого зави-сит только от модуля радиуса-вектора. Будет ли сниматься вырождение первого возбужденного уровня энергии? Указание. Кратность вырождения первого возбужденного уровня энергии равна 3.
\begin{choices}
\choice да, полностью  
\choice да, частично      
\choice нет      
\choice мало информации для ответа
\end{choices}

\question На трехмерный гармонический осциллятор накладываю возмущение, оператор которого зависит только от модуля радиуса-вектора, причем эта зависимость не квадратичная. Произойдет ли расщепле-ние второго возбужденного уровня энергии, и если да, то будет ли оно полным или частичным? Указа-ние. Кратность вырождения второго возбужденного уровня энергии равна 6.
\begin{choices}
\choice да, полным     
\choice да, частичным  
\choice нет      
\choice мало информации для ответа
\end{choices}

\question На трехмерный гармонический осциллятор накладываю возмущение, оператор которого зависит только от модуля радиуса-вектора, причем эта зависимость не квадратичная. На сколько подуровней расщепится второй возбужденный уровень энергии? Указание. Кратность вырождения второго возбуж-денного уровня энергии равна 6.
\begin{choices}
\choice на два      
\choice на три      
\choice на четыре   
\choice на пять
\end{choices}

\question Уровни энергии заряженной частицы в кулоновском поле вырождены по моменту и по проекции. Какое вырождение уровней энергии электрона в атоме будет «снимать» учет неточечности атомного ядра?
\begin{choices}
\choice пропадет вырождение по моменту, по проекции останется 
\choice пропадет вырождение по проекции, по моменту останется
\choice вырождение полностью пропадет
\choice вырождение полностью останется
\end{choices}

\question На атом водорода накладывают возмущение, оператор которого зависит только от модуля радиу-са-вектора, причем эта зависимость не $1/r$. На сколько подуровней расщепится первый возбужден-ный уровень энергии? Указание. Кратность вырождения первого возбужденного уровня энергии равна 4.
\begin{choices}
\choice вырождение не пропадет     
\choice на два
\choice на три            
\choice на четыре
\end{choices}

\question На атом водорода накладывают возмущение, оператор которого зависит только от модуля радиу-са-вектора, причем эта зависимость не $1/r$. Какими будут правильные функции нулевого приближе-ния, отвечающие первому возбужденному уровню невозмущенного атома? Указание. Кратность выро-ждения первого возбужденного уровня энергии равна 4.
\begin{choices}
\choice произвольные комбинации функций ${Y_{00}}$,  ${Y_{10}}$, ${Y_{11}}$, ${Y_{1 - 1}}$
\choice ${\psi _1} \sim {Y_{00}}$, ${\psi _2} \sim {Y_{10}}$, ${\psi _3} \sim {Y_{11}}$, ${\psi _4} \sim {Y_{1 - 1}}$
\choice ${\psi _1} \sim {Y_{00}}$, ${\psi _{2,3,4}}$ произвольные комбинации функций ${Y_{10}}$, ${Y_{11}}$, ${Y_{1 - 1}}$
\choice ${\psi _1} \sim {Y_{00}}$, ${\psi _2} \sim {Y_{10}} + {Y_{11}} + {Y_{1 - 1}}$, ${\psi _3} \sim {Y_{10}} + {Y_{11}} - {Y_{1 - 1}}$, ${\psi _4} \sim {Y_{10}} - {Y_{11}} - {Y_{1 - 1}}$
\end{choices}

\question Частица движется в центральном поле. Чтобы снять вырождение по энергии состояний с проек-циями $m$ и $ - m$ на частицу нужно наложить возмущение, зависящее только от
\begin{choices}
\choice $r$      б. $\vartheta $      в. $r$ и $\vartheta $      
\choice $\varphi $
\end{choices}

\question Бесспиновая заряженная частица находится в кулоновском поле. На сколько подуровней расще-пится первый возбужденный уровень энергии частицы, если наложить на нее слабое магнитное поле? Указание. Кратность вырождения первого возбужденного уровня энергии в кулоновском поле равна 4.
\begin{choices}
\choice расщепления не будет    
\choice на два
\choice на три            
\choice на четыре
\end{choices}

\end{questions}





\section{ Глава 8. Квантовые переходы }


\subsection{ Теория  нестационарных возмущений }

Во всех задачах этой главы ${V_{kn}}$ - матричный элемент оператора возмущения с собст-венными функциями невозмущенного гамильтониана, ${\omega _{kn}} = ({\varepsilon _k} - {\varepsilon _n})/\hbar $ - частота перехода между состояниями $k$ и $n$, ${\varepsilon _k},{\varepsilon _n}$ - невозмущенные энергии. $\vartheta $ и $\varphi $ - полярный и ази-мутальный углы сферической системы координат соответственно. Если это не оговорено особо, вероятности переходов нужно вычислять в первом порядке теории возмущений для волновой функции.

\begin{questions}


\question Под действием каких возмущений квантовые системы могут совершать переходы из одних стационарных состояний в другие?
\begin{choices}
\choice если возмущения – большие
\choice если возмущения снимают вырождение
\choice если возмущения не зависят от времени
\choice если возмущения зависят от времени 
\end{choices}

\question Теория нестационарных возмущений представляет собой приближенный метод реше-ния
\begin{choices}
\choice стационарного уравнения Шредингера
\choice временного уравнения Шредингера
\choice уравнения непрерывности
\choice уравнения на собственные значения оператора импульса
\end{choices}

\question  Какой формулой определяется вероятность перехода из $k$-го в $n$-ое стационарное состояние под действием возмущения $\hat V(x,t)$ в первом порядке нестационарной тео-рии возмущений для волновой функции
\begin{choices}
\choice ${w_{k \to n}} = |\frac{i}{\hbar }\int\limits_{{t_1}}^{{t_2}} {{V_{kn}}(t){e^{i{\omega _{kn}}t}}dt} {|^2}$     
\choice ${w_{k \to n}} = |\frac{i}{\hbar }\int\limits_{{t_1}}^{{t_2}} {{V_{kn}}(t)dt} {|^2}$   
\choice ${w_{k \to n}} = |\frac{i}{\hbar }\int\limits_{{t_1}}^{{t_2}} {\hat V(x,t){e^{i{\omega _{kn}}t}}dxdt} {|^2}$   
\choice ${w_{k \to n}} = |\frac{i}{\hbar }\int\limits_{{t_1}}^{{t_2}} {{\psi _k}^*(x)\hat V(x,t){\psi _n}(x)dxdt} {|^2}$
(здесь ${t_1}$ и ${t_2}$ - моменты начала и окончания действия возмущения).
\end{choices}

\question В каком случае вероятность перехода из $k$-го в $n$-ое стационарное состояние под действием возмущения $\hat V(t)$ равна нулю в первом порядке теории возмущений для волновой функции?
\begin{choices}
\choice если равна нулю частота перехода между этими состояниями
\choice если равно нулю произведения волновых функций этих состояний
\choice если равен нулю матричный элемент возмущения с волновыми функциями этих состоя-ний
\choice если возмущение не снимает вырождение этих состояний
\end{choices}

\question На квантовую систему, находящуюся в стационарном состоянии, накладывают зави-сящее от времени возмущение, которое через некоторое время «выключается». В какие со-стояния совершаются переходы?
\begin{choices}
\choice в стационарные с большей энергией
\choice в стационарные с меньшей энергией
\choice и в те, и в другие
\choice ни в какие из перечисленных
\end{choices}

\question Что такое частота перехода ${\omega _{kn}}$ между $k$-ым и $n$-ым стационарными состояниями?
\begin{choices}
\choice ${\omega _{kn}} = \frac{{{\varphi _k} - {\varphi _n}}}{\hbar }$   
\choice ${\omega _{kn}} = \frac{{{\varepsilon _k} - {\varepsilon _n}}}{\hbar }$ 
\choice ${\omega _{kn}} = \left( {{\varphi _k} - {\varphi _n}} \right)\hbar $      
\choice ${\omega _{kn}} = \left( {{\varepsilon _k} - {\varepsilon _n}} \right)\hbar $
(здесь ${\varphi _i}$ и ${\varepsilon _i}$ - собственные функции и собственные значения не-возмущенного гамильтониана).
\end{choices}

\question Какова размерность частоты перехода между двумя стационарными состояниями?
\begin{choices}
\choice    
\choice    
\choice    
\choice 
\end{choices}

\question На некоторую квантовую систему, находящуюся в $n$-ом стационарном состоянии, накладывают малое, зависящее от времени возмущение $\hat V(x,t) = \hat V(x){e^{ - {t^2}/{\tau ^2}}}$. Известно, что матричные элементы ${V_{nk}}$ оператора $\hat V(x)$ не зависят от индекса $k$. В состояния с какими энергиями ${\varepsilon _k}$ переходы систе-мы будут более вероятными?
\begin{choices}
\choice $\left| {{\varepsilon _k} - {\varepsilon _n}} \right| \le \frac{\hbar }{\tau }$     
\choice $\left| {{\varepsilon _k} - {\varepsilon _n}} \right| \gg \frac{\hbar }{\tau }$
\choice $\left| {{\varepsilon _k} - {\varepsilon _n}} \right| = \frac{\hbar }{\tau }$    
\choice во все состояния с равными вероятностями
\end{choices}

\question На квантовую систему накладывают зависящее от времени возмущение $\alpha \hat V(x,t)$, где $\alpha $ - некоторое число. Как вероятности переходов под действием этого возмущения, вычисленные в первом порядке теории нестационарных возмущений для вол-новой функции, зависят от $\alpha $?
\begin{choices}
\choice как $\alpha $  
\choice как ${\alpha ^2}$ 
\choice как ${\alpha ^3}$ 
\choice как ${\alpha ^4}$
\end{choices}

\question На квантовую систему накладывают зависящее от времени возмущение $\alpha \hat V(x,t)$, где $\alpha $ - некоторое число. Вероятность перехода между некоторыми состоя-ниями, вычисленная в первом порядке теории возмущений, равна нулю. Как вероятность перехода между этими состояниями, вычисленная во втором порядке теории нестационарных возмущений для волновой функции, зависит от $\alpha $?
\begin{choices}
\choice как $\alpha $  
\choice как ${\alpha ^2}$ 
\choice как ${\alpha ^3}$ 
\choice как ${\alpha ^4}$
\end{choices}

\question Каков параметр малости теории нестационарных возмущений?
\begin{choices}
\choice возмущение должно быть мало по сравнению с разностью энергий уровней
\choice возмущение должно быть мало по сравнению с гамильтонианом
\choice вероятности переходов, вычисленные в рамках теории возмущений, должны быть малы
\choice вероятность, что квантовая система останется в начальном состоянии, должна быть мала 
\end{choices}

\question На частицу, находящуюся в бесконечно глубокой прямоугольной потенциальной яме, расположенной между точками $x = 0$ и $x = a$, накладывают возмущение $\hat V(x,t) = {V_0}\cos \frac{{3\pi x}}{a}\;f(t)$, где $f(t)$ - некоторая функция времени. В какие стацио-нарные состояния (основное состояние – первое) возможны переходы из основного состоя-ния? 
\begin{choices}
\choice в 3-е стационарное состояние     
\choice во 2-е и 4-е стационарные состояния
\choice во 2-е стационарное состояние    
\choice в 3-е и 4–е стационарные состояния
\end{choices}

\question На частицу, находящуюся в 5-ом стационарном состоянии в бесконечно глубокой прямоугольной потенциальной яме, расположенной между точками $x = 0$ и $x = a$, накла-дывают возмущение $\hat V(x,t) = {V_0}\cos \frac{{\pi x}}{a}\;f(t)$, где $f(t)$ - некоторая функция времени. В какие стационарные состояния (основное состояние – первое) возможны переходы? 
\begin{choices}
\choice в 4-е стационарное состояние     
\choice в 6-е стационарное состояние
\choice в 4-е и 6-е стационарные состояния  
\choice только в основное
\end{choices}

\question На частицу, находящуюся в $n$-ом стационарном состоянии в бесконечно глубокой прямоугольной потенциальной яме (основное состояние – первое), расположенной между точками $x = 0$ и $x = a$, накладывают возмущение $\hat V(x,t) = {V_0}\cos \frac{{\pi x}}{a}\;f(t)$, где $f(t)$ - некоторая функция времени. В каких стационарных состояниях можно обнаружить частицу после выключения возмущения? 
\begin{choices}
\choice в $n$-ом и $n + 1$-ом            
\choice в $n - 1$-ом и $n + 1$-ом
\choice в $n$-ом, $n + 1$-ом и $n - 1$-ом      
\choice в $n$-ом, $n + 2$-ом и $n - 2$-ом
\end{choices}

\question На частицу, находящуюся в $2$-ом стационарном состоянии в бесконечно глубокой прямоугольной потенциальной яме (основное состояние – первое), расположенной между точками $x = 0$ и $x = a$, накладывают возмущение $\hat V(x,t) = {V_0}\cos \frac{{\pi x}}{a}\;f(t)$, где $f(t)$ - некоторая плавная функция времени. Сравнить вероятности перехо-дов частицы в $1$-ое (${w_{2 \to 1}}$) и $3$-ое (${w_{2 \to 3}}$) состояния, вычисленные в первом порядке теории нестационарных возмущений
\begin{choices}
\choice ${w_{2 \to 1}} > {w_{2 \to 3}}$  
\choice ${w_{2 \to 1}} < {w_{2 \to 3}}$     
\choice ${w_{2 \to 1}} = {w_{2 \to 3}}$  
\choice это зависит от ${V_0}$
\end{choices}

\question На частицу, находящуюся во втором стационарном состоянии (основное состояние - первое) в бесконечно глубокой прямоугольной потенциальной яме, расположенной между точками $x = 0$ и $x = a$, накладывают возмущение $\hat V(x,t) = \alpha (x - a/2)\;f(t)$, где $f(t)$ - некоторая функция времени. В какие стационарные состояния возможен переход? 
\begin{choices}
\choice в основное, третье, пятое и другие нечетные состояния
\choice в четвертое, шестое и другие четные состояния
\choice во все состояния
\choice только в основное
\end{choices}

\question На частицу, находящуюся в основном состоянии бесконечно глубокой потенциаль-ной ямы, расположенной между точками $x = 0$ и $x = a$, накладывают малое возмущение $\hat V(x,t) = {V_0}x(x - a)f(t)$, где $f(t)$ - некоторая функция времени. Для каких состояний вероятность перехода, вычисленная в первом порядке теории нестационарных возмущений, отлична от нуля (основное состояние – первое)?
\begin{choices}
\choice для всех четных         
\choice для всех нечетных
\choice для всех          
\choice только для первого возбужденного
\end{choices}

\question  На одномерный гармонический осциллятор, находящийся в основном состоянии, действует малое, зависящее от времени возмущение $\hat V(x,t) = {V_0}\cos (x/a){e^{ - {t^2}/{\tau ^2}}}$, где ${V_0}$, $a$ и $\tau $ - некоторые постоянные. Чему равна вероят-ность перехода осциллятора в первое возбужденное состояние? 
\begin{choices}
\choice $w = 0$  
\choice $w = {\left( {\frac{{{V_0}}}{{\hbar \omega }}} \right)^2}$     
\choice $w = {\left( {\frac{{{V_0}\tau }}{\hbar }} \right)^2}$      
\choice $w = \left( {\frac{{{a^2}m\omega }}{\hbar }} \right)$
\end{choices}

\question На одномерный гармонический осциллятор, находящийся во втором стационарном состоянии, действует малое, зависящее от времени возмущение $\hat V(x,t) = \alpha \cos (x/\beta )f(t)$, где $\alpha $ и $\beta $ - постоянные, $f(t)$ - функция времени. В какие со-стояния осциллятор может совершить переход? 
\begin{choices}
\choice в основное, второе, четвертое и другие четные состояния
\choice в третье, пятое и другие нечетные состояния
\choice во все состояния
\choice ни в какие состояния
\end{choices}

\question На одномерный гармонический осциллятор, находящийся во втором стационарном состоянии, действует малое, зависящее от времени возмущение $\hat V(x,t) = \alpha \sin (x/\beta )f(t)$, где $\alpha $ и $\beta $ - постоянные, $f(t)$ - функция времени. В какие со-стояния осциллятор может совершить переход? 
\begin{choices}
\choice в основное, второе, четвертое и другие четные состояния
\choice в третье, пятое и другие нечетные состояния
\choice во все состояния
\choice ни в какие состояния
\end{choices}

\question На одномерный гармонический осциллятор, находящийся в 99 стационарном состоя-нии, действует малое, зависящее от времени возмущение $\hat V(x,t) = {V_0}{x^4}{e^{ - {t^2}/{\tau ^2}}}$, где ${V_0}$ и $\tau $ - некоторые постоянные. Чему равна вероятность перехода осциллятора в 100-ое состояние? 
\begin{choices}
\choice $w = 0$  
\choice $w = {\left( {\frac{{{V_0}}}{{\hbar \omega }}} \right)^2}$     
\choice $w = {(\omega \tau )^2}$      
\choice $w = {(\omega \tau )^{ - 2}}$
\end{choices}

\question На одномерный гармонический осциллятор, находящийся в 99 стационарном состоя-нии, действует малое, зависящее от времени возмущение $\hat V(x,t) = {V_0}{x^4}{e^{ - {t^2}/{\tau ^2}}}$, где ${V_0}$ и $\tau $ - некоторые постоянные. В каких состояниях можно обнаружить осциллятор? 
\begin{choices}
\choice в 95, 97, 99, 101, 103     
\choice в 95, 96, 97, 98, 99, 100, 101, 102, 103
\choice во всех четных       
\choice во всех нечетных
\end{choices}

\question На одномерный гармонический осциллятор, находящийся в 99 стационарном состоя-нии, действует малое, зависящее от времени возмущение $\hat V(x,t) = {V_0}{x^3}{e^{ - {t^2}/{\tau ^2}}}$, где ${V_0}$ и $\tau $ - некоторые постоянные. В каких состояниях можно обнаружить осциллятор? 
\begin{choices}
\choice в 96, 98, 100, 102         
\choice в 96, 98, 99, 100, 102
\choice во всех четных       
\choice во всех нечетных
\end{choices}

\question На одномерный гармонический осциллятор, находящийся в $n$ стационарном со-стоянии, действует малое, зависящее от времени возмущение $\hat V(x,t) = \alpha xf(t)$, где $f(t)$ - некоторая функция времени. В какие состояния возможны переходы осциллятора?
\begin{choices}
\choice во все            
\choice только в $(n + 1)$
\choice только в $(n - 1)$         
\choice только в $(n + 1)$ и $(n - 1)$
\end{choices}

\question На одномерный гармонический осциллятор, находящийся в $n$ стационарном со-стоянии, действует малое, зависящее от времени возмущение $\hat V(x,t) = \alpha xf(t)$, где $f(t)$ - некоторая функция времени. В каких состояниях можно обнаружить осциллятор по-сле выключения возмущения (ответ дать в первом порядке теории нестационарных воз-мущений)?
\begin{choices}
\choice во всех           
\choice только в $(n - 1)$ и $(n + 1)$
\choice только в $n$, $(n + 2)$ и $(n - 2)$ 
\choice только в $n$, $(n + 1)$ и $(n - 1)$
\end{choices}

\question На одномерный гармонический осциллятор, находящийся в $n$ стационарном со-стоянии, действует малое, зависящее от времени малое возмущение $\hat V(x,t) = \alpha {x^3}f(t)$, где $f(t)$ - некоторая функция времени. В какие состояния возможны переходы осциллятора?
\begin{choices}
\choice во все
\choice только в $(n + 1)$ и $(n - 1)$
\choice только в $(n + 1)$, $(n - 1)$, $(n + 3)$ и $(n - 3)$
\choice только в $(n + 1)$, $(n - 1)$, $(n + 2)$ и $(n - 2)$
\end{choices}

\question  На одномерный гармонический осциллятор, находящийся в первом возбужденном состоянии, действует зависящее от времени малое возмущение $\hat V(x,t) = \alpha xf(t)$, где $f(t)$ - некоторая функция времени. Чему равно отношение вероятностей перехода ос-циллятора в основное и второе возбужденное состояния? Указание: Матричные элементы оператора координаты с осцилляторными функциями равны: ${x_{nk}} = \sqrt {\frac{{n\hbar }}{{2m\omega }}} {\delta _{k,n - 1}} + \sqrt {\frac{{(n + 1)\hbar }}{{2m\omega }}} {\delta _{k,n + 1}}$.
\begin{choices}
\choice $\frac{{{w_{1 \to 0}}}}{{{w_{1 \to 2}}}} = 2$      
\choice $\frac{{{w_{1 \to 0}}}}{{{w_{1 \to 2}}}} = \frac{1}{2}$     
\choice $\frac{{{w_{1 \to 0}}}}{{{w_{1 \to 2}}}} = 1$      
\choice $\frac{{{w_{1 \to 0}}}}{{{w_{1 \to 2}}}} = \frac{2}{3}$
\end{choices}

\question На заряженную частицу, находящуюся в основном состоянии в некотором централь-ном поле, накладывают малое, зависящее от времени однородное электрическое поле $\vec E(t)$, направленное вдоль оси $z$. Какие значения проекции момента импульса частицы на ось $z$ можно обнаружить в конечном состоянии?
\begin{choices}
\choice только $m = 0, \pm 1$      
\choice только $m = 0, \pm 2$
\choice только $m = 0$    
\choice любые целые
\end{choices}

\question На заряженную частицу, находящуюся в основном состоянии в некотором централь-ном поле, накладывают малое, зависящее от времени однородное электрическое поле $\vec E(t)$. Какие значения момента импульса частицы можно обнаружить в конечном состоя-нии?
\begin{choices}
\choice только $l = 0,1,2$      
\choice только $l = 0,1,2,3$
\choice только $l = 0,1$     
\choice любые целые
\end{choices}

\question На заряженную частицу, находящуюся в основном состоянии в некотором централь-ном поле без случайного вырождения, накладывают малое, зависящее от времени однород-ное электрическое поле $\vec E(t)$. Возможен ли переход частицы на первый возбужденный уровень энергии?
\begin{choices}
\choice да    
\choice нет      
\choice Это зависит от $E(t)$   
\choice Это зависит от поля
\end{choices}

\question На заряженную частицу, находящуюся в основном состоянии в некотором централь-ном поле, накладывают малое, зависящее от времени однородное электрическое поле $\vec E(t)$, направленное вдоль оси $y$. Какие значения проекции момента импульса частицы на ось $z$ можно обнаружить в конечном состоянии?
\begin{choices}
\choice только $m =  \pm 1$     
\choice только $m = 0, \pm 2$
\choice только $m = 0, \pm 1$      
\choice любые целые
\end{choices}

\question На заряженную частицу, находящуюся в основном состоянии в некотором централь-ном поле, накладывают малое, зависящее от времени однородное электрическое поле $\vec E(t)$, направленное вдоль оси $x$. Какие значения проекции момента импульса частицы на ось $x$ можно обнаружить в конечном состоянии?
\begin{choices}
\choice только ${l_x} = 0$      
\choice только ${l_x} = 0, \pm 2$
\choice только ${l_x} = 0, \pm 1$     
\choice любые целые
\end{choices}

\question На частицу, находящуюся в основном состоянии в некотором центральном поле, на-кладывают малое, зависящее от времени возмущение , где $f(t)$ - некоторая функция време-ни. Какие значения проекции момента импульса частицы на ось $z$ можно обнаружить в конечном состоянии?
\begin{choices}
\choice только $m = 0, \pm 4$      
\choice только $m = 0$
\choice только $m = 0, \pm 1$      
\choice любые целые
\end{choices}

\question На частицу, находящуюся в основном состоянии в некотором центральном поле, на-кладывают малое, зависящее от времени возмущение , где $f(t)$ - некоторая функция време-ни. Какие значения проекции момента импульса частицы на ось $y$ можно обнаружить в конечном состоянии?
\begin{choices}
\choice только $m = 0, \pm 4$         
\choice только $m = 0$
\choice только $m = 0, \pm 1, \pm 2, \pm 3, \pm 4$   
\choice любые целые
\end{choices}

\question На частицу, находящуюся в основном состоянии в некотором центрально-симметричном поле, действует малое возмущение , где $f(t)$ - некоторая функция времени. Возможен ли переход частицы в возбужденные состояния с определенным моментом $l = 0$?  
\begin{choices}
\choice да    
\choice нет      
\choice это зависит от $f(t)$   
\choice мало информации для ответа
\end{choices}

\question На частицу, находящуюся в основном состоянии в некотором центрально-симметричном поле, действует малое возмущение , где $f(t)$ - некоторая функция времени. Возможен ли переход частицы в возбужденные состояния с определенным моментом $l = 2$? 
\begin{choices}
\choice да    
\choice нет      
\choice это зависит от $f(t)$   
\choice мало информации для ответа
\end{choices}

\question На частицу, находящуюся в основном состоянии в некотором центрально-симметричном поле без случайного вырождения, действует малое возмущение  (где $f(t)$ - некоторая функция времени). Возможен ли переход частицы на первый возбужденный уро-вень энергии?  
\begin{choices}
\choice да    
\choice нет      
\choice это зависит от $f(t)$   
\choice мало информации для ответа
\end{choices}

\question На частицу, находящуюся в основном состоянии в некотором центрально-симметричном поле со случайным вырождением, действует малое возмущение  (где $f(t)$ - некоторая функция времени). Возможен ли переход частицы на второй возбужденный уро-вень энергии?  
\begin{choices}
\choice да    
\choice нет      
\choice это зависит от $f(t)$   
\choice мало информации для ответа
\end{choices}

\question На частицу, находящуюся в основном состоянии в некотором центрально-симметричном поле без случайного вырождения, действует малое возмущение  (где $f(t)$ - некоторая функция времени). Возможен ли переход частицы на третий возбужденный уро-вень энергии? 
\begin{choices}
\choice да    
\choice нет      
\choice это зависит от $f(t)$   
\choice мало информации для ответа
\end{choices}

\question На частицу, находящуюся в основном состоянии в некотором центрально-симметричном поле, действует малое возмущение  (где $V(r)$ - некоторая функция модуля радиус-вектора, $f(t)$ - некоторая функция времени). Возможен ли переход частицы в воз-бужденные состояния с определенным моментом $l = 0$? 
\begin{choices}
\choice да    
\choice нет      
\choice это зависит от $f(t)$   
\choice мало информации для ответа
\end{choices}

\question На частицу, находящуюся в основном состоянии в некотором центрально-симметричном поле, действует малое возмущение  (где $V(r)$ - некоторая функция модуля радиус-вектора, $f(t)$ - некоторая функция времени). Возможен ли переход частицы в воз-бужденные состояния с определенным моментом $l = 1$?  
\begin{choices}
\choice да    
\choice нет      
\choice это зависит от $f(t)$   
\choice мало информации для ответа
\end{choices}

\question На частицу, находящуюся в основном состоянии в некотором центрально-симметричном поле, действует малое возмущение, оператор которого зависит только от мо-дуля радиус-вектора $\hat V(r,t) = \hat V(r)f(t)$ (где $f(t)$ - некоторая функция времени). Возможен ли переход частицы в возбужденные состояния с определенным моментом $l = 2$?
\begin{choices}
\choice да    
\choice нет      
\choice это зависит от $f(t)$   
\choice мало информации для ответа
\end{choices}

\question На заряженную частицу, находящуюся в основном состоянии в некотором централь-но-симметричном поле, действует зависящее от времени однородное электрическое поле. Возможен ли переход частицы в возбужденные состояния с определенным моментом $l = 1$? 
\begin{choices}
\choice да
\choice нет      
\choice это зависит от ориентации электрического поля
\choice мало информации для ответа
\end{choices}

\question На заряженную частицу, находящуюся в основном состоянии в некотором централь-но-симметричном поле, действует зависящее от времени однородное электрическое поле. Возможен ли переход частицы в возбужденные состояния с определенной проекцией момента импульса на ось $z$ $m \ne 0$? 
\begin{choices}
\choice да
\choice нет      
\choice это зависит от ориентации электрического поля
\choice мало информации для ответа
\end{choices}

\question На частицу, находящуюся в основном состоянии в некотором центральном поле, дей-ствует зависящее от времени малое возмущение  ($f(t)$ - некоторая функция времени). В со-стояния с какими моментами возможны переходы частицы? 
\begin{choices}
\choice только с $l = 1$     
\choice только с $l = 1$ и $l = 2$
\choice только с $l = 0$ и $l = 2$ 
\choice с любыми
\end{choices}

\question На заряженный трехмерный гармонический осциллятор, находящийся в основном состоянии, накладывают малое, зависящее от времени однородное электрическое поле $\vec E(t)$. Возможен ли переход осциллятора на второй возбужденный уровень энергии? Указа-ние. Кратность вырождения второго возбужденного уровня энергии трехмерного осциллятора равна 6.
\begin{choices}
\choice да
\choice нет
\choice это зависит от ориентации электрического поля
\choice мало информации для ответа
\end{choices}

\question На заряженный трехмерный гармонический осциллятор, находящийся в основном состоянии, накладывают малое, зависящее от времени однородное электрическое поле $\vec E(t)$. Возможен ли переход осциллятора на первый возбужденный уровень энергии? Указа-ние. Кратность вырождения первого возбужденного уровня энергии трехмерного осциллятора равна 3.
\begin{choices}
\choice да
\choice нет
\choice это зависит от ориентации электрического поля
\choice мало информации для ответа
\end{choices}

\question На заряженный трехмерный гармонический осциллятор, находящийся в основном состоянии, накладывают малое, зависящее от времени однородное электрическое поле $\vec E(t)$. Возможен ли переход осциллятора на третий возбужденный уровень энергии? Указа-ние. Кратность вырождения третьего возбужденного уровня энергии трехмерного осцилля-тора равна 10.
\begin{choices}
\choice да
\choice нет
\choice это зависит от ориентации электрического поля
\choice мало информации для ответа
\end{choices}

\question На заряженный трехмерный гармонический осциллятор, находящийся в основном состоянии, накладывают малое, зависящее от времени однородное электрическое поле $\vec E(t)$, направленное вдоль оси $z$. В состояния с какой проекцией момента осциллятора на ось $z$ происходят переходы?
\begin{choices}
\choice только с $m =  + 1$        
\choice только с $m =  - 1$
\choice только с $m =  + 1$ и $m =  - 1$ 
\choice только с $m = 0$
\end{choices}

\question На заряженный трехмерный гармонический осциллятор, находящийся в основном состоянии, накладывают малое, зависящее от времени однородное электрическое поле $\vec E(t)$, направленное вдоль оси $y$. В состояния с какой проекцией момента осциллятора на ось $z$ происходят переходы?
\begin{choices}
\choice только с $m =  + 1$        
\choice только с $m =  - 1$
\choice только с $m =  + 1$ и $m =  - 1$ 
\choice только с $m = 0$
\end{choices}

\question На заряженный трехмерный гармонический осциллятор, находящийся в основном состоянии, накладывают малое, зависящее от времени однородное электрическое поле $\vec E(t)$, направленное вдоль оси $y$. В состояния с какой проекцией момента осциллятора на ось $x$ происходят переходы?
\begin{choices}
\choice только с ${l_x} =  + 1$          
\choice только с ${l_x} =  - 1$
\choice только с ${l_x} =  + 1$ и ${l_x} =  - 1$  
\choice только с ${l_x} = 0$
\end{choices}

\question На трехмерный гармонический осциллятор, находящийся в основном состоянии, дей-ствует малое, зависящее от времени возмущение, оператор которого зависит только от моду-ля радиус-вектора $\hat V(\vec r,t) \to \hat V(r,t)$. Возможен ли переход осциллятора на первый возбужденный уровень энергии? Указание. Кратность вырождения первого возбуж-денного уровня энергии трехмерного осциллятора равна 3.
\begin{choices}
\choice да
\choice нет
\choice это зависит от величины возмущения
\choice мало информации для ответа
\end{choices}

\question На трехмерный гармонический осциллятор, находящийся в основном состоянии, дей-ствует малое, зависящее от времени возмущение, оператор которого зависит только от моду-ля радиус-вектора $\hat V(\vec r,t) \to \hat V(r,t)$. Возможен ли переход осциллятора на второй возбужденный уровень энергии? Указание. Кратность вырождения второго возбуж-денного уровня энергии трехмерного осциллятора равна 6.
\begin{choices}
\choice да
\choice нет
\choice это зависит от величины возмущения
\choice мало информации для ответа
\end{choices}

\question  На трехмерный гармонический осциллятор, находящийся в основном состоянии, действует малое, зависящее от времени возмущение, оператор которого зависит только от модуля радиус-вектора $\hat V(\vec r,t) \to \hat V(r,t)$. Возможен ли переход осциллятора на третий возбужденный уровень энергии? Указание. Кратность вырождения второго возбуж-денного уровня энергии трехмерного осциллятора равна 10.
\begin{choices}
\choice да
\choice нет
\choice это зависит от величины возмущения
\choice мало информации для ответа
\end{choices}

\question На трехмерный гармонический осциллятор, находящийся на первом возбужденном уровне энергии, действует малое, зависящее от времени возмущение, оператор которого за-висит только от модуля радиус-вектора $\hat V(\vec r,t) \to \hat V(r,t)$. Возможен ли переход осциллятора в основное состояние? Указание. Кратность вырождения первого возбуж-денного уровня энергии трехмерного осциллятора равна 3.
\begin{choices}
\choice да
\choice нет
\choice это зависит от величины возмущения
\choice мало информации для ответа
\end{choices}

\question На трехмерный гармонический осциллятор, находящийся на первом возбужденном уровне энергии, действует малое, зависящее от времени возмущение, оператор которого за-висит только от модуля радиус-вектора $\hat V(\vec r,t) \to \hat V(r,t)$. Возможен ли переход осциллятора на третий возбужденный уровень энергии? Указание. Кратность вырождения первого возбужденного уровня энергии трехмерного осциллятора равна 3, третьего воз-бужденного – 10.
\begin{choices}
\choice да
\choice нет
\choice это зависит от величины возмущения
\choice мало информации для ответа
\end{choices}

\question На атом водорода, находящийся в основном состоянии, действует зависящее от вре-мени малое возмущение, оператор которого зависит только от модуля радиус-вектора элек-трона $\hat V(\vec r,t) = \hat V(r)f(t)$ (где $f(t)$ - некоторая функция времени). Возможен ли переход электрона на первый возбужденный уровень энергии? 
\begin{choices}
\choice да
\choice нет
\choice это зависит от величины возмущения
\choice мало информации для ответа
\end{choices}

\question На атом водорода, находящийся в основном состоянии, действует зависящее от вре-мени малое возмущение $\hat V(\vec r,t) = {V_0}{\cos ^2}\vartheta f(t)$ (где $f(t)$ - некоторая функция времени). Возможен ли переход электрона на первый возбужденный уровень энергии? 
\begin{choices}
\choice да
\choice нет
\choice это зависит от величины возмущения
\choice мало информации для ответа
\end{choices}

\question На атом водорода, находящийся в основном состоянии, действует зависящее от вре-мени малое возмущение $\hat V(\vec r,t) = {V_0}r{\cos ^2}\vartheta f(t)$ (где $f(t)$ - некото-рая функция времени). Возможен ли переход электрона на первый возбужденный уровень энергии? 
\begin{choices}
\choice да
\choice нет
\choice это зависит от величины возмущения
\choice мало информации для ответа
\end{choices}

\question На атом водорода, находящийся в основном состоянии, действует зависящее от вре-мени малое возмущение $\hat V(\vec r,t) = {V_0}\cos \vartheta f(t)$ (где $f(t)$ - некоторая функция времени). Какие значения может принимать проекция момента импульса электрона на ось $z$ в конечном состоянии? 
\begin{choices}
\choice только $m = 0$    
\choice только $m = 0,1$
\choice только $m = 0, - 1$     
\choice $m = 0, \pm 1$
\end{choices}

\question На атом водорода, находящийся в основном состоянии, действует зависящее от вре-мени малое возмущение $\hat V(\vec r,t) = {V_0}\cos \vartheta f(t)$ (где $f(t)$ - некоторая функция времени). Возможен ли переход электрона на второй возбужденный уровень энер-гии? 
\begin{choices}
\choice да
\choice нет
\choice это зависит от величины возмущения
\choice мало информации для ответа
\end{choices}

\question На атом водорода, находящийся в основном состоянии, действует зависящее от вре-мени малое возмущение $\hat V(\vec r,t) = {V_0}\cos \vartheta f(t)$ (где $f(t)$ - некоторая функция времени). Какие значения может принимать момент импульса электрона в конеч-ном состоянии?  
\begin{choices}
\choice только $l = 0$    
\choice только $l = 1$
\choice только $l = 0$ и $l = 1$   
\choice только $l = 0$ и $l = 2$
\end{choices}

\question На атом водорода, находящийся в основном состоянии, действует зависящее от вре-мени малое возмущение $\hat V(\vec r,t) = {V_0}\cos \varphi f(t)$ (где $f(t)$ - некоторая функция времени). Какие значения может принимать проекция момента импульса электрона на ось $z$ в конечном состоянии?  
\begin{choices}
\choice только $m = 0$    
\choice только $m = 0, - 1$
\choice только $m = 0, \pm 1$      
\choice только $m = 0, \pm 1, \pm 2$
\end{choices}

\question На атом водорода, находящийся в основном состоянии, действует зависящее от вре-мени малое возмущение $\hat V(\vec r,t) = {V_0}\cos 2\varphi f(t)$ (где $f(t)$ - некоторая функция времени). Может ли момент импульса электрона в конечном состоянии принимать значение $l = 1$? 
\begin{choices}
\choice да
\choice нет
\choice это зависит от величины возмущения
\choice мало информации для ответа
\end{choices}

\question Заряженная бесспиновая частица движется в центральном поле. Частица находится в стационарном состоянии с определенной проекцией момента импульса на ось $z$. На части-цу накладывают малое, зависящее от времени однородное магнитное поле, направленное вдоль оси $z$. Какие переходы может совершать частица?
\begin{choices}
\choice только с изменением радиального квантового числа
\choice только с изменением момента
\choice только с изменением проекции момента на ось $z$
\choice частица не будет совершать переходов 
\end{choices}

\question Заряженная бесспиновая частица движется в центральном поле. Частица находится в стационарном состоянии с определенной проекцией момента импульса на ось $z$. На части-цу накладывают малое, зависящее от времени однородное магнитное поле, направленное вдоль оси $y$. Какие переходы может совершать частица?
\begin{choices}
\choice только с изменением радиального квантового числа
\choice только с изменением момента
\choice только с изменением проекции момента на ось $z$
\choice частица не будет совершать переходов 
\end{choices}

\question Заряженная бесспиновая частица находится в стационарном состоянии в центрально-симметричном поле с определенным моментом $l = 0$. На частицу накладывают малое, за-висящее от времени однородное магнитное поле. Будет ли частица совершать при этом квантовые переходы?
\begin{choices}
\choice да
\choice нет
\choice это зависит от величины поля
\choice мало информации для ответа
\end{choices}

\question Заряженная бесспиновая частица находится в стационарном состоянии в центрально-симметричном поле. На частицу накладывают малое, зависящее от времени однородное маг-нитное поле. Будет ли частица совершать при этом квантовые переходы в состояния с дру-гими энергиями?
\begin{choices}
\choice да
\choice нет
\choice это зависит от величины поля
\choice мало информации для ответа
\end{choices}

\question Заряженная бесспиновая частица находится в стационарном состоянии в централь-ном поле с определенным моментом. На частицу накладывают малое, зависящее от времени однородное магнитное поле. Будет ли частица совершать при этом квантовые переходы в состояния с другими моментами?
\begin{choices}
\choice да
\choice нет
\choice это зависит от величины поля
\choice мало информации для ответа
\end{choices}

\question Заряженная бесспиновая частица, движущаяся в центральном поле, находится в ста-ционарном состоянии с неопределенной проекцией момента на ось $z$. На частицу накладывают малое, зависящее от времени однородное магнитное поле, направленное вдоль оси $z$. Будет ли частица совершать при этом квантовые переходы в другие стационарные состояния?
\begin{choices}
\choice да
\choice нет
\choice это зависит от величины поля
\choice мало информации для ответа
\end{choices}

\question Незаряженная и не имеющая магнитного момента частица со спином $s = 1/2$ нахо-дится в стационарном состоянии независящего от спина гамильтониана с определенной про-екцией спина на ось $z$ ${s_z} = 1/2$. На частицу накладывают малое, зависящее от времени однородное магнитное поле, направленное вдоль оси $z$. Какие переходы может совершать частица?
\begin{choices}
\choice только с изменением пространственной части волновой функции
\choice с изменением пространственной части волновой функции и переворотом спина
\choice с переворотом спина, но без изменения пространственной части волновой функции
\choice частица вообще не будет совершать переходы
\end{choices}

\question Незаряженная частица со спином $s = 1/2$, имеющая магнитный момент, находится в стационарном состоянии независящего от спина гамильтониана с определенной проекцией спина на ось $z$ ${s_z} = 1/2$. На частицу накладывают малое, зависящее от времени одно-родное магнитное поле, направленное вдоль оси $z$. Какие переходы может совершать час-тица?
\begin{choices}
\choice только с изменением пространственной части волновой функции
\choice с изменением пространственной части волновой функции и переворотом спина
\choice с переворотом спина, но без изменения пространственной части волновой функции
\choice частица вообще не будет совершать переходов
\end{choices}

\question Незаряженная частица со спином $s = 1/2$, имеющая магнитный момент, находится в стационарном состоянии независящего от спина гамильтониана с определенной проекцией спина на ось $z$ ${s_z} = 1/2$. На частицу накладывают малое, зависящее от времени одно-родное магнитное поле, направленное вдоль оси $x$. Какие переходы может совершать час-тица?
\begin{choices}
\choice только с изменением пространственной части волновой функции
\choice с изменением пространственной части волновой функции и переворотом спина
\choice с переворотом спина, но без изменения пространственной части волновой функции
\choice частица вообще не будет совершать переходов
\end{choices}

\question Незаряженная частица со спином $s = 1/2$, имеющая магнитный момент, находится в стационарном состоянии независящего от спина гамильтониана с определенной проекцией спина на ось $z$ ${s_z} = 1/2$. На частицу накладывают малое однородное магнитное поле, зависящее от времени по закону $\vec B(t) = {\vec B_0}{e^{ - {t^2}/{\tau ^2}}}$. С какой ве-роятностью проекция спина частицы на ось $z$ станет равной ${s_z} =  - 3/2$ к моменту вре-мени $t =  + \infty $?
\begin{choices}
\choice $w = 1/2$   
\choice $w = 0$  
\choice $w = 1$  
\choice это зависит от ${B_0}$
\end{choices}

\question Незаряженная частица со спином $s = 1/2$, имеющая магнитный момент, находится в стационарном состоянии независящего от спина гамильтониана с неопределенной проек-цией спина на ось $z$. На частицу накладывают малое однородное магнитное поле, зави-сящее от времени по закону $\vec B(t) = {\vec B_0}{e^{ - {t^2}/{\tau ^2}}}$. С какой вероятностью спин частицы станет равным $s = 3/2$ к моменту времени $t =  + \infty $?
\begin{choices}
\choice $w = 1/2$   
\choice $w = 0$  
\choice $w = 1$  
\choice это зависит от ${B_0}$
\end{choices}

\end{questions}




\subsection{ Переходы под действием периодических и внезапных возмущений }

Во всех задачах этой главы $\vartheta $ и $\varphi $ - полярный и азимутальный углы сфериче-ской системы координат соответственно. $e$ - элементарный заряд, $a = {\hbar ^2}/m{e^2}$ - бо-ровский радиус. Если это не оговорено особо, вероятности переходов нужно вычислять в первом порядке теории возмущений для волновой функции.

\begin{questions}

\question На частицу, находящуюся в основном состоянии бесконечно глубокой прямоугольной по-тенциальной ямы, расположенной между точками $x =  - a/2$ и $x = a/2$, действует малое перио-дическое возмущение $\hat V(x,t) = \alpha {x^2}\cos \omega t$. При какой частоте возмущения $\omega $ частица сможет перейти в первое возбужденное состояние? Указание. Энергия стацио-нарных состояний частицы в бесконечно глубокой потенциальной яме определяется соотношени-ем ${E_n} = \frac{{{\pi ^2}{\hbar ^2}{n^2}}}{{2m{a^2}}}$, $n = 1,2,3,...$
\begin{choices}
\choice $\omega  = \frac{{3{\pi ^2}\hbar }}{{2m{a^2}}}$    
\choice $\omega  = \frac{{4{\pi ^2}\hbar }}{{2m{a^2}}}$    
\choice $\omega  = \frac{{5{\pi ^2}\hbar }}{{2m{a^2}}}$    
\choice ни при какой
\end{choices}

\question На частицу, находящуюся в основном состоянии бесконечно глубокой прямоугольной по-тенциальной ямы, расположенной между точками $x =  - a/2$ и $x = a/2$, действует малое перио-дическое возмущение $\hat V(x,t) = \alpha x\cos \omega t$. При какой частоте возмущения $\omega $ частица сможет перейти в первое возбужденное состояние? Указание. Энергия стационарных состояний частицы в бесконечно глубокой потенциальной яме определяется соотношением ${E_n} = \frac{{{\pi ^2}{\hbar ^2}{n^2}}}{{2m{a^2}}}$, $n = 1,2,3,...$
\begin{choices}
\choice $\omega  = \frac{{3{\pi ^2}\hbar }}{{2m{a^2}}}$    
\choice $\omega  = \frac{{4{\pi ^2}\hbar }}{{2m{a^2}}}$    
\choice $\omega  = \frac{{5{\pi ^2}\hbar }}{{2m{a^2}}}$    
\choice ни при какой
\end{choices}

\question На частицу, находящуюся в $n$-ом состоянии бесконечно глубокой прямоугольной потен-циальной ямы (основное состояние – первое), расположенной между точками $x = 0$ и $x = a$, действует малое периодическое возмущение $\hat V(x,t) = \alpha \cos \left( {\pi x/a} \right)\cos \omega t$. При какой частоте возмущения $\omega $ частица будет совершать переходы? Ответ дать в первом порядке теории нестационарных возмущений. Указание. Энергия стационарных состояний частицы в бесконечно глубокой потенциальной яме определяется соотношением ${E_n} = \frac{{{\pi ^2}{\hbar ^2}{n^2}}}{{2m{a^2}}}$, $n = 1,2,3,...$
\begin{choices}
\choice $\omega  = \frac{{{\pi ^2}(2n + 1)\hbar }}{{2m{a^2}}}$      
\choice $\omega  = \frac{{{\pi ^2}(2n - 1)\hbar }}{{2m{a^2}}}$ и $\omega  = \frac{{{\pi ^2}(2n + 1)\hbar }}{{2m{a^2}}}$
\choice $\omega  = \frac{{{\pi ^2}(2n - 1)\hbar }}{{2m{a^2}}}$      
\choice $\omega  = \frac{{{\pi ^2}(n - 1)\hbar }}{{m{a^2}}}$ и $\omega  = \frac{{{\pi ^2}(n + 1)\hbar }}{{m{a^2}}}$
\end{choices}

\question На частицу, находящуюся в 3-ом состоянии бесконечно глубокой прямоугольной потенци-альной ямы (основное состояние – первое), расположенной между точками $x = 0$ и $x = a$, дей-ствует малое периодическое возмущение $\hat V(x,t) = \alpha x(x - a)\cos \omega t$. При какой минимальной частоте возмущения $\omega $ частица будет совершать переходы? Ответ дать в первом порядке теории нестационарных возмущений. Указание. Энергия стационарных состояний частицы в бесконечно глубокой потенциальной яме определяется соотношением ${E_n} = \frac{{{\pi ^2}{\hbar ^2}{n^2}}}{{2m{a^2}}}$, $n = 1,2,3,...$
\begin{choices}
\choice $\omega  = \frac{{8{\pi ^2}\hbar }}{{2m{a^2}}}$    б. $\omega  = \frac{{16{\pi ^2}\hbar }}{{2m{a^2}}}$      в. $\omega  = \frac{{5{\pi ^2}\hbar }}{{2m{a^2}}}$    г. $\omega  = \frac{{7{\pi ^2}\hbar }}{{m{a^2}}}$ 
\end{choices}

\question На частицу, находящуюся в основном состоянии бесконечно глубокой прямоугольной по-тенциальной ямы, расположенной между точками $x = 0$ и $x = a$, действует малое периодиче-ское возмущение $\hat V(x,t) = \alpha x\cos \omega t$ (где $\omega  = \frac{{2{\pi ^2}\hbar }}{{m{a^2}}}$). Может ли частица совершить переход, и если да, то в какие состояния (основное состояние – первое)? Указание. Энергия стационарных состояний частицы в бесконечно глубокой потенциальной яме определяется соотношением ${E_n} = \frac{{{\pi ^2}{\hbar ^2}{n^2}}}{{2m{a^2}}}$, $n = 1,2,3,...$
\begin{choices}
\choice да, во второе  
\choice да, в третье      
\choice нет      
\choice мало информации для ответа
\end{choices}

\question На частицу, находящуюся в основном состоянии бесконечно глубокой прямоугольной по-тенциальной ямы, расположенной между точками $x =  - a/2$ и $x = a/2$, действует малое перио-дическое возмущение $\hat V(x,t) = \alpha x\cos \omega t$ (где $\omega  = \frac{{2{\pi ^2}\hbar }}{{m{a^2}}}$). Может ли частица совершить переход, и если да, то в какие состояния? (основное состояние – первое). Указание. Энергия стационарных состояний частицы в бесконечно глубокой потенциальной яме определяется соотношением ${E_n} = \frac{{{\pi ^2}{\hbar ^2}{n^2}}}{{2m{a^2}}}$, $n = 1,2,3,...$
\begin{choices}
\choice да, во второе  
\choice да, в третье      
\choice нет      
\choice мало информации для ответа
\end{choices}

\question На частицу, находящуюся в основном состоянии бесконечно глубокой прямоугольной по-тенциальной ямы, расположенной между точками $x = 0$ и $x = a$, действует малое периодиче-ское возмущение $\hat V(x,t) = \alpha x\cos \omega t$ (где $\omega  = \frac{{3{\pi ^2}\hbar }}{{2m{a^2}}}$). Может ли частица совершить переход, и если да, то в какие состояния (основное состояние – первое)? Указание. Энергия стационарных состояний частицы в бесконечно глу-бокой потенциальной яме определяется соотношением ${E_n} = \frac{{{\pi ^2}{\hbar ^2}{n^2}}}{{2m{a^2}}}$, $n = 1,2,3,...$
\begin{choices}
\choice да, во второе  
\choice да, в третье      
\choice нет      
\choice мало информации для ответа
\end{choices}

\question На частицу, находящуюся в основном состоянии бесконечно глубокой прямоугольной по-тенциальной ямы, расположенной между точками $x = 0$ и $x = a$, действует малое периодиче-ское возмущение $\hat V(x,t) = \alpha {x^2}\cos \omega t$ (где $\omega  = \frac{{3{\pi ^2}\hbar }}{{2m{a^2}}}$). Может ли частица совершить переход, и в какие состояния (основное состояние – первое)?
\begin{choices}
\choice да, во второе  
\choice да, в третье      
\choice нет      
\choice мало информации для ответа
\end{choices}

\question На частицу, находящуюся в основном состоянии бесконечно глубокой прямоугольной по-тенциальной ямы, расположенной между точками $x = 0$ и $x = a$, действует малое периодиче-ское возмущение $\hat V(x,t) = \alpha x(x - a)\cos \omega t$ (где $\omega  = \frac{{3{\pi ^2}\hbar }}{{2m{a^2}}}$). Может ли частица совершить переход, и в какие состояния (основное состояние – первое)?
\begin{choices}
\choice да, во второе  
\choice да, в третье      
\choice нет      
\choice мало информации для ответа
\end{choices}

\question На частицу, находящуюся в основном состоянии бесконечно глубокой прямоугольной по-тенциальной ямы, расположенной между точками $x = 0$ и $x = a$, действует малое периодиче-ское возмущение $\hat V(x,t) = \alpha x\cos \omega t$ (где $\omega  = \frac{{4{\pi ^2}\hbar }}{{m{a^2}}}$). Может ли частица совершить переход, и в какие состояния (основное состояние – первое)? Указание. Энергия стационарных состояний частицы в бесконечно глубокой потенциальной яме определяется соотношением ${E_n} = \frac{{{\pi ^2}{\hbar ^2}{n^2}}}{{2m{a^2}}}$, $n = 1,2,3,...$
\begin{choices}
\choice да, во второе  
\choice да, в третье      
\choice нет      
\choice мало информации для ответа
\end{choices}

\question На частицу, находящуюся в основном состоянии бесконечно глубокой прямоугольной по-тенциальной ямы, расположенной между точками $x = 0$ и $x = a$, действует малое периодиче-ское возмущение $\hat V(x,t) = \alpha {x^3}\cos \omega t$ (где $\omega  = \frac{{4{\pi ^2}\hbar }}{{m{a^2}}}$). Может ли частица совершить переход, и в какие состояния (основное состояние – первое)? Указание. Энергия стационарных состояний частицы в бесконечно глубокой потенциальной яме определяется соотношением ${E_n} = \frac{{{\pi ^2}{\hbar ^2}{n^2}}}{{2m{a^2}}}$, $n = 1,2,3,...$
\begin{choices}
\choice да, во второе  
\choice да, в третье      
\choice нет      
\choice мало информации для ответа
\end{choices}

\question На частицу, находящуюся в $4$-ом состоянии бесконечно глубокой прямоугольной потен-циальной ямы (основное состояние – первое), расположенной между точками $x = 0$ и $x = a$, действует малое периодическое возмущение $\hat V(x,t) = \alpha \cos \left( {2\pi x/a} \right)\cos \omega t$. При какой максимальной частоте возмущения $\omega $ частица будет совершать пе-реходы? Указание. Энергия стационарных состояний частицы в бесконечно глубокой потенциальной яме определяется соотношением ${E_n} = \frac{{{\pi ^2}{\hbar ^2}{n^2}}}{{2m{a^2}}}$, $n = 1,2,3,...$
\begin{choices}
\choice $\omega  = \frac{{6{\pi ^2}\hbar }}{{m{a^2}}}$     
\choice $\omega  = \frac{{7{\pi ^2}\hbar }}{{m{a^2}}}$     
\choice $\omega  = \frac{{8{\pi ^2}\hbar }}{{m{a^2}}}$     
\choice $\omega  = \frac{{9{\pi ^2}\hbar }}{{m{a^2}}}$ 
\end{choices}

\question На частицу, находящуюся в $4$-ом состоянии бесконечно глубокой прямоугольной потен-циальной ямы (основное состояние – первое), расположенной между точками $x = 0$ и $x = a$, действует малое периодическое возмущение $\hat V(x,t) = \alpha \cos \left( {2\pi x/a} \right)\cos \omega t$. При какой минимальной частоте возмущения $\omega $ частица будет совершать пере-ходы? Указание. Энергия стационарных состояний частицы в бесконечно глубокой потенциаль-ной яме определяется соотношением ${E_n} = \frac{{{\pi ^2}{\hbar ^2}{n^2}}}{{2m{a^2}}}$, $n = 1,2,3,...$
\begin{choices}
\choice $\omega  = \frac{{6{\pi ^2}\hbar }}{{m{a^2}}}$     
\choice $\omega  = \frac{{7{\pi ^2}\hbar }}{{m{a^2}}}$     
\choice $\omega  = \frac{{8{\pi ^2}\hbar }}{{m{a^2}}}$     
\choice $\omega  = \frac{{9{\pi ^2}\hbar }}{{m{a^2}}}$ 
\end{choices}

\question Осциллятор находится в $n$-м квантовом состоянии. На осциллятор начинает действовать малое периодическое возмущение $\hat V(x,t) = \hat V(x)\cos \omega t$, частота которого совпада-ет с частотой осциллятора, а оператор $\hat V(x)$ имеет ненулевые матричные элементы для всех состояний осциллятора. В каких состояниях можно обнаружить осциллятор после выключения возмущения? 
\begin{choices}
\choice только в $n$            
\choice только в $(n - 1)$ и $(n + 1)$
\choice только в $n$, $(n + 1)$ и $(n - 1)$ 
\choice во всех стационарных состояниях
\end{choices}

\question Одномерный гармонический осциллятор, находящийся в основном состоянии, дейст-вует малое возмущение $\hat V(x,t) = {V_0}x\cos \omega t$, частота которого равна частоте осциллятора. В какие состояния осциллятор будет совершать переходы? 
\begin{choices}
\choice в первое возбужденное      
\choice во второе возбужденное
\choice в третье возбужденное      
\choice ни в какие
\end{choices}

\question На одномерный гармонический осциллятор, находящийся в основном состоянии, дейст-вует малое возмущение $\hat V(x,t) = {V_0}x\cos \omega t$, частота которого равна удвоенной частоте осциллятора. В какие состояния осциллятор будет совершать переходы? 
\begin{choices}
\choice в первое возбужденное      
\choice во второе возбужденное
\choice в третье возбужденное      
\choice ни в какие
\end{choices}

\question На одномерный гармонический осциллятор, находящийся в $n$ стационарном состоянии, действует малое возмущение $\hat V(x,t) = {V_0}{x^2}\cos \omega t$, частота которого равна час-тоте осциллятора. В каких состояниях можно обнаружить осциллятор? 
\begin{choices}
\choice только в $n$, $(n + 2)$ и $(n - 2)$       
\choice только в $n$, $(n + 1)$ и $(n - 1)$
\choice только в $(n + 1)$ и $(n - 1)$         
\choice только в $n$
\end{choices}

\question На одномерный гармонический осциллятор, находящийся в основном состоянии, дейст-вует малое возмущение $\hat V(x,t) = {V_0}{x^2}\cos \omega t$, частота которого равна удвоенной частоте осциллятора. В какие состояния осциллятор может совершить переходы? Ответ дать в первом порядке теории нестационарных возмущений.
\begin{choices}
\choice только в первое и второе возбужденные
\choice только в первое возбужденное 
\choice только во второе возбужденное
\choice ни в какие
\end{choices}

\question На одномерный гармонический осциллятор с частотой $\omega $, находящийся в основ-ном состоянии, действует малое возмущение $\hat V(x,t) = {V_0}{\rm{tg}}\;x\cos 6\omega t$. Может ли осциллятор совершить переход?
\begin{choices}
\choice да    
\choice нет      
\choice мало информации для ответа    
\choice это зависит от ${V_0}$
\end{choices}

\question На одномерный гармонический осциллятор с частотой $\omega $, находящийся в основ-ном состоянии, действует малое возмущение $\hat V(x,t) = {V_0}{\rm{tg}}\;x\cos 7\omega t$. Может ли осциллятор совершить переход?
\begin{choices}
\choice да    
\choice нет      
\choice мало информации для ответа    
\choice это зависит от ${V_0}$
\end{choices}

\question На одномерный гармонический осциллятор с частотой $\omega $, находящийся в основ-ном состоянии, действует малое возмущение $\hat V(x,t) = {V_0}{\rm{si}}{{\rm{n}}^{\rm{2}}}x\cos 7\omega t$. В какие состояния осциллятор совершает переходы?
\begin{choices}
\choice во все четные  
\choice во все нечетные   
\choice в 7 стационарное  
\choice ни в какие
\end{choices}

\question На одномерный гармонический осциллятор с частотой $\omega $, находящийся в основ-ном состоянии, действует малое возмущение $\hat V(x,t) = {x^5}\cos {\omega _1}t$. При какой минимальной частоте возмущения возможен переход? 
\begin{choices}
\choice ${\omega _1} = \omega $  
\choice ${\omega _1} = 3\omega $ 
\choice ${\omega _1} = 5\omega $ 
\choice ${\omega _1} = 7\omega $
\end{choices}

\question На одномерный гармонический осциллятор с частотой $\omega $, находящийся в основ-ном состоянии, действует малое возмущение $\hat V(x,t) = {x^5}\cos {\omega _1}t$. При какой максимальной частоте возмущения возможен переход? 
\begin{choices}
\choice ${\omega _1} = \omega $  
\choice ${\omega _1} = 3\omega $ 
\choice ${\omega _1} = 5\omega $ 
\choice ${\omega _1} = 7\omega $
\end{choices}

\question На одномерный гармонический осциллятор с частотой $\omega $, находящийся в основ-ном состоянии, действует малое возмущение $\hat V(x,t) = \alpha \hat V(x)\cos (\omega t/2)$, где оператор $\hat V(x)$ имеет ненулевые матричные элементы для всех состояний осциллятора. Как вероятность перехода в первое возбужденное состояние зависит от $\alpha $?
\begin{choices}
\choice как ${\alpha ^2}$    
\choice как ${\alpha ^4}$    
\choice как ${\alpha ^6}$    
\choice как ${\alpha ^8}$
\end{choices}

\question На заряженный трехмерный гармонический осциллятор, находящийся в основном состоя-нии, действует малое однородное периодическое электрическое поле $\vec E(t) = {\vec E_0}\cos \omega t$, частота которого равна частоте осциллятора. Может ли осциллятор совершать перехо-ды? Указание. Кратность вырождения первого возбужденного состояния осциллятора равна 3.
\begin{choices}
\choice да    
\choice нет      
\choice мало информации для ответа    
\choice это зависит от ${E_0}$
\end{choices}

\question На заряженный трехмерный гармонический осциллятор, находящийся в основном состоя-нии, действует малое однородное периодическое электрическое поле $\vec E(t) = {\vec E_0}\cos \omega t$, частота которого равна удвоенной частоте осциллятора. Может ли осциллятор совер-шать переходы? Указание. Кратность вырождения второго возбужденного состояния осциллятора равна 6.
\begin{choices}
\choice да    
\choice нет      
\choice мало информации для ответа    
\choice это зависит от ${E_0}$
\end{choices}

\question На заряженный трехмерный гармонический осциллятор с частотой $\omega $, находя-щийся в основном состоянии, действует малое однородное периодическое электрическое поле $\vec E(t) = {\vec E_0}\cos {\omega _1}t$. При каких из нижеперечисленных частот поля осциллятор будет совершать переходы? Указание. Кратность вырождения первого возбужденного состояния осциллятора равна 3, второго возбужденного – 6, третьего возбужденного – 10, четвертого -15, и все уровни энергии трехмерного осциллятора обладают определенной четностью.
(1) ${\omega _1} = \omega $   (2) ${\omega _1} = 2\omega $  (3) ${\omega _1} = 3\omega $  (4) ${\omega _1} = 4\omega $
\begin{choices}
\choice только (1)        
\choice только (1) и (2)
\choice только (1) и (3)     
\choice и (1), и (2), и (3), и (4)
\end{choices}

\question На трехмерный гармонический осциллятор, находящийся в основном состоянии, действу-ет малое возмущение $\hat V(r,t) = \hat V(r)\cos \omega t$, где оператор $\hat V(r)$ зависит только от модуля радиус-вектора, а частота возмущения равна частоте осциллятора. Может ли осцилля-тор совершить переход, и если да, то в какое состояние? Указание. Кратность вырождения первого возбужденного состояния осциллятора равна 3, второго 6.
\begin{choices}
\choice да, в первое возбужденное  
\choice да, во второе возбужденное
\choice нет               
\choice мало информации для ответа
\end{choices}

\question На трехмерный гармонический осциллятор, находящийся в основном состоянии, действу-ет малое возмущение $\hat V(r,t) = \hat V(r)\cos \omega t$, где оператор $\hat V(r)$ зависит только от модуля радиус-вектора, а частота возмущения равна удвоенной частоте осциллятора. Может ли осциллятор совершить переход, и если да, то в какое состояние? Указание. Кратность вырождения первого возбужденного состояния осциллятора равна 3, второго 6.
\begin{choices}
\choice да, в первое возбужденное  
\choice да, во второе возбужденное
\choice нет               
\choice мало информации для ответа
\end{choices}

\question На трехмерный гармонический осциллятор, находящийся в основном состоянии, действу-ет малое возмущение $\hat V(r,t) = \alpha f(r)\cos \omega t$, где $f(r)$ - функция от модуля радиус-вектора, частота возмущения равна частоте осциллятора. Как вероятность перехода зависит от $\alpha $? Указание. Кратность вырождения первого возбужденного состояния осциллятора равна 3, второго 6.
\begin{choices}
\choice как ${\alpha ^2}$ 
\choice как ${\alpha ^6}$ 
\choice как ${\alpha ^8}$ 
\choice как ${\alpha ^4}$
\end{choices}

\question На трехмерный гармонический осциллятор с частотой $\omega $, находящийся в основном состоянии, действует малое возмущение $\hat V(\vec r,t) = \alpha {Y_{20}}(\vartheta ,\varphi )\cos {\omega _0}t$, где ${Y_{20}}$ - сферическая функция. При какой минимальной частоте возмуще-ния ${\omega _0}$ осциллятор может совершить переход? Ответ дать в первом порядке теории нестационарных возмущений. Указание. Кратность вырождения $N$-го уровня трехмерного гармонического осциллятора равна $(N + 1)(N + 2)/2$ (основному состоянию отвечает $N = 0$).
\begin{choices}
\choice ${\omega _0} = \omega $ 
\choice ${\omega _0} = 4\omega $   
\choice ${\omega _0} = 2\omega $   
\choice ни при какой
\end{choices}

\question На трехмерный гармонический осциллятор с частотой $\omega $, находящийся в основном состоянии, действует малое возмущение $\hat V(\vec r,t) = \alpha {\cos ^3}\vartheta \cos {\omega _0}t$. При какой из нижеперечисленных частот возмущения ${\omega _0}$ осциллятор может со-вершить переход? Указание. Кратность вырождения $N$-го уровня трехмерного гармонического осциллятора равна $(N + 1)(N + 2)/2$ (основному состоянию отвечает $N = 0$)
\begin{choices}
\choice ${\omega _0} = \omega $ 
\choice ${\omega _0} = 2\omega $   
\choice ${\omega _0} = 3\omega $   
\choice ${\omega _0} = 4\omega $
\end{choices}

\question На трехмерный гармонический осциллятор, находящийся в основном состоянии, действу-ет малое возмущение $\hat V = \alpha \sin \varphi \cos \omega t$, частота которого равна частоте осциллятора. Может ли осциллятор совершить переход, и если да, то в какое состояние? Указание. Кратность вырождения первого возбужденного состояния осциллятора равна 3, второго - 6.
\begin{choices}
\choice да, в первое   
\choice нет      
\choice да, во второе  
\choice мало информации для ответа
\end{choices}

\question На трехмерный гармонический осциллятор, находящийся в основном состоянии, действу-ет малое возмущение $\hat V = \alpha {\sin ^2}\varphi \cos \omega t$, частота которого равна часто-те осциллятора. Может ли осциллятор совершить переход, и если да, то в какое состояние? Указа-ние. Кратность вырождения первого возбужденного состояния осциллятора равна 3, второго - 6.
\begin{choices}
\choice да, в первое   
\choice нет      
\choice да, во второе  
\choice мало информации для ответа
\end{choices}

\question На трехмерный гармонический осциллятор, находящийся в основном состоянии, действу-ет малое возмущение $\hat V = \alpha \cos \vartheta \sin \varphi \cos \omega t$, частота которого равна частоте осциллятора. Может ли осциллятор совершить переход, и если да, то в какое состоя-ние? Указание. Кратность вырождения первого возбужденного состояния осциллятора равна 3, второго - 6.
\begin{choices}
\choice да, в первое   
\choice нет      
\choice да, во второе  
\choice мало информации для ответа
\end{choices}

\question На трехмерный гармонический осциллятор, находящийся в основном состоянии, действу-ет малое возмущение $\hat V = \alpha \cos \vartheta \sin \varphi \cos \omega t$, частота которого равна удвоенной частоте осциллятора. Может ли осциллятор совершить переход, и если да, то в какое состояние? Указание. Кратность вырождения первого возбужденного состояния осциллято-ра равна 3, второго - 6.
\begin{choices}
\choice да, в первое   
\choice нет      
\choice да, во второе  
\choice мало информации для ответа
\end{choices}

\question На трехмерный гармонический осциллятор, находящийся в основном состоянии, действу-ет малое возмущение $\hat V = \alpha {\cos ^2}\vartheta \cos \omega t$, частота которого равна час-тоте осциллятора. Может ли осциллятор совершить переход, и если да, то в какое состояние? Ука-зание. Кратность вырождения первого возбужденного состояния осциллятора равна 3, второго - 6.
\begin{choices}
\choice да, в первое   
\choice нет      
\choice да, во второе  
\choice мало информации для ответа
\end{choices}

\question На трехмерный гармонический осциллятор, находящийся в основном состоянии, действу-ет малое возмущение $\hat V = \alpha {\cos ^2}\vartheta \cos \omega t$, частота которого равна удвоенной частоте осциллятора. Может ли осциллятор совершить переход, и если да, то в какое состояние? Указание. Кратность вырождения первого возбужденного состояния осциллятора равна 3, второго - 6.
\begin{choices}
\choice да, в первое   
\choice нет      
\choice да, во второе  
\choice мало информации для ответа
\end{choices}

\question На атом водорода, находящийся в основном состоянии, действует малое возмущение $\hat V(\vec r,t) = \alpha {Y_{10}}(\vartheta ,\varphi )\cos \omega t$, где ${Y_{10}}$ - сферическая функция. При какой минимальной частоте возмущения возможен переход? Указание. Кратность вырождения уровней энергии электрона в атоме равна ${n^2}$, энергия стационарных состояний - ${\varepsilon _n} =  - {e^2}/2{n^2}a$, $n = 1,2,...$
\begin{choices}
\choice $\omega  = \frac{{{e^2}}}{{8a\hbar }}$    
\choice $\omega  = \frac{{2{e^2}}}{{8a\hbar }}$      
\choice $\omega  = \frac{{3{e^2}}}{{8a\hbar }}$      
\choice $\omega  = \frac{{4{e^2}}}{{8a\hbar }}$
\end{choices}

\question На атом водорода, находящийся в основном состоянии, действует малое возмущение $\hat V(\vec r,t) = \alpha {Y_{20}}(\vartheta ,\varphi )\cos \omega t$, где ${Y_{20}}$ - сферическая функция. При какой минимальной частоте возмущения возможен переход? Указание. Кратность вырождения уровней энергии электрона в атоме равна ${n^2}$, энергия стационарных состояний - ${\varepsilon _n} =  - {e^2}/2{n^2}a$, $n = 1,2,...$
\begin{choices}
\choice $\omega  = \frac{{8{e^2}}}{{18a\hbar }}$     
\choice $\omega  = \frac{{9{e^2}}}{{18a\hbar }}$     
\choice $\omega  = \frac{{10{e^2}}}{{18a\hbar }}$    
\choice $\omega  = \frac{{11{e^2}}}{{18a\hbar }}$
\end{choices}

\question На атом водорода, находящийся в основном состоянии, действует малое возмущение $\hat V(\vec r,t) = \alpha {\cos ^2}\vartheta \cos \omega t$. При какой минимальной частоте возмущения возможен переход? Указание. Кратность вырождения уровней энергии электрона в атоме равна ${n^2}$, энергия стационарных состояний - ${\varepsilon _n} =  - {e^2}/2{n^2}a$, $n = 1,2,...$
\begin{choices}
\choice $\omega  = \frac{{3{e^2}}}{{8a\hbar }}$      
\choice $\omega  = \frac{{8{e^2}}}{{18a\hbar }}$     
\choice $\omega  = \frac{{15{e^2}}}{{32a\hbar }}$    
\choice $\omega  = \frac{{24{e^2}}}{{50a\hbar }}$
\end{choices}

\question На атом водорода, находящийся в основном состоянии, действует малое возмущение $\hat V(\vec r,t) = \alpha r{\cos ^2}\vartheta \cos \omega t$. При какой минимальной частоте возмущения возможен переход? Указание. Кратность вырождения уровней энергии электрона в атоме равна ${n^2}$, энергия стационарных состояний - ${\varepsilon _n} =  - {e^2}/2{n^2}a$, $n = 1,2,...$
\begin{choices}
\choice $\omega  = \frac{{3{e^2}}}{{8a\hbar }}$      
\choice $\omega  = \frac{{8{e^2}}}{{18a\hbar }}$     
\choice $\omega  = \frac{{15{e^2}}}{{32a\hbar }}$    
\choice $\omega  = \frac{{24{e^2}}}{{50a\hbar }}$
\end{choices}

\question На атом водорода, находящийся в основном состоянии, действует малое возмущение $\hat V(\vec r,t) = \alpha {\cos ^3}\vartheta \cos \omega t$. При какой минимальной частоте возмущения возможен переход? Указание. Кратность вырождения уровней энергии электрона в атоме равна ${n^2}$, энергия стационарных состояний - ${\varepsilon _n} =  - {e^2}/2{n^2}a$, $n = 1,2,...$
\begin{choices}
\choice $\omega  = \frac{{3{e^2}}}{{8a\hbar }}$      
\choice $\omega  = \frac{{8{e^2}}}{{18a\hbar }}$     
\choice $\omega  = \frac{{15{e^2}}}{{32a\hbar }}$    
\choice $\omega  = \frac{{24{e^2}}}{{50a\hbar }}$
\end{choices}

\question На атом водорода, находящийся в основном состоянии, действует малое возмущение $\hat V(\vec r,t) = V(r)\cos \omega t$, где $V(r)$ зависит только от модуля радиус-вектора. При какой минимальной частоте возмущения возможен переход? Указание. Кратность вырождения уровней энергии электрона в атоме равна ${n^2}$, энергия стационарных состояний - ${\varepsilon _n} =  - {e^2}/2{n^2}a$, $n = 1,2,...$
\begin{choices}
\choice $\omega  = \frac{{3{e^2}}}{{8a\hbar }}$      
\choice $\omega  = \frac{{8{e^2}}}{{18a\hbar }}$     
\choice $\omega  = \frac{{15{e^2}}}{{32a\hbar }}$    
\choice $\omega  = \frac{{24{e^2}}}{{50a\hbar }}$
\end{choices}

\question На атом водорода, находящийся в основном состоянии, действует малое возмущение $\hat V(\vec r,t) = \alpha {\cos ^3}\varphi \cos \omega t$. При какой минимальной частоте возмущения возможен переход? Указание. Кратность вырождения уровней энергии электрона в атоме равна ${n^2}$, энергия стационарных состояний - ${\varepsilon _n} =  - {e^2}/2{n^2}a$, $n = 1,2,...$
\begin{choices}
\choice $\omega  = \frac{{3{e^2}}}{{18a\hbar }}$     
\choice $\omega  = \frac{{8{e^2}}}{{18a\hbar }}$     
\choice $\omega  = \frac{{15{e^2}}}{{32a\hbar }}$    
\choice $\omega  = \frac{{24{e^2}}}{{50a\hbar }}$
\end{choices}

\question На атом водорода, находящийся в основном состоянии, действует малое возмущение $\hat V(\vec r,t) = \alpha \sin \varphi \cos \omega t$, где $\omega  = 8{e^2}/18a\hbar $. Возможен ли пе-реход электрона в состояние с проекцией момента на ось $z$ $m = 1$? Указание. Кратность вы-рождения уровней энергии электрона в атоме равна ${n^2}$, энергии - $ - {e^2}/2{n^2}a$, $n = 1,2,...$
\begin{choices}
\choice да    
\choice нет      
\choice мало информации для ответа    
\choice это зависит от $\alpha $
\end{choices}

\question На атом водорода, находящийся в основном состоянии, действует малое возмущение $\hat V(\vec r,t) = \alpha \sin \varphi \cos \omega t$, где $\omega  = 10{e^2}/18a\hbar $. Возможен ли пе-реход электрона в состояние с проекцией момента на ось $z$ $m = 1$? Указание. Кратность вы-рождения уровней энергии электрона в атоме равна ${n^2}$, энергии - $ - {e^2}/2{n^2}a$, $n = 1,2,...$
\begin{choices}
\choice да    
\choice нет      
\choice мало информации для ответа    
\choice это зависит от $\alpha $
\end{choices}

\question На квантовую систему внезапно накладывают возмущение. Какие из нижеперечисленных величин не успеют измениться за время «включения» возмущения?
\begin{choices}
\choice гамильтониан               
\choice волновая функция
\choice спектр собственных значений энергии    
\choice собственные функции гамильтониана
\end{choices}

\question На одномерную квантовую систему с гамильтонианом ${\hat H_0}$ внезапно накладывают возмущение $\hat V(x)$. Какой формулой описываются вероятности переходов из $n$ состояния гамильтониана ${\hat H_0}$ в $k$ состояние гамильтониана ${\hat H_0} + \hat V(x)$?
\begin{choices}
\choice ${w_{n \to k}} = {\left| {\int {{\varphi _n}^*(x)\hat V(x){\varphi _k}(x)dx} } \right|^2}$      
\choice ${w_{n \to k}} = {\left| {\int {{\psi _n}^*(x){\psi _k}(x)dx} } \right|^2}$
\choice ${w_{n \to k}} = {\left| {\int {{\varphi _n}^*(x)\hat V(x){\psi _k}(x)dx} } \right|^2}$      
\choice ${w_{n \to k}} = {\left| {\int {{\varphi _n}^*(x){\psi _k}(x)dx} } \right|^2}$ 
(здесь $\varphi $ и $\psi $ - собственные функции гамильтонианов ${\hat H_0}$ и ${\hat H_0} + \hat V$ соответственно)
\end{choices}

\question Частица находится в основном состоянии в некотором одномерном потенциале, являю-щимся четной функцией координаты. Внезапно потенциал изменяется, но остается четной функ-цией координаты. Какова вероятность того, что частица окажется в первом возбужденном состоя-нии?
\begin{choices}
\choice $w = 1/2$      
\choice $w = 0$     
\choice $w = 1/4$      
\choice $w = 3/4$
\end{choices}

\question Осциллятор находится в основном состоянии. В некоторый момент времени осциллятор-ная частота $\omega $ мгновенно меняется до некоторого значения ${\omega _1}$. Вероятность перехода осциллятора в первое возбужденное состояние равна
\begin{choices}
\choice $w = 1/2$      
\choice $w = 0$     
\choice $w = \left| {\frac{{\omega  - {\omega _1}}}{\omega }} \right|$    
\choice $w = \left| {\frac{{\omega  - {\omega _1}}}{{\omega  + {\omega _1}}}} \right|$
\end{choices}

\question Одномерный осциллятор находится в основном состоянии. В некоторый момент времени осцилляторная частота $\omega $ мгновенно меняется до некоторого значения ${\omega _1}$. В какие состояния осциллятор будет совершать при этом переходы?
\begin{choices}
\choice во все четные     
\choice во все нечетные
\choice во все         
\choice ни в какие 
\end{choices}

\question Одномерный осциллятор находится в основном состоянии. В некоторый момент времени осциллятор мгновенно перемещается на некоторое малое расстояние $\Delta l$. Сможет ли осцил-лятор совершить при этом переход в первое возбужденное состояние?
\begin{choices}
\choice да
\choice нет
\choice зависит от $\Delta l$
\choice мало информации для ответа
\end{choices}

\question Трехмерный осциллятор находится в основном состоянии. В некоторый момент времени осцилляторная частота $\omega $ мгновенно меняется до некоторого значения ${\omega _1}$. Возможен ли переход осциллятора на первый возбужденный уровень энергии? Указание. Крат-ность вырождения первого возбужденного уровня трехмерного осциллятора равна 3.
\begin{choices}
\choice да    
\choice нет      
\choice если ${\omega _1} > \omega $, то да    
\choice если ${\omega _1} < \omega $, то да
\end{choices}

\question  Трехмерный осциллятор находится в основном состоянии. В некоторый момент времени осцилляторная частота $\omega $ мгновенно меняется до некоторого значения ${\omega _1}$. Возможен ли переход осциллятора в состояние с моментом $l = 2$?
\begin{choices}
\choice да    
\choice нет      
\choice если ${\omega _1} > \omega $, то да    
\choice если ${\omega _1} < \omega $, то да
\end{choices}

\question Трехмерный осциллятор находится в основном состоянии. В некоторый момент времени осцилляторная частота $\omega $ мгновенно меняется до некоторого значения ${\omega _1}$. Возможен ли переход осциллятора на второй возбужденный уровень энергии? Указание. Крат-ность вырождения первого возбужденного уровня трехмерного осциллятора равна 6.
\begin{choices}
\choice да    
\choice нет      
\choice если ${\omega _1} > \omega $, то да    
\choice если ${\omega _1} < \omega $, то да
\end{choices}

\question Трехмерный осциллятор находится в основном состоянии. В некоторый момент времени осцилляторная частота $\omega $ мгновенно меняется до некоторого значения ${\omega _1}$. Возможен ли переход осциллятора на третий возбужденный уровень энергии? Указание. Крат-ность вырождения третьего возбужденного уровня трехмерного осциллятора равна 10.
\begin{choices}
\choice да    
\choice нет      
\choice если ${\omega _1} > \omega $, то да    
\choice если ${\omega _1} < \omega $, то да
\end{choices}

\question Трехмерный осциллятор находится на первом возбужденном уровне энергии. Внезапно на осциллятор накладывается возмущение, зависящее только от модуля радиус-вектора. Может ли осциллятор совершить переход в основное состояние? Указание. Кратность вырождения первого возбужденного уровня трехмерного осциллятора равна 3.
\begin{choices}
\choice да
\choice нет
\choice это зависит от величины возмущения
\choice мало информации для ответа
\end{choices}

\question Трехмерный осциллятор находится на втором возбужденном уровне энергии. Внезапно на осциллятор накладывается возмущение, зависящее только от модуля радиус-вектора. Может ли осциллятор совершить переход в основное состояние? Указание. Кратность вырождения второго возбужденного уровня трехмерного осциллятора равна 6.
\begin{choices}
\choice да
\choice нет
\choice это зависит от величины возмущения
\choice мало информации для ответа
\end{choices}

\question Трехмерный осциллятор находится на втором возбужденном уровне в состоянии с неопре-деленным моментом. Внезапно на осциллятор накладывается возмущение, зависящее только от модуля радиус-вектора. Может ли осциллятор совершить переход в основное состояние? Указа-ние. Кратность вырождения второго возбужденного уровня трехмерного осциллятора равна 6.
\begin{choices}
\choice да
\choice нет
\choice это зависит от величины возмущения
\choice мало информации для ответа
\end{choices}

\question Трехмерный осциллятор находится на втором возбужденном уровне энергии в состоянии с неопределенной проекцией момента на ось $z$. Внезапно на осциллятор накладывается возмуще-ние, зависящее только от модуля радиус-вектора. Может ли осциллятор совершить переход в основное состояние? Указание. Кратность вырождения второго возбужденного уровня трехмерного осциллятора равна 6.
\begin{choices}
\choice да
\choice нет
\choice это зависит от величины возмущения
\choice мало информации для ответа
\end{choices}

\question Трехмерный осциллятор находится в основном состоянии. Внезапно осцилляторная часто-та изменяется. Возможен ли переход осциллятора в состояние с проекцией момента на ось $z$ $m = 1$?
\begin{choices}
\choice да
\choice нет
\choice это зависит от величины возмущения
\choice мало информации для ответа
\end{choices}

\question  Незаряженная частица со спином $s = 1/2$, имеющая магнитный момент, находится в ста-ционарном состоянии некоторого независящего от спина гамильтониана с определенной проекцией спина на ось $z$: ${s_z} = 1/2$. Внезапно включается магнитное поле, направленное вдоль оси $x$. Возможен ли переход частицы в состояние с проекцией спина на ось $z$ ${s_z} =  - 1/2$?
\begin{choices}
\choice да
\choice нет
\choice это зависит от величины возмущения
\choice мало информации для ответа
\end{choices}

\question  Незаряженная частица со спином $s = 1/2$, имеющая магнитный момент, находится в ста-ционарном состоянии некоторого гамильтониана с определенной проекцией спина на ось $z$: ${s_z} = 1/2$. Внезапно включается магнитное поле, направленное вдоль оси $x$. Сразу после этого измеряют проекцию спина частицы на ось $x$. Какие значения будут обнаружены и с какими вероятностями?
\begin{choices}
\choice только ${s_x} = 1/2$
\choice только ${s_x} =  - 1/2$
\choice ${s_x} = 1/2$ и ${s_x} =  - 1/2$ с одинаковыми вероятностями
\choice ${s_x} = 1/2$ с вероятностью $w = 3/4$ и ${s_x} =  - 1/2$ с вероятностью $w = 1/4$
\end{choices}

\question Электрон в атоме водорода находится на первом возбужденном уровне энергии. Внезапно на атом накладывается возмущение, оператор которого зависит только от модуля радиус-вектора. Может ли электрон совершить переход в основное состояние? Указание. Кратность вырождения первого возбужденного уровня энергии атома водорода равна 4.
\begin{choices}
\choice да
\choice нет
\choice это зависит от величины возмущения
\choice мало информации для ответа
\end{choices}

\question Электрон в атоме водорода находится на первом возбужденном уровне энергии в состоя-нии с неопределенным моментом. Внезапно на атом накладывается возмущение, оператор которого зависит только от модуля радиус-вектора. Может ли электрон совершить переход в основное состояние? Указание. Кратность вырождения первого возбужденного уровня энергии атома водорода равна 4.
\begin{choices}
\choice да
\choice нет
\choice это зависит от величины возмущения
\choice мало информации для ответа
\end{choices}

\question Электрон в атоме водорода находится на первом возбужденном уровне энергии в состоя-нии с неопределенной проекцией момента на ось $z$. Внезапно на атом накладывается возмуще-ние, оператор которого зависит только от модуля радиус-вектора. Может ли электрон совершить переход в основное состояние? Указание. Кратность вырождения первого возбужденного уровня энергии атома водорода равна 4.
\begin{choices}
\choice да
\choice нет
\choice это зависит от величины возмущения
\choice мало информации для ответа
\end{choices}

\question Электрон в водородоподобном ионе находится в основном состоянии. Внезапно заряд ядра изменяется (это происходит при $\beta $-распаде ядер). Может ли электрон оказаться на первом возбужденном уровне энергии иона?
\begin{choices}
\choice да
\choice нет
\choice это зависит от величины изменения заряда
\choice мало информации для ответа
\end{choices}

\question Электрон в водородоподобном ионе находится в основном состоянии. Внезапно заряд ядра изменяется (это происходит при $\beta $-распаде ядер). Может ли электрон перейти при этом в $p$-состояние?
\begin{choices}
\choice да
\choice нет
\choice это зависит от величины изменения заряда
\choice мало информации для ответа
\end{choices}

\question Электрон в водородоподобном ионе находится в основном состоянии. Внезапно заряд ядра изменяется (это происходит при $\beta $-распаде ядер). Может ли электрон перейти при этом в третье $s$-состояние?
\begin{choices}
\choice да
\choice нет
\choice это зависит от величины изменения заряда
\choice мало информации для ответа
\end{choices}

\question На атом водорода, находящийся в основном состоянии мгновенно накладывается однород-ное электрическое поле. Может ли при этом электрон совершить переход в $p$-состояние?
\begin{choices}
\choice да
\choice нет
\choice это зависит от величины поля
\choice мало информации для ответа
\end{choices}

\end{questions}


\section{ Системы тождественных частиц }

\subsection{ Перестановочная симметрия волновой функции }



\begin{questions}

\question Выбрать правильное утверждение
\begin{choices}
\choice фермионы – это частицы с четным спином
\choice бозоны имеют полуцелый спин
\choice спин всех бозонов равен 2
\choice если частица имеет спин, равный 1/2, то это фермион
\end{choices}

\question Выбрать правильное утверждение
\begin{choices}
\choice фермионы – это частицы с целым спином
\choice бозоны – это частицы с нечетным спином
\choice фермионы – это частицы с полуцелым спином
\choice бозоны – это частицы с четным спином
\end{choices}

\question Частица имеет спин $s = 3/4$. Какое утверждение – А, Б, В или Г - относительно свойств этой частицы справедливо?
\begin{choices}
\choice эта частица – фермион
\choice эта частица – бозон
\choice эта частица - суперпозиция состояний бозона со спином $s = 1$ и фермиона со спином $s = 1/2$
\choice спин такого значения принимать не может
\end{choices}

\question Волновая функция системы тождественных фермионов 
\begin{choices}
\choice симметрична относительно перестановок аргументов, относящихся к разным частицам
\choice антисимметрична относительно перестановок аргументов, относящихся к разным частицам
\choice симметрична относительно замены ${\vec r_i} \to  - {\vec r_i}$
\choice антисимметрична относительно замены ${\vec r_i} \to  - {\vec r_i}$
\end{choices}

\question Волновая функция системы тождественных бозонов
\begin{choices}
\choice симметрична относительно перестановок аргументов, относящихся к разным частицам
\choice антисимметрична относительно перестановок аргументов, относящихся к разным частицам
\choice симметрична относительно перестановок пространственных координат частиц
\choice антисимметрична относительно перестановок пространственных координат частиц
\end{choices}

\question Гамильтониан системы тождественных фермионов
\begin{choices}
\choice симметричен относительно перестановок координат частиц, так как это фермионы
\choice антисимметричен относительно перестановок координат частиц, так это фермионы
\choice симметричен относительно перестановок координат частиц, так как частицы тождественные
\choice антисимметричен относительно перестановок координат, так как частицы тождественные
\end{choices}

\question Гамильтониан системы тождественных бозонов
\begin{choices}
\choice симметричен относительно перестановок координат частиц, так как это бозоны
\choice антисимметричен относительно перестановок координат, так это бозоны
\choice симметричен относительно перестановок координат частиц, так как частицы тождественные
\choice антисимметричен относительно перестановок координат, так как частицы тождественные
\end{choices}

\question Гамильтониан системы тождественных невзаимодействующих частиц
\begin{choices}
\choice равен сумме одинаковых операторов, каждый из которых действует на координаты только од-ной частицы
\choice равен произведению одинаковых операторов, каждый из которых действует на координаты только одной частицы
\choice равен сумме одинаковых операторов, каждый из которых действует на координаты только од-ной пары частиц
\choice равен произведению одинаковых операторов, каждый из которых действует на координаты только одной пары частиц
\end{choices}

\question Гамильтониан системы из двух тождественных частиц имеет вид: $\hat H({x_1},{x_2}) = \hat h({x_1}) + \hat h({x_2})$. Собственные значения ${\varepsilon _n}$ и собственные функции ${f_n}(x)$ одночастичного гамильтониана $\hat h(x)$ известны. Какая (какие) из нижеперечис-ленных функций не будет собственной функцией гамильтониана системы $\hat H({x_1},{x_2})$?
\begin{choices}
\choice ${f_i}({x_1}){f_k}({x_2}) + 5{f_k}({x_1}){f_i}({x_2})$      
\choice ${f_k}({x_1}){f_i}({x_2})$
\choice ${f_k}({x_1}){f_i}({x_2}) + {f_j}({x_1}){f_n}({x_2})$    
\choice $2{f_i}({x_1}){f_k}({x_2}) - {f_k}({x_1}){f_i}({x_2})$
\end{choices}

\question Гамильтониан системы из двух тождественных частиц имеет вид: $\hat H({x_1},{x_2}) = \hat h({x_1}) + \hat h({x_2})$. Собственные значения ${\varepsilon _n}$ и собственные функции ${f_n}(x)$ одночастичного гамильтониана $\hat h(x)$ известны. Какая (какие) из нижеперечис-ленных функций будет одновременно и собственной функцией гамильтониана системы $\hat H({x_1},{x_2})$ и являться антисимметричной по отношению к перестановкам координат частиц?
\begin{choices}
\choice ${f_i}({x_1}){f_k}({x_2}) - 5{f_k}({x_1}){f_i}({x_2})$      
\choice ${f_k}({x_1}) - {f_i}({x_2})$
\choice ${f_k}({x_1}){f_i}({x_2}) - {f_j}({x_1}){f_n}({x_2})$    
\choice ${f_i}({x_1}){f_k}({x_2}) - {f_k}({x_1}){f_i}({x_2})$
\end{choices}

\question Что такое определитель Слэттера?
\begin{choices}
\choice это определитель секулярного уравнения
\choice волновая функция системы тождественных бозонов 
\choice волновая функция системы тождественных фермионов
\choice определитель матриц Паули
\end{choices}

\question Какая из нижеследующих формул представляет собой определитель Слэттера для системы двух тождественных невзаимодействующих фермионов, один из которых находится в одночас-тичном состоянии ${f_1}$, второй – в одночастичном состоянии ${f_2}$?
\begin{choices}
\choice $\frac{1}{{\sqrt 2 }}\left| {\begin{array}{*{20}{c}}
{{f_1}({{\vec r}_1})}&{{f_1}({{\vec r}_2})}\\
{{f_1}({{\vec r}_1})}&{{f_1}({{\vec r}_2})}
\end{array}} \right|$               
\choice $\frac{1}{{\sqrt 2 }}\left| {\begin{array}{*{20}{c}}
{{f_1}({{\vec r}_1})}&{{f_2}({{\vec r}_2})}\\
{{f_1}({{\vec r}_1})}&{{f_2}({{\vec r}_2})}
\end{array}} \right|$
\choice $\frac{1}{{\sqrt 2 }}\left| {\begin{array}{*{20}{c}}
{{f_1}({{\vec r}_1})}&{{f_1}({{\vec r}_2})}\\
{{f_2}({{\vec r}_1})}&{{f_2}({{\vec r}_2})}
\end{array}} \right|$               
\choice $\frac{1}{{\sqrt 2 }}\left| {\begin{array}{*{20}{c}}
{{f_2}({{\vec r}_1})}&{{f_1}({{\vec r}_2})}\\
{{f_2}({{\vec r}_1})}&{{f_1}({{\vec r}_2})}
\end{array}} \right|$
\end{choices}

\question Имеется система тождественных невзаимодействующих фермионов со спином $s = 99/2$. Какое максимальное количество частиц могут находиться в одинаковом пространственном со-стоянии?
\begin{choices}
\choice 99    
\choice 100      
\choice 101      
\choice любое
\end{choices}

\question Имеется система тождественных невзаимодействующих бозонов со спином $s = 99$. Какое максимальное количество частиц могут находиться в одинаковом пространственном состоянии?
\begin{choices}
\choice 99    
\choice 100      
\choice 101      
\choice любое
\end{choices}

\question Может ли волновая функция $\Psi ({\vec r_1},{\vec r_2}) \sim {f_1}({\vec r_1}){f_2}({\vec r_2}) + {f_2}({\vec r_1}){f_1}({\vec r_2})$, где ${f_1}$ и ${f_2}$ - функции одной переменной, опи-сывать пространственное состояние системы из двух тождественных фермионов со спином $s = 1/2$ каждый?
\begin{choices}
\choice да, поскольку эта функция симметрична относительно перестановок
\choice нет, поскольку эта функция симметрична относительно перестановок
\choice да, если спиновая часть волновой функции антисимметрична
\choice это зависит от ${f_1}$, ${f_2}$
\end{choices}

\question Может ли волновая функция $\Psi ({\vec r_1},{\vec r_2}) \sim \left( {{f_1}({{\vec r}_1}){f_2}({{\vec r}_2}) + {f_2}({{\vec r}_1}){f_1}({{\vec r}_2})} \right)\left[ {{{\left( {\begin{array}{*{20}{c}}
1\\
0
\end{array}} \right)}_1}{{\left( {\begin{array}{*{20}{c}}
0\\
1
\end{array}} \right)}_2} - {{\left( {\begin{array}{*{20}{c}}
0\\
1
\end{array}} \right)}_1}{{\left( {\begin{array}{*{20}{c}}
1\\
0
\end{array}} \right)}_2}} \right]$ описывать состояние системы из двух тождественных фермионов со спином $s = 1/2$ каждый?
\begin{choices}
\choice да    
\choice нет      
\choice зависит от ${f_1}$, ${f_2}$      
\choice мало информации для от-вета
\end{choices}

\question Может ли волновая функция $\Psi ({\vec r_1},{\vec r_2}) \sim {f_1}({\vec r_1}){f_2}({\vec r_2}) - {f_2}({\vec r_1}){f_1}({\vec r_2})$, где ${f_1}$ и ${f_2}$ - функции одной переменной, опи-сывать пространственное состояние системы из двух тождественных бозонов со спином $s = 0$ каждый?
\begin{choices}
\choice да, поскольку эта функция антисимметрична относительно перестановок
\choice нет, поскольку эта функция антисимметрична относительно перестановок
\choice да, если спиновая часть волновой функции антисимметрична
\choice это зависит от ${f_1}$, ${f_2}$
\end{choices}

\question Система тождественных бозонов с нулевым спином имеет волновую функцию:
$\Psi  \sim {f_i}({\vec r_1}){f_k}({\vec r_2}) + {f_k}({\vec r_1}){f_i}({\vec r_2}) + {f_i}({\vec r_1}){f_n}({\vec r_2}) + {f_n}({\vec r_1}){f_i}({\vec r_2})$, 
где $i,\;k,\;n$- квантовые числа одночастичных состояний. Сколько частиц входят в систему?
\begin{choices}
\choice одна  
\choice две      
\choice три      
\choice эта система с неопределенным числом частиц
\end{choices}

\question Два тождественных бозона со спином $s = 0$ каждый связаны потенциалом, зависящим только от модуля относительного радиуса-вектора: $U({\vec r_1},{\vec r_2}) = U(|{\vec r_1} - {\vec r_2}|)$. Какие значения может принимать орбитальный момент относительного движения?
\begin{choices}
\choice только $l = 0$          
\choice только $l = 1$
\choice только четные значения        
\choice только нечетные значения
\end{choices}

\question Система из трех тождественных бозонов со спином $s = 0$ находится в состоянии, в кото-ром две частицы находятся в одночастичном состоянии $i$, одна частица - в одночастичном со-стоянии $k$. Какой формулой – А., Б., 
\begin{choices}
\choice или 
\choice – определяется волновая функция системы (${f_i}(\vec r)$ и ${f_k}(\vec r)$ - одночастичные волновые функции)
\choice $\Psi ({\vec r_1},{\vec r_2},{\vec r_3}) = {f_i}({\vec r_1}){f_i}({\vec r_2}){f_k}({\vec r_3})$ 
\choice $\Psi ({\vec r_1},{\vec r_2},{\vec r_3}) = \frac{1}{{\sqrt 2 }}\left( {{f_i}({{\vec r}_1}){f_i}({{\vec r}_2}){f_k}({{\vec r}_3}) + {f_i}({{\vec r}_1}){f_i}({{\vec r}_3}){f_k}({{\vec r}_2})} \right)$   
\choice $\Psi ({\vec r_1},{\vec r_2},{\vec r_3}) = \frac{1}{{\sqrt 3 }}\left( {{f_i}({{\vec r}_1}){f_i}({{\vec r}_2}){f_k}({{\vec r}_3}) + {f_i}({{\vec r}_1}){f_i}({{\vec r}_3}){f_k}({{\vec r}_2}) + {f_i}({{\vec r}_3}){f_i}({{\vec r}_2}){f_k}({{\vec r}_1})} \right)$
\choice такого состояния в системе бозонов быть не может
\end{choices}

\question Три тождественных невзаимодействующих фермиона со спином $s = 1/2$ движутся в неко-тором потенциале. Собственные функции одночастичного гамильтониана ${f_i}(x)$ известны ($x = \vec r,{\sigma _z}$, индекс $i$ включает в себя и «спиновые» квантовые числа). Какой формулой определяется волновая функция состояния системы частиц, в котором два фермиона находятся в одночастичном состоянии $i$, один – в одночастичном состоянии $k$?
\begin{choices}
\choice $\Psi ({x_1},{x_2},{x_3}) = {f_i}({x_1}){f_i}({x_2}){f_k}({x_3})$
\choice $\Psi ({x_1},{x_2},{x_3}) = \frac{1}{{\sqrt 2 }}\left( {{f_i}({x_1}){f_i}({x_2}){f_k}({x_3}) - {f_i}({x_1}){f_i}({x_3}){f_k}({x_2})} \right)$
\choice $\Psi ({x_1},{x_2},{x_3}) = \frac{1}{{\sqrt 3 }}\left( {{f_i}({x_1}){f_i}({x_2}){f_k}({x_3}) - {f_i}({x_1}){f_i}({x_3}){f_k}({x_2}) - {f_i}({x_3}){f_i}({x_2}){f_k}({x_1})} \right)$
\choice такого состояния в системе фермионов быть не может
\end{choices}

\question Десять тождественных невзаимодействующих фермионов со спином $s = 1/2$ находятся в потенциале одномерного гармонического осциллятора с частотой $\omega $. Какова энергия ос-новного состояния системы? 
\begin{choices}
\choice $24\hbar \omega $ 
\choice $25\hbar \omega $ 
\choice $26\hbar \omega $ 
\choice $27\hbar \omega $
\end{choices}

\question Десять тождественных невзаимодействующих фермионов со спином $s = 3/2$ находятся в потенциале одномерного гармонического осциллятора с частотой $\omega $. Какова энергия ос-новного состояния системы? 
\begin{choices}
\choice $10\hbar \omega $ 
\choice $11\hbar \omega $ 
\choice $12\hbar \omega $ 
\choice $13\hbar \omega $
\end{choices}

\question Десять тождественных невзаимодействующих бозонов со спином $s = 1$ находятся в по-тенциале одномерного гармонического осциллятора с частотой $\omega $. Какова энергия основ-ного состояния системы? 
\begin{choices}
\choice $5\hbar \omega $  
\choice $6\hbar \omega $  
\choice $7\hbar \omega $  
\choice $8\hbar \omega $
\end{choices}

\question Десять тождественных невзаимодействующих фермионов со спином $s = 1/2$ находятся в потенциале трехмерного гармонического осциллятора с частотой $\omega $. Какова энергия ос-новного состояния системы? Указание. Кратности вырождения уровней энергии трехмерного ос-циллятора равны: 1 - для основного состояния, 3 – для первого возбужденного, 6 – для второго возбужденного. Энергии стационарных состояний трехмерного осциллятора описываются соот-ношением: ${\varepsilon _n} = \hbar \omega (n + 3/2)$, $n = 0,1,2,...$,
\begin{choices}
\choice $28\hbar \omega $ 
\choice $30\hbar \omega $ 
\choice $32\hbar \omega $ 
\choice $34\hbar \omega $
\end{choices}

\question Десять тождественных невзаимодействующих фермионов со спином $s = 3/2$ находятся в потенциале трехмерного гармонического осциллятора с частотой $\omega $. Какова энергия ос-новного состояния системы? Указание. Кратности вырождения уровней энергии трехмерного ос-циллятора равны: 1 - для основного состояния, 3 – для первого возбужденного, 6 – для второго возбужденного. Энергии стационарных состояний трехмерного осциллятора описываются соот-ношением: ${\varepsilon _n} = \hbar \omega (n + 3/2)$, $n = 0,1,2,...$,
\begin{choices}
\choice $20\hbar \omega $ 
\choice $21\hbar \omega $ 
\choice $22\hbar \omega $ 
\choice $23\hbar \omega $
\end{choices}

\question Два тождественных невзаимодействующих бозона со спином $s = 0$ находятся в потенциале одномерного гармонического осциллятора. Какова кратность вырождения второго возбужденного уровня энергии системы? 
\begin{choices}
\choice 1     
\choice 2     
\choice 3     
\choice 4
\end{choices}

\question Два тождественных невзаимодействующих фермиона со спином $s = 1/2$ находятся в свя-занном состоянии в некотором одномерном потенциале. Какова кратность вырождения основ-ного состояния системы? 
\begin{choices}
\choice 1     
\choice 2     
\choice 3     
\choice 4
\end{choices}

\question Два тождественных невзаимодействующих бозона со спином $s = 1$ находятся в связан-ном состоянии в некотором одномерном потенциале. Какова кратность вырождения основного состояния системы? 
\begin{choices}
\choice 4     
\choice 5     
\choice 6     
\choice 7
\end{choices}

\question Два тождественных невзаимодействующих фермиона со спином $s = 1/2$ находятся в свя-занном состоянии в некотором одномерном потенциале. Какова кратность вырождения первого возбужденного состояния системы? 
\begin{choices}
\choice 1     
\choice 2     
\choice 3     
\choice 4
\end{choices}

\question Два тождественных невзаимодействующих бозона со спином $s = 1$ находятся в связан-ном состоянии в некотором одномерном потенциале. Какова кратность вырождения первого воз-бужденного состояния системы? 
\begin{choices}
\choice 6     
\choice 7     
\choice 8     
\choice 9
\end{choices}

\question Два тождественных невзаимодействующих фермиона со спином $s = 1/2$ находятся в по-тенциале одномерного гармонического осциллятора. Какова кратность вырождения второго воз-бужденного уровня энергии системы? 
\begin{choices}
\choice 3     
\choice 4     
\choice 5     
\choice 6
\end{choices}

\question Два тождественных невзаимодействующих бозона со спином $s = 1$ находятся в потенциале одномерного гармонического осциллятора. Какова кратность вырождения второго возбужденного уровня энергии системы? 
\begin{choices}
\choice 15    
\choice 16    
\choice 17    
\choice 18
\end{choices}

\question Два тождественных невзаимодействующих бозона со спином $s = 0$ находятся в потенциале сферического гармонического осциллятора. Какова кратность вырождения первого возбужденного состояния системы? Указание. Кратности вырождения уровней энергии трехмерного осциллятора равны: 1 - для основного состояния, 3 – для первого возбужденного. 
\begin{choices}
\choice 1     
\choice 2     
\choice 3     
\choice 4
\end{choices}

\question Два тождественных невзаимодействующих фермиона со спином $s = 1/2$ находятся в по-тенциале сферического гармонического осциллятора. Какова кратность вырождения первого воз-бужденного уровня энергии системы? Указание. Кратности вырождения уровней энергии трех-мерного осциллятора равны: 1 - для основного состояния, 3 – для первого возбужденного.
\begin{choices}
\choice 9     
\choice 12    
\choice 15    
\choice 18
\end{choices}

\question Два тождественных невзаимодействующих бозона со спином $s = 1$ находятся в потенциале сферического гармонического осциллятора. Какова кратность вырождения первого возбужденного уровня энергии системы? Указание. Кратности вырождения уровней энергии трехмерного осциллятора равны: 1 - для основного состояния, 3 – для первого возбужденного, 6 – для второго возбужденного.
\begin{choices}
\choice 21    
\choice 24    
\choice 27    
\choice 30
\end{choices}

\question Два тождественных невзаимодействующих бозона со спином $s = 0$ находятся в потенциале сферического гармонического осциллятора. Какова кратность вырождения второго возбужденного уровня энергии системы? Указание. Кратности вырождения уровней энергии трехмерного осциллятора равны: 1 - для основного состояния, 3 – для первого возбужденного, 6 – для второго возбужденного.
\begin{choices}
\choice 10    
\choice 12    
\choice 14    
\choice 16
\end{choices}

\question Два тождественных невзаимодействующих фермиона со спином $s = 1/2$ находятся в по-тенциале сферического гармонического осциллятора. Какова кратность вырождения второго воз-бужденного уровня энергии системы? Указание. Кратности вырождения уровней энергии трех-мерного осциллятора равны: 1 - для основного состояния, 3 – для первого возбужденного, 6 – для второго возбужденного.
\begin{choices}
\choice 27    
\choice 32    
\choice 36    
\choice 39
\end{choices}

\question Два тождественных невзаимодействующих бозона со спином $s = 1$ находятся в потенциале сферического гармонического осциллятора. Какова кратность вырождения второго возбужденного уровня энергии системы? Указание. Кратности вырождения уровней энергии трехмерного осциллятора равны: 1 - для основного состояния, 3 – для первого возбужденного, 6 – для второго возбужденного.
\begin{choices}
\choice 81    
\choice 87    
\choice 83    
\choice 99
\end{choices}

\question Имеется два фермиона со спином $s = 1/2$. Что можно сказать о спиновой волновой функ-ции состояния с определенным суммарным спином $S = 1$?
\begin{choices}
\choice эта функция антисимметрична относительно перестановок спиновых координат частиц, так как частицы – фермионы
\choice эта не обладает определенной симметрией по отношению к перестановкам спиновых координат
\choice эта функция антисимметрична относительно перестановок спиновых координат частиц по по-строению независимо от того, являются частицы тождественными, или нет
\choice эта функция симметрична относительно перестановок спиновых координат частиц по построе-нию независимо от того, являются частицы тождественными, или нет
\end{choices}

\question Имеется два фермиона со спином $s = 1/2$. Что можно сказать о спиновой волновой функ-ции состояния с определенным суммарным спином $S = 0$?
\begin{choices}
\choice эта функция антисимметрична относительно перестановок спиновых координат частиц, так как частицы – фермионы
\choice эта не обладает определенной симметрией по отношению к перестановкам спиновых координат
\choice эта функция антисимметрична относительно перестановок спиновых координат частиц по по-строению независимо от того, являются частицы тождественными, или нет
\choice эта функция симметрична относительно перестановок спиновых координат частиц по построе-нию независимо от того, являются частицы тождественными, или нет
\end{choices}

\question Имеется два фермиона со спином $s = 1/2$. Фермионы находятся в состоянии, в котором проекция суммарного спина на ось $z$ имеет определенное значение ${S_z} = 0$, а при измере-нии суммарного спина с определенными вероятностями могут быть получены два значения. Какое утверждение относительно спиновой функции этого состояния является верным?
\begin{choices}
\choice эта функция антисимметрична относительно перестановок спиновых координат частиц, так как частицы – фермионы
\choice эта функция не обладает определенной симметрией по отношению к перестановкам спиновых координат
\choice эта функция антисимметрична относительно перестановок спиновых координат частиц по по-строению независимо от того, являются частицы тождественными, или нет
\choice эта функция симметрична относительно перестановок спиновых координат частиц по построе-нию независимо от того, являются частицы тождественными, или нет
\end{choices}

\question Имеется два фермиона со спином $s = 1/2$. Фермионы находятся в состоянии, в котором проекция суммарного спина на ось $z$ имеет определенное значение ${S_z} = 1$. Будет ли спино-вая волновая функция этого состояния обладать определенной симметрией по отношению к пере-становкам спиновых координат, и если да, то какой?
\begin{choices}
\choice да, симметрична         
\choice нет
\choice да, антисимметрична     
\choice мало информации для ответа
\end{choices}

\question Имеется два фермиона со спином $s = 1/2$. Фермионы находятся в состоянии, в котором проекция суммарного спина на ось $z$ имеет определенное значение ${S_z} = 0$. Будет ли спино-вая волновая функция этого состояния обладать определенной симметрией по отношению к пере-становкам спиновых координат, и если да, то какой?
\begin{choices}
\choice да, симметрична         
\choice нет
\choice да, антисимметрична     
\choice мало информации для ответа
\end{choices}

\question Имеется два фермиона со спином $s = 1/2$. Фермионы находятся в состоянии, в котором суммарный спин имеет определенное значение $S = 1$, а при измерении проекции суммарного спина на ось $z$ с определенными вероятностями могут быть получены два значения. Будет ли спиновая волновая функция этого состояния обладать определенной симметрией по отношению к перестановкам спиновых координат, и если да, то какой?
\begin{choices}
\choice да, симметрична         
\choice нет
\choice да, антисимметрична     
\choice мало информации для ответа
\end{choices}

\question Два тождественных невзаимодействующих фермиона со спином $s = 1/2$ находятся в по-тенциале одномерного гармонического осциллятора. Будет ли суммарный спин системы иметь определенное значение в основном состоянии, и если да, то какое? 
\begin{choices}
\choice да, $S = 1$     
\choice да, $S = 0$     
\choice нет      
\choice мало информации для ответа
\end{choices}

\question Два тождественных невзаимодействующих бозона со спином $s = 1$ находятся в потенциале одномерного гармонического осциллятора. Будет ли суммарный спин системы иметь определенное значение в основном состоянии, и если да, то какое? 
\begin{choices}
\choice да, $S = 1$     
\choice да, $S = 0$     
\choice нет      
\choice да, $S = 2$
\end{choices}

\question Имеется два фермиона со спином $s = 99/2$. Что можно сказать о симметрии спиновой волновой функции состояния, в котором суммарный спин и его проекция на ось $z$ имеют опре-деленные значения $S = 99$, ${S_z} = 99$?
\begin{choices}
\choice эта функция симметрична относительно перестановок спиновых координат частиц
\choice эта функция антисимметрична относительно перестановок спиновых координат частиц
\choice эта не обладает определенной симметрией по отношению к перестановкам спиновых коорди-нат
\choice мало информации для ответа
\end{choices}

\question Имеется два фермиона со спином $s = 99/2$. Что можно сказать о симметрии спиновой волновой функции состояния, в котором суммарный спин и его проекция на ось $z$ имеют опре-деленные значения $S = 99$, ${S_z} = 98$?
\begin{choices}
\choice эта функция симметрична относительно перестановок спиновых координат частиц
\choice эта функция антисимметрична относительно перестановок спиновых координат частиц
\choice эта не обладает определенной симметрией по отношению к перестановкам спиновых коорди-нат
\choice мало информации для ответа
\end{choices}

\question Имеется два фермиона со спином $s = 99/2$. Что можно сказать о симметрии спиновой волновой функции состояния, в котором суммарный спин и его проекция на ось $z$ имеют опре-деленные значения $S = 98$, ${S_z} = 98$?
\begin{choices}
\choice эта функция симметрична относительно перестановок спиновых координат частиц
\choice эта функция антисимметрична относительно перестановок спиновых координат частиц
\choice эта не обладает определенной симметрией по отношению к перестановкам спиновых коорди-нат
\choice мало информации для ответа
\end{choices}

\question Имеется два фермиона со спином $s = 99/2$. Что можно сказать о симметрии спиновой волновой функции состояния, в котором суммарный спин имеет определенное значение $S = 95$, а его проекция на ось $z$ определенного значения не имеет? 
\begin{choices}
\choice эта функция симметрична относительно перестановок спиновых координат частиц
\choice эта функция антисимметрична относительно перестановок спиновых координат частиц
\choice эта не обладает определенной симметрией по отношению к перестановкам спиновых коорди-нат
\choice мало информации для ответа
\end{choices}

\question Имеется два фермиона со спином $s = 99/2$. Что можно сказать о симметрии спиновой волновой функции состояния, в котором проекция суммарного спина на ось $z$ имеет определен-ное значение ${S_z} = 95$, а сам суммарный спин определенного значения не имеет? 
\begin{choices}
\choice эта функция симметрична относительно перестановок спиновых координат частиц
\choice эта функция антисимметрична относительно перестановок спиновых координат частиц
\choice эта не обладает определенной симметрией по отношению к перестановкам спиновых коорди-нат
\choice мало информации для ответа
\end{choices}

\question Имеется два бозона со спином $s$. Что можно сказать о спиновой волновой функции со-стояния, в котором суммарный спин имеет определенное значение $S = 2s$?
\begin{choices}
\choice эта функция симметрична относительно перестановок спиновых координат частиц, так как частицы – бозоны
\choice эта не обладает определенной симметрией по отношению к перестановкам спиновых координат 
\choice эта функция антисимметрична относительно перестановок координат частиц по построению независимо от того, являются частицы тождественными, или нет
\choice эта функция симметрична относительно перестановок координат частиц по построению независимо от того, являются частицы тождественными, или нет
\end{choices}

\question Имеется два бозона со спином $s$. Бозоны находятся в состоянии, в котором суммарный спин имеет определенное значение $S = 2s - 1$. Будет ли спиновая волновая функция этого со-стояния обладать определенной симметрией по отношению к перестановкам спиновых коорди-нат, и если да, то какой?
\begin{choices}
\choice да, симметрична         
\choice нет
\choice да, антисимметрична     
\choice мало информации для ответа
\end{choices}

\question Два тождественных фермиона со спином $s = 3/2$ находятся в состоянии с определенным суммарным спином $S$. Будет ли пространственная часть волновой функции системы обладать определенной симметрией по отношению к перестановкам?
\begin{choices}
\choice да             
\choice нет
\choice будет, если $S = 3$ или $S = 1$  
\choice будет, если $S = 2$ или $S = 0$
\end{choices}

\question Система из двух тождественных фермионов со спином $s = 1/2$ находится в состоянии, в котором суммарный спин не имеет определенного значения. Будет ли пространственная часть волновой функции системы обладать определенной симметрией по отношению к перестановкам, и если да, то какой?
\begin{choices}
\choice да, симметрична      
\choice да, антисимметрична
\choice нет            
\choice мало информации для ответа
\end{choices}

\question Система из двух тождественных невзаимодействующих фермионов со спином $s = 1/2$ находится в состоянии, в котором суммарный спин имеет определенное значение $S = 1$. Каков характер обменного взаимодействия между частицами в этом состоянии?
\begin{choices}
\choice притяжение
\choice отталкивание
\choice взаимодействия между частицами по условию нет
\choice мало информации для ответа
\end{choices}

\question Система из двух тождественных невзаимодействующих фермионов со спином $s = 1/2$ находится в состоянии, в котором суммарный спин имеет определенное значение $S = 0$. Каков характер обменного взаимодействия между частицами в этом состоянии?
\begin{choices}
\choice притяжение
\choice отталкивание
\choice взаимодействия между частицами по условию нет
\choice мало информации для ответа
\end{choices}

\question Система из двух тождественных фермионов со спином $s = 1/2$ находится в состоянии с пространственной волновой функцией $\Psi ({\vec r_1},{\vec r_2}) \sim {f_1}({\vec r_1}){f_2}({\vec r_2}) + {f_2}({\vec r_1}){f_1}({\vec r_2})$. Имеет ли суммарный спин системы определенное зна-чение в этом состоянии, и если да, то какое?
\begin{choices}
\choice да, $S = 1$    
\choice нет      
\choice да, $S = 0$    
\choice мало информации для ответа
\end{choices}

\question Система из двух тождественных фермионов со спином $s = 1/2$ находится в состоянии с пространственной волновой функцией $\Psi ({\vec r_1},{\vec r_2}) \sim {f_1}({\vec r_1}){f_2}({\vec r_2}) - {f_2}({\vec r_1}){f_1}({\vec r_2})$. Имеет ли проекция суммарного спина системы на ось $z$ определенное значение в этом состоянии, и если да, то какое?
\begin{choices}
\choice да, ${S_z} = 1$      
\choice нет      
\choice да, ${S_z} = 0$      
\choice мало информации для от-вета
\end{choices}

\question Система из двух тождественных невзаимодействующих бозонов со спином $s = 1$ нахо-дится в состоянии, в котором пространственная часть волновой функции симметрична относи-тельно перестановок координат. Имеет ли суммарный спин такой системы определенное значе-ние, и если да, то какое? 
\begin{choices}
\choice да, $S = 2$    
\choice нет      
\choice да, $S = 1$    
\choice мало информации для ответа
\end{choices}

\question Система из двух тождественных невзаимодействующих бозонов со спином $s = 1$ нахо-дится в состоянии, в котором пространственная часть волновой функции антисимметрична отно-сительно перестановок координат. Имеет ли суммарный спин такой системы определенное значе-ние, и если да, то какое? 
\begin{choices}
\choice да, $S = 2$    
\choice нет      
\choice да, $S = 1$    
\choice мало информации для ответа
\end{choices}

\question Два тождественных невзаимодействующих бозона со спином $s = 2$ находятся в одинако-вых пространственных состояниях. Какие значения суммарного спина системы можно обнару-жить при измерениях?
\begin{choices}
\choice только $S = 0$, $S = 2$ и $S = 4$
\choice только $S = 4$
\choice только $S = 2$ 
\choice все возможные при сложении двух спинов $s = 2$ значения $S = 0$, $S = 1$, $S = 2$, $S = 3$ и $S = 4$
\end{choices}

\question Два тождественных невзаимодействующих фермиона со спином $s = 3/2$ находятся в оди-наковых пространственных состояниях. Какие значения суммарного спина системы можно об-наружить при измерениях?
\begin{choices}
\choice только $S = 3$ и $S = 1$
\choice только $S = 2$
\choice только $S = 2$ и $S = 0$
\choice все возможные при сложении двух спинов $s = 3/2$ значения $S = 0$, $S = 1$, $S = 2$, $S = 3$
\end{choices}

\question Известно, что волновая функция системы двух тождественных частиц, описывающая со-стояние с определенным суммарным спином $S = 0$, является симметричной относительно пере-становок пространственных координат частиц. Какие это частицы?
\begin{choices}
\choice фермионы
\choice бозоны
\choice информации недостаточно
\choice это зависит от пространственного состояния частиц
\end{choices}

\question Известно, что волновая функция системы двух тождественных частиц, описывающая со-стояние с определенным суммарным спином $S = 0$, является антисимметричной относительно перестановок пространственных координат частиц. Какие это частицы?
\begin{choices}
\choice фермионы
\choice бозоны
\choice информации недостаточно
\choice это зависит от пространственного состояния частиц
\end{choices} 

\end{questions}




\section{ Метод вторичного квантования }

\begin{questions}

\question Метод вторичного квантования дает возможность
\begin{choices}
\choice проквантовать пространство («первичное») и время («вторичное» квантование)
\choice находить собственные значения и собственные функции гамильтониана системы тождествен-ных частиц
\choice вычислять матричные элементы операторов с волновыми функциями систем тождественных частиц
\choice находить решения временного уравнения Шредингера для систем тождественных частиц
\end{choices}

\question Что такое числа заполнения одночастичных состояний?
\begin{choices}
\choice это квантовые числа этих состояний
\choice это количество «мест» для частиц в этих состояниях
\choice это число частиц, которые находятся в этих состояниях
\choice это доля заполнения этих «мест» в этих состояниях частицами
\end{choices}

\question Имеется система тождественных, невзаимодействующих частиц. Какие два из трех терми-нов, относящихся к данной системе – «волновая функция системы», «одночастичная волновая функция», «многочастичная волновая функция» - обозначают одно и то же?
\begin{choices}
\choice первый и второй      
\choice первый и третий
\choice второй и третий      
\choice все разные
\end{choices}

\question Система тождественных бозонов имеет волновую функцию: $\Psi  \sim {f_i}({\vec r_1}){f_k}({\vec r_2}) + {f_k}({\vec r_1}){f_i}({\vec r_2})$, где ${f_i}(\vec r)$ и ${f_k}(\vec r)$ - соб-ственные функции гамильтониана одной частицы (одночастичного гамильтониана), $i$ и $k$ - одночастичные квантовые числа. Какое утверждение относительно чисел заполнения одночастич-ных состояний в этом состоянии справедливо?
\begin{choices}
\choice числа заполнения состояний $k$ и $i$ имеют определенные значения, равные 1, остальных од-ночастичных состояний – определенных значений не имеют
\choice числа заполнения состояний $k$ и $i$ имеют определенные значения, равные 1/2, остальных одночастичных состояний – определенных значений не имеют
\choice числа заполнения состояний $k$ и $i$ не имеют определенных значений, остальных одночас-тичных состояний – определенные значения, равные 0
\choice числа заполнения состояний $k$ и $i$ имеют определенные значения, равные 1, остальных од-ночастичных состояний – определенные значения, равные 0
\end{choices}

\question Система тождественных невзаимодействующих частиц находится в таком состоянии, в ко-тором числа заполнения состояний одночастичного гамильтониана системы ${n_i}$ имеют опре-деленные значения. Будет ли это состояние собственным состоянием гамильтониана системы?
\begin{choices}
\choice да
\choice нет
\choice это зависит от состояния
\choice это зависит от значений чисел заполнения
\end{choices}

\question Система тождественных невзаимодействующих частиц находится в одном из собственных состояний гамильтониана системы. Будут ли в этом состоянии числа заполнения одночастичных состояний иметь определенные значения?
\begin{choices}
\choice да
\choice нет
\choice да, если отсутствует вырождение одночастичных состояний
\choice зависит от того, являются частицы фермионами или бозонами
\end{choices}

\question Система шести тождественных невзаимодействующих фермионов находится в состоянии, в котором числа заполнения состояний одночастичного гамильтониана имеют следующие значе-ния: ${n_i} = 3$, ${n_k} = 2$, ${n_m} = 1$. Будет ли это состояние собственным для оператора Гамильтона системы и если да, то какому собственному значению оно отвечает? Собственные значения ${\varepsilon _i}$ одночастичного гамильтониана известны.
\begin{choices}
\choice да, собственному значению $3{\varepsilon _i} + 2{\varepsilon _k} + {\varepsilon _m}$
\choice нет
\choice да, собственному значению $6{\varepsilon _i} + 4{\varepsilon _k} + 2{\varepsilon _m}$
\choice да, если состояния $i,\;k,\;m$ вырождены 
\end{choices}

\question Сравнить энергию основных состояний системы невзаимодействующих тождественных бо-зонов ${E_b}$ и системы невзаимодействующих тождественных фермионов ${E_f}$, если у этих систем одинаковые одночастичные гамильтонианы, а число частиц в системах одинаковое и большое
\begin{choices}
\choice ${E_b} > {E_f}$         
\choice ${E_b} < {E_f}$
\choice ${E_b} = {E_f}$         
\choice сравнить эти энергии невозможно
\end{choices}

\question Система тождественных частиц находится в состоянии, в котором числа заполнения одно-частичных состояний имеют определенные значения ${n_i}$. Какой смысл имеет сумма всех чи-сел заполнения $\sum\limits_i {{n_i}} $?
\begin{choices}
\choice эта сумма рана единице (условие нормировки волновой функции)   
\choice эта сумма равна числу частиц в системе
\choice эта сумма равна энергии системы
\choice эта сумма равна полной массе системы
\end{choices}

\question Система тождественных бозонов со спином $s = 0$ имеет волновую функцию:
$\Psi  \sim {f_i}({\vec r_1}){f_k}({\vec r_2}) + {f_k}({\vec r_1}){f_i}({\vec r_2}) + {f_i}({\vec r_1}){f_n}({\vec r_2}) + {f_n}({\vec r_1}){f_i}({\vec r_2})$, где $f(\vec r)$ - собственные функции одночастичного гамильтониана, $i,\;k,\;n$- одночастичные квантовые числа. Будут ли в этом со-стоянии числа заполнения одночастичных состояний $i,\;k,\;n$ иметь определенные значения?
\begin{choices}
\choice состояния $k$- да, состояний $i,\;n$ - нет      
\choice состояния $i$- да, состояний $k,\;n$ - нет
\choice состояния $n$- да, состояний $k,\;i$ - нет      
\choice да для всех состояний $i,\;k,\;n$
\end{choices}

\question Система тождественных невзаимодействующих бозонов со спином $s = 0$ имеет волно-вую функцию: $\Psi  \sim {f_i}({\vec r_1}){f_k}({\vec r_2}) + {f_k}({\vec r_1}){f_i}({\vec r_2}) + {f_i}({\vec r_1}){f_n}({\vec r_2}) + {f_n}({\vec r_1}){f_i}({\vec r_2})$, где $f(\vec r)$ - собственные функции одночастичного гамильтониана, $i,\;k,\;n$- одночастичные квантовые числа. Найти среднее значения числа заполнения одночастичного состояния $k$ в этом состоянии
\begin{choices}
\choice $\overline {{n_k}}  = 0$      
\choice $\overline {{n_k}}  = 1/3$    
\choice $\overline {{n_k}}  = 1/2$    
\choice $\overline {{n_k}}  = 1$
\end{choices}

\question Система тождественных невзаимодействующих бозонов со спином $s = 0$ имеет волно-вую функцию: $\Psi  \sim {f_i}({\vec r_1}){f_k}({\vec r_2}) + {f_k}({\vec r_1}){f_i}({\vec r_2}) + {f_i}({\vec r_1}){f_n}({\vec r_2}) + {f_n}({\vec r_1}){f_i}({\vec r_2})$, где $f(\vec r)$ - собственные функции одночастичного гамильтониана системы, $i,\;k,\;n$- одночастичные квантовые числа. Будет ли это состояние собственным состоянием гамильтониана системы?
\begin{choices}
\choice да
\choice нет
\choice да, если состояния $i$ и $n$ вырождены
\choice да, если состояния $k$ и $n$ вырождены 
\end{choices}

\question Система тождественных невзаимодействующих бозонов со спином $s = 0$ имеет волно-вую функцию: $\Psi  \sim {f_i}({\vec r_1}){f_k}({\vec r_2}) + {f_k}({\vec r_1}){f_i}({\vec r_2}) + {f_i}({\vec r_1}){f_n}({\vec r_2}) + {f_n}({\vec r_1}){f_i}({\vec r_2})$, где $f(\vec r)$ - собственные функции одночастичного гамильтониана, $i,\;k,\;n$- одночастичные квантовые числа. Чему равна средняя энергия системы в этом состоянии?
\begin{choices}
\choice $\overline E  = {\varepsilon _i} + {\varepsilon _k}/2 + {\varepsilon _n}/2$      
\choice $\overline E  = {\varepsilon _k} + {\varepsilon _i}/2 + {\varepsilon _n}/2$
\choice $\overline E  = {\varepsilon _n} + {\varepsilon _k}/2 + {\varepsilon _i}/2$      
\choice $\overline E  = ({\varepsilon _i} + {\varepsilon _k} + {\varepsilon _n})/3$
(здесь ${\varepsilon _j}$ - энергии одночастичных состояний).
\end{choices}

\question Пусть разложение волновой функции системы тождественных частиц $\Psi ({\vec r_1},{\vec r_2},...)$ по состояниям, в которых числа заполнения имеют определенные значения ${\Psi _{{n_1},{n_2},...}}({\vec r_1},{\vec r_2},...)$, где ${n_1},{n_2},...$ - числа заполнения одночастичных состояний 1, 2,…  имеет вид
$\Psi ({\vec r_1},{\vec r_2},...) = \sum\limits_{{n_1},{n_2},..} {{C_{{n_1},{n_2},..}}{\Psi _{{n_1},{n_2},...}}({{\vec r}_1},{{\vec r}_2},...)} $,
где ${C_{{n_1},{n_2},..}}$ - коэффициенты разложения. Какая величина в этом равенстве пред-ставляет собой волновую функцию рассматриваемого состояния в представлении чисел заполнения?
\begin{choices}
\choice $\Psi ({\vec r_1},{\vec r_2},...)$     
\choice ${\Psi _{{n_1},{n_2},...}}({\vec r_1},{\vec r_2},...)$      
\choice ${C_{{n_1},{n_2},...}}$ 
\choice ни одна из этих величин
\end{choices}

\question Аргументами волновой функции системы тождественных частиц в представлении чисел заполнения являются
\begin{choices}
\choice координаты частиц
\choice числа заполнения одночастичных состояний 
\choice импульсы всех частиц
\choice у волновой функции в представлении чисел заполнения нет аргументов
\end{choices}

\question Значениями волновой функции системы тождественных частиц в представлении чисел заполнения являются
\begin{choices}
\choice вероятности различных значений координат частиц
\choice возможные значения координат частиц
\choice возможные значения чисел заполнения одночастичных состояний
\choice вероятности различных значений чисел заполнения одночастичных состояний 
\end{choices}

\question Система тождественных фермионов имеет волновую функцию:
$\Psi  = \frac{{{C_1}}}{{\sqrt 2 }}\left| {\begin{array}{*{20}{c}}
{{f_i}({x_1})}&{{f_k}({x_1})}\\
{{f_i}({x_2})}&{{f_k}({x_2})}
\end{array}} \right| + \frac{{{C_2}}}{{\sqrt 2 }}\left| {\begin{array}{*{20}{c}}
{{f_i}({x_1})}&{{f_n}({x_1})}\\
{{f_i}({x_2})}&{{f_n}({x_2})}
\end{array}} \right|$, 
где $f(x)$ - собственные функции одночастичного гамильтониана,$i,\;k,\;n$- одночастичные кван-товые числа, ${x_1}$ и ${x_2}$ включают в себя как пространственные, так и спиновые перемен-ные, ${C_1}$ и ${C_2}$ числа. Какое из нижеперечисленных равенств относительно вероятно-стей различных значений чисел заполнения одночастичного состояния $i$ является верным?
\begin{choices}
\choice $w({n_i} = 1) = 1/2$ 
\choice $w({n_i} = 0) = {\left| {{C_2}} \right|^2}$  
\choice $w({n_i} = 2) = 0$   
\choice $w({n_i} = 1) = {\left| {{C_1}} \right|^2}$
\end{choices}

\question Система тождественных фермионов имеет волновую функцию:
$\Psi  = \frac{{{C_1}}}{2}\left| {\begin{array}{*{20}{c}}
{{f_i}({x_1})}&{{f_k}({x_1})}\\
{{f_i}({x_2})}&{{f_k}({x_2})}
\end{array}} \right| + \frac{{{C_2}}}{2}\left| {\begin{array}{*{20}{c}}
{{f_i}({x_1})}&{{f_n}({x_1})}\\
{{f_i}({x_2})}&{{f_n}({x_2})}
\end{array}} \right|$, 
где $f(x)$ - собственные функции одночастичного гамильтониана,$i,\;k,\;n$- одночастичные кван-товые числа, ${x_1}$ и ${x_2}$ включают в себя как пространственные, так и спиновые перемен-ные, ${C_1}$ и ${C_2}$ числа. Какое из нижеперечисленных равенств относительно вероятно-стей различных значений чисел заполнения одночастичного состояния $n$ является верным?
\begin{choices}
\choice $w({n_n} = 1) = 1$   
\choice $w({n_n} = 0) = {\left| {{C_1}} \right|^2}$  
\choice $w({n_n} = 1) = 1/2$ 
\choice $w({n_n} = 1) = {\left| {{C_1}} \right|^2}$
\end{choices}

\question Система тождественных фермионов имеет волновую функцию:
$\Psi  = \frac{{{C_1}}}{2}\left| {\begin{array}{*{20}{c}}
{{f_i}({x_1})}&{{f_k}({x_1})}\\
{{f_i}({x_2})}&{{f_k}({x_2})}
\end{array}} \right| + \frac{{{C_2}}}{2}\left| {\begin{array}{*{20}{c}}
{{f_i}({x_1})}&{{f_n}({x_1})}\\
{{f_i}({x_2})}&{{f_n}({x_2})}
\end{array}} \right|$, 
где $f(x)$ - собственные функции одночастичного гамильтониана,$i,\;k,\;n$- одночастичные кван-товые числа, ${x_1}$ и ${x_2}$ включают в себя как пространственные, так и спиновые перемен-ные, ${C_1}$ и ${C_2}$ числа. Какое из нижеперечисленных равенств относительно вероятно-стей различных значений чисел заполнения одночастичного состояния $k$ является верным?
\begin{choices}
\choice $w({n_k} = 1) = 1 - |{C_1}{|^2}$ 
\choice $w({n_k} = 0) = 0$   
\choice $w({n_k} = 1) = 1$   
\choice $w({n_k} = 1) = {\left| {{C_1}} \right|^2}$
\end{choices}

\question Система тождественных фермионов имеет волновую функцию:
$\Psi  = \frac{{{C_1}}}{2}\left| {\begin{array}{*{20}{c}}
{{f_i}({x_1})}&{{f_k}({x_1})}\\
{{f_i}({x_2})}&{{f_k}({x_2})}
\end{array}} \right| + \frac{{{C_2}}}{2}\left| {\begin{array}{*{20}{c}}
{{f_i}({x_1})}&{{f_n}({x_1})}\\
{{f_i}({x_2})}&{{f_n}({x_2})}
\end{array}} \right|$, 
где $f(x)$ - собственные функции одночастичного гамильтониана,$i,\;k,\;n$- одночастичные кван-товые числа, ${x_1}$ и ${x_2}$ включают в себя как пространственные, так и спиновые перемен-ные, ${C_1}$ и ${C_2}$ числа. Какое из нижеперечисленных равенств относительно вероятно-стей различных значений чисел заполнения одночастичных состояний является неверным?
\begin{choices}
\choice $w({n_k} = 1) = 0$   
\choice $w({n_n} = 1) = {\left| {{C_2}} \right|^2}$  
\choice $w({n_i} = 1) = 1$   
\choice $w({n_k} = 0) = 1 - |{C_1}{|^2}$
\end{choices}

\question Система тождественных фермионов имеет волновую функцию:
$\Psi  = \frac{{{C_1}}}{2}\left| {\begin{array}{*{20}{c}}
{{f_i}({x_1})}&{{f_k}({x_1})}\\
{{f_i}({x_2})}&{{f_k}({x_2})}
\end{array}} \right| + \frac{{{C_2}}}{2}\left| {\begin{array}{*{20}{c}}
{{f_i}({x_1})}&{{f_n}({x_1})}\\
{{f_i}({x_2})}&{{f_n}({x_2})}
\end{array}} \right|$, 
где $f(x)$ - собственные функции одночастичного гамильтониана, $i,\;k,\;n$- одночастичные квантовые числа, ${x_1}$ и ${x_2}$ включают в себя как пространственные, так и спиновые пере-менные, ${C_1}$ и ${C_2}$ числа. Среднее значение числа заполнения одночастичного состояния $n$ равно
\begin{choices}
\choice $\overline {{n_n}}  = 1$   
\choice $\overline {{n_n}}  = 0$   
\choice $\overline {{n_n}}  = {\left| {{C_1}} \right|^2}$     
\choice $\overline {{n_n}}  = {\left| {{C_2}} \right|^2}$
\end{choices}

\question Как действует бозонный оператор рождения частицы в $i$-м одночастичном состоянии на волновые функции состояний с определенными значениями чисел заполнения ${\Psi _{{n_1},{n_2},...{n_i},...}}$?
\begin{choices}
\choice ${\hat a_i}^ + {\Psi _{{n_1},{n_2},...{n_i},...}} = \sqrt {{n_i}} \;{\Psi _{{n_1},{n_2},...{n_i} + 1,...}}$    
\choice ${\hat a_i}^ + {\Psi _{{n_1},{n_2},...{n_i},...}} = \sqrt {{n_i} + 1} \;{\Psi _{{n_1},{n_2},...{n_i} + 1,...}}$
\choice ${\hat a_i}^ + {\Psi _{{n_1},{n_2},...{n_i},...}} = \sqrt {{n_i}} \;{\Psi _{{n_1},{n_2},...{n_i} - 1,...}}$    
\choice ${\hat a_i}^ + {\Psi _{{n_1},{n_2},...{n_i},...}} = \sqrt {{n_i} + 1} \;{\Psi _{{n_1},{n_2},...{n_i} - 1,...}}$
\end{choices}

\question Как действует бозонный оператор уничтожения частицы в $i$-м одночастичном состоянии на волновые функции состояний с определенными значениями чисел заполнения ${\Psi _{{n_1},{n_2},...{n_i},...}}$?
\begin{choices}
\choice ${\hat a_i}{\Psi _{{n_1},{n_2},...{n_i},...}} = \sqrt {{n_i}} \;{\Psi _{{n_1},{n_2},...{n_i} + 1,...}}$     
\choice ${\hat a_i}{\Psi _{{n_1},{n_2},...{n_i},...}} = \sqrt {{n_i} + 1} \;{\Psi _{{n_1},{n_2},...{n_i} + 1,...}}$
\choice ${\hat a_i}{\Psi _{{n_1},{n_2},...{n_i},...}} = \sqrt {{n_i}} \;{\Psi _{{n_1},{n_2},...{n_i} - 1,...}}$     
\choice ${\hat a_i}{\Psi _{{n_1},{n_2},...{n_i},...}} = \sqrt {{n_i} + 1} \;{\Psi _{{n_1},{n_2},...{n_i} - 1,...}}$
\end{choices}

\question Оператор рождения частицы в $i$-м одночастичном состоянии является
\begin{choices}
\choice эрмитовым   
\choice унитарным      
\choice линейным    
\choice совпадающим с обратным 
\end{choices}

\question Какое из приведенных ниже перестановочных соотношений операторов рождения и унич-тожения для фермионов является верным?
\begin{choices}
\choice ${\hat a_i}^ + {\hat a_k} + {\hat a_k}{\hat a_i}^ +  = {\delta _{ik}}$        
\choice ${\hat a_i}^ + {\hat a_k} - {\hat a_k}{\hat a_i}^ +  = {\delta _{ik}}$
\choice ${\hat a_i}^ + {\hat a_k}^ +  + {\hat a_k}^ + {\hat a_i}^ +  = {\delta _{ik}}$      
\choice ${\hat a_i}{\hat a_k} - {\hat a_k}{\hat a_i} = {\delta _{ik}}$
\end{choices}

\question Каковы перестановочные соотношения операторов рождения и уничтожения для бозонов?
\begin{choices}
\choice ${\hat a_i}^ + {\hat a_k} + {\hat a_k}{\hat a_i}^ +  = {\delta _{ik}}$        
\choice ${\hat a_i}^ + {\hat a_k} - {\hat a_k}{\hat a_i}^ +  = {\delta _{ik}}$
\choice ${\hat a_i}^ + {\hat a_k}^ +  + {\hat a_k}^ + {\hat a_i}^ +  = {\delta _{ik}}$      
\choice ${\hat a_i}{\hat a_k} - {\hat a_k}{\hat a_i} = {\delta _{ik}}$
\end{choices}

\question Какое из нижеперечисленных операторных равенств, содержащее фермионные операторы рождения и уничтожения, является верным?
\begin{choices}
\choice ${\hat a_i}{\hat a_k} = \hat 0$  
\choice ${\hat a_i}{\hat a_i}^ +  = \hat 1$ 
\choice ${\hat a_i}^ + {\hat a_i}^ +  = \hat 1$   
\choice ${\hat a_i}{\hat a_i} = \hat 0$ 
(где $\hat 0$ - нулевой, а $\hat 1$ - единичный оператор)
\end{choices}

\question Какой формулой – А., Б., или – определяется оператор числа частиц в представлении чисел заполнения (индекс $i$ нумерует одночастичные состояния)? 
\begin{choices}
\choice $\hat N = \sum\limits_i {{{\hat a}_i}^ + } {\hat a_i}$   
\choice $\hat N = \sum\limits_i {{{\hat a}_i}{{\hat a}_i}^ + } $ 
\choice $\hat N = \sum\limits_i {{{\hat a}_i}^ + } {\hat a_{i + 1}}$   
\choice $\hat N = \sum\limits_i {{{\hat a}_{i + 1}}{{\hat a}_i}^ + } $
\end{choices}

\question Оператор числа заполнения одночастичного состояния ${\hat N_i}$ в представлении чисел заполнения равен
\begin{choices}
\choice ${\hat N_i} = {\hat a_i}^ + {\hat a_i}$      
\choice ${\hat N_i} = {\hat a_i}{\hat a_i}^ + $      
\choice ${\hat N_i} = {\hat a_i}^ + {\hat a_i} - 1$  
\choice ${\hat N_i} = {\hat a_i}{\hat a_i}^ +  + 1$
\end{choices}

\question Какими являются быть собственные значения фермионного оператора ${\hat N_i} = {\hat a_i}^ + {\hat a_i}$?
\begin{choices}
\choice любыми целыми числами
\choice 1 и 2
\choice 0 и 1
\choice 0, 1 и 2
\end{choices}

\question Оператор ${\hat a_i}^ + {\hat a_i}$ является
\begin{choices}
\choice эрмитовым   
\choice унитарным      
\choice нелинейным  
\choice совпадающим с обратным 
\end{choices}

\question Система тождественных невзаимодействующих бозонов со спином $s = 0$ имеет волно-вую функцию: $\Psi  \sim {f_i}({\vec r_1}){f_k}({\vec r_2}) + {f_k}({\vec r_1}){f_i}({\vec r_2}) + {f_i}({\vec r_1}){f_n}({\vec r_2}) + {f_n}({\vec r_1}){f_i}({\vec r_2})$, где $f(\vec r)$ - собственные функции одночастичного гамильтониана, $i,\;k,\;n$- одночастичные квантовые числа. Будет ли эта функция собственной функцией оператора числа заполнения одночастичного состояния ${\hat N_i}$, и если да, то какому собственному значению она будет отвечать?
\begin{choices}
\choice да, ${n_i} = 1$      
\choice да, ${n_i} = 2$      
\choice да, ${n_i} = 3$      
\choice нет
\end{choices}

\question Система тождественных невзаимодействующих бозонов со спином $s = 0$ имеет волно-вую функцию: $\Psi  \sim {f_i}({\vec r_1}){f_k}({\vec r_2}) + {f_k}({\vec r_1}){f_i}({\vec r_2}) + {f_i}({\vec r_1}){f_n}({\vec r_2}) + {f_n}({\vec r_1}){f_i}({\vec r_2})$, где $f(\vec r)$ - собственные функции одночастичного гамильтониана, $i,\;k,\;n$- одночастичные квантовые числа. Будет ли эта функция собственной функцией оператора числа заполнения одночастичного состояния ${\hat N_k}$, и если да, то какому собственному значению она будет отвечать?
\begin{choices}
\choice да, ${n_k} = 1$      
\choice да, ${n_k} = 2$      
\choice да, ${n_k} = 3$      
\choice нет
\end{choices}

\question Система тождественных невзаимодействующих бозонов со спином $s = 0$ имеет волно-вую функцию: $\Psi  \sim {f_i}({\vec r_1}){f_k}({\vec r_2}) + {f_k}({\vec r_1}){f_i}({\vec r_2}) + {f_i}({\vec r_1}){f_n}({\vec r_2}) + {f_n}({\vec r_1}){f_i}({\vec r_2})$, где $f(\vec r)$ - собственные функции одночастичного гамильтониана, где $i,\;k,\;n$- одночастичные квантовые числа. Будет ли эта функция собственной функцией оператора числа частиц, и если да, то какому собственному значению она будет отвечать?
\begin{choices}
\choice да, $N = 1$    
\choice да, $N = 2$    
\choice да, $N = 3$    
\choice нет
\end{choices}

\question Система шести тождественных невзаимодействующих бозонов находится в состоянии, в котором числа заполнения состояний одночастичного гамильтониана имеют следующие значе-ния: ${n_1} = 3$, ${n_2} = 2$, ${n_3} = 1$ (квантовые числа одночастичных состояний $1,\;2,\;3$ включают в себя и спиновые квантовые числа). Будет ли это состояние собственным для оператора ${\hat a_1}^ + {\hat a_1}$ и если да, то какому собственному значению оно будет отвечать?
\begin{choices}
\choice да, собственному значению 3
\choice нет
\choice да, собственному значению 2
\choice да, собственному значению 1
\end{choices}

\question Система шести тождественных невзаимодействующих фермионов находится в состоянии, в котором числа заполнения состояний одночастичного гамильтониана имеют следующие значе-ния: ${n_1} = 3$, ${n_2} = 2$, ${n_3} = 1$ (квантовые числа одночастичных состояний $1,\;2,\;3$ включают в себя и спиновые квантовые числа). Будет ли это состояние собственным для оператора ${\hat a_2}^ + {\hat a_2}$ и если да, то какому собственному значению оно будет отвечать?
\begin{choices}
\choice да, собственному значению 3
\choice нет
\choice да, собственному значению 2
\choice да, собственному значению 1
\end{choices}

\question Система шести тождественных невзаимодействующих фермионов находится в состоянии, в котором фермионы заполняют шесть состояний одночастичного гамильтониана с квантовыми числами $1,\;2,\;3,\;4,\;5,\;6$. Будет ли это состояние собственным для оператора ${\hat a_6}^ + {\hat a_6}$ и если да, то какому собственному значению оно будет отвечать?
\begin{choices}
\choice да, собственному значению 6
\choice да, собственному значению 0
\choice да, собственному значению 1
\choice такого состояния быть не может
\end{choices}

\question Система шести тождественных невзаимодействующих фермионов находится в состоянии, в котором фермионы заполняют шесть состояний одночастичного гамильтониана с квантовыми числами $1,\;2,\;3,\;4,\;5,\;6$. Будет ли это состояние собственным для оператора ${\hat a_7}^ + {\hat a_7}$ и если да, то какому собственному значению оно будет отвечать?
\begin{choices}
\choice да, собственному значению 7
\choice да, собственному значению 0
\choice да, собственному значению 1
\choice такого состояния быть не может
\end{choices}

\question Определим вакуумное состояние (или вакуум) системы частиц как состояние, в котором частицы отсутствуют. Какой будет волновая функция вакуума (обозначается как $\left| 0 \right\rangle $) в представлении чисел заполнения?
\begin{choices}
\choice такую функцию определить невозможно: так как частиц нет, нет и аргументов у функции
\choice ее значение равно 1 при всех аргументах, равных нулю, и равно 0 для остальных аргументов
\choice ее значение равно 1 при всех аргументах, равных единице, и равно 0 для остальных аргумен-тов
\choice ее значение равно 0 при всех аргументах, равных единице, и равно 1 для остальных аргументов
\end{choices}

\question Собственной функцией каких из нижеперечисленных операторов является вакуумное со-стояние системы частиц $\left| 0 \right\rangle $ и каким собственным значениям оно отвечает?
\begin{choices}
\choice ${\hat a_i}^ + {\hat a_i}$, для всех значений индекса $i$, собственные значения равны 0
\choice ${\hat a_i}^ + {\hat a_i}$, для всех значений индекса $i$, собственные значения равны 1
\choice $\sum\limits_i {{{\hat a}_i}^ + {{\hat a}_i}} $, собственное значение равно 1
\choice ни для каких из перечисленных
(здесь $i$ - квантовые числа одночастичных состояний)
\end{choices}

\question Рассмотрим состояние $\psi  = \hat a_k^ + \left| 0 \right\rangle $, где $\left| 0 \right\rangle $ - вакуум системы тождественных невзаимодействующих частиц, $k$ - одночастичные квантовые числа (квантовые числа собственного состояния одночастичного гамильтониана). Является ли это состояние собственным для оператора $\sum\limits_i {{{\hat a}_i}^ + {{\hat a}_i}} $, и если да, то какому собственному значению оно отвечает?
\begin{choices}
\choice да, собственное значение равно 0    
\choice да, собственное значение равно 1
\choice да, собственное значение равно 2    
\choice нет
\end{choices}

\question Рассмотрим состояние $\psi  = \hat a_k^ + \left| 0 \right\rangle $, где $\left| 0 \right\rangle $ - вакуум системы тождественных невзаимодействующих частиц, $k$ - квантовые числа одного из собственных состояний одночастичного гамильтониана. Будет ли это состояние собственным для операторов ${\hat a_i}^ + {\hat a_i}$ ($i$ - квантовые числа одночастичных состояний), и если да, то какому собственному значению оно отвечает?
\begin{choices}
\choice да, для всех $i$, собственные значения равны: 0 для $i = k$ и 1 для всех $i \ne k$
\choice да, для всех $i$, собственные значения равны: 1 для $i = k$ и 0 для всех $i \ne k$
\choice да, для $i = k$, собственное значение равно 1, для всех $i \ne k$ это состояние не собственное
\choice не является собственным ни для каких $i$
\end{choices}

\question Рассмотрим состояние $\psi  = \hat a_k^ + \left| 0 \right\rangle $, где $\left| 0 \right\rangle $ - вакуум системы тождественных невзаимодействующих частиц, $k$ - квантовые числа одного из собственных состояний одночастичного гамильтониана. Какой является волновая функция этого состояния в координатном представлении? Собственные функции одночастичного гамильто-ниана ${f_i}(x)$ известны.
\begin{choices}
\choice ${f_1}({x_1}){f_2}({x_2})...$ (произведение функций с квантовыми числами, не равными $k$)
\choice ${f_1}({x_1}){f_2}({x_2})...{f_k}({x_k})$ (произведение функций с квантовыми числами, мень-шими $k$)
\choice ${f_1}({x_1}) + {f_2}({x_2}) + ... + {f_k}({x_k})$ (сумма функций с квантовыми числами, мень-шими $k$)
\choice ${f_k}(x)$
\end{choices}

\question Рассмотрим состояние $\psi  \sim \hat a_k^ + \left| 0 \right\rangle  + \hat a_m^ + \left| 0 \right\rangle $, где $\left| 0 \right\rangle $ - вакуум системы тождественных невзаимодействующих частиц, $k$ и $m$- квантовые числа двух собственных состояний одночастичного гамильто-ниана ($k \ne m$). Является ли это состояние состоянием с определенным значением числа час-тиц в системе, и если да, то с каким?
\begin{choices}
\choice да, две частицы         
\choice да, три частицы
\choice да, одна частица        
\choice нет
\end{choices}

\question Рассмотрим состояние $\psi  \sim \hat a_k^ + \left| 0 \right\rangle  + \hat a_m^ + \left| 0 \right\rangle $, где $\left| 0 \right\rangle $ - вакуум системы тождественных невзаимодействующих частиц, $k$ и $m$- квантовые числа двух собственных состояний одночастичного гамильто-ниана ($k \ne m$). Является ли это состояние состоянием с определенными значениями чисел за-полнения всех одночастичных состояний, и если да, то с какими?
\begin{choices}
\choice да, все равны 1 
\choice да, все, кроме состояний $k$ и $m$, равны 0, числа заполнения состояний $k$ и $m$ равны 1
\choice да, все, кроме состояний $k$ и $m$, равны 0, числа заполнения состояний $k$ и $m$ равны 2
\choice нет
\end{choices}

\question Рассмотрим состояние $\psi  \sim \hat a_k^ + \hat a_m^ + \left| 0 \right\rangle $, где $\left| 0 \right\rangle $ - вакуум системы тождественных невзаимодействующих бозонов, $k$ и $m$- квантовые числа двух собственных состояний одночастичного гамильтониана. Является ли это состояние состоянием с определенным значением числа частиц в системе, и если да, то с каким?
\begin{choices}
\choice да, две частицы         
\choice да, три частицы
\choice да, одна частица        
\choice нет
\end{choices}

\question Рассмотрим состояние $\psi  = \alpha \hat a_k^ + \hat a_m^ + \left| 0 \right\rangle $, где $\left| 0 \right\rangle $ - вакуум системы тождественных невзаимодействующих частиц, $k$ и $m$- квантовые числа двух собственных состояний одночастичного гамильтониана ($k \ne m$). Чему равен множитель $\alpha $, если данная функция нормирована на единицу?
\begin{choices}
\choice $\alpha  = 1$            
\choice $\alpha  = \sqrt 2 $
\choice $\alpha  = 1/\sqrt 2 $            
\choice эту функцию нельзя нормировать
\end{choices}

\question Рассмотрим состояние $\psi  \sim \hat a_k^ + \hat a_m^ + \left| 0 \right\rangle $, где $\left| 0 \right\rangle $ - вакуум системы тождественных невзаимодействующих бозонов, $k$ и $m$- квантовые числа двух собственных состояний одночастичного гамильтониана ($k \ne m$). Явля-ется ли это состояние собственным для операторов ${\hat a_i}^ + {\hat a_i}$, и если да, то каким собственным значениям оно отвечает?
\begin{choices}
\choice да, для $i = k$ или $i = m$ собственные значения - 0, для $i \ne k,m$ собственные значения – 1
\choice да, для $i = k$ или $i = m$ собственные значения - 1, для $i \ne k,m$ собственные значения – 0
\choice да, для всех $i$ собственные значения – 1
\choice нет
\end{choices}

\question Рассмотрим состояние $\psi  \sim \hat a_k^ + \hat a_m^ + \left| 0 \right\rangle $, где $\left| 0 \right\rangle $ - вакуум системы тождественных невзаимодействующих бозонов, $k$ и $m$- квантовые числа двух собственных состояний одночастичного гамильтониана ($k \ne m$). Явля-ется ли это состояние собственным для оператора $\sum\limits_i {{{\hat a}_i}^ + {{\hat a}_i}} $, и если да, то какому собственному значению оно отвечает?
\begin{choices}
\choice да, собственное значение равно 0
\choice да, собственное значение равно 1
\choice да, собственное значение равно 2
\choice нет
\end{choices}

\question Рассмотрим состояние $\psi  \sim \hat a_k^ + \hat a_m^ + \left| 0 \right\rangle $, где $\left| 0 \right\rangle $ - вакуум системы тождественных невзаимодействующих бозонов, $k$ и $m$- квантовые числа двух собственных состояний одночастичного гамильтониана ($k \ne m$). Явля-ется ли это состояние состоянием с определенным значением числа частиц в системе, и если нет, то какие значения числа частиц можно обнаружить в системе и с какими вероятностями?
\begin{choices}
\choice да, 1             
\choice да, 3
\choice да, 2          
\choice нет
\end{choices}

\question Рассмотрим состояние $\psi  \sim \hat a_j^ + \left| 0 \right\rangle  + \hat a_k^ + \hat a_m^ + \left| 0 \right\rangle $, где $\left| 0 \right\rangle $ - вакуум системы тождественных невзаимодей-ствующих бозонов, $k$ и $m$- квантовые числа двух собственных состояний одночастичного гамильтониана ($k \ne m$). Является ли это состояние собственным для оператора $\sum\limits_i {{{\hat a}_i}^ + {{\hat a}_i}} $, и если да, то какому собственному значению оно отвечает
\begin{choices}
\choice да, 1             
\choice да, 3
\choice да, 2          
\choice нет
\end{choices}

\question Рассмотрим состояние $\psi  \sim \hat a_k^ + \hat a_m^ + \left| 0 \right\rangle $, где $\left| 0 \right\rangle $ - вакуум системы тождественных невзаимодействующих бозонов, $k$ и $m$- квантовые числа двух собственных состояний одночастичного гамильтониана ($k \ne m$). Какой является волновая функция этого состояния в координатном представлении? Собственные функ-ции одночастичного гамильтониана ${f_i}(x)$ известны.
\begin{choices}
\choice такого состояния не существует      
\choice ${f_k}({x_1}){f_m}({x_2}) - {f_k}({x_2}){f_m}({x_1})$
\choice ${f_k}({x_1}){f_m}({x_2}) + {f_k}({x_2}){f_m}({x_1})$     
\choice ${f_k}({x_1}){f_m}({x_2})$
\end{choices}

\question Рассмотрим состояние $\psi  \sim \hat a_k^ + \hat a_m^ + \left| 0 \right\rangle $, где $\left| 0 \right\rangle $ - вакуум системы тождественных невзаимодействующих фермионов, $k$ и $m$- квантовые числа двух собственных состояний одночастичного гамильтониана ($k \ne m$). Какой является волновая функция этого состояния в координатном представлении? Собственные функ-ции одночастичного гамильтониана ${f_i}(x)$ известны.
\begin{choices}
\choice такого состояния не существует      
\choice ${f_k}({x_1}){f_m}({x_2}) - {f_k}({x_2}){f_m}({x_1})$
\choice ${f_k}({x_1}){f_m}({x_2}) + {f_k}({x_2}){f_m}({x_1})$     
\choice ${f_k}({x_1}){f_m}({x_2})$
\end{choices}

\question Рассмотрим состояние $\psi  \sim \hat a_k^ + \hat a_k^ + \left| 0 \right\rangle $, где $\left| 0 \right\rangle $ - вакуум системы тождественных невзаимодействующих фермионов, $k$ - кванто-вые числа собственного состояния одночастичного гамильтониана. Какой является волновая функция этого состояния в координатном представлении? Собственные функции одночастичного гамильтониана ${f_i}(x)$ известны.
\begin{choices}
\choice такого состояния не существует      
\choice ${f_k}({x_1}) - {f_k}({x_2})$
\choice ${f_k}({x_1}){f_k}({x_2}) + {f_k}({x_2}){f_k}({x_1})$     
\choice ${f_k}({x_1}){f_k}({x_2})$
\end{choices}

\question Рассмотрим состояние $\psi  = {C_1}\hat a_j^ + \left| 0 \right\rangle  + {C_2}(1/\sqrt 2 )\hat a_k^ + \hat a_m^ + \left| 0 \right\rangle $, где $\left| 0 \right\rangle $ - вакуум системы тождест-венных невзаимодействующих бозонов, $k$ и $m$- квантовые числа двух собственных состоя-ний одночастичного гамильтониана ($k \ne m$). Является ли это состояние состоянием с опреде-ленным числом частиц в системе, и если нет, то какое количество частиц и с какими вероятностя-ми может быть обнаружено в системе?
\begin{choices}
\choice да, в системе одна частица
\choice нет, с вероятностью ${C_1}$ в системе одна частица, с вероятностью ${C_2}$ - две
\choice нет, с вероятностью $|{C_1}{|^2}$ в системе одна частица, с вероятностью $|{C_2}{|^2}$ - две
\choice нет, с вероятностью $|{C_2}{|^2}$ в системе одна частица, с вероятностью $|{C_1}{|^2}$ - две
\end{choices}

\question Классическое выражение физической величины $A$ для системы тождественных частиц представляет собой сумму слагаемых, каждое из которых относится к одной частице $A = \sum\limits_a {A({{\vec r}_a})} $, где индекс $a$ нумерует частицы. Каким будет квантовомехани-ческий оператор физической величины $A$ в представлении чисел заполнения?
\begin{choices}
\choice $\hat A = \sum\limits_{i,k} {{A_{ik}}{{\hat a}_i}^ + {{\hat a}_k}} $ 
\choice $\hat A = \sum\limits_{i,k} {{A_{ik}}{{\hat a}_i}{{\hat a}_k}^ + } $ 
\choice $\hat A = \sum\limits_{i,k} {{A_{ki}}{{\hat a}_i}^ + {{\hat a}_k}} $ 
\choice $\hat A = \sum\limits_{i,k} {{A_{ki}}{{\hat a}_i}{{\hat a}_k}^ + } $,
(здесь индексы $i,k$ нумеруют собственные состояния одночастичного гамильтониана, ${A_{ik}}$ - матричные элементы оператора $\hat A({\vec r_a})$ с одночастичными волновыми функциями).
\end{choices}

\question Какой из нижеследующих формул определяется оператор Гамильтона системы тождест-венных невзаимодействующих частиц в представлении чисел заполнения?
\begin{choices}
\choice $\hat H = \sum\limits_i {{\varepsilon _i}^2{{\hat a}_i}^ + } {\hat a_i}$   
\choice $\hat H = \sum\limits_i {{\varepsilon _i}} {\hat a_i}{\hat a_i}^ + $ 
\choice $\hat H = \sum\limits_i {{\varepsilon _i}{{\hat a}_i}^ + } {\hat a_i}$  
\choice $\hat H = {\sum\limits_i {{\varepsilon _i}} ^2}{\hat a_i}{\hat a_i}^ + $
(здесь индекс $i$ нумерует собственные состояния одночастичного гамильтониана).
\end{choices}

\question Какой из нижеследующих формул определяется оператор Гамильтона системы тождест-венных невзаимодействующих частиц в представлении чисел заполнения?
\begin{choices}
\choice $\hat H = \sum\limits_{ik} {{\varepsilon _i}{\varepsilon _k}{{\hat a}_i}^ + } {\hat a_k}$ 
\choice $\hat H = \sum\limits_i {{\varepsilon _i}} {\hat a_i}^ + {\hat a_i}$ 
\choice $\hat H = \sum\limits_i {{\varepsilon _i}{{\hat a}_i}} {\hat a_i}^ + $  
\choice $\hat H = \sum\limits_{ik} {\sqrt {{\varepsilon _i}{\varepsilon _k}} {{\hat a}_i}^ + } {\hat a_k}$
(здесь индекс $i$ нумерует собственные состояния одночастичного гамильтониана).
\end{choices}

\question Оператор физической величины $A$, относящейся к системе тождественных невзаимодей-ствующих частиц, представляет собой сумму слагаемых $\hat A({x_i})$, каждое из которых дейст-вует на координаты одной частицы. Пусть матричные элементы оператора $\hat A({x_i})$ с собст-венными функциями одночастичного гамильтониана ${A_{mk}}$ известны. Найти среднее значе-ние величины $A$ в состоянии с определенными значениями чисел заполнения ${n_k}$
\begin{choices}
\choice $\overline A  = \sum\limits_{km} {{n_k}{n_m}{A_{km}}} $  
\choice $\overline A  = \sum\limits_{km} {\sqrt {{n_k}{n_m}} {A_{km}}} $  
\choice $\overline A  = \sum\limits_k {{n_k}{A_{kk}}} $ 
\choice $\overline A  = \sum\limits_k {{n_k}^2{A_{kk}}} $
\end{choices}

\question Система тождественных невзаимодействующих частиц находится в состоянии с опреде-ленными значениями чисел заполнения ${n_k}$. На систему действует зависящее от времени внешнее поле. В какие из нижеперечисленных состояний возможны переход системы? (ответ дать в первом порядке теории нестационарных возмущений)
\begin{choices}
\choice только в состояния с теми же самыми числами заполнения
\choice только в состояния с такими числами заполнения, одно из которых увеличилось на единицу, а второе на единицу уменьшилось
\choice только в состояния с такими числами заполнения, одно из которых увеличилось на два, а второе на два уменьшилось
\choice только в состояния с такими числами заполнения, два из которых увеличились на единицу, а два на единицу уменьшились
\end{choices}

\question Система тождественных невзаимодействующих частиц находится в состоянии с опреде-ленными значениями чисел заполнения ${n_k}$. На систему накладывают малое возмущение, оператор которого имеет вид: $\hat V = \sum\limits_a {\hat V({{\vec r}_a})} $, где индекс $a$ ну-мерует частицы. Какой формулой определяется поправка первого порядка к энергии рассматри-ваемого состояния?
\begin{choices}
\choice $\Delta {E^{(1)}} = \sum\limits_k {{n_k}{V_{kk}}} $   
\choice $\Delta {E^{(1)}} = \sum\limits_{ik} {{n_i}{n_k}{V_{ik}}} $ 
\choice $\Delta {E^{(1)}} = \sum\limits_{ik} {\sqrt {{n_i}{n_k}} {V_{ik}}} $ 
\choice $\Delta {E^{(1)}} = \sum\limits_k {{n_k}{V_{kk}}^2} $
\end{choices} 

\end{questions}


\section{ Задача рассеяния }

\subsection{ Задача рассеяния. Постановка и принципы решения }


\begin{questions}

\question Каков спектр собственных значений стационарного уравнения Шредингера для свободной частицы ($U(\vec r) = 0$)?
\begin{choices}
\choice любые положительные значения     
\choice любые отрицательные значения
\choice целые положительные значения        
\choice любые 
\end{choices}

\question Какова кратность вырождения собственных значений свободного трехмерного уравнения Шредингера?
\begin{choices}
\choice 1     
\choice 2     
\choice $2l + 1$    
\choice $\infty $
\end{choices}

\question Потенциальная энергия частицы равна нулю. Какая из нижеследующих функций является решением стационарного уравнения Шредингера при энергии $E$ ($k = \sqrt {2mE/{\hbar ^2}} $, $m$ - масса частицы)
\begin{choices}
\choice ${e^{i{k_1}x + i{k_2}y}}$, ${k_1} + {k_2} = k$     
\choice ${e^{i{k_1}x + i{k_2}z}}$, ${k_1}^2 + {k_2}^2 = {k^2}$
\choice ${e^{i{k_1}y + i{k_2}z}}$, ${k_1}^2 - {k_2}^2 = {k^2}$      
\choice никакая из пере-численных
\end{choices}

\question Потенциальная энергия частицы равна нулю. Какая из нижеследующих функций является решением стационарного уравнения Шредингера при энергии $E$ ($k = \sqrt {2mE/{\hbar ^2}} $, $m$ - масса частицы)
\begin{choices}
\choice ${e^{ikx}} + {e^{3iky}}$      
\choice ${e^{ikx}} - 2{e^{ikz}}$      
\choice ${e^{ky}} + {e^{ - ky}}$      
\choice никакая из перечисленных
\end{choices}

\question Потенциальная энергия частицы равна нулю. Каков физический смысл решения ${e^{ikx}} - 2{e^{ikz}}$ стационарного уравнения Шредингера при энергии $E$ ($k = \sqrt {2mE/{\hbar ^2}} $, $m$ - масса частицы)
\begin{choices}
\choice описывает два потока частиц, распространяющихся вдоль осей $x$ и $z$. Плотность второго потока вдвое больше плотности первого
\choice описывает два потока частиц, распространяющихся вдоль осей $x$ и $z$. Плотность второго потока вчетверо больше плотности первого
\choice описывает поток частиц, распространяющийся в направлении вектора $\vec i + 2\vec k$
\choice не является решением свободного стационарного уравнения Шредингера
\end{choices}

\question Потенциальная энергия частицы равна нулю. Каков физический смысл решения ${e^{i{k_1}x + i{k_2}y}}$, ${k_1}^2 + {k_2}^2 = {k^2}$ стационарного уравнения Шредингера при энергии $E$ ($k = \sqrt {2mE/{\hbar ^2}} $)?
\begin{choices}
\choice описывает два потока частиц, распространяющихся вдоль осей $x$ и $y$
\choice описывает поток частиц, распространяющихся в направлении вектора $\vec i{k_1} + \vec j{k_2}$
\choice описывает поток частиц, распространяющихся в направлении вектора $\vec i{k_1}^2 + \vec j{k_2}^2$   
\choice не является решением свободного стационарного уравнения Шредингера
\end{choices}

\question Потенциальная энергия частицы равна нулю. Какие из перечисленных функций будут ре-шениями стационарного уравнения Шредингера при энергии $E$ ($k = \sqrt {2mE/{\hbar ^2}} $, $m$ - масса частицы)?
\begin{choices}
\choice ${e^{ikr}}$    
\choice ${e^{ - ikr}}$    
\choice ${e^{ikr}}/r$, $r \ne 0$   
\choice ${e^{ - ikr}}/{r^2}$, $r \ne 0$
\end{choices}

\question Потенциальная энергия частицы равна нулю. Какое из перечисленных решений стационарного уравнения Шредингера является плоской волной?
\begin{choices}
\choice ${e^{i{k_1}z + i{k_2}x}}$  
\choice ${e^{ikr}}/r$  
\choice ${e^{ - ikz}} + {e^{ikx}}$    
\choice ${e^{ - ikr}}/r$ 
\end{choices}

\question Потенциальная энергия частицы равна нулю. Какое из перечисленных решений стационарного уравнения Шредингера является расходящейся сферической волной?
\begin{choices}
\choice ${e^{ikr}}$    
\choice ${e^{ikr}}/r$  
\choice ${e^{ - ikr}}$ 
\choice ${e^{ - ikr}}/r$ 
\end{choices}

\question Потенциальная энергия частицы равна нулю. Какое из перечисленных решений стацио-нарного уравнения Шредингера является сходящейся сферической волной?
\begin{choices}
\choice ${e^{ikr}}$    
\choice ${e^{ikr}}/r$  
\choice ${e^{ - ikr}}$ 
\choice ${e^{ - ikr}}/r$ 
\end{choices}

\question Потенциальная энергия частицы равна нулю. Какая из перечисленных функций является решением стационарного уравнения Шредингера, описывающим суперпозицию двух потоков частиц?
\begin{choices}
\choice ${e^{i{k_1}z + i{k_2}x}}$  
\choice ${e^{ikz}} + 2{e^{ - ikx}}$   
\choice ${e^{ - kz}} + {e^{kx}}$   
\choice ${e^{i{k_1}z}} + 2{e^{ - i{k_2}x}}$ (${k_1} \ne {k_2}$)
\end{choices}

\question Потенциальная энергия частицы равна нулю. Каков физический смысл решения ${e^{ikr}}/r$ стационарного уравнения Шредингера при энергии $E$ ($k = \sqrt {2mE/{\hbar ^2}} $)? 
\begin{choices}
\choice описывает изотропный сходящийся поток частиц с определенной энергией, распространяю-щихся из бесконечности в направлении начала координат
\choice описывает изотропный расходящийся поток частиц с определенной энергией, распространяю-щихся из начала координат в бесконечность
\choice описывает суперпозицию двух потоков частиц с определенной энергией, распространяющихся из бесконечности в направлении начала координат и из начала координат в бесконечность
\choice эта функция не является решением свободного стационарного уравнения Шредингера
\end{choices}

\question Потенциальная энергия частицы равна нулю. На $ - \infty $ по оси $z$ расположен источ-ник частиц, который излучает частицы с определенной энергией $E$ в положительном направле-нии оси $z$. Какой волновой функцией описывается поток этих частиц ($k = \sqrt {2mE/{\hbar ^2}} $)?
\begin{choices}
\choice ${e^{ - ikz}}$ 
\choice ${e^{ikz}} + {e^{ - ikz}}$ 
\choice ${e^{ikz}}$    
\choice ни одной из пере-численных
\end{choices}

\question Потенциальная энергия частицы равна нулю. Какая из нижеперечисленных функций бу-дет точным решением стационарного уравнения Шредингера при любых $r \ne 0$ ($k = \sqrt {2mE/{\hbar ^2}} $)?
\begin{choices}
\choice $\sin \vartheta {e^{ikr}}/r$        
\choice $\cos \vartheta \sin \varphi {e^{ - ikr}}/r$
\choice ${e^{i\varphi }}{e^{ikr}}/r$        
\choice никакая из перечисленных
\end{choices}

\question Потенциальная энергия частицы равна нулю. Какие из нижеперечисленных функций бу-дут приближенными решениями стационарного уравнения Шредингера при $r \to \infty $?
\begin{choices}
\choice $\sin \vartheta r{e^{ikr}}$         
\choice $\cos \vartheta \sin \varphi \sin kr/r$   
\choice ${e^{i\varphi }}{e^{kr}}/r$         
\choice никакая из перечисленных
\end{choices}

\question Потенциальная энергия частицы равна нулю. Как направлен вектор потока частиц на больших расстояниях от начала координат, если состояние частиц описывается следующим при-ближенным (при $r \to \infty $) решением уравнения Шредингера $\psi (r,\vartheta ,\varphi ) = \cos \vartheta \sin \varphi {e^{ikr}}/r$? 
\begin{choices}
\choice в каждой точке вдоль единичного вектора ${\vec e_\varphi }$
\choice в каждой точке вдоль единичного вектора ${\vec e_\vartheta }$
\choice в каждой точке вдоль единичного вектора ${\vec e_r}$
\choice вдоль некоторой линейной комбинации этих единичных векторов
\end{choices}

\question Потенциальная энергия частицы не равна нулю, но обращается в нуль при $r \to \infty $. Будет ли функция ${e^{ikx}}$ решением стационарного уравнения Шредингера при энергии $E$ ($k = \sqrt {2mE/{\hbar ^2}} $)?
\begin{choices}
\choice да             
\choice нет, даже приближенно
\choice приближенно при $r \to \infty $     
\choice приближенно при $r \to 0$
\end{choices}

\question Потенциальная энергия частицы не равна нулю, но обращается в нуль при $r \to \infty $. Существует ли такое решение стационарного уравнения Шредингера при энергии $E$ ($k = \sqrt {2mE/{\hbar ^2}} $), которое имеет асимптотику ${e^{iky}}$ при $r \to \infty $?
\begin{choices}
\choice да
\choice нет
\choice это зависит от потенциала
\choice мало информации для ответа
\end{choices}

\question Потенциальная энергия частицы не равна нулю, но обращается в нуль при $r \to \infty $. Что такое волновая функция задачи рассеяния, описывающая состояние с определенной энергией?
\begin{choices}
\choice произвольное решение временного уравнения Шредингера
\choice произвольное решение стационарного уравнения Шредингера
\choice решение стационарного уравнения Шредингера, имеющее асимптотику ${e^{ikr}}/r$ при $r \to \infty $
\choice решение стационарного уравнения Шредингера, имеющее асимптотику ${e^{ikz}} + f(\vartheta ,\varphi ){e^{ikr}}/r$ при $r \to \infty $
\end{choices}

\question Потенциальная энергия частицы не равна нулю, но обращается в нуль при $r \to \infty $. На $ - \infty $ по оси $z$ расположен источник частиц, который излучает частицы с определенной энергией $E$ в положительном направлении оси $z$. Какой волновой функцией описывается поток этих частиц в области $r \to \infty $?
\begin{choices}
\choice ${e^{ikz}}$    
\choice ${e^{ikz}} + f(\vartheta ){e^{ikr}}/r$    
\choice ${e^{ikz}} + f(\vartheta ){e^{ - ikr}}/r$    
\choice ${e^{ - ikz}} + f(\vartheta ){e^{ikr}}/r$
(здесь $k = \sqrt {2mE/{\hbar ^2}} $, $f(\vartheta )$ - некоторая функция $\vartheta $).
\end{choices}

\question Потенциальная энергия частицы не равна нулю, но обращается в нуль при $r \to \infty $. Какая из нижеперечисленных функций не описывает асимптотику (при $r \to \infty $) волновой функции задачи рассеяния частиц с определенной энергией $E$?
\begin{choices}
\choice ${e^{ikz}} + f(\vartheta ){e^{ - ikr}}/r$       
\choice ${e^{ikx}} + f(\vartheta ,\varphi ){e^{ikr}}/r$
\choice ${e^{i{k_1}z + i{k_2}y}} + f(\vartheta ,\varphi ){e^{ikr}}/r$     
\choice ${e^{ - ikz}} + f(\vartheta ){e^{ikr}}/r$
(здесь $k = \sqrt {2mE/{\hbar ^2}} $, ${k_1}^2 + {k_2}^2 = {k^2}$,$f$ - некоторая функция углов).
\end{choices}

\question Потенциальная энергия частицы не равна нулю, но обращается в нуль при $r \to \infty $. Каков физический смысл решения стационарного уравнения Шредингера, имеющего асимптоти-ческое поведение ${e^{ikz}} + f(\vartheta ){e^{i{k_1}r}}/r$ при $r \to \infty $?
\begin{choices}
\choice описывает падающие вдоль оси $z$ и рассеянные частицы
\choice описывает частицы, излученные источником при $r = 0$ и движущиеся вдоль оси $z$
\choice описывает частицы, движущиеся противоположно оси $z$ и рассеянные потенциалом
\choice не является решением
(здесь $k$ и ${k_1}$ - некоторые числа, $k \ne {k_1}$, $f(\vartheta )$ - некоторая функция $\vartheta $)?
\end{choices}

\question Потенциальная энергия частицы не равна нулю, но обращается в нуль при $r \to \infty $. Рассмотрим решение стационарного уравнения Шредингера, которое имеет асимптотику $A({e^{ikz}} + f(\vartheta ){e^{ikr}}/r)$ при $r \to \infty $ (здесь $A$ и $k$ - числа, $f(\vartheta )$ - некоторая функция $\vartheta $). Что в этом выражении есть амплитуда рассеяния?
\begin{choices}
\choice $A$      
\choice $k$      
\choice $f(\vartheta )$   
\choice ничего 
\end{choices}

\question Потенциальная энергия частицы не равна нулю, но обращается в нуль при $r \to \infty $. Рассмотрим решение стационарного уравнения Шредингера, которое имеет асимптотику $A{e^{ikz}} + B{e^{ikx}} + f(\vartheta ,\varphi ){e^{ikr}}/r$ при $r \to \infty $ (здесь $A$, $B$ и $k$ - числа, $f(\vartheta ,\varphi )$ - некоторая функция углов). Что в этом выражении есть амплитуда рассеяния?
\begin{choices}
\choice $A$      
\choice $B$      
\choice $f(\vartheta ,\varphi )$   
\choice ничего 
\end{choices}

\question Потенциальная энергия частицы не равна нулю, но обращается в нуль при $r \to \infty $. Сколько линейно независимых решений стационарного уравнения Шредингера имеют асимптотику ${e^{ikz}} + f(\vartheta ){e^{ikr}}/r$ ($f(\vartheta )$ - некоторая функция $\vartheta $)?
\begin{choices}
\choice одно  
\choice два      
\choice бесконечно много     
\choice это зависит от потенциала
\end{choices}

\question Потенциальная энергия частицы не равна нулю, но обращается в нуль при $r \to \infty $. Как волновая функция задачи рассеяния, которая имеет асимптотику $A{e^{ikz}} + f(\vartheta ){e^{ikr}}/r$ при $r \to \infty $ (здесь $A$ и $k$ - числа, $f(\vartheta )$ - некоторая функция $\vartheta $), ведет себя в области действия потенциала?
\begin{choices}
\choice как ${e^{ikz}}$
\choice как $f(\vartheta ){e^{ikr}}/r$
\choice как ${e^{ikz}} + f(\vartheta ){e^{ikr}}/r$
\choice по-другому
\end{choices}

\question Потенциальная энергия частицы не равна нулю, но обращается в нуль при $r \to \infty $. Некоторое решение уравнения Шредингера асимптотически при $r \to \infty $ ведет себя как  $f(\vartheta ,\varphi ){e^{ikr}}/r$, где $f(\vartheta ,\varphi )$ - некоторая функция углов. При на-хождении асимптотики плотности потока при $r \to \infty $ необходимо дифференцировать
\begin{choices}
\choice только функцию $f(\vartheta ,\varphi )$      
\choice только функцию ${e^{ikr}}$
\choice только функцию $1/r$    
\choice все функции
\end{choices}

\question Потенциальная энергия зависит только от модуля радиус-вектора. Частицы падают на по-тенциал вдоль оси $z$. От каких переменных зависит в этом случае амплитуда рассеяния?
\begin{choices}
\choice только от $\varphi $  
\choice только от $\vartheta $   
\choice только от $\varphi $ и $\vartheta $   
\choice от $\varphi $, $\vartheta $ и $r$
\end{choices}

\question Какова размерность амплитуды рассеяния?
\begin{choices}
\choice       
\choice       
\choice       
\choice 
\end{choices}

\question Как дифференциальное сечение рассеяния связано с потоками падающих  и рассеянных  частиц?
\begin{choices}
\choice       
\choice    
\choice    
\choice 
\end{choices}

\question Как плотность потока частиц в рассеянной сферической волне $f(\vartheta ){e^{ikr}}/r$ зависит от $r$ на больших расстояниях от рассеивающего центра?
\begin{choices}
\choice как $1/r$    
\choice как $r$      
\choice как $1/{r^2}$      
\choice как ${r^2}$
\end{choices}

\question Как дифференциальное сечение рассеяния выражается через амплитуду рассеяния $f(\vartheta )$?
\begin{choices}
\choice $\frac{{d\sigma }}{{d\Omega }} = \left| {f(\theta )} \right|$  
\choice $\frac{{d\sigma }}{{d\Omega }} = {\left| {f(\theta )} \right|^2}$    
\choice $\frac{{d\sigma }}{{d\Omega }} = 1 - {\left| {f(\theta )} \right|^2}$      
\choice $\frac{{d\sigma }}{{d\Omega }} = \frac{1}{{{{\left| {f(\theta )} \right|}^2}}}$
\end{choices}

\question Как полное сечение рассеяния выражается через амплитуду рассеяния $f(\vartheta )$?
\begin{choices}
\choice $\sigma  = 2\pi \int\limits_0^\pi  {{{\left| {f(\vartheta )} \right|}^2}} d\vartheta $       
\choice $\sigma  = \int\limits_0^\pi  {{{\left| {f(\vartheta )} \right|}^2}} \sin \vartheta d\vartheta $
\choice $\sigma  = \int\limits_0^\pi  {{{\left| {f(\vartheta )} \right|}^2}} d\vartheta $         
\choice $\sigma  = 2\pi \int\limits_0^\pi  {{{\left| {f(\vartheta )} \right|}^2}\sin \vartheta } d\vartheta $
\end{choices}

\question Частицы рассеиваются на потенциале, быстро спадающем с расстоянием. Как дифферен-циальное сечение рассеяния зависит от угла рассеяния?
\begin{choices}
\choice растет с ростом угла    
\choice убывает с ростом угла
\choice не зависит от угла рассеяния  
\choice бессмысленный вопрос
\end{choices}

\question Частицы рассеиваются на потенциале, быстро спадающем с расстоянием. Как полное се-чение рассеяния зависит от угла рассеяния?
\begin{choices}
\choice растет с ростом угла    
\choice убывает с ростом угла
\choice не зависит от угла рассеяния  
\choice бессмысленный вопрос
\end{choices}

\question В каких пределах изменяется аргумент $\vartheta $ у амплитуды рассеяния?
\begin{choices}
\choice $0 < \vartheta  < \infty $    
\choice $0 < \vartheta  < 2\pi $      
\choice $0 < \vartheta  < \pi $    
\choice $ - \infty  < \vartheta  <  + \infty $
\end{choices}

\question На рисунках приведены четыре зависимости амплитуды рассеяния от угла рассеяния $\vartheta $. Какой из этих графиков отвечает изотропному рассеянию?
\begin{choices}
\choice 1
\choice 2
\end{choices}

\question Какая формула является математическим выражением оптической теоремы?
\begin{choices}
\choice ${\mathop{\rm Im}\nolimits} f(0) = \frac{k}{{4\pi }}\sigma $   
\choice ${\mathop{\rm Re}\nolimits} f(0) = \frac{k}{{4\pi }}\sigma $   
\choice ${\mathop{\rm Im}\nolimits} f(\pi ) = \frac{k}{{4\pi }}\sigma $   
\choice ${\mathop{\rm Re}\nolimits} f(\pi ) = \frac{k}{{4\pi }}\sigma $
(здесь ${\mathop{\rm Im}\nolimits} ...$ - мнимая часть, ${\mathop{\rm Re}\nolimits} $… - дейст-вительная часть, $k$ - волновой вектор, $\sigma $ - полное сечение рассеяния)
\end{choices}

\question Частицы, падающие на потенциал вдоль оси $z$, рассеиваются некоторым потенциалом. Может ли амплитуда рассеяния быть действительной при $\vartheta  = 0$?
\begin{choices}
\choice да             
\choice нет
\choice обязательна действительна  
\choice это зависит от потенциала
\end{choices}

\question Частицы рассеиваются некоторым потенциалом. «Убыль» частиц  в налетающем потоке определяется
\begin{choices}
\choice ${\mathop{\rm Im}\nolimits} f(0)$      
\choice ${\mathop{\rm Re}\nolimits} f(0)$      
\choice $|f(0){|^2}$      
\choice отношением ${\mathop{\rm Im}\nolimits} f(0)/{\mathop{\rm Re}\nolimits} f(0)$
\end{choices}

\question Частицы рассеиваются некоторым потенциалом. «Убыль» частиц  в налетающем потоке связана с
\begin{choices}
\choice поглощением 
\choice рассеянием     
\choice затуханием  
\choice захватом частиц
\end{choices}

\question Частицы рассеиваются некоторым потенциалом. Что можно сказать о знаке мнимой части амплитуды рассеяния вперед ${\mathop{\rm Im}\nolimits} f(\vartheta  = 0)$?
\begin{choices}
\choice всегда «+»           
\choice всегда «-»
\choice зависит от потенциала      
\choice бессмысленный вопрос
\end{choices}

\question Частицы рассеиваются некоторым потенциалом. Что можно сказать о знаке действитель-ной части амплитуды рассеяния вперед ${\mathop{\rm Im}\nolimits} f(\vartheta  = 0)$?
\begin{choices}
\choice всегда «+»           
\choice всегда «-»
\choice зависит от потенциала      
\choice бессмысленный вопрос
\end{choices}

\question Оптическая теорема утверждает, что
\begin{choices}
\choice уменьшение количества частиц в падающем потоке равно количеству частиц, рассеянных назад
\choice увеличение количества частиц в падающем потоке равно количеству частиц, пришедших из других каналов
\choice уменьшение количества частиц в падающем потоке равно количеству рассеянных частиц
\choice движение частиц с большими волновых векторами можно рассматривать в рамках геометриче-ской оптики
\end{choices}

\question Какое из нижеследующих утверждений называется условием унитарности для рассеяния?
\begin{choices}
\choice энергия рассеянных частиц равна энергии падающих
\choice количество падающих частиц равно количеству рассеянных
\choice момент падающих частиц равен моменту рассеянных
\choice другое 
\end{choices}

\question Оптическая теорема есть следствие условия
\begin{choices}
\choice унитарности    
\choice эрмитовости    
\choice обратимости 
\choice других физических принципов
\end{choices}

\question Как зависит от угла рассеяния $\vartheta $ сечение рассеяния заряженной частицы на ку-лоновском потенциале (резерфордовское сечение)?
\begin{choices}
\choice как ${\sin ^2}(\vartheta /2)$ 
\choice как $\frac{1}{{{{\sin }^2}(\vartheta /2)}}$  
\choice как $\frac{1}{{{{\sin }^4}(\vartheta /2)}}$  
\choice как ${\sin ^4}(\vartheta /2)$
\end{choices}

\question Какая из нижеприведенных формул определяет амплитуду рассеяния частиц с энергией $E$ потенциалом $U(r)$?
\begin{choices}
\choice $f(\vartheta ,\varphi ) =  - \frac{m}{{2\pi {\hbar ^2}}}\int {U(r')\Psi (\vec r'){e^{ - i\vec k\vec r'}}d\vec r'} $      
\choice $f(\vartheta ,\varphi ) =  - \frac{m}{{2\pi {\hbar ^2}}}\int {U(r')\Psi (\vec r'){e^{ - ikr'}}d\vec r'} $
\choice $f(\vartheta ,\varphi ) =  - \frac{m}{{2\pi {\hbar ^2}}}\int {U(r'){e^{ikz'}}{e^{ - i\vec k\vec r'}}d\vec r'} $    
\choice никакая из них
(здесь $m$ - масса рассеивающихся частиц, $\Psi (\vec r)$ - волновая функция задачи рассеяния, $\vec k$ - волновой вектор рассеянных частиц, $k = \sqrt {2mE/{\hbar ^2}} $).
\end{choices}

\end{questions}

\subsection{ Борновское приближение и фазовая теория рассеяния }

\begin{questions}

\question Частицы рассеиваются на потенциале $U(\vec r)$. Какая из нижеследующих формул опре-деляет амплитуду рассеяния в борновском приближении?
\begin{choices}
\choice $f(\vartheta ) \sim \int {U(\vec r)d\vec r} $         
\choice $f(\vartheta ) \sim \int {U(\vec r){e^{ - i\vec k\vec r}}d\vec r} $
\choice $f(\vartheta ) \sim \int {U(\vec r){e^{ - i\vec k'\vec r}}d\vec r} $    
\choice $f(\vartheta ) \sim \int {U(\vec r){e^{ - i\vec q\vec r}}d\vec r} $
(здесь $\vec k$ - волновой вектор падающих частиц, $\vec k'$ - волновой вектор рассеянных час-тиц, $\vec q = \vec k' - \vec k$ - переданный импульс)
\end{choices}

\question Какая из нижеперечисленных формул для переданного импульса является правильной?
\begin{choices}
\choice $q = 2k\sin \vartheta /2$  
\choice $q = 2k\cos \vartheta /2$  
\choice $q = 2k\,{\rm{tg}}\vartheta /2$  
\choice $q = 2k\,{\rm{ctg}}\vartheta /2$
\end{choices}

\question Частицы рассеиваются на потенциале $U(\vec r)$. Через какую из нижеследующих вели-чин в формулу Борна для амплитуды рассеяния $f(\vartheta ) =  - (m/2\pi {\hbar ^2})\int {U(\vec r){e^{ - i\vec q\vec r}}d\vec r} $ ($\vec q$ - переданный импульс) входит угол рассеяния $\vartheta $?
\begin{choices}
\choice через общий множитель
\choice через $U(\vec r)$, так как $\vartheta $ - угол между $\vec r$ и осью $z$
\choice через $\vec q$, так как $\vartheta $ - угол между $\vec q$ и осью $z$
\choice через $\vec q$, так как $q = 2k\sin \vartheta /2$
\end{choices}

\question Частицы рассеиваются на потенциале $U(\vec r)$. Чтобы борновское приближение работа-ло, потенциал $U(\vec r)$ должен быть
\begin{choices}
\choice «большим»      
\choice «малым»     
\choice резким      
\choice осциллирующим
\end{choices}

\question Частицы рассеиваются на потенциале $U(\vec r)$, радиус действия которого равен $a$. Ка-кое условие для амплитуды рассеяния частиц на этом потенциале будет справедливым, если выполнены условия применимости борновского приближения?
\begin{choices}
\choice $f(\vartheta ) \gg a$
\choice $f(\vartheta ) \sim a$
\choice $f(\vartheta ) \ll a$
\choice амплитуда рассеяния и условия применимости борновского приближения не связаны
\end{choices}

\question Частицы рассеиваются на потенциале $U(\vec r)$, радиус действия которого равен $a$, ве-личина - ${U_0}$. При выполнении какого условия работает борновское приближение, если час-тицы медленные $ka \ll 1$ ($k = \sqrt {2mE/{\hbar ^2}} $)?
\begin{choices}
\choice ${U_0} \ll \frac{{{\hbar ^2}}}{{m{a^2}}}$    
\choice ${U_0} \gg \frac{{{\hbar ^2}}}{{m{a^2}}}$    
\choice ${U_0} \ll \frac{{{\hbar ^2}k}}{{ma}}$    
\choice ${U_0} \ll \frac{{{\hbar ^2}{k^2}}}{m}$
\end{choices}

\question Частицы рассеиваются на потенциале $U(\vec r)$, радиус действия которого равен $a$, ве-личина - ${U_0}$. При выполнении какого неравенства работает борновское приближение, если частицы быстрые $ka \gg 1$ ($k = \sqrt {2mE/{\hbar ^2}} $)?
\begin{choices}
\choice ${U_0} \ll \frac{{{\hbar ^2}}}{{m{a^2}}}$    
\choice ${U_0} \gg \frac{{{\hbar ^2}}}{{m{a^2}}}$    
\choice ${U_0} \ll \frac{{{\hbar ^2}k}}{{ma}}$    
\choice ${U_0} \ll \frac{{{\hbar ^2}{k^2}}}{m}$
\end{choices}

\question Частицы рассеиваются на потенциале $U(\vec r) = {U_0}{e^{ - \kappa r}}$, где ${U_0}$ и $\kappa $- числа. При выполнении какого условия работает борновское приближение, если час-тицы медленные $k/\kappa  \ll 1$ ($k = \sqrt {2mE/{\hbar ^2}} $)?
\begin{choices}
\choice ${U_0} \ll \frac{{{\hbar ^2}k\kappa }}{m}$   
\choice ${U_0} \gg \frac{{{\hbar ^2}{\kappa ^2}}}{m}$      
\choice ${U_0} \ll \frac{{{\hbar ^2}{\kappa ^2}}}{m}$      
\choice ${U_0} \ll \frac{{{\hbar ^2}{k^2}}}{m}$
\end{choices}

\question Частицы рассеиваются на потенциале $U(\vec r) = {U_0}{e^{ - \kappa r}}$, где ${U_0}$ и $\kappa $- числа. При выполнении какого условия работает борновское приближение, если час-тицы быстрые $k/\kappa  \gg 1$ ($k = \sqrt {2mE/{\hbar ^2}} $)?
\begin{choices}
\choice  ${U_0} \ll \frac{{{\hbar ^2}k\kappa }}{m}$  
\choice ${U_0} \gg \frac{{{\hbar ^2}{\kappa ^2}}}{m}$      
\choice ${U_0} \ll \frac{{{\hbar ^2}{\kappa ^2}}}{m}$      
\choice ${U_0} \ll \frac{{{\hbar ^2}{k^2}}}{m}$
\end{choices}

\question Частицы массой $m$ рассеиваются на потенциале $U(\vec r)$, радиус действия которого равен $a$, величина - ${U_0}$. Для параметров потенциала выполнено условие ${U_0} \gg \frac{{{\hbar ^2}}}{{m{a^2}}}$. Для каких энергий частиц работает борновкое приближение?
\begin{choices}
\choice ни для каких   
\choice для любых      
\choice $E \gg {U_0}$     
\choice $E \gg \frac{{m{a^2}{U_0}^2}}{{{\hbar ^2}}}$
\end{choices}

\question Частицы массой $m$ рассеиваются на потенциале $U(\vec r)$, радиус действия которого равен $a$, величина - ${U_0}$. Для параметров потенциала выполнено условие ${U_0} \ll \frac{{{\hbar ^2}}}{{m{a^2}}}$. Для каких энергий частиц работает борновкое приближение?
\begin{choices}
\choice ни для каких   
\choice для любых      
\choice $E \gg {U_0}$     
\choice $E \gg \frac{{{\hbar ^2}{U_0}^2}}{{m{a^2}}}$
\end{choices}

\question Частицы рассеиваются на некотором потенциале $U(\vec r)$, радиус действия которого равен $a$,. Для каких – быстрых или медленных – частиц лучше работает борновское приближе-ние?
\begin{choices}
\choice для медленных $ka \ll 1$
\choice для быстрых $ka \gg 1$
\choice одинаково
\choice это зависит от потенциала
\end{choices}

\question Частицы рассеиваются на потенциале $U(\vec r) = \alpha /{r^2}$. Каково условие приме-нимости борновского приближения для медленных частиц?
\begin{choices}
\choice $\alpha  \ll \frac{{{\hbar ^2}}}{m}$    
\choice $\alpha  \ll \frac{\hbar }{{{m^2}}}$    
\choice $\alpha  \ll \frac{{{\hbar ^3}}}{{{m^2}}}$    
\choice $\alpha  \ll \frac{{{\hbar ^2}}}{{{m^3}}}$
\end{choices}

\question Частицы рассеиваются на потенциале $U(\vec r) = \alpha /{r^2}$. Как полное сечение рас-сеяния зависит от энергии частиц
\begin{choices}
\choice не зависит     
\choice как $E$
\choice как ${E^2}$     
\choice полное сечение расходится из-за медленного спадания потенциала
\end{choices}

\question Частицы рассеиваются на некотором потенциале $U(\vec r)$, радиус действия которого равен $a$,. Частицы медленные $ka \ll 1$, и выполнены условия применимости борновского при-ближения. Как амплитуда рассеяния зависит от угла рассеяния?
\begin{choices}
\choice возрастает с ростом угла рассеяния     
\choice убывает с ростом угла рассеяния
\choice не зависит от угла рассеяния        
\choice это зависит от потенциала
\end{choices}

\question Частицы рассеиваются на некотором потенциале $U(\vec r)$, радиус действия которого равен $a$. Частицы медленные $ka \ll 1$, и выполнены условия применимости борновского при-ближения. Как амплитуда рассеяния зависит от энергии частиц?
\begin{choices}
\choice возрастает с ростом энергии         
\choice убывает с ростом энергии
\choice не зависит от энергии            
\choice это зависит от потенциала
\end{choices}

\question Частицы рассеиваются на некотором потенциале $U(\vec r)$, радиус действия которого равен $a$. Частицы быстрые $ka \gg 1$. Какое утверждение относительности зависимости ампли-туды от угла рассеяния справедливо?
\begin{choices}
\choice является резко возрастающей функцией угла
\choice является резко убывающей функцией угла
\choice не зависит от угла рассеяния
\choice это зависит от потенциала
\end{choices}

\question Частицы рассеиваются на некотором потенциале $U(\vec r)$, радиус действия которого равен $a$,. Частицы быстрые $ka \gg 1$. На какие углы в основном происходит рассеяние?
\begin{choices}
\choice вперед, в узкий конус с углом раствора $\Delta \vartheta  = 1/ka$
\choice назад, в узкий конус с углом раствора $\Delta \vartheta  = 1/ka$
\choice вперед, в узкий конус с углом раствора $\Delta \vartheta  = 1/{(ka)^2}$ 
\choice рассеяние является изотропным
\end{choices}

\question Частицы рассеиваются на некотором потенциале $U(\vec r)$, радиус действия которого равен $a$. Частицы быстрые $ka \gg 1$. Как сечение рассеяния на малые углы $\Delta \vartheta  \le 1/ka$зависит от энергии?
\begin{choices}
\choice растет с ростом энергии
\choice убывает с ростом энергии
\choice не зависит от энергии
\choice зависит от потенциала
\end{choices}

\question Частицы рассеиваются на некотором потенциале $U(\vec r)$, радиус действия которого равен $a$, характерный масштаб - ${U_0}$. Чему равна амплитуда рассеяния частиц на нулевой угол, если выполнены условия применимости борновского приближения?
\begin{choices}
\choice $f(\vartheta ) = \frac{{m{a^3}{U_0}}}{{2\pi {\hbar ^2}}}$      
\choice $f(\vartheta ) =  - \frac{{m{a^3}{U_0}}}{{2\pi {\hbar ^2}}}$
\choice $f(\vartheta ) = \frac{{m{a^2}{U_0}}}{{2\pi {\hbar ^2}}}$      
\choice $f(\vartheta ) =  - \frac{{m{a^2}{U_0}}}{{2\pi {\hbar ^2}}}$
\end{choices}

\question Выполняется ли в рамках первого борновского приближения оптическая теорема? 
\begin{choices}
\choice да
\choice нет
\choice только для медленных частиц, $ka \ll 1$
\choice только для быстрых частиц, $ka \gg 1$ 
\end{choices}

\question В рамках фазовой теории рассеяния проводится разложение волновой функции задачи рассеяния по состояниям
\begin{choices}
\choice с определенным импульсом
\choice с определенной энергией
\choice с определенным моментом
\choice с определенной координатой
\end{choices}

\question Частицы рассеиваются на некотором потенциале $U(\vec r)$. Фазовую теорию рассеяния можно использовать, если потенциальная энергия 
\begin{choices}
\choice мала
\choice велика
\choice резкая функция координаты
\choice независимо от потенциальной энергии
\end{choices}

\question Поток свободных частиц описывается волновой функцией ${e^{ikz}}$. Измеряют момент импульса частиц. Какие значения можно при этом получить?
\begin{choices}
\choice единственное значение $l = 0$
\choice все неотрицательные четные значения
\choice все положительные нечетные значения
\choice все неотрицательные целые значения 
\end{choices}

\question Поток свободных частиц описывается волновой функцией ${e^{ikz}}$. Измеряют проек-цию момента импульса частиц. Какие значения можно при этом получить?
\begin{choices}
\choice единственное значение $m = 0$
\choice положительные и отрицательные четные значения
\choice положительные и отрицательные нечетные значения
\choice все целые значения 
\end{choices}

\question Какие сферические функции ${Y_{lm}}$ входят в разложение волновой функции задачи рассеяния, имеющей следующую асимптотику ${e^{ikz}} + f(\vartheta ){e^{ikr}}/r$?
\begin{choices}
\choice ${Y_{l0}}$ ($l$ - любое целое неотрицательное число)
\choice ${Y_{0m}}$ ($m$ - любое целое число)
\choice ${Y_{lm}}$ ($l$ - четные целые неотрицательные числа, $m$ - нечетные)
\choice ${Y_{lm}}$ ($l$ и $m$ -все возможные значения)
\end{choices}

\question Потенциальная энергия частицы равна нулю. Какие из перечисленных функций являются решениями стационарного уравнения Шредингера при энергии $E$ и описывают частицы с определенным моментом и его проекцией на ось $z$ ($k = \sqrt {2mE/{\hbar ^2}} $, $m$ - масса частицы, $r = \sqrt {{x^2} + {y^2} + {z^2}} $)
\begin{choices}
\choice ${e^{ikz}}$    
\choice $\frac{{{\rm{tg}}\,kr}}{r}$   
\choice $\frac{{\sin kr}}{r}$   
\choice $\frac{{\cos kr}}{r}$
\end{choices}

\question Потенциальная энергия частицы равна нулю. Какие из нижеперечисленных функций яв-ляются приближенными решениями стационарного уравнения Шредингера при $r \to \infty $ и описывают состояния с определенным моментом $l$ и проекцией $\mu $ на ось $z$?
\begin{choices}
\choice $\frac{1}{r}\sin \left( {kr - \frac{{l\pi }}{2}} \right){Y_{l\mu }}(\vartheta ,\varphi )$    
\choice $\frac{1}{r}\sin \left( {kr + \frac{{l\pi }}{2}} \right){Y_{l\mu }}(\vartheta ,\varphi )$
\choice $\frac{1}{r}\cos \left( {kr - \frac{{l\pi }}{2}} \right){Y_{l\mu }}(\vartheta ,\varphi )$    
\choice $\frac{1}{r}\cos \left( {kr + \frac{{l\pi }}{2}} \right){Y_{l\mu }}(\vartheta ,\varphi )$
\end{choices}

\question Если потенциальная энергия частицы равна нулю, то функции $\frac{C}{r}\sin \left( {kr - \frac{{l\pi }}{2}} \right){Y_{l\mu }}(\vartheta ,\varphi )$ являются приближенными решениями ра-диального уравнения Шредингера с определенным моментом $l$ и проекцией $\mu $ при $r \to \infty $ ($C$ - постоянная, $k = \sqrt {\frac{{2mE}}{{{\hbar ^2}}}} $). Как изменятся эти решения, если рассматривается движение частиц в потенциале, быстро убывающем с расстоянием?
\begin{choices}
\choice изменится постоянная $C$
\choice эта функция останется асимптотическим решением, поскольку при $r \to \infty $ $U(r) = 0$
\choice в аргументе синуса появится дополнительный (к $l\pi /2$) фазовый сдвиг
\choice как изменится эта функция зависит от потенциальной энергии
\end{choices}

\question Частицы рассеиваются некоторым потенциалом. Фазы рассеяния ${\delta _0}$ и ${\delta _1}$ известны. Какой формулой определяется асимптотика радиальной волновой функции с мо-ментом $l = 0$ при $r \to \infty $?
\begin{choices}
\choice $\frac{1}{r}\sin \left( {kr + {\delta _0}} \right)$   
\choice $\frac{1}{r}\sin \left( {kr - \frac{\pi }{2} + {\delta _0}} \right)$    
\choice $\frac{1}{r}\sin \left( {kr + {\delta _1}} \right)$   
\choice $\frac{1}{r}\sin \left( {kr - \frac{\pi }{2} + {\delta _1}} \right)$
\end{choices}

\question Частицы рассеиваются некоторым потенциалом. Фазы рассеяния ${\delta _0}$ и ${\delta _1}$ известны. Какой формулой определяется асимптотика радиальной волновой функции с мо-ментом $l = 1$ при $r \to \infty $?
\begin{choices}
\choice $\frac{1}{r}\sin \left( {kr + {\delta _0}} \right)$   
\choice $\frac{1}{r}\sin \left( {kr - \frac{\pi }{2} + {\delta _0}} \right)$    
\choice $\frac{1}{r}\sin \left( {kr + {\delta _1}} \right)$   
\choice $\frac{1}{r}\sin \left( {kr - \frac{\pi }{2} + {\delta _1}} \right)$
\end{choices}

\question Что такое фазы рассеяния?
\begin{choices}
\choice фазовые множители в волновой функции задачи рассеяния
\choice сдвиг аргумента (фазы) амплитуды рассеяния по сравнению со случаем нулевого потенциала
\choice отдельные этапы (начальная фаза, собственно рассеяние и конечная фаза) процесса рассеяния
\choice сдвиг аргумента (фазы) синуса, описывающего асимптотику радиальной части волновой функ-ции задачи рассеяния с определенным моментом по сравнению со случаем нулевого потенциала
\end{choices}

\question Частицы рассеиваются на некотором потенциале. Сколько фаз рассеяния точно характеризуют этот процесс?
\begin{choices}
\choice три – начальная, промежуточная и конечная    
\choice одна – для оси $z$
\choice бесконечно много – для каждого момента    
\choice это зависит от потенциала
\end{choices}

\question Каким является выражение для сечения рассеяния через фазы рассеяния ${\delta _l}$?
\begin{choices}
\choice $f(\vartheta ) \sim \sum\limits_l {(2l + 1){P_l}(\cos \vartheta ){e^{2i{\delta _l}}}} $      
\choice $f(\vartheta ) \sim \sum\limits_l {(2l + 1){H_l}(\cos \vartheta ){e^{2i{\delta _l}}}} $
\choice $f(\vartheta ) \sim \sum\limits_l {{\delta _l}\,(2l + 1){H_l}(\cos \vartheta )} $      
\choice $f(\vartheta ) \sim \sum\limits_l {{\delta _l}\,(2l + 1){P_l}(\cos \vartheta )} $
(здесь ${P_l}(\cos \vartheta )$ - полиномы Лежандра, ${H_l}(\cos \vartheta )$ - полиномы Эрмита)
\end{choices}

\question Какой формулой определяется полное сечение упругого рассеяния?
\begin{choices}
\choice $\sigma  \sim \sum\limits_l {(2l + 1){{\sin }^2}{\delta _l}} $    
\choice $\sigma  \sim \sum\limits_l {(2l + 1){{\cos }^2}{\delta _l}} $
\choice $\sigma  \sim \sum\limits_l {(2l + 1){\rm{t}}{{\rm{g}}^2}{\delta _l}} $    
\choice $\sigma  \sim \sum\limits_l {(2l + 1){\rm{ct}}{{\rm{g}}^2}{\delta _l}} $
\end{choices}

\question Что означает индекс $k$ в фазе рассеяния ${\delta _k}$?
\begin{choices}
\choice волновой вектор         
\choice момент
\choice проекцию момента     
\choice номер фазы
\end{choices}

\question Как фаза рассеяния зависит от угла рассеяния?
\begin{choices}
\choice растет      
\choice убывает     
\choice не зависит     
\choice это зависит от потенциала
\end{choices}

\question Частицы рассеиваются на некотором потенциале отталкивания $U(\vec r)$, радиус дейст-вия которого равен $a$. Частицы медленные $ka \ll 1$. Как фаза $s$-рассеяния ${\delta _0}$ зави-сит от волнового вектора $k$?
\begin{choices}
\choice не зависит от $k$    
\choice как $k$  
\choice как ${k^2}$ 
\choice как ${k^{ - 1}}$ 
\end{choices}

\question Частицы рассеиваются на потенциале $U(\vec r) = \left\{ {\begin{array}{*{20}{c}}
{\infty ,\quad r < R}\\
{0,\quad r > R}
\end{array}} \right.$. Найти фазу рассеяния $s$-волны.
\begin{choices}
\choice ${\delta _0} = kR$       
\choice ${\delta _0} = {(kR)^2}$ 
\choice ${\delta _0} =  - kR$ 
\choice ${\delta _0} =  - {(kR)^2}$
\end{choices}

\question Частицы рассеиваются на потенциале $U(\vec r) = \left\{ {\begin{array}{*{20}{c}}
{\infty ,\quad r < R}\\
{0,\quad r > R}
\end{array}} \right.$. Найти дифференциальное сечение рассеяния медленных частиц ($E \to 0$).
\begin{choices}
\choice $\frac{{d\sigma }}{{d\Omega }}\left( \vartheta  \right) = {R^2}$   
\choice $\frac{{d\sigma }}{{d\Omega }}\left( \vartheta  \right) = {R^2}/2$ 
\choice $\frac{{d\sigma }}{{d\Omega }}\left( \vartheta  \right) = \pi {R^2}$  
\choice $\frac{{d\sigma }}{{d\Omega }}\left( \vartheta  \right) = \pi {R^2}/2$
\end{choices}

\question Частицы рассеиваются на потенциале $U(\vec r) = \left\{ {\begin{array}{*{20}{c}}
{\infty ,\quad r < R}\\
{0,\quad r > R}
\end{array}} \right.$. Найти полное сечение рассеяния медленных частиц ($E \to 0$).
\begin{choices}
\choice $\sigma  = 2\pi {R^2}$      
\choice $\sigma  = 2{\pi ^2}{R^2}$     
\choice $\frac{{d\sigma }}{{d\Omega }}\left( \vartheta  \right) = 2{\pi ^2}{R^2}$   
\choice $\sigma  = 4\pi {R^2}$
\end{choices}

\question Частицы рассеиваются на некотором потенциале. Известно, что фаза рассеяния ${\delta _0}$ не равна нулю, все остальные фазы рассеяния равны нулю. Как зависит от угла рассеяния $\vartheta $ дифференциальное сечение рассеяния?
\begin{choices}
\choice не зависит        
\choice как $\cos \vartheta $
\choice как $1/\cos \vartheta $    
\choice как $A + B\cos \vartheta $ 
(здесь $A$ и $B$ - постоянные)
\end{choices}

\question Частицы рассеиваются на некотором потенциале. Известно, что фазы рассеяния ${\delta _0}$ и ${\delta _1}$ не равна нулю, все остальные фазы рассеяния равны нулю. Как зависит от уг-ла рассеяния $\vartheta $ дифференциальное сечение рассеяния?
\begin{choices}
\choice не зависит        
\choice как $\cos \vartheta $
\choice как $1/\cos \vartheta $    
\choice как $A + B\cos \vartheta $
(здесь $A$ и $B$ - постоянные)
\end{choices}

\question Частицы рассеиваются на некотором потенциале $U(\vec r)$, радиус действия которого равен $a$. Частицы медленные $ka \ll 1$. Как ведут себя фазы рассеяния ${\delta _l}$ в зависимо-сти от $l$?
\begin{choices}
\choice растут с ростом $l$     
\choice убывают с ростом $l$
\choice не зависят от $l$    
\choice зависит от потенциала
\end{choices}

\question Частицы рассеиваются на некотором потенциале $U(\vec r)$, радиус действия которого равен $a$. Частицы – не медленные $ka \ge 1$. Как ведут себя фазы рассеяния ${\delta _l}$ в зави-симости от $l$?
\begin{choices}
\choice растут с ростом $l$     
\choice убывают с ростом $l$
\choice не зависят от $l$    
\choice это зависит от потенциала
\end{choices}

\question $S$-оператор ($S$ - матрица) рассеяния связывает
\begin{choices}
\choice энергию падающей и рассеянной частиц
\choice амплитуду падающих и рассеянных частиц
\choice момент падающих и рассеянных частиц
\choice волновой вектор падающих и рассеянных частиц
\end{choices}

\question Частицы рассеиваются на некотором потенциале $U(\vec r)$. Рассмотрим решение ста-ционарного уравнения Шредингера, которое имеет асимптотику (при $r \to \infty $) $F(\vartheta ,\varphi )\frac{{{e^{ - ikr}}}}{r} - {F_1}(\vartheta ,\varphi )\frac{{{e^{ikr}}}}{r}$, где $F(\vartheta ,\varphi )$ и ${F_1}(\vartheta ,\varphi )$ - амплитуды падающей и рассеянной волн. Как называет-ся оператор, связывающий функции $F(\vartheta ,\varphi )$ и ${F_1}(\vartheta ,\varphi )$?
\begin{choices}
\choice $Q$ - оператор       
\choice $R$ - оператор
\choice $S$ - оператор       
\choice $T$ - оператор
\end{choices}

\question Каким является $S$ - оператор ($S$-матрица), если потенциальная энергия равна нулю?
\begin{choices}
\choice единичным         
\choice нулевым
\choice «минус единичным»    
\choice в этом случае $S$ - оператор невозможно определить
\end{choices}

\question $S$ - оператор ($S$-матрица) является
\begin{choices}
\choice эрмитовым
\choice унитарным   
\choice совпадающим со своим обратным
\choice другим
\end{choices}

\question Частицы рассеиваются на некотором потенциале $U(\vec r)$. Рассмотрим решение ста-ционарного уравнения Шредингера, которое имеет асимптотику (при $r \to \infty $) $F(\vartheta ,\varphi )\frac{{{e^{ - ikr}}}}{r} - {F_1}(\vartheta ,\varphi )\frac{{{e^{ikr}}}}{r}$, где $F(\vartheta ,\varphi )$ и ${F_1}(\vartheta ,\varphi )$ - амплитуды падающей и рассеянной волн. К унитарно-сти $S$ - оператора ($S$-матрицы), приводит то обстоятельство, что
\begin{choices}
\choice энергии первого и второго слагаемого одинаковы
\choice импульсы первого и второго слагаемого одинаковы
\choice моменты первого и второго слагаемого одинаковы
\choice нормировки первого и второго слагаемого одинаковы
\end{choices}

\question Как связаны друг с другом фазы рассеяния ${\delta _l}$ и диагональные матричные эле-менты $S$ - оператора ($S$-матрицы) ${S_l}$?
\begin{choices}
\choice ${S_l} = \sin {\delta _l}$    
\choice ${S_l} = {e^{i{\delta _l}}}$     
\choice ${S_l} = \sin 2{\delta _l}$      
\choice ${S_l} = {e^{2i{\delta _l}}}$
\end{choices}

\question Частицы рассеиваются некоторым потенциалом. Фазы рассеяния являются
\begin{choices}
\choice действительными
\choice чисто мнимыми
\choice их мнимая часть обязательно положительна
\choice их мнимая часть обязательно отрицательна
\end{choices}

\question Что можно сказать о диагональных матричных элементах $S$-матрицы при наличии не-упругих процессов (каналов реакций)?
\begin{choices}
\choice $|{S_l}| > 1$        
\choice $|{S_l}| < 1$
\choice ${S_l}$ - действительны 
\choice $|{S_l}| = 1$
\end{choices}

\question Какое выражение является правильным обобщением оптической теоремы на случай неуп-ругого рассеяния?
\begin{choices}
\choice ${\mathop{\rm Im}\nolimits} f(0) = \frac{k}{{4\pi }}{\sigma _e}$     
\choice ${\mathop{\rm Im}\nolimits} f(0) = \frac{k}{{4\pi }}{\sigma _r}$     
\choice ${\mathop{\rm Im}\nolimits} f(0) = \frac{k}{{4\pi }}{\sigma _t}$
\choice в случае неупругого рассеяния оптическая теорема ни в каком виде не имеет места
(здесь ${\sigma _e}$, ${\sigma _r}$ и ${\sigma _t}$ - сечение упругого рассеяния, сечение реакций и полное сечение рассеяния)?
\end{choices}

\question Пусть диагональные матричные элементы $S$-матрицы ${S_l}$ при наличии неупругих процессов (каналов реакций) известны. Какой формулой определяется полное сечение упругого рассеяния ${\sigma _e}$?
\begin{choices}
\choice $ \sim \sum\limits_l {(2l + 1)\left( {1 - {{\left| {{S_l}} \right|}^2}} \right)} $  
\choice $ \sim \sum\limits_l {(2l + 1){{\left| {1 - {S_l}} \right|}^2}} $    
\choice $ \sim \sum\limits_l {(2l + 1)\left( {1 - {\mathop{\rm Re}\nolimits} {S_l}} \right)} $ 
\choice другой
\end{choices}

\question Пусть диагональные матричные элементы $S$-матрицы ${S_l}$ при наличии неупругих процессов (каналов реакций) известны. Какой формулой определяется сечение реакций ${\sigma _r}$?
\begin{choices}
\choice $ \sim \sum\limits_l {(2l + 1)\left( {1 - {{\left| {{S_l}} \right|}^2}} \right)} $  
\choice $ \sim \sum\limits_l {(2l + 1){{\left| {1 - {S_l}} \right|}^2}} $    
\choice $ \sim \sum\limits_l {(2l + 1)\left( {1 - {\mathop{\rm Re}\nolimits} {S_l}} \right)} $ 
\choice другой
\end{choices}

\question Пусть диагональные матричные элементы $S$-матрицы ${S_l}$ при наличии неупругих процессов (каналов реакций) известны. Какой формулой определяется полное сечение упругого рассеяния плюс реакций ${\sigma _t}$?
\begin{choices}
\choice $ \sim \sum\limits_l {(2l + 1)\left( {1 - {{\left| {{S_l}} \right|}^2}} \right)} $  
\choice $ \sim \sum\limits_l {(2l + 1){{\left| {1 - {S_l}} \right|}^2}} $    
\choice $ \sim \sum\limits_l {(2l + 1)\left( {1 - {\mathop{\rm Re}\nolimits} {S_l}} \right)} $ 
\choice другой
\end{choices}

\question Значение диагонального матричного элемента ${S_l} = 1$ отвечает
\begin{choices}
\choice отсутствию рассеяния    
\choice полному поглощению частиц
\choice рассеянию всех частиц      
\choice это зависит от потенциала
\end{choices}

\question Значение диагонального матричного элемента ${S_l} = 0$ отвечает
\begin{choices}
\choice отсутствию рассеяния    
\choice полному поглощению частиц
\choice рассеянию всех частиц      
\choice это зависит от потенциала
\end{choices}

\question «Закон $1/v$» говорит о том, что
\begin{choices}
\choice сечение упругого рассеяния при малых энергиях обратно пропорционально скорости
\choice сечение упругого рассеяния при больших энергиях обратно пропорционально скорости
\choice сечение неупругого рассеяния при малых энергиях обратно пропорционально скорости
\choice сечение неупругого рассеяния при больших энергиях обратно пропорционально скорости
\end{choices}

\question В чем причина возрастания сечения неупругого рассеяния при малых энергиях как $1/v$:
\begin{choices}
\choice при малой скорости легче попасть в ту область, где происходят реакции
\choice при малой скорости частицы сильнее взаимодействуют
\choice частица дольше находится в области действия потенциала, и возрастает вероятность реакций
\choice Другая
\end{choices}

\end{questions}

\end{document}


